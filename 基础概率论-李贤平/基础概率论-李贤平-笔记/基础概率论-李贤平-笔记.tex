\documentclass[lang = cn, scheme = chinese, thmcnt = section]{elegantbook}
% elegantbook      设置elegantbook文档类
% lang = cn        设置中文环境
% scheme = chinese 设置标题为中文
% thmcnt = section 设置计数器


%% 1.封面设置

\title{基础概率论 - 李贤平 - 笔记}                % 文档标题

\author{若水}                        % 作者

\myemail{ethanmxzhou@163.com}       % 邮箱

\homepage{helloethanzhou.github.io} % 主页

\date{\today}                       % 日期

\logo{PiCreatures_happy.pdf}        % 设置Logo

\cover{阿基米德螺旋曲线.pdf}          % 设置封面图片

% 修改标题页的色带
\definecolor{customcolor}{RGB}{135, 206, 250} 
% 定义一个名为customcolor的颜色,RGB颜色值为(135, 206, 250)

\colorlet{coverlinecolor}{customcolor}     % 将coverlinecolor颜色设置为customcolor颜色

%% 2.目录设置
\setcounter{tocdepth}{3}  % 目录深度为3

%% 3.引入宏包
\usepackage[all]{xy}
\usepackage{bbm, svg, graphicx, float, extpfeil, amsmath, amssymb, mathrsfs, mathalpha, hyperref}


%% 4.定义命令
\newcommand{\N}{\mathbb{N}}            % 自然数集合
\newcommand{\R}{\mathbb{R}}            % 实数集合
\newcommand{\C}{\mathbb{C}}  		   % 复数集合
\newcommand{\Q}{\mathbb{Q}}            % 有理数集合
\newcommand{\Z}{\mathbb{Z}}            % 整数集合
\newcommand{\sub}{\subset}             % 包含
\newcommand{\im}{\text{im }}           % 像
\newcommand{\lang}{\langle}            % 左尖括号
\newcommand{\rang}{\rangle}            % 右尖括号
\newcommand{\bs}{\boldsymbol}          % 向量加黑
\newcommand{\dd}{\mathrm{d}}           % 微分d
\newcommand{\toP}{\xlongrightarrow{\mathrm{P}}}
\newcommand{\toas}{\xlongrightarrow{\mathrm{a.s.}}}
\newcommand{\tod}{\xlongrightarrow{\mathrm{d}}}
\newcommand{\tov}{\xlongrightarrow{\mathrm{v}}}
\newcommand{\tow}{\xlongrightarrow{\mathrm{w}}}
\newcommand{\toLp}{\xlongrightarrow{L^p}}
\newcommand{\dis}{\displaystyle}
\newcommand{\pll}{\kern 0.56em/\kern -0.8em /\kern 0.56em} % 平行
\newcommand{\function}[5]{
	\begin{align*}
		#1:\begin{aligned}[t]
			#2 &\longrightarrow #3\\
			#4 &\longmapsto #5
		\end{aligned}
	\end{align*}
}                                     % 函数

\newcommand{\lhdneq}{%
	\mathrel{\ooalign{$\lneq$\cr\raise.22ex\hbox{$\lhd$}\cr}}} % 真正规子群

\newcommand{\rhdneq}{%
	\mathrel{\ooalign{$\gneq$\cr\raise.22ex\hbox{$\rhd$}\cr}}} % 真正规子群

%% 5.参考文献

\addbibresource[location=local]{reference.bib} % 添加本地的参考文献文件reference.bib

\begin{document}
	
\maketitle       % 创建标题页

\frontmatter     % 开始前言部分

\chapter*{致谢}

\markboth{致谢}{致谢}

\vspace*{\fill}
	\begin{center}
		
		\large{感谢 \textbf{ 勇敢的 } 自己}
		
	\end{center}
\vspace*{\fill}

\tableofcontents % 创建目录

\mainmatter      % 开始正文部分

\chapter{事件与概率}

\section{随机现象与统计规律性}

\textbf{必然事件}:在一定条件下,必然会发生的事情。

\textbf{不可能事件}:在一定条件下,不然不会发生的事情。

\textbf{随机现象}:在基本条件不变的情况下,一系列试验或观察会得到不同的结果的现象。

\textbf{随机事件}:随机现象出现的结果。

\textbf{频率}:对于随机事件$A$,若在$N$次试验中出现了$n$次,则称$F_N(A)=n/N$为随机事件$A$在$N$次试验中出现的频率。

\textbf{概率}:对于随机事件$A$,表示该事件发生的可能性大小的数$P(A)$称为该事件的概率。

\begin{proposition}{频率的性质}
	\begin{enumerate}
		\item $F_N(A)\ge 0$
		\item 若记必然事件为$\Omega$,则有$F_N(\Omega)=1$。
		\item 若$A\cap B=\varnothing$,则$F_N(A\cup B)=F_N(A)+F_N(B)$
	\end{enumerate}
\end{proposition}

\begin{proposition}{概率的性质}
	\begin{enumerate}
		\item 非负性:$P(A)\ge 0$
		\item 规范性:$P(\Omega)=1$
		\item 有限可加性:若$A_1,\cdots,A_n$两两不相容,则
		$$
		P\left(\sum_{k=1}^{n}{A_k}\right)=\sum_{k=1}^{n}{P(A_k)}
		$$
	\end{enumerate}
\end{proposition}

\section{样本空间与事件}

\textbf{样本点}:随机试验可能出现的结果称为样本点,记作$\omega$。

\textbf{样本空间}:样本点全体构成样本空间,记作$\Omega$。

\textbf{样本空间的类型}:

\begin{enumerate}
	\item 有限个样本点
	\item 无穷可列个样本点
	\item 无穷不可列个样本点
	\item 三维空间
	\item 函数空间
\end{enumerate}

\textbf{事件}:某些样本点构成的集合。

\section{古典概型}

\begin{proposition}{模型与计算公式}{模型与计算公式}
	\begin{enumerate}
		\item $P(\omega_k)=\frac{1}{n}$,其中$k=1,\cdots,n$。
		\item 若$A=\omega_{k_1}+\cdots+\omega_{k_m}$,则$P(A)=\frac{m}{n}$。
	\end{enumerate}
\end{proposition}

\begin{proposition}{组合公式}
	\begin{enumerate}
		\item 排列
		\begin{enumerate}
			\item 有放回的选取并排列:$n^r$
			\item 不放回的选取并排列:$A_n^r=n(n-1)\cdots(n-r+1)=\frac{n!}{(n-r)!}$
			\item 全排列:$n!$
			\item 圆排列:$(n-1)!$
		\end{enumerate}
		\item 组合
		\begin{enumerate}
			\item 不放回的选取:$C_n^r={n\choose r}=\frac{n(n-1)\cdots(n-r+1)}{r!}=\frac{n!}{r!(n-r)!}$
			\item 分为多个部分:$\frac{n!}{r_1!\cdots r_k!}$
			\item 有放回的选取:${n+r-1}\choose r$
			\item 全错排:$a_n=n!(\frac{1}{0!}-\frac{1}{1!}+\frac{1}{2!}-\ldots+(-1)^n\frac{1}{n!})$
		\end{enumerate}
		\item 组合公式
		\begin{enumerate}
			\item ${n\choose k}={n\choose{n-k}}$
			\item ${n\choose 0}+\cdots+{n\choose n}=2^n$
			\item ${a\choose 0}{b\choose n}+\cdots+{a\choose n}{b\choose 0}={{a+b}\choose n}$
			\item ${{-\alpha}\choose{k}}=(-1)^k{{\alpha+k-1}\choose{k}}$
		\end{enumerate}
	\end{enumerate}
\end{proposition}

\section{几何概率}

\textbf{计算公式}:$P(A)=m(A)/m(\Omega)$

\begin{proposition}{几何概率的性质}
	\begin{enumerate}
		\item 非负性:$P(A)\ge 0$
		\item 规范性:$P(\Omega)=1$
		\item 可数可加性:若$A_1,A_2,\cdots$两两不相容,则%
		$$
		P\left(\sum_{n=1}^{\infty}{A_n}\right)=\sum_{n=1}^{\infty}{P(A_n)}
		$$
	\end{enumerate}
\end{proposition}

\section{概率空间}

\begin{definition}{$\sigma$-域}
	称集族$\Sigma\sub\mathscr{P}(\Omega)$为样本空间$\Omega$上的$\sigma$-域,如果成立
	\begin{enumerate}
		\item $\varnothing\in\Sigma$
		\item 如果$A\in\Sigma$,那么$\Omega\setminus A\in\Sigma$;
		\item 如果$\{ A_n \}_{n=1}^{\infty}\sub \Sigma$,那么
		$$
		\bigcup_{n=1}^{\infty}{A_n}\in\Sigma
		$$
	\end{enumerate}
\end{definition}

\begin{definition}{概率}
	对于样本空间$\Omega$上的$\sigma$-域$\Sigma\sub\mathscr{P}(\Omega)$,称映射$P:\Sigma\to [0,1]$为概率,如果成立如下命题。
	\begin{enumerate}
		\item $P(\Omega)=1$
		\item 对于不相容序列$\{A_n\}_{n=1}^{\infty}\sub \Sigma$,成立
		$$
		P\left(\sum_{n=1}^{\infty}A_n\right)=\sum_{n=1}^{\infty}P(A_n)
		$$
	\end{enumerate}
\end{definition}

\begin{definition}{概率空间}
	称$(\Omega,\Sigma,P)$为测度空间,如果$\Omega$为样本空间,$\Sigma\sub\mathscr{P}(\Omega)$为$\Omega$上的$\sigma$-代数,映射$P:\Sigma\to \R$为概率。
\end{definition}

\begin{definition}{事件域}
	称$\Sigma$为概率空间$(\Omega,\Sigma,P)$的事件域,其中的元素称为事件。
\end{definition}

\begin{proposition}{概率的性质}
	\begin{enumerate}
		\item 对于任意$A\in\Omega$,成立$P(A^c)=1-P(A)$。
		\item 如果$B\sub A$,那么$P(A\setminus B)=P(A)-P(B)$。
		\item $P(A\cup B)+P(A\cap B)=P(A)+P(B)$
		\item Bool不等式:$P(A\cup B)\le P(A)+P(B)$
		\item Bonferroni不等式:$P(A\cap B)\ge P(A)+P(B)-1$
		\item 下连续性:对于单调递减事件序列$\{A_n\}_{n=1}^{\infty}$,成立
		$$
		P\left(\lim_{n\to\infty}{A_n}\right)=\lim_{n\to\infty}{P(A_n)}
		$$
		\item 上连续性:对于单调递增事件序列$\{A_n\}_{n=1}^{\infty}$,成立
		$$
		P\left(\lim_{n\to\infty}{A_n}\right)=\lim_{n\to\infty}{P(A_n)}
		$$
		\item $\dis P\left(\sum_{n=1}^{\infty}{A_n}\right)\le\sum_{n=1}^{\infty}{P(A_n)}$
	\end{enumerate}
\end{proposition}

\chapter{条件概率与统计独立性}

\section{条件概率,全概率公式,Bayes公式}

\begin{definition}{条件概率}
	对于概率空间$(\Omega,\Sigma,P)$,如果事件$B\in\Sigma$成立$P(B)>0$,那么对于事件$A\in\Sigma$,称%
	$$
	P(A\mid B)=\frac{P(AB)}{P(B)}
	$$
	为在事件$B$发生的条件下事件$A$发生的条件概率。
\end{definition}

\begin{proposition}{条件概率的性质}
	\begin{enumerate}
		\item 非负性:$P(A \mid B)\ge 0$
		\item 规范性:$P(\Omega \mid B)=1$
		\item 可数可加性:若$A_1,A_2,\cdots$两两不相容,则$\dis P\left(\sum_{k=1}^{\infty}{A_k} \mid B\right)=\sum_{k=1}^{\infty}{P(A_k \mid B)}$
		\item 乘法原理:
		$$
		P(AB)=P(B)P(A \mid B)
		$$
		\item 推广的乘法公式:
		$$
		P(A_1\cdots A_n)=P(A_1)P(A_2 \mid A_1)P(A_3 \mid A_1A_2)\cdots P(A_n \mid A_1A_2\cdots A_{n-1})
		$$
	\end{enumerate}
\end{proposition}

\begin{theorem}{全概率公式}{全概率公式}
	对于样本空间$\Omega$的划分%
	$$
	\Omega=\bigsqcup_{n=1}^{\infty}A_n
	$$
	那么对于事件$B$,成立全概率公式%
	$$
	P(B)=\sum_{k=1}^{\infty}{P(A_k)P(B \mid A_k)}
	$$
\end{theorem}

\begin{theorem}{Bayes公式}
	若事件$A_1,A_2,\cdots$两两不相容,且对于事件$B=\sum_{k=1}^{\infty}{A_kB}$,存在Bayes公式
	$$
	P(A_k \mid B)=\frac{P(A_k)P(B \mid A_k)}{\dis\sum_{i=1}^{\infty}{P(A_i)P(B \mid A_i)}}
	$$
\end{theorem}

\section{事件独立性}

\begin{definition}{两个事件的独立性}
	称事件$A$和$B$是统计独立的,如果
	$$
	P(AB)=P(A)P(B)
	$$
\end{definition}

\begin{definition}{三个事件的独立性}
	称事件$A,B,C$是统计独立的,如果
	$$
	\begin{cases}
		P(AB)=P(A)P(B)\\
		P(BC)=P(B)P(C)\\
		P(CA)=P(C)P(A)\\
		P(ABC)=P(A)P(B)P(C)
	\end{cases}
	$$
\end{definition}

\begin{proposition}
	\begin{enumerate}
		\item 若事件$A,B$独立,且$P(B)>0$,则$P(A \mid B)=P(A)$。
		\item 若事件$A,B$独立,则事件$A,B^c$独立。
	\end{enumerate}
\end{proposition}

\begin{definition}{试验的独立性}
	令$\mathscr{A}_k$为第$k$次试验有关的事件全体。称试验$\mathscr{A}_1,\cdots,\mathscr{A}_n$是相互独立的,如果对于任意的$A_k\in \mathscr{A}_k$,成立
	$$
	P(A_1\cdots A_n)=P(A_1)\cdots P(A_n)
	$$
\end{definition}

\section{Bernoulli试验与直线上的随机游动}

\begin{definition}{Bernoulli试验}
	只存在两种可能结果的试验称为Bernoulli试验。
\end{definition}

\begin{definition}{$n$重Bernoulli试验}
	称$n$次Bernoulli试验为$n$重Bernoulli试验,如果成立如下命题。
	\begin{enumerate}
		\item 每次实验至多出现两个可能结果$A$与$A^c$之一。
		\item $A$在每次试验中出现的概率$p$不变。
		\item 各次试验相互独立。
	\end{enumerate}
\end{definition}

\begin{proposition}{几何分布的无记忆性}
	在Bernoulli试验中,已知在前$n$次试验中没有成功,则首次成功所在需要的时间满足几何分布。在离散型分布中,仅有几何分布满足此性质。
\end{proposition}

\begin{theorem}{分赌注问题}
	甲、乙两个赌徒中止赌博,若甲再胜$n$场则可赢得赌注,乙再胜$m$场则可赢得赌注。甲在每局获胜的概率为$p$,则甲赢得赌注的概率为
	\begin{align*}
		p_{\text{甲}}&=\sum_{k=0}^{m-1}{{{n+k-1}\choose{k}}p^n(1-p)^k}\\
		&=\sum_{k=n}^{\infty}{{{m+k-1}\choose{k}}p^k(1-p)^m}\\
		&=\sum_{k=n}^{n+m-1}{{{n+m-1}\choose{k}}p^k(1-p)^{n+m-1-k}}
	\end{align*}
\end{theorem}

\begin{theorem}{直线上的随机游动}
	考虑$x$轴上的一个质点,规定其只能位于整数点,在$t=0$时刻,位于初始位置$x=a\in \Z$,以后每隔单位时间,分别以概率$p$及$1-p$向正的或负的方向移动一个单位。
	\begin{enumerate}
		\item 无限制随机游动:若质点在$t=0$时刻从原点出发,质点在$t=n$时刻位于$k$的概率为
		$$
		p(n,k)=\begin{cases}
			{{n}\choose{\frac{n+k}{2}}}p^{\frac{n+k}{2}}(1-p)^{\frac{n-k}{2}},\qquad & n\equiv k\mod 2\\
			0,\qquad & n\not\equiv k\mod 2
		\end{cases}
		$$
		\item 两端带有吸收壁的随机游动:若质点在$t=0$时刻从$x=n\in(a,b)$出发,而在$x=a\in \Z$和$x=b\in \Z$处各有一个吸收壁,质点碰到吸收壁后将不再运动,则质点最终在$a$点被吸收的概率为
		$$
		f_a(n)=\begin{cases}
			\frac{(\frac{1-p}{p})^n-(\frac{1-p}{p})^b}{(\frac{1-p}{p})^a-(\frac{1-p}{p})^b},&\qquad p\ne\frac{1}{2}\\
			\frac{n-b}{a-b},&\qquad p=\frac{1}{2}
		\end{cases}
		$$
		质点最终在$b$点被吸收的概率为
		$$
		f_b(n)=\begin{cases}
			\frac{(\frac{1-p}{p})^n-(\frac{1-p}{p})^a}{(\frac{1-p}{p})^b-(\frac{1-p}{p})^a},&\qquad p\ne\frac{1}{2}\\
			\frac{n-a}{b-a},&\qquad p=\frac{1}{2}
		\end{cases}
		$$
	\end{enumerate}
\end{theorem}

\begin{definition}{多项分布}
	$n$次重复独立试验且每次试验出现的可能结果为$A_1,\cdots,A_r$,其中$P(A_k)=p_k\ge 0$且$p_1+\cdots+p_r=1$,则$n$次试验中$A_k$出现$n_k$次的概率为
	$$
	P=\frac{n!}{n_1!\cdots n_r!}p_1^{n_1}\cdots p_r^{n_r}
	$$
	其中$n_k\ge 0$且$n_1+\cdots+n_r=n$。
\end{definition}

\section{二项分布与Poisson分布}

\begin{proposition}
	函数$b(k;n,p)$关于$k$先递增后递减,且
	$$
	b_{\max}(k;n,p)=b([(n+1)p];n,p)
	$$
\end{proposition}

\begin{theorem}{Poisson定理}
	在独立实验中,以与$n$有关的常数$p_n$代表事件在实验中出现的概率。若$np_n\to\lambda$,则当$n\to\infty$时,
	$$
	b(k;n,p)\to\frac{\lambda^k}{k!}\mathrm{e}^{-\lambda}
	$$
\end{theorem}

\chapter{随机变量与分布函数}

\section{随机变量与其分布}

\subsection{随机变量}

\begin{definition}{随机变量}
	对于概率空间$(\Omega,\Sigma,P)$,称函数$\xi:\Omega\to \R$随机变量,如果
	$$
	\xi^{-1}(\mathscr{B})\sub \Sigma
	$$
\end{definition}

\begin{definition}{概率分布}
	对于概率空间$(\Omega,\Sigma,P)$,定义随机变量$\xi$在Borel集$B\in \mathscr{B}$上的概率分布为$P\{ \xi^{-1}(B) \}$。
\end{definition}

\begin{definition}{分布函数}
	对于概率空间$(\Omega,\Sigma,P)$,定义随机变量$\xi$的分布函数为%
	$$
	F(x)=P\{ \xi^{-1}(-\infty,x) \},\qquad x\in\overline{\R}
	$$
\end{definition}

\begin{proposition}{分布函数的性质}
	\begin{enumerate}
		\item $F(-\infty)=0$,$F(+\infty)=1$
		\item 单调性:若$a<b$,则$F(a)\le F(b)$。
		\item 左连续性:$F(x^-)=F(x)$
		\item 分布函数至多仅有可列个不连续点。
		\item 对于分布函数$F(x)$存在Lebesgue分解
		$$
		F(x)=c_1F_1(x)+c_2F_2(x)+c_3F_3(x)
		$$
		其中$F_1(x)$为跳跃函数,$F_1(x)$为绝对连续函数,$F_1(x)$为奇异函数。
	\end{enumerate}
\end{proposition}

\begin{theorem}{随机变量的存在性定理}
	如果$F(x)$是左连续的单调不减函数,且$F(-\infty)=0,F(+\infty)=1$,那么存在概率空间$(\Omega,\Sigma,P)$及其上的随机变量$\xi$,使$\xi$的分布函数为$F(x)$。
\end{theorem}

\subsection{离散型随机变量}

\begin{definition}{概率分布}
	令$\{x_k\}_{k=1}^{\infty}$为离散型随机变量$\xi$的所有可能值,$p(x_k)$是$\xi$取$x_k$的概率,称$\{p(x_k):k\in\N^*\}$为随机变量$\xi$的概率分布。
\end{definition}

\begin{definition}{分布函数}
	$$
	F(x)=\sum_{x_k<x}{p(x_k)}
	$$
\end{definition}

\subsection{连续型随机变量}

\begin{definition}{分布密度函数}
	$$
	p(x),\qquad x\in\R
	$$
\end{definition}

\begin{definition}{分布函数}
	$$
	F(x)=\int_{-\infty}^{x}p(t)\mathrm{d}t,\qquad x\in\R
	$$
\end{definition}

\begin{proposition}{连续型随机变量的性质}
	\begin{enumerate}
		\item $p(x)\ge 0$
		\item $\int_{-\infty}^{\infty}{p(x)\mathrm{d}x}=1$
		\item $P\{a\le \xi <b\}=F(b)-F(a)=\int_{a}^{b}{p(x)\mathrm{d}x}$
		\item $P\{\xi=c\}=0$
	\end{enumerate}
\end{proposition}

\begin{definition}{正态分布$N(\mu,\sigma^2)$}
	\begin{enumerate}
		\item 密度函数:
		$$
		p(x)=\frac{1}{\sqrt{2\pi}\sigma}\mathrm{e}^{-\frac{(x-\mu)^2}{2\sigma^2}},\qquad x\in\R
		$$
		\item 分布函数:
		$$
		F(x)=\frac{1}{\sqrt{2\pi}\sigma}\int_{-\infty}^{x}{\mathrm{e}^{-\frac{(t-\mu)^2}{2\sigma^2}}\mathrm{d}t},\qquad x\in\R
		$$
	\end{enumerate}
\end{definition}

\begin{definition}{标准正态分布$N(0,1)$}
	记标准正态分布$N(0,1)$分布密度函数与分布函数分别为$\varphi(x)$和$\Phi(x)$,那么
	\begin{align*}
		& \varphi(x)=\frac{1}{\sqrt{2\pi}}\mathrm{e}^{-\frac{x^2}{2}},&& x\in\R\\
		& \Phi(x)=\frac{1}{\sqrt{2\pi}}\int_{-\infty}^{x}{\mathrm{e}^{-\frac{t^2}{2}}\mathrm{d}t},&& x\in\R
	\end{align*}
	那么对于正态分布$N(\mu,\sigma^2)$的分布密度函数$p(x)$与分布函数$F(x)$,成立
	$$
	p(x)=\frac{1}{\sigma}\varphi\left(\frac{x-\mu}{\sigma}\right),\qquad 
	F(x)=\Phi\left(\frac{x-\mu}{\sigma}\right)
	$$
\end{definition}

\section{随机向量,随机变量的独立性}

\subsection{随机向量及其分布}

\begin{definition}{$n$维随机向量}
	对于概率空间$(\Omega,\Sigma,P)$上的随机变量$\xi_1,\cdots,\xi_n$,定义$n$维随机向量为%
	$$
	\boldsymbol{\xi}=(\xi_1,\cdots,\xi_n)
	$$
\end{definition}

\begin{definition}{联合分布函数}
	对于概率空间$(\Omega,\Sigma,P)$,称$n$元函数
	$$
	F(x_1,\cdots,x_n)=P\{ \boldsymbol{\xi}^{-1}((-\infty,x_1)\times\cdots\times(-\infty,x_n)) \}
	$$
	为随机向量$\boldsymbol{\xi}=(\xi_1,\cdots,\xi_n)$的联合分布函数。
\end{definition}

\begin{proposition}{多元分布函数的性质}
	\begin{enumerate}
		\item 单调性:关于每个变元是单调不减函数。
		\item $F(x_1,\cdots,-\infty,\cdots,x_n)=0$
		$F(+\infty,\cdots,+\infty)=1$
		\item 左连续性:关于每个变元左连续。
	\end{enumerate}
\end{proposition}

\begin{definition}{多元分布密度函数}
	对于连续型随机向量$\boldsymbol{\xi}=(\xi_1,\cdots,\xi_n)$,称$p(x_1,\cdots,x_n)$称为多元分布密度函数,如果%
	$$
	F(x_1,\cdots,x_n)=\int_{-\infty}^{x_1}{\cdots{\int_{-\infty}^{x_n}{p(t_1,\cdots,t_n)\mathrm{d}{t_1}\cdots\mathrm{d}{t_n}}}}
	$$
\end{definition}

\begin{definition}{多项分布}
	$$
	P\{\xi_1=k_1,\cdots,\xi_r=k_r\}=\frac{n!}{k_1!\cdots k_r!}p_1^{k_1}\cdots p_r^{k_r}
	$$
	其中$p_1+\cdots+p_r=1$且$k_1+\cdots+k_r=n$。
\end{definition}

\begin{definition}{多元超几何分布}
	$$
	P\{\xi_1=n_1,\cdots,\xi_r=n_r\}=\frac{{N_1\choose n_1}\cdots{N_r\choose n_r}}{{N\choose n}}
	$$
	其中$n_1+\cdots+n_r=n$。
\end{definition}

\begin{definition}{均匀分布}
	$$
	p(x_1,\cdots,x_n)=\begin{cases}
		\frac{1}{S},&\qquad (x_1,\cdots,x_n)\in G\\
		0,&\qquad (x_1,\cdots,x_n)\notin G
	\end{cases}
	$$
\end{definition}

\begin{definition}{多元正态分布$N(\boldsymbol{\mu},\Sigma)$}
	$$
	p(\boldsymbol{x})=
	\frac{1}{(2\pi)^{\frac{n}{2}}(\det{\Sigma})^{\frac{1}{2}}}
	\exp\left(-\frac{1}{2}(\boldsymbol{x}-\boldsymbol{\mu})^T{\Sigma}^{-1}(\boldsymbol{x}-\boldsymbol{\mu})\right)
	$$
	其中$\boldsymbol{\mu}$为$n$阶向量,$\Sigma$为$n$阶正定对称矩阵。
\end{definition}

\subsection{边际分布}

\begin{definition}{边际分布}
	考虑二维随机向量$(\xi,\eta)$,令
	\begin{align*}
		& P\{\xi=x,\eta=y\}=p(x,y)\\
		& P\{\xi=x\}=p_1(x)\\
		& P\{\eta=y\}=p_2(y)
	\end{align*}
	则%
	$$
	p_1(x)=\sum_{y\in\R}p(x,y),\qquad 
	p_2(y)=\sum_{x\in\R}p(x,y)
	$$
	称$\{ p_1(x):x\in\R \}$与$\{ p_2(y):y\in\R \}$为$\{ p(x,y):x,y\in\R \}$的边际分布。
\end{definition}

\begin{definition}{边际分布函数}
	考虑二维随机向量$(\xi,\eta)$,其分布函数为$F(x,y)$,称
	$$
	F_1(x)=F(x,+\infty),\qquad F_2(y)=F(+\infty,y)
	$$
	为$F(x,y)$的边际分布函数。
\end{definition}

\begin{definition}{边际分布密度函数}
	考虑二维连续型随机向量$(\xi,\eta)$,其分布函数为$F(x,y)$,密度函数为$p(x,y)$,称
	$$
	p_1(x)=\int_{-\infty}^{+\infty}{p(x,y)\mathrm{d}y},\qquad p_2(y)=\int_{-\infty}^{+\infty}{p(x,y)\mathrm{d}x}
	$$
	为$p(x,y)$的边际分布密度函数。
\end{definition}

\begin{definition}{二元正态分布$N(\mu_1,\mu_2,\sigma_1^2,\sigma_2^2,\rho)$}
	$$
	p(x,y)=\frac{1}{2\pi\sigma_1\sigma_2\sqrt{1-\rho^2}}
	\exp\left(-\frac{1}{2(1-\rho^2)}(\frac{(x-\mu_1)^2}{\sigma_1^2}-2\rho\frac{(x-\mu_1)(y-\mu_2)}{\sigma_1\sigma_2}+\frac{(y-\mu_2)^2}{\sigma_2^2})\right)
	$$
	其中$\mu_1,\mu_2$为两个边际分布的数学期望,$\sigma_1,\sigma_2$为两个边际分布的标准差,$\rho$为二元正态分布的相关系数,且构成协方差矩阵
	$$
	\Sigma=\left(
	\begin{matrix}
		\sigma_1^2&\rho\sigma_1\sigma_2\\
		\rho\sigma_1\sigma_2&\sigma_2^2
	\end{matrix}
	\right)
	$$
\end{definition}

\begin{proposition}{二元正态分布密度函数的典型分解}
	二元正态分布$N(\mu_1,\mu_2,\sigma_1^2,\sigma_2^2,\rho)$存在如下两个分解
	\begin{align*}
		p(x,y)
		&=\frac{1}{\sqrt{2\pi}\sigma_1}
		\mathrm{e}^{-\frac{(x-\mu_1)^2}{2\sigma_1^2}}
		\cdot
		\frac{1}{\sqrt{2\pi}\sigma_2\sqrt{1-\rho^2}}
		\mathrm{e}^{-\frac{(y-(\mu_2+\rho\frac{\sigma_2}{\sigma_1}(x-\mu_1)))^2}{2\sigma_2^2(1-\rho^2)}}\\
		&=\frac{1}{\sqrt{2\pi}\sigma_2}
		\mathrm{e}^{-\frac{(y-\mu_2)^2}{2\sigma_2^2}}
		\cdot
		\frac{1}{\sqrt{2\pi}\sigma_1\sqrt{1-\rho^2}}
		\mathrm{e}^{-\frac{(x-(\mu_1+\rho\frac{\sigma_1}{\sigma_2}(x-\mu_2)))^2}{2\sigma_1^2(1-\rho^2)}}
	\end{align*}
	\begin{enumerate}
		\item 第一式的第一部分为$N(\mu_1,\sigma_1)$的密度函数,第二部分为$N(\mu_2+\rho\frac{\sigma_1}{\sigma_2}(x-\mu_1),\sigma_2^2(1-\rho^2))$的密度函数。
		\item 第二式的第一部分为$N(\mu_2,\sigma_2)$的密度函数,第二部分为$N(\mu_1+\rho\frac{\sigma_2}{\sigma_1}(y-\mu_2),\sigma_1^2(1-\rho^2))$的密度函数。
	\end{enumerate}
\end{proposition}

\begin{definition}{二元正态分布的边际分布}
	二元正态分布$N(\mu_1,\mu_2,\sigma_1^2,\sigma_2^2,\rho)$的边际分布密度函数为
	$$
	p_1(x)=\frac{1}{\sqrt{2\pi}\sigma_1}\mathrm{e}^{-\frac{(x-\mu_1)^2}{2\sigma_1^2}},\qquad 
	p_2(y)=\frac{1}{\sqrt{2\pi}\sigma_2}\mathrm{e}^{-\frac{(x-\mu_2)^2}{2\sigma_2^2}}
	$$
	因此二元正态分布的边际分布仍为正态分布。
\end{definition}

\subsection{条件分布}

\begin{definition}{离散型随机变量的条件分布}
	对于二维离散型随机向量$(\xi,\eta)$,$\eta$关于$\xi$的条件分布为%
	$$
	P\{\eta=y \mid \xi=x\}=\frac{p(x,y)}{p_1(x)}
	$$
\end{definition}

\begin{definition}{连续型随机变量的条件分布}
	对于二维连续型随机向量$(\xi,\eta)$,$\eta$关于$\xi$的条件分布为%
	\begin{align*}
		P\{\eta<y \mid \xi=x\}
		&=\lim_{\Delta x\to 0}{\frac{F(x+\Delta x,y)-F(x,y)}{F(x+\Delta x,\infty)-F(x,\infty)}}\\
		&=\lim_{\Delta x\to 0}{\frac{\dis\int_{x}^{x+\Delta x}{\dis\int_{-\infty}^{y}{p(u,v)\mathrm{d}u\mathrm{d}v}}}{\int_{x}^{x+\Delta x}{\dis\int_{-\infty}^{\infty}{p(u,v)\mathrm{d}u\mathrm{d}v}}}}
	\end{align*}
	若$p_1(x)\ne 0$,则
	$$
	p(y \mid x)=\frac{p(x,y)}{p_1(x)}
	$$
\end{definition}

\subsection{随机变量的独立性}

\begin{definition}{离散型随机变量的独立性}
	称离散型随机变量$\xi_1,\cdots,\xi_n$是相互独立的,如果对于任意的$x_1,\cdots,x_n$,成立
	$$
	P\{\xi_1=x_1,\cdots,\xi_n=x_n\}=P\{\xi_1=x_1\}\cdots P\{\xi_n=x_n\}
	$$
\end{definition}

\begin{definition}{随机变量的独立性}
	称随机变量$\xi_1,\cdots,\xi_n$相互独立,如果成立如下命题之一。
	\begin{enumerate}
		\item 对于任意的$x_1,\cdots,x_n$,成立
		$$
		P\{\xi_1<x_1,\cdots,\xi_n<x_n\}=P\{\xi_1<x_1\}\cdots P\{\xi_n<x_n\}
		$$
		\item 对于任意的$x_1,\cdots,x_n$,成立%
		$$
		F(x_1,\cdots,x_n)=F_1(x_1)\cdots F_n(x_n)
		$$
	\end{enumerate}
\end{definition}

\begin{corollary}
	如果随机变量$\xi_1,\cdots,\xi_n$相互独立,那么对于任意的$x_1,\cdots,x_n$,成立%
	$$
	p(x_1,\cdots,x_n)=p_1(x_1)\cdots p_n(x_n)
	$$
\end{corollary}

\section{随机变量的函数及其分布}

\subsection{Borel函数与随机变量的函数}

\begin{definition}{Borel函数}
	称函数$f:\R^n\to\R$为Borel函数,如果%
	$$
	f^{-1}(\mathscr{B}_1)\sub \mathscr{B}_n
	$$
\end{definition}

\begin{proposition}{随机变量在Borel函数下的像为随机变量}
	对于概率空间$(\Omega,\Sigma,P)$,如果$(\xi_1,\cdots,\xi_n)$为随机向量,且$f:\R^n\to\R$为Borel函数,那么$f(\xi_1,\cdots,\xi_n)$为随机变量。
\end{proposition}

\begin{theorem}{随机变量的函数的独立性}
	如果随机变量$\xi_1,\cdots,\xi_n$相互独立,那么对于任意Borel函数$f_k:\R\to\R$,随机变量$f_1(\xi_1),\cdots,f_n(\xi_n)$相互独立。
\end{theorem}

\subsection{单个随机变量的函数的分布律}

\textbf{目标}:已知随机变量$\xi$的分布函数$F(x)$或密度函数$p(x)$,求解$\eta=g(\xi)$的分布函数$G(y)$或密度函数$q(y)$,即
$$
G(y)=\int_{g(x)<y}{p(x)\mathrm{d}x}
$$

\begin{proposition}
	若$g(x)$严格单调,其反函数$g^{-1}(y)$存在连续导函数,则$\eta=g(\xi)$具有密度函数
	$$
	q(y)=p(g^{-1}(y)) \mid (g^{-1}(y))^{\prime} \mid 
	$$
\end{proposition}

\begin{proposition}{倍数分布}
	如果随机变量$\xi$的分布函数为$F(x)$,密度函数为$p(x)$,那么$\eta=c\xi$的分布函数为
	$$
	G(y)=\begin{cases}
		F(y/c),&\qquad c>0\\
		1-F(y/c),&\qquad c<0
	\end{cases}
	$$
	密度函数为
	$$
	q(y)=\frac{p(y/c)}{|c|}
	$$
\end{proposition}

\begin{proposition}{平方分布}
	如果随机变量$\xi$的分布函数为$F(x)$,密度函数为$p(x)$,则$\eta=\xi^2$的分布函数为
	$$
	G(y)=\begin{cases}
		F(\sqrt{y})-F(-\sqrt{y}),&\qquad y>0\\
		0,&y\qquad \le 0
	\end{cases}
	$$
	密度函数为
	$$
	q(y)=\begin{cases}
		\frac{p(\sqrt{y})+p(-\sqrt{y})}{2\sqrt{y}},&\qquad y>0\\
		0,&\qquad y<0
	\end{cases}
	$$
\end{proposition}

\subsection{多个随机变量的函数的分布律}

\textbf{目标}:已知随机向量$(\xi_1,\cdots,\xi_n)$的分布函数$F(x_1,\cdots,x_n)$或密度函数$p(x_1,\cdots,x_n)$,求解$\eta=g(\xi_1,\cdots,\xi_n)$的分布函数$G(y)$或密度函数$q(y)$,即
$$
G(y)=\int_{g(x_1,\cdots,x_n)<y}p(x_1,\cdots,x_n)\dd x_1\cdots \dd x_n
$$

\begin{proposition}{和的分布}
	如果随机向量$(\xi_1,\xi_2)$的密度函数为$p(x_1,x_2)$,那么$\eta=\xi_1+\xi_2$的分布函数为%
	$$
	G(y) = \iint\limits_{x_1+x_2< y}p(x_1,x_2)\dd x_1\dd x_2
	$$
	若$(\xi_1,\xi_2)$相互独立,则%
	$$
	G(y)
	=\int_{-\infty}^{y}\dd v\int_{-\infty}^{+\infty}p(u,v-u)\dd u
	=\int_{-\infty}^{y}\dd v\int_{-\infty}^{+\infty}p(v-u,u)\dd u
	$$
	此时密度函数为%
	$$
	q(y)=\int_{-\infty}^{+\infty}p(x,y-x)\dd x
	=\int_{-\infty}^{+\infty}p(y-x,x)\dd x
	$$
\end{proposition}

\begin{proposition}{商的分布}
	如果随机向量$(\xi_1,\xi_2)$的密度函数为$p(x_1,x_2)$,那么$\eta=\xi_1/\xi_2$的分布函数为%
	$$
	G(y) = \iint\limits_{x_1/x_2< y}p(x_1,x_2)\dd x_1\dd x_2
	$$
	若$(\xi_1,\xi_2)$相互独立,则%
	$$
	G(y)=\int_{0}^{+\infty}\mathrm{d}z\int_{-\infty}^{zy}{p(x,z)\mathrm{d}x}
	+\int_{-\infty}^{0}\mathrm{d}z\int_{zy}^{+\infty}{p(x,z)\mathrm{d}x}
	$$
	此时密度函数为%
	$$
	q(y)=\int_{-\infty}^{+\infty}{|z|p(zy,z)\mathrm{d}z}
	$$
\end{proposition}

\begin{proposition}{顺序统计量的分布}
	假设随机变量$\xi_1,\cdots,\xi_n$相互独立,且具有相同的分布函数$F(x)$和密度函数$p(x)$。
	\begin{enumerate}
		\item 极小值的分布函数与密度函数:%
		$$
		G(x)=1-(1-F(x))^n,\qquad q(x)=np(x)(1-F(x))^{n-1}
		$$
		\item 极大值的分布函数与密度函数:
		$$
		G(x)=G(x)=(F(x))^n,\qquad q(x)=np(x)(F(x))^{n-1}
		$$
		\item 极小值与极大值的联合分布函数与联合密度函数:%
		\begin{align*}
			&G(x,y)=\begin{cases}
				(F(y))^n,&x\ge y\\
				(F(y))^n-(F(y)-F(x))^n,&x<y
			\end{cases}\\
			&q(x,y)=\begin{cases}
				0,&x\ge y\\
				n(n-1)q(x)q(y)(F(y)-F(x))^{n-2},&x<y
			\end{cases}
		\end{align*}
	\end{enumerate}
\end{proposition}

\begin{theorem}{随机向量的变换的分布}
	如果随机向量 $(\xi_1,\cdots,\xi_n)$的密度函数为$p(x_1,\cdots,x_n)$,随机变量$\eta_1,\cdots,\eta_n$
	$$
	\eta_1=g_1(\xi_1,\cdots,\xi_n)\qquad \cdots
	\qquad \eta_n=g_n(\xi_1,\cdots,\xi_n)
	$$
	的密度函数为$q(y_1,\cdots,y_n)$,且对于$y_k=g_k(x_1,\cdots,x_n)$,存在且存在唯一的反函数$x_k=x_k(y_1,\cdots,y_n)$,那么
	$$
	q(y_1,\cdots,y_n)=p(x_1(y_1,\cdots,y_n),\cdots,x_n(y_1,\cdots,y_n))|J|
	$$
	其中$J$为Jacobi行列式
	$$
	J=\begin{vmatrix}
		\frac{\partial x_1}{\partial y_1}&\cdots&\frac{\partial x_1}{\partial y_n}\\
		\vdots&\ddots&\vdots\\
		\frac{\partial x_n}{\partial y_1}&\cdots&\frac{\partial x_n}{\partial y_n}
	\end{vmatrix}
	$$
\end{theorem}

\chapter{数字特征与特征函数}

\section{数学期望}

\begin{definition}{离散型随机变量的数学期望}
	对于离散型随机变量$\xi$,令$P\{\xi=x_k\}=p_k$,称
	$$
	E(\xi)=\sum_{k=1}^{\infty}{x_k p_k}
	$$
	为$\xi$的数学期望,如果该级数绝对收敛。
\end{definition}

\begin{definition}{连续型随机变量的数学期望}
	对于离散型随机变量$\xi$,其密度函数为$p(x)$,称
	$$
	E(\xi)=\int_{-\infty}^{\infty}{xp(x)\mathrm{d}x}
	$$
	为$\xi$的数学期望,如果该积分绝对可积。
\end{definition}

\begin{definition}{数学期望}
	\begin{enumerate}
		\item 如果随机变量$\xi$的分布函数为$F(x)$,且积分
		$$
		\int_{-\infty}^{\infty}{x\mathrm{d}F(x)}
		$$
		绝对可积,那么定义
		$$
		E(\xi)=\int_{-\infty}^{\infty}{x\mathrm{d}F(x)}
		$$
		为$\xi$的数学期望。
		\item 随机向量$(\xi_1,\cdots,\xi_n)$的数学期望为$(E(\xi_1),\cdots,E(\xi_n))$,其中
		$$
		E(\xi_k)=\int_{-\infty}^{\infty}{x_k\mathrm{d}F_k(x_k)},\qquad k=1,\cdots,n
		$$
	\end{enumerate}
\end{definition}

\begin{proposition}{随机变量的函数的数学期望}
	\begin{enumerate}
		\item 如果随机变量$\xi$的分布函数为$F(x)$,且$g:\R\to\R$为Borel函数,那么随机变量$\eta=g(\xi)$的数学期望为
		$$
		E(\eta)=\int_{-\infty}^{\infty}{g(x)\mathrm{d}F(x)}
		$$
		\item 如果随机向量$(\xi_1,\cdots,\xi_n)$的分布函数为$F(x_1,\cdots,x_n)$,且$g:\R^n\to\R$为Borel函数,那么随机变量$\eta=g(\xi_1,\cdots,\xi_n)$的数学期望为
		$$
		E(\eta)=\int_{-\infty}^{\infty}\cdots\int_{-\infty}^{\infty}{g(x_1,\cdots,x_n)\mathrm{d}F(x_1,\cdots,x_n)}
		$$
	\end{enumerate}
\end{proposition}

\begin{proposition}{数学期望的性质}
	\begin{enumerate}
		\item 若$a\le\xi\le b$,则$a\le E(\xi)\le b$。
		\item 线性性:
		$$
		E(\xi+\eta)=E(\xi)+E(\eta),\qquad 
		E(c\xi)=cE(\xi)
		$$
	\end{enumerate}
\end{proposition}

\begin{theorem}{Cauchy-Schwarz不等式}
	对随机变量$\xi$和$\eta$,成立
	$$
	(E(\xi\eta))^2\le E(\xi^2)E(\eta^2)
	$$
	当且仅当存在常数$a,b$,使得成立
	$$
	P\{a\xi=b\eta\}=1
	$$
	时等号成立。
\end{theorem}

\section{方差,相关系数,矩}

\subsection{方差}

\begin{definition}{方差}
	对于随机变量$\xi$,如果
	$$
	D(\xi)
	=E((\xi-E(\xi))^2)
	=E(\xi^2)-(E(\xi))^2
	$$
	存在,那么称$D(\xi)$为$\xi$的方差。
\end{definition}

\begin{definition}{标准差}
	称随机变量的方差的方根为标准差,即$\sqrt{D(\xi)}$。
\end{definition}

\begin{definition}{随机变量的标准化}
	对于随机变量$\xi$,若其数学期望$E(\xi)$及方差$D(\xi)$均存在,且$D(\xi)>0$,则可标准化为
	$$
	\xi^*=\frac{\xi-E(\xi)}{D(\xi)},\qquad 
	E(\xi^*)=0,\qquad 
	D(\xi^*)=1
	$$
\end{definition}

\begin{proposition}{方差的性质}
	\begin{enumerate}
		\item 非线性性质:%
		$$
		D(c\xi)=c^2D(\xi),\qquad 
		D(\xi+\eta)=D(\xi)+D(\eta)+2(E(\xi\eta)-E(\xi) E(\eta))
		$$
		\item $D(\xi)\le E(\xi-c)^2$,当且仅当$E(\xi)=c$时等号成立。
	\end{enumerate}
\end{proposition}

\begin{theorem}{Chebyshev不等式}
	如果随机变量$\xi$存在数学期望$E(\xi)$与方差$D(\xi)$,那么对于任意$\varepsilon>0$,成立
	$$
	P\{ \mid \xi-E(\xi) \mid \ge\varepsilon\}\le\frac{D(\xi)}{\varepsilon^2}
	$$
\end{theorem}

\subsection{协方差}

\begin{definition}{协方差}
	对于随机变量$\xi$和$\eta$,如果
	$$
	\mathrm{cov}(\xi,\eta)
	=E((\xi-E(\xi))(\eta-E(\eta)))
	=E(\xi\eta)-E(\xi)E(\eta)
	$$
	存在,那么称$\mathrm{cov}(\xi,\eta)$为$\xi$与$\eta$的协方差。
\end{definition}

\begin{proposition}{协方差的性质}
	\begin{enumerate}
		\item 与方差的关系:$\mathrm{cov}(\xi,\xi)=D(\xi)$
		\item 对称性:$\mathrm{cov}(\xi,\eta)=\mathrm{cov}(\eta,\xi)$
		\item 线性性:%
		$$
		\mathrm{cov}(\xi,c\eta)=c\mathrm{cov}(\xi,\eta),\qquad 
		\mathrm{cov}(\xi,\eta+\zeta)=\mathrm{cov}(\xi,\eta)+\mathrm{cov}(\xi,\zeta)
		$$
	\end{enumerate}
\end{proposition}

\begin{definition}{协方差矩阵}
	定义随机向量$\bs{\xi}=(\xi_1,\cdots,\xi_n)$的协方差矩阵为
	$$
	D(\bs{\xi})=\left(
	\begin{matrix}
		\mathrm{cov}(\xi_1,\xi_1)&\cdots&\mathrm{cov}(\xi_1,\xi_n)\\
		\vdots&\ddots&\vdots\\
		\mathrm{cov}(\xi_n,\xi_1)&\cdots&\mathrm{cov}(\xi_n,\xi_n)
	\end{matrix}
	\right)
	$$
\end{definition}

\subsection{相关系数与相关性}

\begin{definition}{相关系数}
	\begin{enumerate}
		\item 对于随机变量$\xi$与$\eta$,定义其相关系数为%
		$$
		\rho=\begin{cases}
			\frac{\mathrm{cov}(\xi,\eta)}{\sqrt{D(\xi)}\sqrt{D(\eta)}},&,\qquad\xi\text{ 与 }\eta\text{ 均不为常数}\\
			0,&\qquad \xi\text{ 或 }\eta\text{ 为常数}
		\end{cases}
		$$
		\item 对于事件$A$和$B$,定义其相关系数为%
		$$
		\rho=\frac{P(AB)-P(A)P(B)}{\sqrt{P(A)P(A^c)P(B)P(B^c)}}
		$$
	\end{enumerate}
\end{definition}

\begin{proposition}{相关系数的性质}
	\begin{enumerate}
		\item 对称性:$\rho(\xi,\eta)=\rho(\eta,\xi)$
		\item $\rho(\xi,c\eta)=\mathrm{sgn}(c)\rho(\xi,\eta)$
	\end{enumerate}
\end{proposition}

\begin{proposition}{相关系数的相关性}
	令随机变量$\xi$和$\eta$的相关系数为$\rho$。
	\begin{enumerate}
		\item $\rho>0$:正相关
		\item $\rho<0$:负相关
		\item $\rho=1$:完全正相关
		\item $\rho=-1$:完全负相关
		\item $\rho=0$:不相关
	\end{enumerate}
\end{proposition}

\begin{proposition}
	对于随机变量$\xi$和$\eta$的相关系数$\rho$,成立%
	$$
	|\rho|\le 1
	$$
	并且$\rho=1$当且仅当
	$$
	P\left\{\frac{\xi-E(\xi)}{\sqrt{D(\xi)}}=\frac{\eta-E(\eta)}{\sqrt{D(\eta)}}\right\}=1
	$$
	$\rho=-1$当且仅当
	$$
	P\left\{\frac{\xi-E(\xi)}{\sqrt{D(\xi)}}+\frac{\eta-E(\eta)}{\sqrt{D(\eta)}}=0\right\}=1
	$$
\end{proposition}

\begin{definition}{相关性}
	称随机变量$\xi$和$\eta$不相关,如果成立如下命题之一。
	\begin{enumerate}
		\item $\rho=0$
		\item $\mathrm{cov}(\xi,\eta)=0$
		\item $E(\xi\eta)=E(\xi)E(\eta)$
		\item $D(\xi+\eta)=D(\xi)+D(\eta)$
	\end{enumerate}
\end{definition}

\begin{proposition}{独立性与相关性}
	\begin{enumerate}
		\item 如果随机变量$\xi$和$\eta$独立,那么$\xi$和$\eta$不相关。
		\item 对于二元正态分布,不相关性与独立性是等价的。
		\item 对于二值随机变量,不相关性与独立性是等价的。
	\end{enumerate}
\end{proposition}

\subsection{矩}

\begin{definition}{原点矩}
	定义随机变量$\xi$的$n$阶原点矩为%
	$$
	m_n=E(\xi^n)
	$$
\end{definition}

\begin{definition}{中心矩}
	定义随机变量$\xi$的$n$阶中心矩为%
	$$
	c_n=E((\xi-E(\xi))^n)
	$$
\end{definition}

\begin{note}
	由于$|\xi|^{n-1}\le 1+|\xi|^n$,因此若$n$阶矩存在,则$n-1$阶矩存在。
\end{note}

\begin{proposition}{原点矩与中心距的关系}
	$$
	c_n=\sum_{k=0}^{n}{{n\choose k}(-m_1)^{n-k}m_k},\qquad 
	m_n=\sum_{k=0}^{n}{{n\choose k}c_{n-k}m_1^k}
	$$
\end{proposition}

\subsection{分位数}

\begin{definition}{$p$分位数}
	对于$0<p<1$,称$x_p$为分布函数$F(x)$的$p$分位数,如果
	$$
	F(x_p)\le p\le F(x_p^+)
	$$
\end{definition}

\begin{definition}{中位数}
	称$x_{0.5}$为中位数。
\end{definition}

\subsection{条件数学期望与最小二乘回归}

\begin{definition}{条件数学期望}
	对于随机向量$(\xi,\eta)$,定义$\eta$在$\xi=x$的条件下的条件数学期望为
	$$
	E\{\eta \mid \xi=x\}=\int_{-\infty}^{\infty}{yp(y \mid x)\mathrm{d}y}
	$$
\end{definition}

\begin{proposition}{重期望公式}
	$$
	E(\eta)=E(E(\eta \mid \xi))
	$$
\end{proposition}

\begin{definition}{最小二乘法}
	对于随机变量$\xi$和$\eta$,优化问题
	$$
	\inf_{h}E(\eta-h(\xi))
	$$
	的解为
	$$
	h(x)=E\{\eta \mid \xi=x\}
	$$
\end{definition}

\begin{definition}{最小二乘回归}
	称
	$$
	y=E\{\eta \mid \xi=x\}
	$$
	为$\eta$关于$\xi$的最小二乘回归。
\end{definition}

\begin{definition}{线性回归}
	如果$\xi$和$\eta$的数学期望分别为$\mu_1$和$\mu_2$,标准差分别为$\sigma_1$和$\sigma_2$及相关系数为$\rho$,那么$\eta$关于$\xi$的线性回归为
	$$
	L(x)=\mu_2+\rho\frac{\sigma_2}{\sigma_1}(x-\mu_1)
	$$
	其均方误差为
	$$
	E(\eta-L(\xi)^2)=\sigma_2^2(1-\rho^2)
	$$
\end{definition}

\section{母函数}

\begin{definition}{母函数}
	对于离散型随机变量$\xi$,概率分布为$P\{\xi=n\}=p_n$,定义$\xi$的母函数为
	$$
	P(s)=\sum_{n=0}^{\infty}{p_n s^n}
	$$
\end{definition}

\begin{proposition}{母函数的性质}
	\begin{enumerate}
		\item 唯一性:母函数与概率分布函数一一对应。
		\item 数学期望:$E(\xi)=P'(1)$
		\item 方差:$D(\xi)=P''(1)+P'(1)-(P'(1))^2$
	\end{enumerate}
\end{proposition}

\begin{proposition}{独立随机变量之和的母函数}
	若随机变量$\xi$和$\eta$相互独立,其相应的母函数分别为$A(s)$和$B(s)$,则随机变量$\xi+\eta$的母函数$C(s)$为
	$$
	C(s)=A(s)B(s)
	$$
\end{proposition}

\begin{proposition}{随机个随机变量之和的母函数}
	设$\xi_1,\xi_2,\cdots$是相互独立的具有相同概率分布$P\{\xi_i=j\}=f_j$的随机变量,其母函数为
	$$
	F(s)=\sum_{n=0}^{\infty}{f_n s^n}
	$$
	随机变量$\nu$取整数值,且$P\{\nu=n\}=g_n$,其母函数为
	$$
	G(s)=\sum_{n=0}^{\infty}{g_n s^n}
	$$
	若$\{\xi_n\}$与$\nu$独立,则随机变量$\eta=\xi_1+\cdots+\xi_{\nu}$的母函数为
	$$
	G(F(s))
	$$
\end{proposition}

\section{特征函数}

\subsection{特征函数}

\begin{definition}{特征函数}
	对于分布函数为$F(x)$的随机变量$\xi$,定义其特征函数为%
	$$
	f(t)=E(\mathrm{e}^{it\xi})=\int_{-\infty}^{\infty}{\mathrm{e}^{itx}\mathrm{d}F(x)}
	$$
\end{definition}

\begin{definition}{连续型随机变量的特征函数}
	对于密度函数为$p(x)$的连续型随机变量$\xi$,定义其特征函数为%
	$$
	f(t)=\int_{-\infty}^{\infty}{\mathrm{e}^{itx}p(x)\mathrm{d}x}
	$$
\end{definition}

\begin{proposition}{特征函数的性质}
	\begin{enumerate}
		\item %
		$$
		f(0)=1,\qquad 
		|f(t)|\le f(0),\qquad 
		f(-t)=\overline{f(t)}
		$$
		\item 一致连续性:特征函数$f(t)$在$\R$上一致连续。
		\item 非负定性:对于任意$n\in \N^*$与实数$t_1,\cdots,t_n$及复数$\lambda_1,\cdots,\lambda_n$,成立
		$$
		\sum_{i,j=1}^{n}{f(t_i-t_j)\lambda_i\overline{\lambda_j}}\ge 0
		$$
		\item 随机变量和的特征函数:若随机变量$\xi$和$\eta$相互独立,且对应的特征函数分别为$f_{\xi}(t)$和$f_{\eta}(t)$,则其和$\xi+\eta$的特征函数$f_{\xi+\eta}(t)$满足
		$$
		f_{\xi+\eta}(t)=f_{\xi}(t)f_{\eta}(t)
		$$
		\item 可微性:若随机变量$\xi$存在$n$阶矩,则其特征函数可微分$n$次,且当$k\le n$时,成立
		$$
		f^{(k)}(0)=i^kE(\xi^k)
		$$
		\item 位移性:
		$$
		f_{a\xi+b}(t)=\mathrm{e}^{ibt}f_{\xi}(at)
		$$
	\end{enumerate}
\end{proposition}

\subsection{特征函数的唯一性}

\begin{theorem}{逆转公式}
	若分布函数$F(x)$的特征函数为$f(t)$,且$x_1,x_2$是$F(x)$的连续点,则成立逆转公式
	$$
	F(x_2)-F(x_1)=\frac{1}{2\pi}\int_{-\infty}^{\infty}{\frac{\mathrm{e}^{-itx_1}-\mathrm{e}^{-itx_2}}{it}f(t)\mathrm{d}t}
	$$
\end{theorem}

\begin{theorem}{唯一性定理}
	分布函数$F(x)$由特征函数$f(t)$唯一确定,且
	$$
	F(x)=\frac{1}{2\pi}\lim_{y\to\infty}{\int_{-\infty}^{\infty}{\frac{\mathrm{e}^{-ity}-\mathrm{e}^{-itx}}{it}f(t)\mathrm{d}t}}
	$$
	特别的,若$f(t)$绝对可积,则相应的分布函数$F(x)$存在并连续,且
	$$
	F'(x)=\frac{1}{2\pi}\int_{-\infty}^{\infty}{\mathrm{e}^{-itx}f(t)\mathrm{d}t}
	$$
\end{theorem}

\chapter{极限定理}

\section{Bernoulli试验场合的极限定理}

\subsection{大数定律与中心极限定理}

\begin{definition}{大数定律}
	称随机变量序列$\{ \xi_n \}_{n=1}^{\infty}$服从大数定律,如果存在数列$\{ a_n \}_{n=1}^{\infty}$,使得对于任意$\varepsilon>0$,成立%
	$$
	\lim_{n\to\infty}{P\{ | \eta_n-a_n | <\varepsilon\}}=1
	$$
	其中
	$$
	\eta_n=\frac{\xi_1+\cdots+\xi_n}{n},\qquad n\in\N^*
	$$
\end{definition}

\begin{definition}{强大数定律}
	称独立随机变量序列$\{ \xi_n \}_{n=1}^{\infty}$服从强大数定律,如果
	$$
	P\left\{\lim_{n\to\infty}\frac{1}{n}{\sum_{k=1}^{n}{(\xi_k-E(\xi_k))}}=0\right\}=1
	$$
\end{definition}

\begin{definition}{中心极限定理}
	称存在数学期望与方差的相互独立的随机变量序列$\{ \xi_n \}_{n=1}^{\infty}$服从中心极限定理,如果成立%
	$$
	\lim_{n\to\infty}{P\{\zeta_n<x\}}=\frac{1}{\sqrt{2\pi}}\int_{-\infty}^{x}{\mathrm{e}^{-\frac{t^2}{2}}\mathrm{d}t},\qquad x\in\R
	$$
	其中
	$$
	\zeta_n=\frac{\dis\sum_{k=1}^{n}{\xi_k}-\sum_{k=1}^{n}{E(\xi_k)}}{\dis\sqrt{\sum_{k=1}^{n}{D(\xi_k)}}},\qquad n\in\N^*
	$$
\end{definition}

\subsection{大数定律}

\begin{theorem}{Chebyshev大数定律}{Chebyshev大数定律}
	如果互不相关的随机变量序列$\{ \xi_n \}_{n=1}^{\infty}$的方差一致有界,那么对于任意$\varepsilon>0$,成立%
	$$
	\lim_{n\to\infty}P\left\{\left|\frac{1}{n}\sum_{k=1}^{n}{\xi_k}-\frac{1}{n}\sum_{k=1}^{n}{E(\xi_k)}\right|<\varepsilon\right\}=1
	$$
\end{theorem}

\begin{theorem}{Markov大数定理}{Markov大数定理}
	对于随机变量序列$\{ \xi_n \}_{n=1}^{\infty}$,如果
	$$
	\lim_{n\to\infty}D\left(\frac{\xi_1+\cdots+\xi_n}{n}\right)=0
	$$
	那么对于任意$\varepsilon>0$,成立%
	$$
	\lim_{n\to\infty}P\left\{\left|\frac{1}{n}\sum_{k=1}^{n}{\xi_k}-\frac{1}{n}\sum_{k=1}^{n}{E(\xi_k)}\right|<\varepsilon\right\}=1
	$$
\end{theorem}

\begin{theorem}{Bernoulli大数定律}{Bernoulli大数定律}
	如果$\mu_n$为$n$次Bernoulli试验中某事件出现的次数,$p$为该事件在每次试验中出现的概率,那么对于任意$\varepsilon>0$,成立
	$$
	\lim_{n\to\infty}P\left\{\left|\frac{\mu_n}{n}-p\right|<\varepsilon\right\}=1
	$$
\end{theorem}

\begin{theorem}{Poisson大数定律}{Poisson大数定律}
	如果在一个独立实验序列中,某事件在第$k$次试验中出现的概率为$p_k$,以$\mu_n$记前$n$次试验中该事件出现的次数,那么对于任意的$\varepsilon>0$,恒成立
	$$
	\lim_{n\to\infty}P\left\{\left|\frac{\mu_n}{n}-\frac{p_1+\cdots+p_n}{n}\right|<\varepsilon\right\}=1
	$$
\end{theorem}

\subsection{De Moivre—Laplace极限定理}

\begin{theorem}{De Moivre—Laplace极限定理}{De Moivre—Laplace极限定理}
	\begin{enumerate}
		\item 局部极限定理:如果$\mu_n$是$n$次Bernoulli试验中某事件出现的次数,令$x_k=\frac{k-np}{\sqrt{np(1-p)}}$,其中$0<p<1$,且满足$a\le x_k\le b$,那么
		$$
		P\{\mu_n=k\}\sim \frac{1}{\sqrt{np(1-p)}}\frac{1}{\sqrt{2\pi}}\mathrm{e}^{-\frac{x_k^2}{2}}\qquad(n\to\infty)
		$$
		\item 积分极限定理:如果$\mu_n$是$n$次Bernoulli试验中某事件出现的次数,那么对于$0<p<1$,成立
		$$
		\lim_{n\to\infty}P\left\{a\le\frac{\mu_n-np}{\sqrt{np(1-p)}}\le b\right\}=\frac{1}{\sqrt{2\pi}}\int_{a}^{b}{\mathrm{e}^{-\frac{x^2}{2}}\mathrm{d}x}
		$$
	\end{enumerate}
\end{theorem}

\begin{corollary}
	\begin{enumerate}
		\item 用频率估计概率:
		$$
		P\left\{\left|\frac{\mu_n}{n}-p\right|<\varepsilon\right\}
		\approx\frac{2}{\sqrt{2\pi}}\int_{-\infty}^{\varepsilon\sqrt{\frac{n}{p(1-p)}}}{\mathrm{e}^{-\frac{x^2}{2}}\mathrm{d}x}-1
		$$
		\item 二项式分布计算:
		$$
		{{n}\choose{k}}p^k(1-p)^{n-k}\approx\frac{1}{\sqrt{np(1-p)}}\frac{1}{\sqrt{2\pi}}\mathrm{e}^{-\frac{1}{2}(\frac{k-np}{\sqrt{np(1-p)}})^2}
		$$
		\item 二项分布计算:
		$$
		P\{a\le\mu_n\le b\}\approx\frac{1}{\sqrt{2\pi}}\int_{\frac{a-np}{\sqrt{np(1-p)}}}^{\frac{b-np}{\sqrt{np(1-p)}}}{\mathrm{e}^{-\frac{x^2}{2}}\mathrm{d}x}
		$$
	\end{enumerate}
\end{corollary}

\section{收敛性}

\subsection{分布函数弱收敛}

\begin{definition}{弱收敛}
	称分布函数序列$\{F_n(x)\}$弱收敛于单调递增函数$F(x)$,并记作$F_n(x)\xlongrightarrow{\mathrm{W}}F(x)$,如果对于任意$F(x)$的连续点$x$,成立
	$$
	\lim_{n\to\infty}{F_n(x)}=F(x)
	$$
	在$F(x)$的每一连续点上都成立,则称$F_n(x)$,。
\end{definition}

\begin{theorem}{Helly第一定理}{Helly第一定理}
	一致有界的单调递增函数序列$\{F_n(x)\}$存在子序列$\{F_{n_k}(x)\}$弱收敛于有界单调递增函数$F(x)$。
\end{theorem}

\begin{theorem}{Helly第二定理}{Helly第二定理}
	如果$f(x)$是$[a,b]$上的连续函数,且一致有界单调递增函数序列$\{F_n(x)\}$在$[a,b]$上弱收敛于函数$F(x)$,同时$a$和$b$是$F(x)$的连续点,那么
	$$
	\lim_{n\to\infty}{\int_a^b{f(x)\mathrm{d}F_n(x)}}=\int_a^b{f(x)\mathrm{d}F(x)}
	$$
\end{theorem}

\begin{theorem}{Helly第二定律的推广}
	如果$f(x)$是$\R$上的连续有界函数,且一致有界单调递增函数序列$\{F_n(x)\}$在$[a,b]$上弱收敛于函数$F(x)$,同时
	$$
	\lim_{n\to\infty}{F_n(-\infty)}=F(-\infty),\qquad
	\lim_{n\to\infty}{F_n(+\infty)}=F(+\infty)
	$$
	那么
	$$
	\lim_{n\to\infty}{\int_{-\infty}^{+\infty}{f(x)\mathrm{d}F_n(x)}}=\int_{-\infty}^{+\infty}{f(x)\mathrm{d}F(x)}
	$$
\end{theorem}

\subsection{Lévy—Cramer定理}

\begin{theorem}{Lévy—Cramer正极限定理}{Lévy—Cramer正极限定理}
	如果分布函数序列$\{F_n(x)\}$弱收敛于分布函数$F(x)$,那么$\{F_n(x)\}$的特征函数序列$\{f_n(t)\}$内闭一致收敛于$F(x)$的特征函数$f(t)$。
\end{theorem}

\begin{theorem}{Lévy—Cramer逆极限定理}{Lévy—Cramer逆极限定理}
	如果特征函数序列$\{f_n(t)\}$收敛于特征函数$f(t)$,且$f(t)$在$t=0$处连续,那么$\{f_n(t)\}$的分布函数序列$\{F_n(x)\}$弱收敛于$f(t)$的分布函数$F(x)$。
\end{theorem}

\subsection{随机变量的收敛性}

\begin{definition}{依分布收敛}
	对于随机变量序列$\{\xi_n\}$与随机变量$\xi$,称$\{\xi_n\}$依分布收敛于$\xi$,并记作$\xi_n\xlongrightarrow{\mathrm{L}}\xi$,如果$F_n(x)\xlongrightarrow{\mathrm{W}}F(x)$,其中$\{\xi_n\}$与$\xi$的分布函数分别为$\{F_n(x)\}$与$F(x)$。
\end{definition}

\begin{definition}{依概率收敛}
	对于随机变量序列$\{\xi_n\}$与随机变量$\xi$,称$\{\xi_n\}$依概率收敛于$\xi$,并记作$\xi_n\xlongrightarrow{\mathrm{P}}\xi$,如果对于任意$\varepsilon>0$,成立
	$$
	\lim_{n\to\infty}{P\{|\xi_n-\xi|\ge\varepsilon\}}=0
	$$
\end{definition}

\begin{definition}{$r$阶收敛}
	对于存在$r$阶矩的随机变量序列$\{\xi_n\}$与随机变量$\xi$,称$\{\xi_n\}$$r$阶收敛于$\xi$,并记作$\xi_n\xlongrightarrow{r}\xi$,如果
	$$
	\lim_{n\to\infty}E(|\xi_n-\xi|^r)=0
	$$
\end{definition}

\begin{definition}{以概率$1$收敛}
	对于存在$r$阶矩的随机变量序列$\{\xi_n\}$与随机变量$\xi$,称$\{\xi_n\}$以概率$1$收敛于$\xi$,并记作$\xi_n\xlongrightarrow{\mathrm{a.s.}}\xi$,如果
	$$
	P\left\{\lim_{n\to\infty}{\xi_n}=\xi\right\}=1
	$$
\end{definition}

\begin{proposition}
	\begin{align*}
		& \xi_n\xlongrightarrow{r}\xi
		\implies\xi_n\xlongrightarrow{\mathrm{P}}\xi
		\implies\xi_n\xlongrightarrow{\mathrm{L}}\xi\\
		& \xi_n\xlongrightarrow{\mathrm{a.s.}}\xi
		\implies\xi_n\xlongrightarrow{\mathrm{P}}\xi\\
		& \text{对于常数}C,\xi_n\xlongrightarrow{\mathrm{P}}C\iff\xi_n\xlongrightarrow{\mathrm{L}}C
	\end{align*}
\end{proposition}

\subsection{Bochner—Khinchin定理}

\begin{theorem}{Bochner—Khinchin定理}{Bochner—Khinchin定理}
	函数$f(t)$是特征函数的充分必要条件是:$f(t)$非负定,连续,且$f(0)=1$。
\end{theorem}

\begin{theorem}{Herglotz定理}{Herglotz定理}
	复数列$\{C_n\}_{n=1}^{\infty}$可以表示为
	$$
	C_n=\int_{-\pi}^{\pi}{\mathrm{e}^{inx}\mathrm{d}G(x)}
	$$
	的充分必要条件是其为非负定的,即对于任意$n\in\N^*$及$\{\lambda_n\}_{n=1}^{\infty}\sub\C$均有
	$$
	\sum_{i,j=1}^{n}{C_{i-j}\lambda_i\overline{\lambda_j}}\ge 0
	$$
	其中$G(x)$是$[-\pi,\pi]$上有界、单调非减、左连续函数。
\end{theorem}

\section{独立同分布场合的极限定理}

\subsection{Khinchin大数定律}

\begin{theorem}{Khinchin大数定律}{Khinchin大数定律}
	对于相互独立的随机变量序列$\{ \xi_n \}_{n=1}^{\infty}$,如果$\xi_n$服从相同的分布,且
	$$
	E(\xi_n)=\mu,\qquad n\in\N^*
	$$
	那么对于任意$\varepsilon>0$,成立
	$$
	\lim_{n\to\infty}P\left\{\left|\frac{1}{n}\sum_{k=1}^{n}{\xi_k}-\mu\right|<\varepsilon\right\}=1
	$$
\end{theorem}

\subsection{中心极限定理}

\begin{theorem}{Lindeberg—Lévy中心极限定理}{Lindeberg—Lévy中心极限定理}
	对于相互独立的随机变量序列$\{ \xi_n \}_{n=1}^{\infty}$,如果$\xi_n$服从相同的分布,且
	$$
	E(\xi_n)=\mu,\qquad 
	D(\xi_n)=\sigma^2,\qquad
	n\in\N^*
	$$
	那么标准化随机变量
	$$
	\zeta_n=\frac{1}{\sigma\sqrt{n}}\sum_{k=1}^{n}{(\xi_k-\mu)}
	$$
	成立
	$$
	\lim_{n\to\infty}{P\{\zeta_n<x\}}=\frac{1}{\sqrt{2\pi}}\int_{-\infty}^{x}{\mathrm{e}^{-\frac{t^2}{2}}\mathrm{d}t}
	$$
\end{theorem}

\begin{corollary}
	对于相互独立的随机变量序列$\{ \xi_n \}_{n=1}^{\infty}$,如果$\xi_n$服从相同的分布,且
	$$
	E(\xi_n)=\mu,\qquad 
	D(\xi_n)=\sigma^2,\qquad
	n\in\N^*
	$$
	那么
	$$
	P\left\{a\le\sum_{k=1}^{n}{\xi_k}\le b\right\}\approx\frac{1}{\sqrt{2\pi}}\int_{\frac{a-n\mu}{\sigma\sqrt{n}}}^{\frac{b-n\mu}{\sigma\sqrt{n}}}{\mathrm{e}^{-\frac{t^2}{2}}\mathrm{d}t}
	$$
\end{corollary}

\section{强大数定律}

\subsection{Borel强大数定律}

\begin{theorem}{Borel—Cantelli引理}{Borel—Cantelli引理}
	\begin{enumerate}
		\item 对于随机事件序列$\{A_n\}_{n=1}^{\infty}$,如果
		$$
		\sum_{n=1}^{\infty}{P(A_n)}<\infty,\qquad n\in\N^*
		$$
		那么
		$$
		P\left\{\limsup_{n\to\infty}{A_n}\right\}=0,
		P\left\{\liminf_{n\to\infty}A^c\right\}=1
		$$
		\item 对于相互独立的随机事件序列$\{A_n\}_{n=1}^{\infty}$,成立
		$$
		\sum_{n=1}^{\infty}{P(A_n)}=\infty,\qquad n\in\N^*
		\iff
		P\left\{\limsup_{n\to\infty}{A_n}\right\}=0
		\text{或}
		P\left\{\liminf_{n\to\infty}{A_n^c}\right\}=1
		$$
	\end{enumerate}
\end{theorem}

\begin{theorem}{Borel强大数定律}{Borel强大数定律}
	如果$\mu_n$为某事件在$n$次独立试验中出现的次数,且在每次试验中该事件出现的概率均为$p$,那么
	$$
	\lim_{n\to\infty}P\left\{\frac{\mu_n}{n}\to p\right\}=1
	$$
\end{theorem}

\subsection{Kolmogorov强大数定律}

\begin{theorem}{Kolmogorov不等式}{Kolmogorov不等式}
	对于相互独立的随机变量序列$\{ \xi_n \}_{n=1}^{\infty}$,如果对于任意$n\in\N^*$,$D(\xi_n)$存在且有限,那么对于任意$\varepsilon>0$,成立
	$$
	P\left\{\max_{1\le k\le n}{\left|\sum_{i=1}^{k}{(\xi_i-E(\xi_i))}\right|}\ge\varepsilon\right\}\le\frac{1}{\varepsilon^2}\sum_{k=1}^{n}{D(\xi_k)}
	$$
\end{theorem}

\begin{theorem}{Hājek—Rényi不等式}{Hājek—Rényi不等式}
	对于相互独立的随机变量序列$\{ \xi_n \}_{n=1}^{\infty}$,如果对于任意$n\in\N^*$,$D(\xi_n)=\sigma_n^2<\infty$,且$\{C_n\}_{n=1}^{\infty}\sub\R^+$为单调递减序列,那么对于任意$m<n\in\N^*$与$\varepsilon>0$,成立
	$$
	P\left\{\max_{m\le k\le n}{\left|\sum_{i=1}^{k}{(\xi_i-E(\xi)_i)}\right|}\ge\varepsilon\right\}\le\frac{1}{\varepsilon^2}\left(C_m^2\sum_{i=1}^{m}{\sigma_i^2}+\sum_{i=m+1}^{n}{C_i^2\sigma_i^2}\right)
	$$
\end{theorem}

\begin{theorem}{Kolmogorov强大数定律}{Kolmogorov强大数定律}
	对于相互独立的随机变量序列$\{ \xi_n \}_{n=1}^{\infty}$,如果%
	$$
	\sum_{n=1}^{\infty}{\frac{D(\xi_n)}{n^2}}<\infty
	$$
	那么%
	$$
	P\left\{\lim_{n\to\infty}{\frac{1}{n}{\sum_{k=1}^{n}{(\xi_k-E(\xi_k))}}}=0\right\}=1
	$$
\end{theorem}

\begin{theorem}{独立同分布场合的强大数定律}{独立同分布场合的强大数定律}
	对于相互独立且服从相同分布的随机变量序列$\{ \xi_n \}_{n=1}^{\infty}$,成立%
	$$
	\frac{1}{n}\sum_{k=1}^{n}{\xi_k}\xlongrightarrow{\mathrm{a.s.}}\mu\iff
	E(\xi_n)=\mu,n\in\N^*
	$$
\end{theorem}

\section{中心极限定理}

\textbf{目标}:对于相互独立且服从相同分布的随机变量序列$\{ \xi_n \}_{n=1}^{\infty}$,作标准化
$$
\zeta_n=\frac{\dis\sum_{k=1}^{n}{\xi_k}-\sum_{k=1}^{n}{E(\xi_k)}}{\dis\sqrt{\sum_{k=1}^{n}{D(\xi_k)}}},\qquad n\in\N^*
$$
本节寻找$\zeta_n$的分布函数为正态分布函数的充分必要条件。

\begin{definition}{Lindeberg条件}
	对于随机变量序列$\{ \xi_n \}_{n=1}^{\infty}$,令%
	$$
	\mu_n=E(\xi_n),\qquad 
	\sigma_n^2=D(\xi_n),\qquad 
	\Sigma_n^2=\sum_{k=1}^{n}\sigma_k^2,\qquad n\in\N^*
	$$
	对于任意$\tau>0$,成立
	$$
	\lim_{n\to\infty}{\frac{1}{\Sigma_n^2}\sum_{k=1}^{n}{\int_{|x-\mu_j|>\tau \Sigma_n}{(x-\mu_k)^2\mathrm{d}F_k(x)}}}=0
	$$
	其中$F_n(x)$为$\xi_n$的分布函数。
\end{definition}

\begin{definition}{Felelr条件}
	对于随机变量序列$\{ \xi_n \}_{n=1}^{\infty}$,令%
	$$
	\mu_n=E(\xi_n),\qquad 
	\sigma_n^2=D(\xi_n),\qquad 
	\Sigma_n^2=\sum_{k=1}^{n}\sigma_k^2,\qquad n\in\N^*
	$$
	成立%
	$$
	\lim_{n\to\infty}\max_{1\le k\le n}\frac{\sigma_k}{\Sigma_n}=0
	\iff \lim_{n\to\infty}{\Sigma_n}=\infty\text{且}\lim_{n\to\infty}\frac{\sigma_n}{\Sigma_n}=0
	$$
\end{definition}

\begin{theorem}{Lindeberg—Felelr定理}{Lindeberg—Felelr定理}
	对于相互独立且服从相同分布的随机变量序列$\{ \xi_n \}_{n=1}^{\infty}$,作标准化
	$$
	\zeta_n=\frac{\dis\sum_{k=1}^{n}{\xi_k}-\sum_{k=1}^{n}{E(\xi_k)}}{\dis\sqrt{\sum_{k=1}^{n}{D(\xi_k)}}},\qquad n\in\N^*
	$$
	那么成立
	$$
	\lim_{n\to\infty}{P\{\zeta_n<x\}}=\frac{1}{\sqrt{2\pi}}\int_{-\infty}^{x}{\mathrm{e}^{-\frac{t^2}{2}}\mathrm{d}t}
	$$
	与Felelr条件的充分必要条件是成立Lindeberg条件。
\end{theorem}

\begin{corollary}
	对于相互独立的随机变量序列$\{ \xi_n \}_{n=1}^{\infty}$,作标准化
	$$
	\zeta_n=\frac{\dis\sum_{k=1}^{n}{\xi_k}-\sum_{k=1}^{n}{E(\xi_k)}}{\dis\sqrt{\sum_{k=1}^{n}{D(\xi_k)}}},\qquad n\in\N^*
	$$
	令%
	$$
	\mu_n=E(\xi_n),\qquad 
	\sigma_n^2=D(\xi_n),\qquad 
	\Sigma_n^2=\sum_{k=1}^{n}\sigma_k^2,\qquad n\in\N^*
	$$
	如果存在常数$K_n$,使得成立
	$$
	\max_{1\le k\le n}{|\xi_k|}\le K_n,\qquad n\in\N^*
	$$
	且
	$$
	\lim_{n\to\infty}{\frac{K_n}{\Sigma_n}}=0
	$$
	那么
	$$
	\lim_{n\to\infty}{P\{\zeta_n<x\}}=\frac{1}{\sqrt{2\pi}}\int_{-\infty}^{x}{\mathrm{e}^{-\frac{t^2}{2}}\mathrm{d}t}
	$$
\end{corollary}

\begin{corollary}{Lyapunov定理}{Lyapunov定理}
	对于相互独立的随机变量序列$\{ \xi_n \}_{n=1}^{\infty}$,作标准化
	$$
	\zeta_n=\frac{\dis\sum_{k=1}^{n}{\xi_k}-\sum_{k=1}^{n}{E(\xi_k)}}{\dis\sqrt{\sum_{k=1}^{n}{D(\xi_k)}}},\qquad n\in\N^*
	$$
	令%
	$$
	\mu_n=E(\xi_n),\qquad 
	\sigma_n^2=D(\xi_n),\qquad 
	\Sigma_n^2=\sum_{k=1}^{n}\sigma_k^2,\qquad n\in\N^*
	$$
	如果存在$\delta>0$,使得成立
	$$
	\lim_{n\to\infty}\frac{1}{\Sigma_n^{2+\delta}}\sum_{k=1}^{n}{E(|\xi_k-\mu_k|^{2+\delta})}=0
	$$
	那么
	$$
	\lim_{n\to\infty}{P\{\zeta_n<x\}}=\frac{1}{\sqrt{2\pi}}\int_{-\infty}^{x}{\mathrm{e}^{-\frac{t^2}{2}}\mathrm{d}t}
	$$
\end{corollary}

\appendix

\chapter{收敛性,大数定律与中心极限定理}

\section{收敛性}

\subsection{分布函数}

\begin{definition}{Lebesgue-Stieltjes函数}
	称函数$F:\R^n\to\R$为Lebesgue-Stieltjes函数,如果$F$在$\R^n$上处处上连续,且$F$在任一方体$(\bs{a},\bs{b}]$上具有非负增量。
\end{definition}

\begin{definition}{分布函数 distributio function}
	称Lebesgue-Stieltjes函数$F:\R^n\to\R$为分布函数,如果$F$在$\R^n$上单调递增,且对于任意$1\le k \le n$,成立%
	$$
	F(x_1,\cdots,x_{k-1},-\infty,x_{k+1},\cdots,x_n)=0,\qquad
	F(\bs{+\infty})=1
	$$
\end{definition}

\subsection{随机变量的四大收敛性}

\begin{definition}{依$L^p$收敛 converge in $L^p$}
	对于随机变量序列$\{ X_n \}_{n=1}^{\infty}$与随机变量$X$,称$\{ X_n \}_{n=1}^{\infty}$依$L^p$收敛于$X$,并记作$X_n\toLp X$,如果
	$$
	\lim_{n\to\infty}E(|X_n-X|^p)=0
	$$
\end{definition}

\begin{definition}{几乎必然收敛 converge almost surely}
	对于随机变量序列$\{ X_n \}_{n=1}^{\infty}$与随机变量$X$,称$\{ X_n \}_{n=1}^{\infty}$几乎必然收敛于$X$,并记作$X_n\toas X$,如果对于任意$\varepsilon>0$,成立%
	$$
	P\left\{ \lim_{n\to\infty}|X_n-X| \right\}=1
	$$
\end{definition}

\begin{definition}{依概率收敛 converge in probability}
	对于随机变量序列$\{ X_n \}_{n=1}^{\infty}$与随机变量$X$,称$\{ X_n \}_{n=1}^{\infty}$依概率收敛于$X$,并记作$X_n\toP X$,如果对于任意$\varepsilon>0$,成立%
	$$
	\lim_{n\to\infty}P\{ |X_n-X|\ge \varepsilon \}=0
	$$
\end{definition}

\begin{definition}{弱收敛 converge weakly}
	对于有界Lebesgue-Stieltjes函数序列$\{ F_n:\R\to\R \}_{n=1}^{\infty}$与有界Lebesgue-Stieltjes函数$F$,称$\{ F_n:\R\to\R \}_{n=1}^{\infty}$弱收敛于$F$,并记作$F_n\tow F$,如果对于任意$F$的连续点$x$,成立$F_n(x)\to F(x)$,且$F_n(\pm\infty)\to F(\pm\infty)$。
\end{definition}

\begin{definition}{依分布收敛 converge in distribution}
	对于随机变量序列$\{ X_n \}_{n=1}^{\infty}$与随机变量$X$,$F_n$为$X_n$的分布函数,$F$为$X$的分布函数,称$\{ X_n \}_{n=1}^{\infty}$依分布收敛于$X$,并记作$X_n\tod X$,如果$F_n\tow F$。
\end{definition}

\begin{proposition}
	$$
	X_n \toLp X\implies
	X_n \toP X\implies
	X_n \tod X,\qquad
	X_n \tod X\implies
	X_n \toP X
	$$
\end{proposition}

\section{大数定律}

\subsection{大数定理的定义}

\begin{definition}{大数定律 weak law of large numbers}
	\begin{enumerate}
		\item 对于随机变量序列$\{ X_n \}_{n=1}^{\infty}$,$\dis S_n=\sum_{k=1}^{n}X_k$为其部分和,称$\{ X_n \}_{n=1}^{\infty}$服从古典意义下的大数定律,如果诸期望$E(X_n)$存在且有限,同时%
		$$
		\frac{S_n-E(S_n)}{n}\toP 0
		$$
		\item 对于随机变量序列$\{ X_n \}_{n=1}^{\infty}$,$\dis S_n=\sum_{k=1}^{n}X_k$为其部分和,称$\{ X_n \}_{n=1}^{\infty}$服从现代意义下的大数定律,如果存在中心化数列$\{ a_n \}_{n=1}^{\infty}$与正则化数列$\{ b_n \}_{n=1}^{\infty}$,其中$0<b_n\to \infty$,使得成立
		$$
		\frac{S_n-a_n}{b_n}\toP 0
		$$
	\end{enumerate}
\end{definition}

\subsection{经典大数定律}

\begin{theorem}{Khinchin大数定律}
	对于随机变量序列$\{ X_n \}_{n=1}^{\infty}$,$\dis S_n=\sum_{k=1}^{n}X_k$为其部分和,如果$\{ X_n \}_{n=1}^{\infty}$相互独立且同分布于随机变量$X$,同时存在有限期望$E(X)$,那么%
	$$
	\frac{S_n}{n}\toP E(X)
	$$
\end{theorem}

\begin{theorem}{Markov大数定律}
	对于随机变量序列$\{ X_n \}_{n=1}^{\infty}$,$\dis S_n=\sum_{k=1}^{n}X_k$为其部分和,如果诸方差$D(X_n)$存在且有限,且存在实数列$\{ b_n \}_{n=1}^{\infty}$,使得成立Markov条件%
	$$
	\frac{D(S_n)}{b_n^2}\to 0
	$$
	那么%
	$$
	\frac{S_n-E(S_n)}{b_n}\toP 0
	$$
\end{theorem}

\begin{theorem}{Chebyshëv大数定律}
	对于随机变量序列$\{ X_n \}_{n=1}^{\infty}$,$\dis S_n=\sum_{k=1}^{n}X_k$为其部分和,如果$\{ X_n \}_{n=1}^{\infty}$互不相关且方差一致有界,那么
	$$
	\frac{S_n-E(S_n)}{n}\toP 0
	$$
\end{theorem}

\begin{theorem}{Poisson大数定律}
	对于随机变量序列$\{ X_n \}_{n=1}^{\infty}$,$\dis S_n=\sum_{k=1}^{n}X_k$为其部分和,如果$\{ X_n \}_{n=1}^{\infty}$相互独立且
	$$
	P\{ X_n=1 \}=p_n\in(0,1),\qquad
	P\{ X_n=0 \}=1-p_n,\qquad
	n\in\N^*
	$$
	那么
	$$
	\frac{S_n-E(S_n)}{n}\toP 0
	$$
\end{theorem}

\begin{theorem}{Bernoulli大数定律}
	在每次成功概率为$p$的Bernoulli试验序列中,若以$\mu_n$表示前$n$次试验中成功的次数,那么%
	$$
	\frac{\mu_n}{n}\toP p
	$$
\end{theorem}

\begin{table}[htbp]
	\centering
	\caption{经典大数定律}
	\renewcommand{\arraystretch}{3}
	\resizebox{\linewidth}{!}{\begin{tabular}{|c|c|c|c|c|c|c|c|}
			\hline
			名称 & 分布 & 期望 & 方差 & 相关性 & 独立性 & 条件 & 结论 \\
			\hline
			Khinchin大数定律 & 同分布于$X$ & $E(X)$存在且有限 & & & 相互独立 & & $\frac{S_n}{n}\toP E(X)$ \\
			\hline
			Markov大数定理 & & $E(X_n)$存在且有限 & $D(X_n)$存在且有限 & & & $\frac{D(S_n)}{b_n^2}\to 0$ & $\frac{S_n-E(S_n)}{b_n}\toP 0$ \\
			\hline
			Chebyshëv大数定律 & & $E(X_n)$存在且有限 & $D(X_n)$存在且有限 & 互不相关 & & $D(X_n)$一致有界 &  $\frac{S_n-E(S_n)}{n}\toP 0$  \\
			\hline
			Poisson大数定律 & $P\{ X_n=1 \}=p_n,P\{ X_n=0 \}=1-p_n$ & & & & 相互独立 & & $\frac{S_n-E(S_n)}{n}\toP 0$ \\
			\hline
			Bernoulli大数定律 & 每次成功概率为$p$的Bernoulli试验序列 & & & & & 以$\mu_n$表示前$n$次试验中成功的次数 & $\frac{\mu_n}{n}\toP p$     \\
			\hline
	\end{tabular}}
\end{table}

\section{强大数定律}

\subsection{强大数定律的定义}

\begin{definition}{强大数定律 strong law of large numbers}
	\begin{enumerate}
		\item 对于随机变量序列$\{ X_n \}_{n=1}^{\infty}$,$\dis S_n=\sum_{k=1}^{n}X_k$为其部分和,称$\{ X_n \}_{n=1}^{\infty}$服从古典意义下的强大数定律,如果诸期望$E(X_n)$存在且有限,同时%
		$$
		\frac{S_n-E(S_n)}{n}\toas 0
		$$
		\item 对于随机变量序列$\{ X_n \}_{n=1}^{\infty}$,$\dis S_n=\sum_{k=1}^{n}X_k$为其部分和,称$\{ X_n \}_{n=1}^{\infty}$服从现代意义下的强大数定律,如果存在中心化数列$\{ a_n \}_{n=1}^{\infty}$与正则化数列$\{ b_n \}_{n=1}^{\infty}$,其中$0<b_n\to \infty$,使得成立
		$$
		\frac{S_n-a_n}{b_n}\toas 0
		$$
	\end{enumerate}
\end{definition}

\subsection{经典强大数定律}

\begin{theorem}{Kolmogorov强大数定律}
	对于相互独立且同分布于$X$的随机变量序列$\{ X_n \}_{n=1}^{\infty}$,$\dis S_n=\sum_{k=1}^{n}X_k$为其部分和,成立如下命题。
	\begin{enumerate}
		\item 若$E(|X|)$存在且有限,那么%
		$$
		\frac{S_n}{n}\toas E(X)
		$$
		\item 若$\frac{S_n}{n}\toas \mu$,那么$E(|X|)$存在且有限,同时$E(X)=\mu$。
	\end{enumerate}
\end{theorem}

\begin{theorem}{Marcinkiewicz-Zygmund强大数定律}
	对于相互独立且同分布于$X$的随机变量序列$\{ X_n \}_{n=1}^{\infty}$,$\dis S_n=\sum_{k=1}^{n}X_k$为其部分和,如下命题等价。
	\begin{enumerate}
		\item 存在常数$a\in\R$与$p\in(0,2)$,使得成立%
		$$
		\frac{S_n-na}{n^{-\frac{1}{p}}}\toas 0
		$$
		\item $p$阶矩$E(|X|^p)$存在且有限。
	\end{enumerate}
	此时当$0<p<1$时,$a$可取任意常数;当$1\le p<2$时,$a=E(X)$。
\end{theorem}

\begin{theorem}{独立但不同分布的强大数定律}
	对于相互独立的随机变量序列$\{ X_n \}_{n=1}^{\infty}$,$\dis S_n=\sum_{k=1}^{n}X_k$为其部分和,实数列$\{ b_n \}_{n=1}^{\infty}$成立$b_n>0$且单调递增趋于$\infty$,成立如下命题。
	\begin{enumerate}
		\item 如果存在$0<p\le 1$,使得诸$p$阶矩$E(|X|^p)$存在,且%
		$$
		\sum_{n=1}^{\infty}\frac{E(|X_n|^p)}{b_n^p}<\infty
		$$
		那么%
		$$
		\frac{S_n}{b_n}\toas 0
		$$
		\item 如果存在$1<p\le 2$,使得诸$p$阶矩$E(|X|^p)$存在,且%
		$$
		\sum_{n=1}^{\infty}\frac{E(|X_n|^p)}{b_n^p}<\infty
		$$
		那么%
		$$
		\frac{S_n-E(S_n)}{b_n}\toas 0
		$$
	\end{enumerate}
\end{theorem}

\section{中心极限定理}

\subsection{中心极限定理的定义}

\begin{definition}{中心极限定理 central limit theorem}
	\begin{enumerate}
		\item 对于随机变量序列$\{ X_n \}_{n=1}^{\infty}$,$\dis S_n=\sum_{k=1}^{n}X_k$为其部分和,称$\{ X_n \}_{n=1}^{\infty}$服从古典意义下的中心极限定理,如果诸方差$D(X_n)$存在且有限,同时%
		$$
		\frac{S_n-E(S_n)}{\sqrt{D(S_n)}}\tod N(0,1)
		$$
		\item 对于随机变量序列$\{ X_n \}_{n=1}^{\infty}$,$\dis S_n=\sum_{k=1}^{n}X_k$为其部分和,称$\{ X_n \}_{n=1}^{\infty}$服从现代意义下的中心极限定理,如果存在中心化数列$\{ a_n \}_{n=1}^{\infty}$与正则化数列$\{ b_n \}_{n=1}^{\infty}$,其中$0<b_n\to \infty$,使得成立
		$$
		\frac{S_n-a_n}{b_n}\tod N(0,1)
		$$
	\end{enumerate}
\end{definition}

\subsection{经典中心极限定理}

\begin{theorem}{Dé Moivre-Laplace中心极限定理}
	对于相互独立且同分布于$X$的随机变量序列$\{ X_n \}_{n=1}^{\infty}$,$\dis S_n=\sum_{k=1}^{n}X_k$为其部分和,如果%
	$$
	P\{ X=1 \}=p\in(0,1),\qquad
	P\{ X=0 \}=1-p
	$$
	那么%
	$$
	\frac{S_n-np}{\sqrt{np(1-p)}}\tod N(0,1)
	$$
\end{theorem}

\begin{theorem}{Lindeberg-Lévy中心极限定理}
	对于相互独立且同分布于$X$的随机变量序列$\{ X_n \}_{n=1}^{\infty}$,$\dis S_n=\sum_{k=1}^{n}X_k$为其部分和,如果期望$E(X)$存在且有限,方差$D(X)$存在且有限,那么%
	$$
	\frac{S_n-n E(X)}{\sqrt{nD(X)}}\tod N(0,1)
	$$
\end{theorem}

\begin{theorem}{Lindeberg-Feller中心极限定理}
	对于相互独立的随机变量序列$\{ X_n \}_{n=1}^{\infty}$,$\dis S_n=\sum_{k=1}^{n}X_k$为其部分和,诸期望$E(X_n)$存在且有限,诸方差$D(X_n)$存在且有限,令%
	$$
	\mu_n=E(X_n),\qquad
	\sigma_n^2=D(X_n),\qquad
	B_n^2=\sum_{k=1}^{n}\sigma_k^2
	$$
	那么
	\begin{enumerate}
		\item Feller条件:$\dis\lim_{n\to\infty}\max_{1\le k\le n}\frac{\sigma_k^2}{B_n^2}=0$
		\item $\dis\frac{S_n-E(S_n)}{B_n}\tod N(0,1)$
	\end{enumerate}
	成立的充分必要条件为Lindeberg条件,即对于任意$\varepsilon>0$,成立%
	$$
	\lim_{n\to\infty}\frac{1}{B_n^2}\sum_{k=1}^{n}E(X_k-\mu_k)^2I\{ |X_k-\mu_k|\ge\varepsilon B_n \}=0
	$$
\end{theorem}

\begin{theorem}{Lyapunov中心极限定理}
	对于相互独立的随机变量序列$\{ X_n \}_{n=1}^{\infty}$,$\dis S_n=\sum_{k=1}^{n}X_k$为其部分和,如果$\{ X_n \}_{n=1}^{\infty}$成立Lyapunov条件,即存在$\delta>0$,使得成立%
	$$
	\lim_{n\to\infty}\frac{1}{B_n^{2+\delta}}\sum_{k=1}^{n}E(|X_k-E(X_k)|^{2+\delta})=0
	$$
\end{theorem}

\chapter{概率模型}

\begin{table}[htbp]
	\centering
	\renewcommand{\arraystretch}{3}
	\resizebox{\linewidth}{!}{\begin{tabular}{|c|c|c|c|c|c|}
			\hline
			概率模型 & 密度函数$p(x)$ & 参数范围 & 数学期望$E(\xi)$ & 方差$D(\xi)$ & 特征函数$f(t)$ \\
			\hline
			退化分布$I_c(x)$ & $p(x)=\begin{cases}1,&x=c\\0,&x\ne c\end{cases}$ &  & $c$ & $0$ & $\mathrm{e}^{ict}$ \\
			\hline
			Bernoulli分布 & $p(x)=\begin{cases}1-p,&x=0\\p,&x=1\end{cases}$ & $0<p<1$ & $p$ & $p(1-p)$ & $p\mathrm{e}^{it}+1-p$ \\
			\hline
			二项分布$B(n,p)$ & $b(k;n,p)={{n}\choose{k}}p^k(1-p)^{n-k}$ & $0\le k \le n;0<p<1$ & $np$ & $np(1-p)$ & $(p\mathrm{e}^{it}+1-p)^n$ \\
			\hline
			Poisson分布$P(\lambda)$ & $p(k;\lambda)=\frac{\lambda^k}{k!}\mathrm{e}^{-\lambda}$ & $k\in\N;\lambda>0$ & $\lambda$ & $\lambda$ & $\mathrm{e}^{\lambda(\mathrm{e}^{it}-1)}$ \\
			\hline
			几何分布 & $g(k;p)=p(1-p)^{k-1}$ & $k\in\N^*,0<p<1$ & $\frac{1}{p}$ & $\frac{q}{p^2}$ & $\frac{p\mathrm{e}^{it}}{1-(1-p)\mathrm{e}^{it}}$ \\
			\hline
			超几何分布 & $p_k = \frac{{\binom{M}{k} \binom{N-M}{n-k}}}{{\binom{N}{n}}}$ & $M,n\le N;0\le  k\le \min\{M,n\}$ & $\frac{nM}{N}$ & $\frac{nM(N-M)(N-n)}{N^2(N-1)}$ & $\dis\sum_{k=0}^{n} \frac{{\binom{M}{k} \binom{N-M}{n-k}}}{{\binom{N}{n}}} e^{ikt}
			$ \\
			\hline
			Pascal分布 & $p_k={{k-1}\choose{r-1}}p^r(1-p)^{k-r}$ & $k\ge r,0<p<1$ & $\frac{r}{p}$ & $\frac{r(1-p)}{p^2}$ & $(\frac{(1-p)\mathrm{e}^{it}}{1-(1-p)\mathrm{e}^{it}})^r$ \\
			\hline
			负二项分布 & $p_k={{-r}\choose{k}}p^r(p-1)^k$ & $k\in\N,0<p<1,r>0$ & $\frac{r(1-p)}{p}$ & $\frac{r(1-p)}{p^2}$ & $(\frac{p}{1-(1-p)\mathrm{e}^{it}})^r$ \\
			\hline
			正态分布$N(\mu,\sigma^2)$ & $p(x)=\frac{1}{\sqrt{2\pi}\sigma}\mathrm{e}^{-\frac{(x-\mu)^2}{2\sigma^2}}$ &  & $\mu$ & $\sigma$ & $\mathrm{e}^{i\mu t-\frac{1}{2}\sigma^2t^2}$ \\
			\hline
			均匀分布$U[a,b]$ & $p(x)=\begin{cases}\frac{1}{b-a},&a\le x\le b\\0,&\text{其他}\end{cases}$ & $a<b$ & $\frac{a+b}{2}$ & $\frac{(b-a)^2}{12}$ & $\frac{\mathrm{e}^{ibt}-\mathrm{e}^{iat}}{i(b-a)t}$ \\
			\hline
			指数分布$\mathrm{Exp}(\lambda)$ & $p(x)=\begin{cases}\lambda\mathrm{e}^{-\lambda x},&x\ge 0\\0,&x<0\end{cases}$ & $\lambda>0$ & $\lambda^{-1}$ & $\lambda^{-2}$ & $(1-\frac{it}{\lambda})^{-1}$ \\
			\hline
			$\chi^2$分布 & $p(x)=\begin{cases}\frac{1}{2^{\frac{n}{2}}\Gamma(\frac{n}{2})}x^{\frac{n}{2}-1}\mathrm{e}^{-\frac{x}{2}},&x\ge 0\\0,&x<0\end{cases}$ & $n\in \N^*$ & $n$ & $2n$ & $(1-2it)^{-\frac{n}{2}}$ \\
			\hline
			$\Gamma$分布$\Gamma(r,\lambda)$ & $p(x)=\begin{cases}\frac{\lambda^r}{\Gamma(r)}x^{r-1}\mathrm{e}^{-\lambda x},&x\ge 0\\0,&x<0\end{cases}$ & $r,\lambda>0$ & $\frac{r}{\lambda}$ & $\frac{r}{\lambda^2}$ & $(1-\frac{it}{\lambda})^{-r}$ \\
			\hline
			Cauchy分布 & $p(x)=\frac{1}{\pi}\frac{\lambda}{\lambda^2+(x-\mu)^2}$ & $\mu\in\R,\lambda>0$ & 不存在 & 不存在 & $\mathrm{e}^{i\mu t-\lambda|t|}$ \\
			\hline
			$t$分布 & $p(x)=\frac{\Gamma(\frac{n+1}{2})}{\sqrt{n\pi}\Gamma(\frac{n}{2})}(1+\frac{x^2}{n})^{-\frac{n+1}{2}}$ & $n\in\N^*$ & $0(n>1)$ & $\frac{n}{n-2}(n>2)$ &  \\
			\hline
			Pareto分布 & $p(x)=\begin{cases}rA^r\frac{1}{x^{r+1}},&x\ge A\\0,&x<A\end{cases}$ & $r,A>0$ & ($r>1$时存在) & ($r>2$时存在) &  \\
			\hline
			$F$分布 & $p(x)=\begin{cases}\frac{\Gamma(\frac{m+n}{2})}{\Gamma(\frac{m}{2})\Gamma(\frac{n}{2})}m^{\frac{m}{2}}n^{\frac{n}{2}}\frac{x^{\frac{m}{2}-1}}{(n+mx)^{\frac{m+n}{2}}},&x\ge 0\\0,&x<0\end{cases}$ & $m,n\in \N^*$ & $\frac{n}{n-2}(n>2)$ & $\frac{2n^2(m+n-2)}{m(n-2)^2(n-4)}(n>4)$ &  \\
			\hline
			$\beta$分布 & $p(x)=\begin{cases}\frac{\Gamma(p+q)}{\Gamma(p)\Gamma(q)}x^{p-1}(1-x)^{q-1},&0<x<1\\0,&\text{其他}\end{cases}$ & $p,q>0$ & $\frac{p}{p+q}$ & $\frac{pq}{(p+q)^2(p+q+1)}$ & $\dis\frac{\Gamma(p+q)}{\Gamma(p)}\sum_{k=0}^{\infty}{\frac{\Gamma(p+k)(it)^k}{\Gamma(p+q+k)\Gamma(k+1)}}$ \\
			\hline
			对数正态分布 & $p(x)=\begin{cases}\frac{1}{\sqrt{2\pi}\sigma x}\mathrm{e}^{\frac{(\ln x-\alpha)^2}{2\sigma^2}},&x>0\\0,&x\le0\end{cases}$ & $\alpha,\sigma>0$ & $\mathrm{e}^{\alpha+\frac{\sigma^2}{2}}$ & $\mathrm{e}^{2\alpha+\sigma^2}(\mathrm{e}^{\sigma^2}-1)$ &  \\
			\hline
			Weibull分布 & $p(x)=\begin{cases}\alpha\lambda x^{\alpha-1}\mathrm{e}^{-\lambda x^{\alpha}},&x>0\\0,&x\le0\end{cases}$ & $\lambda,\alpha>0$ & $\Gamma(\frac{1}{\alpha}+1)\lambda^{-\frac{1}{\alpha}}$ & $\lambda^{-\frac{2}{\alpha}}(\Gamma(\frac{2}{\alpha}+1)-(\Gamma(\frac{1}{\alpha}+1))^2)$ &  \\
			\hline
			Rayleigh分布 & $p(x)=\begin{cases}x\mathrm{e}^{-\frac{x^2}{2}},&x\ge 0\\0,&x<0\end{cases}$ &  & $\sqrt{\frac{\pi}{2}}$ & $2-\frac{\pi}{2}$ &  \\
			\hline
	\end{tabular}}
\end{table}

\end{document}