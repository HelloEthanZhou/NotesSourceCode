\documentclass[lang = cn, scheme = chinese, 10pt]{elegantbook}
% elegantbook:     设置elegantbook文档类
% lang = cn:       设置中文环境
% scheme = chinese:设置标题为中文
% 10pt:            设置字体大小


%% 1.封面设置

\title{泛函分析 - 江泽坚 -课后习题}                % 文档标题

\author{若水}                        % 作者

\myemail{ethanmxzhou@163.com}       % 邮箱

\homepage{helloethanzhou.github.io} % 主页

\date{\today}                       % 日期

\logo{PiCreatures_happy.pdf}        % 设置Logo

\cover{阿基米德螺旋曲线.pdf}          % 设置封面图片

% 修改标题页的橙色带
\definecolor{customcolor}{RGB}{32,178,170} % 定义一个名为customcolor的颜色,RGB颜色值为(32, 178, 170)

\colorlet{coverlinecolor}{customcolor}     % 将coverlinecolor颜色设置为customcolor颜色

%% 2.目录设置
\setcounter{tocdepth}{3}  % 目录深度为3

%% 3.引入宏包
\usepackage[all]{xy}
\usepackage{bbm, svg, graphicx, float, extpfeil, amsmath, amssymb, mathrsfs, mathalpha}


%% 4.定义命令
\newcommand{\N}{\mathbb{N}}  % 自然数集合
\newcommand{\R}{\mathbb{R}}  % 实数集合
\newcommand{\C}{\mathbb{C}}  % 复数集合
\newcommand{\Q}{\mathbb{Q}}  % 有理数集合
\newcommand{\Z}{\mathbb{Z}}  % 整数集合
\newcommand{\sub}{\subset}   % 包含
\newcommand{\im}{\text{im }}   % 像
\newcommand{\lang}{\langle}   % 左尖括号
\newcommand{\rang}{\rangle}   % 右尖括号

\begin{document}

\maketitle       % 创建标题页

\frontmatter     % 开始前言部分

\chapter*{致谢}

\markboth{致谢}{致谢}

\vspace*{\fill}
\begin{center}
	
	\large{由衷感谢 \textbf{ 胡前锋 } 老师对于本课程的帮助}
	
\end{center}
\vspace*{\fill}

\tableofcontents % 创建目录

\mainmatter      % 开始正文部分

\chapter{线性度量空间}

\begin{proposition}
	证明:在线性空间中,对于任意向量$x$和数$\alpha$,成立
	\nonumber\begin{align}
		& 0x=0\\
		& (-1)x=-x\\
		&\alpha0=0
	\end{align}
\end{proposition}

\begin{proof}
	对于$x$,注意到
	$$
	x+0x=1x+0x=(1+0)x=1x=x
	$$
	因此由零元的唯一性,$0x=0$。于是
	$$
	x+(-1)x=1x+(-1)x=(1-1)x=0x=0
	$$
	因此由逆元的唯一性,$(-1)x=-x$。又注意到
	$$
	\alpha0=\alpha(0x)=(\alpha0)x=0x=0
	$$
	因此原命题得证!
\end{proof}

\begin{proposition}
	证明:下述消去律在线性空间中成立。
	\nonumber\begin{align}
		&x+y=x+z\implies y=z\\
		&\alpha x=\alpha y,\alpha\ne0\implies x=y\\
		&\alpha x=\beta x,x\ne0\implies \alpha=\beta
	\end{align}
\end{proposition}

\begin{proof}
	对于第一式,注意到
	\nonumber\begin{align}
		&x+y=x+z\\
		\implies & -x+x+y=-x+x+z \\
		\implies & (-x+x)+y=(-x+x)+z \\
		\implies & 0+y=0+z \\
		\implies & y=z
	\end{align}
	下面两式蕴含于如下命题
	$$
	\alpha x=0\implies 或\alpha=0,或x=0
	$$
	如果$\alpha=0$,那么命题成立。如果$\alpha\ne0$,那么
	$$
	\alpha x=0\implies  \frac{1}{\alpha}(\alpha x)=\frac{1}{\alpha}0\implies (\frac{1}{\alpha}\alpha) x=0 \implies 1x=0\implies x=0
	$$
	于是命题得证!
\end{proof}

\begin{proposition}
	证明:在空间$(s)$中,序列$\{x_n\}_{n=1}^{\infty}$按坐标收敛于$x_0$,当且仅当$\{x_n\}_{n=1}^{\infty}$按度量$d$收敛于$x_0$。
\end{proposition}

\begin{proof}
	令$x_n=\{ x_i^{(n)} \}_{i=1}^{\infty}$,$x_0=\{ x_i^{(0)} \}_{i=1}^{\infty}$。
	
	对于充分性,如果$\{x_n\}_{n=1}^{\infty}$按度量$d$收敛于$x_0$,即$x_n\xrightarrow{d}x_0$,我们的目标是证明对于任意$n\in\N^*$,成立$x_i^{(n)}\to x_i^{(0)}$。
	
	任取$n\in\N^*$以及$\varepsilon>0$,由于$x_n\xrightarrow{d}x_0$,那么存在$M\in\N^*$,使得对于任意$m>M$,成立
	$$
	\sum_{i=m}^{\infty}\frac{1}{2^i}\frac{|x_i^{(n)}-x_i^{(0)}|}{1+|x_i^{(n)}-x_i^{(0)}|}<\varepsilon
	\implies 
	\frac{|x_m^{(n)}-x_m^{(0)}|}{1+|x_m^{(n)}-x_m^{(0)}|}<\varepsilon
	\implies
	|x_m^{(n)}-x_m^{(0)}|<\varepsilon
	$$
	因此$\{x_n\}_{n=1}^{\infty}$按坐标收敛于$x_0$。
	
	对于必要性,如果序列$\{x_n\}_{n=1}^{\infty}$按坐标收敛于$x_0$,即对于任意$n\in\N^*$,成立$x_i^{(n)}\to x_i^{(0)}$,我们的目标是证明$x_n\xrightarrow{d}x_0$,即
	$$
	d(\{x_i^{(n)}\},\{x_i^{(0)}\})=\sum_{i=1}^{\infty}\frac{1}{2^i}\frac{|x_i^{(n)}-x_i^{(0)}|}{1+|x_i^{(n)}-x_i^{(0)}|}\to0
	$$
	任取$\varepsilon>0$,由于$\displaystyle\sum_{i=n}^{\infty}\frac{1}{2^i}\to0$且对于任意$i\in\N^*$,成立$x_i^{(n)}\to x_i^{(0)}$,那么存在$M,N\in\N^*$,使得成立$\displaystyle\sum_{i=M+1}^{\infty}\frac{1}{2^i}<\varepsilon/2$,且对于任意$1\le i \le M$,以及任意$n>N$,成立$|x_i^{(n)}-x_i^{(0)}|<\varepsilon/2$,于是
	\nonumber\begin{align}
		&\sum_{i=1}^{\infty}\frac{1}{2^i}\frac{|x_i^{(n)}-x_i^{(0)}|}{1+|x_i^{(n)}-x_i^{(0)}|}\\
		= & \sum_{i=1}^{M}\frac{1}{2^i}\frac{|x_i^{(n)}-x_i^{(0)}|}{1+|x_i^{(n)}-x_i^{(0)}|}+
		\sum_{i=M+1}^{\infty}\frac{1}{2^i}\frac{|x_i^{(n)}-x_i^{(0)}|}{1+|x_i^{(n)}-x_i^{(0)}|}\\
		\le & \sum_{i=1}^{M}\frac{\varepsilon}{2^{i+1}}+\sum_{i=M+1}^{\infty}\frac{1}{2^i}\\
		< & \frac{\varepsilon}{2}+\frac{\varepsilon}{2}\\
		= & \varepsilon
	\end{align}
	因此$\{x_n\}_{n=1}^{\infty}$按如下度量$d$收敛于$x_0$。
	
	原命题得证!
\end{proof}

\begin{proposition}
	证明:空间$(c)$是可分的。
\end{proposition}

\begin{proof}
	定义
	$$
	S=\{ x_r=\{ r_1,\cdots,r_n,r_n,\cdots \} \}:r_i\in\Q \}
	$$
	显然$S$为可数集。
	
	下面证明$S$为稠密集,任取$\varepsilon>0$以及$x=\{x_n\}_{n=1}^{\infty}\in (c)$,记$x_n\to x_0$,那么存在$N\in\N^*$,使得当$n>N$时,$|x_n-x_0|<\varepsilon/2$。当$1\le n \le N$时,取$r_n$满足$|r_n-x_n|<\varepsilon$;当$n>N$时,取$r_{N+1}$满足$|r_{N+1}-x_0|<\varepsilon/4$,此时成立
	$$
	|r_{N+1}-x_n|\le|x_n-x_0|+|r_{N+1}-x_0|<\varepsilon
	$$
	那么令$x_r=\{ r_1,\cdots,r_N,r_{N+1},r_{N+1},\cdots \}$,于是
	$$
	d(x_r,x)=\sup_{n\in\N^*}|r_n-x_n|=\max( \sup_{1\le n \le N}|r_n-x_n|,\sup_{n>N}|r_{N+1}-x_n| )\le \varepsilon
	$$
	
	因此$S$为稠密集。
\end{proof}

\begin{proposition}
	证明:如果$\{x_n\}_{n=1}^{\infty}$和$\{y_n\}_{n=1}^{\infty}$是度量空间$( X,d )$中的两个Cauchy序列,那么$\{d(x_n,y_n)\}_{n=1}^{\infty}$是Cauchy序列。
\end{proposition}

\begin{proof}
	由于$\{x_n\}_{n=1}^{\infty}$和$\{y_n\}_{n=1}^{\infty}$是度量空间$( X,d )$中的两个Cauchy序列,那么任取$\varepsilon>0$,存在$N\in\N^*$,使得对于任意$m,n > N$,成立
	$$
	d(x_m,x_n)<\varepsilon/2,\qquad d(y_m,y_n)<\varepsilon/2
	$$
	进而
	\nonumber\begin{align}
		&|d(x_m,y_m)-d(x_n,y_n)|\\
		= & |d(x_m,y_m)-d(y_m,x_n)+d(y_m,x_n)-d(x_n,y_n)|\\
		\le & |d(x_m,y_m)-d(y_m,x_n)|+|d(y_m,x_n)-d(x_n,y_n)|\\
		\le & d(x_m,x_n)+d(y_m,y_n)\\
		<& \varepsilon
	\end{align}
	因此$\{d(x_n,y_n)\}_{n=1}^{\infty}$是Cauchy序列。
\end{proof}

\begin{proposition}
	证明:度量空间中的Cauchy序列有界。
\end{proposition}

\begin{proof}
	设$\{ x_n \}_{n=1}^{\infty}\sub X$是度量空间$( X,d )$中的Cauchy序列,那么存在$N\in\N^*$,使得对于任意$n>N$时,成立$d(x_n,d_N)<1$,因此$\{ x_n \}_{n=N}^{\infty}$有界,进而$\{ x_n \}_{n=1}^{\infty}$有界。
\end{proof}

\begin{proposition}
	对于度量空间$(X,d)$,给定子集$A\sub X$,定义
	$$
	d(x)=\inf\{ d(x,y):y\in A \},\qquad x\in X
	$$
	证明:$d(x)$连续。
\end{proposition}

\begin{proof}
	事实上,$d$为Lipschitz连续的。
	
	任取$x,y,z\in\R^n$,那么由三角不等式
	$$
	||x-z|-|y-z||\le |x-y|
	$$
	对$z\in E$取下确界,那么
	$$
	|f(x)-f(y)|\le |x-y|
	$$
	于是$f$为Lipschitz连续的,进而$f$为连续的。
\end{proof}

\begin{proposition}
	对于$S\sub\R^n$,$C(S)$为$S$上的全体有界连续函数构成的线性空间,定义度量为
	$$
	d(f,g)=\sup_{x\in S}|f(x)-g(x)|
	$$
	证明:$C(S)$是完备的度量线性空间。
\end{proposition}

\begin{proof}
	首先,证明$C(S)$为度量空间。
	
	$1.$对于正则性,显然成立$d(f,g)\ge 0$。如果$f=g$,那么显然$d(f,g)=0$;如果$d(f,g)=0$,那么$\sup_{x\in S}|f(x)-g(x)|=0$,显然有$f=g$。
	
	$2.$对于对称性,显然成立。
	
	$3.$对于三角不等式,任取$f,g,h\in C(S)$,注意到,任取$x\in S$
	\nonumber\begin{align}
		&|f(x)-h(x)|\\
		\le&|f(x)-g(x)|+|g(x)-h(x)|\\
		\le&\sup_{x\in S}|f(x)-g(x)|+\sup_{x\in S}|g(x)-h(x)|\\
		=&d(f,g)+d(g,h)
	\end{align}
	由$x\in S$的任意性
	$$
	d(f,h)=\sup_{x\in S}|f(x)-h(x)|\le d(f,g)+d(g,h)
	$$
	因此$C(S)$为度量空间。
	
	其次,证明$C(S)$为度量线性空间。任取$f_n\xrightarrow{d}f,g_n\xrightarrow{d}g,\alpha_n\to\alpha$。
	
	$1.$对于加法运算的连续性,对于任意$\varepsilon>0$,存在$N\in\N^*$,使得当$n>N$时,成立
	$$
	d(f_n,f)<\varepsilon/2,\qquad 
	d(g_n,g)<\varepsilon/2
	$$
	因此当$n>N$时,成立
	$$
	d(f_n+g_n,f+g)\le d(f_n+g_n,f+g_n)+d(f+g_n,f+g)=d(f_n,f)+d(g_n,g)<\varepsilon
	$$
	于是
	$$
	\lim_{n\to\infty}d(f_n+g_n,f+g)=0
	$$
	
	$2.$对于数乘运算的连续性,任取$\varepsilon>0$,存在$N\in\N^*$,使得对于任意$n>N$,成立
	$$
	|\alpha_n-\alpha|<\varepsilon,\qquad 
	d(f_n,f)<\varepsilon
	$$
	由于$f$有界,那么存在$M>0$,使得成立$|f|<M$,因此当$n>N$时,成立
	$$
	|f_n|\le|f|+|f_n-f|<M+\varepsilon
	$$
	进而当$n>N$时,成立
	$$
	d(\alpha_n f_n,\alpha f)\le d(\alpha_nf_n,\alpha f_n)+d(\alpha f_n,\alpha f)=|\alpha|d(f_n,f)+|\alpha_n-\alpha|\sup|f_n|<\varepsilon(|\alpha|+M+\varepsilon)
	$$
	因此
	$$
	\lim_{n\to\infty}d(\alpha_n f_n,\alpha f)=0
	$$
	
	最后,证明$C(S)$为完备的,即证明对于Cauchy序列封闭。任取Cauchy序列$\{f_n(x)\}_{n=1}^{\infty}\sub C(S)$,那么对于任意$\varepsilon>0$,存在$N\in\N^*$,使得对于任意$m,n\ge N$,成立
	\begin{equation}\label{eq1}
		\small{d(f_n(x),f_m(x))<\varepsilon\implies \sup_{x\in S}|f_n(x)-f_m(x)|<\varepsilon\implies 
			|f_n(x)-f_m(x)|<\varepsilon,\forall x\in S}
	\end{equation}
	因此对于任意$x\in S$,数列$\{f_n(x)\}_{n=1}^{\infty}$收敛,记$f_n(x)\to f(x)$,在$(\ref{eq1})$式中令$m\to\infty$,可得
	\begin{equation}\label{eq2}
		|f_n(x)-f(x)|<\varepsilon,\qquad \forall n\ge N,\forall x\in S
	\end{equation}
	下面证明$f(x)\in C(S)$,即证明$f(x)$有界且连续。
	
	对于有界性。由于$\{f_n(x)\}_{n=1}^{\infty}$有界,那么对于任意$n\in\N^*$,存在$M_n>0$,使得对于任意$s\in S$,成立$|f_n(x)|<M_n$。由(\ref{eq2})式
	$$
	|f(x)|\le |f(x)-f_{N}(x)|+|f_{N}(x)|\le \varepsilon+M_{N},\qquad \forall x\in S
	$$
	于是$f(x)$有界。
	
	对于连续性。由(\ref{eq2})式,$f_n(x)$一致收敛于$f(x)$,因此$f(x)$连续。事实上,任取$x_0\in S$,在(\ref{eq2})式中,令$x_0+h\in S$,那么成立
	$$
	|f_N(x_0)-f(x_0)|<\varepsilon,\qquad |f_N(x_0+h)-f(x_0+h)|<\varepsilon
	$$
	又由于$f_n(x)$连续,那么存在$\delta>0$,使得当$|h|<\delta$且$x_0+h\in S$时
	$$
	|f_N(x_0+h)-f_N(x_0)|<\varepsilon
	$$
	于是由三角不等式
	$$
	\small{
		|f(x_0+h)-f(x_0)|\le |f(x_0+h)-f_N(x_0+h)|+|f_N(x_0+h)-f_N(x_0)|+|f_N(x_0)-f(x_0)|<3\varepsilon
	}
	$$
	因此$f(x)$在$x_0$处连续,由$x_0$的任意性,$f(x)$为连续函数,于是$f(x)\in C(S)$,由(\ref{eq2})式
	$$
	\sup_{x\in S}|f_n(x)-f(x)|<\varepsilon,\forall n\ge N\implies d(f_n(x),f(x))<\varepsilon,\forall n\ge N\implies f_n(x)\xrightarrow{d}f(x)
	$$
	因此$C(S)$为完备的。
	
	综上所述,原命题得证!
\end{proof}

\begin{proposition}
	证明:$l^p$空间是完备的度量空间,其中$1\le p <\infty$。
\end{proposition}

\begin{proof}
	我们只证完备性。任取Cauchy序列$\{ \{x_n^{(m)}\}_{n=1}^\infty \}_{m=1}^\infty\sub l^p$,于是对于任意$\varepsilon$,存在$M>0$,使得对于任意$i,j\ge M$,成立
	$$
	d(\{x_n^{(i)}\}_{n=1}^\infty,\{x_n^{(j)}\}_{n=1}^\infty)<\varepsilon
	\implies \sum_{n=1}^{\infty}|x_n^{(i)}-x_n^{(j)}|^p<\varepsilon^p
	\implies |x_n^{(i)}-x_n^{(j)}|^p<\varepsilon^p,\forall n\in\N^*
	$$
	于是对于任意$n\in\N^*$,数列$\{x_n^{(m)}\}_{m=1}^{\infty}$为Cauchy序列,记$\displaystyle \lim_{m\to\infty} x_n^{(m)}=x_n$,下面我们证明$\{x_n\}_{n=1}^{\infty}\in l^p$且$\displaystyle \lim_{m\to\infty}d(\{x_n^{(m)}\}_{n=1}^\infty,\{x_n\}_{n=1}^\infty)=0$。
	
	注意到,对于任意$m\ge M$,成立$\displaystyle\sum_{n=1}^{\infty}|x_n^{(M)}-x_n^{(m)}|^p<\varepsilon^p$,令$m\to\infty$,可得$\displaystyle\sum_{n=1}^{\infty}|x_n^{(M)}-x_n|^p<\varepsilon^p$,于是
	\nonumber\begin{align}
		&\sum_{n=1}^{\infty}|x_n|^p\\
		\le&\left( \left(\sum_{n=1}^{\infty}|x_n^{(M)}-x_n|^p\right)^{1/p}+\left(\sum_{n=1}^{\infty}|x_n^{(M)}|^p\right)^{1/p} \right)^p\\
		<&\left( \varepsilon+\left(\sum_{n=1}^{\infty}|x_n^{(M)}|^p\right)^{1/p} \right)^p\\
		<&\infty
	\end{align}
	因此$\{x_n\}_{n=1}^{\infty}\in l^p$。注意到,对于任意$k,m\ge M$,成立$\displaystyle\sum_{n=1}^{\infty}|x_n^{(m)}-x_n^{(k)}|^p<\varepsilon^p$,令$k\to\infty$,可得$\displaystyle\sum_{n=1}^{\infty}|x_n^{(m)}-x_n|^p<\varepsilon^p$,于是$d(\{x_n^{(m)}\}_{n=1}^\infty,\{x_n\}_{n=1}^\infty)<\varepsilon$,因此$\displaystyle \lim_{m\to\infty}d(\{x_n^{(m)}\}_{n=1}^\infty,\{x_n\}_{n=1}^\infty)=0$。
	
	综上所述,$l^p$空间是完备的度量空间。
\end{proof}

\begin{proposition}
	对于赋范线性空间$X$,$A\sub X$为有界子集,证明:$A$为完全有界的$\iff$对于任意$\varepsilon>0$,存在$X$中的有限维子空间$M$,使得对于任意$a\in A$,$d(a,M)<\varepsilon$。
\end{proposition}

\begin{proof}
	对于必要性,如果$A$为完全有界的,那么对于任意$\varepsilon>0$,存在$X$中的有限维子空间$M$,使得对于任意$a\in A$,存在$m\in M$,成立$d(a,m)<\varepsilon$,进而$d(a,M)\le d(a,m)<\varepsilon$。
	
	对于充分性,如果对于任意$\varepsilon>0$,存在$X$中的有限维子空间$M$,使得对于任意$a\in A$,$d(a,M)<\varepsilon$,那么存在$m\in M$,成立$d(a,m)<\varepsilon$,因此$A$为完全有界的。
\end{proof}

\begin{proposition}
	证明:对于赋范线性空间$X$,成立
	$$
	X\text{为有限维赋范线性空间}\iff
	B_r=\{ x\in X:\|x\|< r \}\text{为列紧子集}
	$$
\end{proposition}

\begin{proof}
	如果$X$为$n$维赋范线性空间,那么$X\cong E^n$,其中$E^n$为$n$为欧式空间,而$B_r$为有界子集,那么$B_r$为列紧子集。
	
	如果$X$为无限维赋范线性空间,那么任取$x_1\in B_r\setminus\{0\}$,取$x_2=-x_1/(2\|x_1\|)\in B_r$,那么$\|x_1-x_2\|=\|x_1\|+1/2>1/2$。
	
	假设已经选取$\{ x_k \}_{k=1}^{n}\subset B_r$,使得对于任意$i\ne j$,成立$\|x_i-x_j\|>1/2$,那么记$M_n=\mathrm{Sp}\{ x_k \}_{k=1}^{n}$,于是$M_n$为有限维子空间,因此$M_n$为完备度量空间,进而$M_n$是闭的真线性子空间。由Riesz引理,存在$x_{n+1}\in B_r$,使得成立$\|x_{n+1}\|=1$,且对于任意$1\le k\le n$,成立$\|x_{n+1}-x_k\|>1/2$。
	
	递归的,存在$\{ x_n \}_{n=1}^{\infty}\subset B_r$,使得对于任意$i\ne j$,$\|x_i-x_j\|>1/2$,因此$\{ x_n \}_{n=1}^{\infty}$没有收敛子列,进而$B_r$不为列紧子集。
\end{proof}

\begin{proposition}
	$\R^n$以$\|\cdot\|_2$形成赋范线性空间。
\end{proposition}

\begin{proof}
	显然!
\end{proof}

\begin{proposition}
	在$\R^n$中定义度量
	$$
	\rho(\{x_k\}_{k=1}^n,\{y_k\}_{k=1}^n)=\max_{1\le k\le n}\{ |x_k-y_k| \}
	$$
	其中$x=(x_1,\cdots,x_n),y_n=(y_1,\cdots,y_n)$。证明:$( \R^n,\rho )$为完备的度量线性空间。
\end{proposition}

\begin{proof}
	我们只证完备性。任取Cauchy序列$\{ \{ x_k^{(m)} \}_{k=1}^{n} \}_{m=1}^{\infty}\sub\R^n$,于是对于任意$\varepsilon$,存在$M>0$,使得对于任意$i,j\ge M$,成立
	$$
	\rho(\{x_k^{(i)}\}_{k=1}^n,\{x_k^{(j)}\}_{k=1}^n)<\varepsilon
	\implies \max_{1\le k\le n}\{ |x_k^{(i)}-x_k^{(j)}| \}<\varepsilon
	\implies |x_k^{(i)}-x_k^{(j)}|<\varepsilon,\forall 1\le k\le n
	$$
	于是对于任意$1\le k\le n$,数列$\{ x_k^{m} \}_{m=1}^\infty$为Cauchy序列,记$\displaystyle \lim_{m\to\infty} x_k^{(m)}=x_k$,于是$\{x_k\}_{k=1}^{n}\in \R^n$,下面我们证明$\displaystyle \lim_{m\to\infty}\rho(\{x_k^{(m)}\}_{k=1}^n,\{x_k\}_{k=1}^n)=0$。
	
	注意到,对于任意$r,m\ge M$,成立$\displaystyle\max_{1\le k\le n}\{ |x_k^{(m)}-x_k^{(r)}| \}<\varepsilon$,令$r\to\infty$,可得$\displaystyle\max_{1\le k\le n}\{ |x_k^{(m)}-x_k| \}<\varepsilon$,于是$\rho(\{x_k^{(m)}\}_{k=1}^n,\{x_k\}_{k=1}^n)<\varepsilon$,因此$\displaystyle \lim_{m\to\infty}\rho(\{x_k^{(m)}\}_{k=1}^n,\{x_k\}_{k=1}^n)=0$。
	
	综上所述,$(\R^n,\rho)$为完备的度量空间。
\end{proof}

\begin{proposition}
	对于完备度量空间$(X,d)$,如果$F\sub X$为闭集,那么$(F,d)$为完备的度量空间。
\end{proposition}

\begin{proof}
	$( F,d )$为度量空间是显然的,下面我们证明$( F,d )$的完备性。
	
	任取Cauchy序列$\{ x_n \}_{n=1}^{\infty}\sub F\sub X$,那么由于$X$的完备性,存在$x\in X$,使得成立$x_n\xrightarrow{d}x$。任取$r>0$,由于$x_n\xrightarrow{d}x$,那么存在$N>0$,使得当$n\ge N$时,成立$d(x_n,x)<r$,即$x_n\in B_r(x)$。如果对于任意$n\ge N$,成立$x_n=x$,那么$x=x_N\in F$;如果存在$n_0\ge N$,使得成立$x_{n_0}\ne x$,那么$B_r(x)\cap F\supset \{ x_{n_0} \}\ne\varnothing$,于是$x$为$F$的极限点,由于$F$为闭集,那么$x\in F$,于是Cauchy序列在$F$中依度量$d$收敛于$x\in F$,进而$( F,d )$完备的,原命题得证!
\end{proof}

\begin{proposition}
	对于$1\le p<\infty$,子集$X\sub l^p$为列紧的,当且仅当如下命题同时成立。
	
	1.存在$M>0$,使得对于任意数列$\{x_n\}_{n=1}^{\infty}\in X$,成立$\displaystyle\sum_{n=1}^{\infty}|x_n|^p<M$。
	
	2.对于任意$\varepsilon>0$,存在$N\in\N^*$,使得对于任意数列$\{x_n\}_{n=1}^{\infty}\in X$,成立$\displaystyle\sum_{n=N+1}^{\infty}|x_n|^p<\varepsilon$。
\end{proposition}

\begin{proof}
	对于必要性,任取$\varepsilon>0$,如果$X$是列紧的,那么$X$是完全有界的,于是存在有限数列序列$\{ \{x_n^{(1)}\}_{n=1}^{\infty},\cdots,\{x_n^{(m)}\}_{n=1}^{\infty} \}\sub X$,使得对于任意数列$\{x_n\}_{n=1}^{\infty}\in X$,存在$k=1,\cdots,m$,使得成立$\displaystyle\sum_{n=1}^{\infty}|x_n-x_n^{(k)}|^p<\frac{\varepsilon}{2^p}$。
	
	对于数列序列$\{ \{x_n^{(1)}\}_{n=1}^{\infty},\cdots,\{x_n^{(m)}\}_{n=1}^{\infty} \}$,存在$N\in\N^*$,使得对于任意$k=1,\cdots,n$,成立$\displaystyle\sum_{n=N}^{\infty}|x_n^{(k)}|^p<\frac{\varepsilon}{2^p}$。
	
	记$\displaystyle M^{1/p}=\frac{\varepsilon^{1/p}}{2}+\max_{1\le k\le m}\left(\sum_{n=1}^{\infty}|x_n^{(k_0)}|^p\right)^{1/p}$,任取数列$\{x_n\}_{n=1}^{\infty}\in X$,于是存在$k_0=1,\cdots,m$,使得成立$\displaystyle\sum_{n=1}^{\infty}|x_n-x_n^{(k_0)}|^p<\frac{\varepsilon}{2^p}$,因此
	\nonumber\begin{align}
		&\sum_{n=1}^{\infty}|x_n|^p\\
		\le &\left(\left(\sum_{n=1}^{\infty}|x_n-x_n^{(k_0)}|^p\right)^{1/p}+\left(\sum_{n=1}^{\infty}|x_n^{(k_0)}|^p\right)^{1/p}\right)^p\\
		<&\left(\left(\frac{\varepsilon}{2^p}\right)^{1/p}+\max_{1\le k\le m}\left(\sum_{n=1}^{\infty}|x_n^{(k_0)}|^p\right)^{1/p}\right)^p\\
		=&M
	\end{align}
	\nonumber\begin{align}
		&\sum_{n=N}^{\infty}|x_n|^p\\
		\le &\left(\left(\sum_{n=N}^{\infty}|x_n-x_n^{(k_0)}|^p\right)^{1/p}+\left(\sum_{n=N}^{\infty}|x_n^{(k_0)}|^p\right)^{1/p}\right)^p\\
		\le &\left(\left(\sum_{n=1}^{\infty}|x_n-x_n^{(k_0)}|^p\right)^{1/p}+\left(\sum_{n=N}^{\infty}|x_n^{(k_0)}|^p\right)^{1/p}\right)^p\\
		<&\left(\left(\frac{\varepsilon}{2^p}\right)^{1/p}+\left(\frac{\varepsilon}{2^p}\right)^{1/p}\right)^p\\
		=&\varepsilon
	\end{align}
	
	由紧致性证明2,注意到$\displaystyle X\sub\bigcup_{x\in X}B_{\frac{\varepsilon^{1/p}}{2}}(x)$,那么存在$x^{(1)},\cdots,x^{(m)}\in X$,使得成立$\displaystyle X\sub\bigcup_{k=1}^{m}B_{\frac{\varepsilon^{1/p}}{2}}(x^{(k)})$,不妨仍设为$x^{(k)}=\{x_n^{(k)}\}_{n=1}^{\infty}$,那么存在$N\in\N^*$,使得对于任意$1\le k\le m$,成立$\displaystyle\sum_{n=N}^{\infty}|x_n^{(k)}|^p<\frac{\varepsilon^{1/p}}{2}$,于是任取数列$\{x_n\}_{n=1}^{\infty}\in X$,存在$k_0$,使得$\{x_n\}_{n=1}^{\infty}\in B_{\frac{\varepsilon^{1/p}}{2}}(x^{(k_0)})$,进而
	\nonumber\begin{align}
		&\sum_{n=N+1}^{\infty}|x_n|^p\\
		\le &\left(\left(\sum_{n=N}^{\infty}|x_n-x_n^{(k_0)}|^p\right)^{1/p}+\left(\sum_{n=N}^{\infty}|x_n^{(k_0)}|^p\right)^{1/p}\right)^p\\
		<&\left(\frac{\varepsilon^{1/p}}{2}+\frac{\varepsilon^{1/p}}{2}\right)^p\\
		=&\varepsilon
	\end{align}
	
	对于充分性,任取数列序列$\{ \{x_n^{(m)}\}_{n=1}^\infty \}_{m=1}^{\infty}\sub X$,由$2$,存在$M>0$,使得对于任意$m\in\N^*$,成立$\displaystyle\sum_{n=1}^{\infty}|x_n^{(m)}|^p<M$,因此对于任意$n\in\N^*$,数列$\{x_n^{(m)}\}_{m=1}^{\infty}$以$M$为界,于是可依对角线方法找到正整数子列$\{m_k\}_{k=1}^{\infty}\sub\N^*$,使得对于任意$n\in\N^*$,存在$x_n$,使得成立$\displaystyle\lim_{k\to\infty}x_n^{(m_k)}=x_n$。由于对于任意$k\in\N^*$,成立$\displaystyle\sum_{n=1}^{\infty}|x_n^{(m_k)}|^p<M$,那么令$k\to\infty$,可得$\displaystyle\sum_{n=1}^{\infty}|x_n|^p<M$,于是$\{x_n\}_{n=1}^{\infty}\in l^p$。
	
	任取$\varepsilon>0$,由$2$,存在$N\in\N^*$,使得对于数列$\{x_n\}_{n=1}^{\infty}\in l^p$,成立$\displaystyle\sum_{n=N+1}^{\infty}|x_n|^p<\frac{\varepsilon}{2^{p+1}}$,以及对于任意$k\in\N^*$,成立$\displaystyle\sum_{n=N+1}^{\infty}|x_n^{(m_k)}|^p<\frac{\varepsilon}{2^{p+1}}$。因为对于任意$1\le n\le N$,成立$\displaystyle\lim_{k\to\infty}x_n^{(m_k)}=x_n$,所以存在$K\in\N^*$,使得对于任意$k\ge K$,以及任意$1\le n\le N$,成立$|x_n^{(m_k)}-x_n|<(\varepsilon/(2N))^{1/p}$,因此对于任意$k\ge K$,成立
	\nonumber\begin{align}
		& \sum_{n=1}^{\infty}|x_n^{(m_k)}-x_n|^p\\
		= & \sum_{n=1}^{N}|x_n^{(m_k)}-x_n|^p+\sum_{n=N+1}^{\infty}|x_n^{(m_k)}-x_n|^p\\
		\le & \sum_{n=1}^{N}|x_n^{(m_k)}-x_n|^p+\left(\left( \sum_{n=1}^{\infty}|x_n^{(m_k)}|^p \right)^{1/p}+\left( \sum_{n=1}^{\infty}|x_n|^p \right)^{1/p}\right)^p\\
		< & \frac{\varepsilon}{2}+\frac{\varepsilon}{2}\\
		= & \varepsilon
	\end{align}
\end{proof}

\begin{proposition}
	$\{\{x_n\}_{n=1}^{\infty}\}$空间中的子集$S$是列紧的,当且仅当对于任意$n\in\N^*$,存在$M_n$,对于任意$\{x_n\}_{n=1}^{\infty}\in S$,成立$|x_n|\le M_n$,其中
	$$
	d(\{x_n\}_{n=1}^{\infty},\{y_n\}_{n=1}^{\infty})=\sum_{n=1}^{\infty}\frac{1}{2^n}\frac{|x_n-y_n|}{1+|x_n-y_n|}
	$$
\end{proposition}

\begin{proof}
	对于必要性,如果$S$是列紧的,那么$S$是完全有界的,因此对于任意$\varepsilon>0$,存在有限个数列$\{ \{x_n^{(k)}\}_{n=1}^{\infty} \}_{k=1}^{m}$,使得对于任意$\{x_n\}_{n=1}^{\infty}\in S$,存在$1\le k\le m$,使得成立$\displaystyle\sum_{n=1}^{\infty}\frac{1}{2^n}\frac{|x_n-x_n^{(k)}|}{1+|x_n-x_n^{(k)}|}<\varepsilon$。任取$n\in\N^*$,令$\displaystyle M_n=\frac{2^n\varepsilon}{1-2^n\varepsilon}+\max_{1\le k\le m}|x_n^{(k)}|$,且$\varepsilon<2^{-n}$,那么对于任意$\{x_n\}_{n=1}^{\infty}\in S$,成立
	\nonumber\begin{align}
		&\sum_{n=1}^{\infty}\frac{1}{2^n}\frac{|x_n-x_n^{(k)}|}{1+|x_n-x_n^{(k)}|}<\varepsilon\\
		\implies&\frac{|x_n-x_n^{(k)}|}{1+|x_n-x_n^{(k)}|}<\varepsilon\\
		\implies&|x_n-x_n^{(k)}|<\frac{2^n\varepsilon}{1-2^n\varepsilon}\\
		\implies&|x_n|\le|x_n-x_n^{(k)}|+|x_n^{(k)}|<\frac{2^n\varepsilon}{1-2^n\varepsilon}+\max_{1\le k\le m}|x_n^{(k)}|=M_n
	\end{align}
	
	对于充分性,任取序列$\{ \{x_n^{(m)}\}_{n=1}^{\infty} \}_{m=1}^{\infty}$,那么对于任意$n\in\N^*$,存在$M_n$,数列$\{x_n^{(m)}\}_{m=1}^{\infty}$以$M_n$为界,于是可依对角线方法找到正整数子列$\{m_k\}_{k=1}^{\infty}\sub\N^*$,使得对于任意$n\in\N^*$,存在$x_n$,使得成立$\displaystyle\lim_{k\to\infty}x_n^{(m_k)}=x_n$,因此任取$\varepsilon>0$,存在$K\in\N^*$,使得对于任意$k>K$,成立$|x_n^{(m_k)}-x_n|<\varepsilon$,因此
	$$
	d(\{ x_n^{(m_k)} \}_{n=1}^{\infty},\{x_n\}_{n=1}^{\infty})=\sum_{n=1}^{\infty}\frac{1}{2^n}\frac{|x_n^{(m_k)}-x_n|}{1+|x_n^{(m_k)}-x_n|}<\frac{\varepsilon}{1+\varepsilon}\sum_{n=1}^{\infty}\frac{1}{2^n}=\frac{\varepsilon}{1+\varepsilon}<\varepsilon
	$$
	进而$\displaystyle \lim_{k\to\infty}d(\{ x_n^{(m_k)} \}_{n=1}^{\infty},\{x_n\}_{n=1}^{\infty})=0$,即序列$\{ \{x_n^{(m)}\}_{n=1}^{\infty} \}_{m=1}^{\infty}$存在收敛子列$\{ \{x_n^{(m_k)}\}_{n=1}^{\infty} \}_{k=1}^{\infty}$,$S$是列紧的。
	
	综上所述,原命题得证!
\end{proof}

\begin{proposition}
	定义线性空间
	$$
	M[a,b]=\{ f:\exists K_f,\text{ s.t. }\|f\|<M_f \},\qquad 
	\|f\|=\sup_{[a,b]}|f|
	$$
	证明:$M[a,b]$为Banach空间。
\end{proposition}

\begin{proof}
	首先,证明$M[a,b]$为赋范线性空间。
	
	正定性:显然$\|f\|\ge 0$,且
	$$
	\|f\|=0\iff\sup_{[a,b]}|f|=0\iff f=0
	$$
	绝对齐性:
	$$
	\|af\|=\sup_{[a,b]}|af|=|a|\sup_{[a,b]}|f|=|a|\|f\|
	$$
	三角不等式:任取$x\in[a,b]$,注意到
	$$
	|f(x)+g(x)|\le |f(x)|+|g(x)|\le\sup_{[a,b]}|f|+\sup_{[a,b]}|g|
	$$
	由$x$的任意性,$\|f+g\|\le \|f\|+\|g\|$。
	
	综合如上三点,$M[a,b]$为赋范线性空间。
	
	首先,证明$M[a,b]$的完备性。任取Cauchy序列$\{f_n\}\sub M[a,b]$,于是对于任意$\varepsilon>0$,存在$N\in\N^*$,使得对于任意$m,n\ge N$,成立
	$$
	\label{公式11}\|f_m-f_n\|<\varepsilon \implies|f_m(x),f_n(x)|<\varepsilon,\quad \forall x\in[a,b]
	$$
	于是对于任意$x\in[a,b]$,数列$\{f_n(x)\}$为Cauchy序列,记$\displaystyle f(x)=\lim_{n\to\infty}f_n(x)$。由于对于任意$n\in\N^*$,$f_n\in M[a,b]$,于是存在$K_n$,使得成立$\|f_n\|<K_n$。在式$(\ref{公式11})$中令$m\to\infty$,可得
	$$
	\|f-f_n\|<\varepsilon\implies\|f\|\le \|f_n-f\|+\|f_n\|<1+M_n<\infty
	$$
	因此$f\in M[a,b]$,同时可得$\|f_n\|\to\|f\|$,于是$M[a,b]$为完备空间。
	
	综上所述,$M[a,b]$为完备的赋范线性空间,即Banach空间。
\end{proof}

\begin{proposition}
	证明:有界变差函数空间
	$$
	V[a,b]=\{ f:[a,b]\to \C:V_a^b(f)<\infty \}
	$$
	依范数
	$$
	\|f\|=|f(a)|+V_a^b(f)
	$$
	构成Banach空间。
\end{proposition}

\begin{proposition}
	举例说明:在一般的度量空间中,完全有界集不一定是列紧的。
\end{proposition}

\begin{proof}
	在度量空间$\Q$中,取子集$M=[0,1]\cap\Q$。
	
	首先,任取$\varepsilon>0$,存在$n\in\N^*$,使得成立$\frac{1}{n}<\varepsilon$。令$N=\{ \frac{k}{n}:0\le k\le n,k\in\N \}\sub M$,于是对于任意$x\in M$,存在$y\in N$,使得成立$|x-y|<\frac{1}{n}<\varepsilon$,于是$M$是完全有界集。
	
	其次,存在$\{x_n\}_{n=1}^{\infty}\sub M$,使得成立$x_n\to \frac{1}{\sqrt{2}}\notin \Q$,因此$\{x_n\}_{n=1}^{\infty}$在$\Q$中不存在收敛子列。
	
	综上所述,在$\Q$中,完全有界集不一定是列紧的。
\end{proof}

\begin{proposition}
	对于度量空间$(X,d)$中的Cauchy序列$\{x_n\}_{n=1}^{\infty}\sub X$,证明:如果$\{x_n\}_{n=1}^{\infty}$存在依距离$d$收敛至$x$的子序列$\{x_{k_n}\}_{n=1}^{\infty}$,那么$\{x_n\}_{n=1}^{\infty}$收敛至$x$。
\end{proposition}

\begin{proof}
	任取$\varepsilon>0$,由于$\{x_n\}_{n=1}^{\infty}$为Cauchy序列,且$\{x_{k_n}\}_{n=1}^{\infty}$依距离$d$收敛于$x$,那么存在$N>0$,使得对于任意$n>N$,成立
	$$
	d(x_n,x_{k_n})<\varepsilon/2,\qquad d(x_{k_n},x)<\varepsilon/2
	$$
	于是
	$$
	d(x_n,x)<d(x_n,x_{k_n})+d(x_{k_n},x)<<\varepsilon
	$$
	因此$\{x_n\}_{n=1}^{\infty}$依距离$d$收敛于$x$。
\end{proof}

\begin{proposition}
	$f$连续,当且仅当闭集的原像是闭集。
\end{proposition}

\begin{proof}
	显然!
\end{proof}

\begin{proposition}
	对于$\C$上的$n$维线性空间$X$,$\{e_k\}_{k=1}^{\infty}\sub X$为$X$的一组基,那么$\R$上的线性空间$X$的维数是多少?
\end{proposition}

\begin{proof}
	容易注意到$\{e_k\}_{k=1}^{\infty}\cup \{ie_k\}_{k=1}^{\infty}\sub X$为$\R$上的线性空间$X$的一组基。
\end{proof}

\begin{proposition}
	对于赋范线性空间$(X,\|\cdot\|)$,以及序列$\{x_n\}_{n=1}^{\infty}\sub X$,称序列级数$\displaystyle\sum_{n=1}^{\infty}x_n$收敛,如果序列$\displaystyle\left\{\sum_{k=1}^{n}x_k\right\}_{n=1}^{\infty}$收敛;称序列级数$\displaystyle\sum_{n=1}^{\infty}x_n$绝对收敛,如果数列级数$\displaystyle\sum_{n=1}^{\infty}\|x_n\|$收敛。
	
	证明:赋范线性空间$X$中绝对收敛序列级数为收敛序列级数$\iff X$为Banach空间。
\end{proposition}

\begin{proof}
	对于必要性,任取Cauchy序列$\{x_n\}_{n=1}^{\infty}\sub X$,我们来递归的寻找子序列$\{ n_k \}_{k=1}^{\infty}\sub\mathbb{N}^*$,使得对于任意$k\in\mathbb{N}^*$,成立$\| x_{n_{k+1}}-x_{n_k} \|<2^{-k}$。
	
	1. 取$\varepsilon=2^{-1}$,于是存在$N_1\in\mathbb{N}^*$,使得对于任意$m,n\ge N_1$,成立$\|x_m-x_n\|<2^{-1}$。取$n_1=N_1$。
	2. 如果已取$n_1,\cdots,n_k$,那么取$\varepsilon=2^{-(k+1)}$,于是存在$N_{k+1}\in\mathbb{N}^*$,使得对于任意$m,n\ge N_{k+1}$,成立$\|x_m-x_n\|<2^{-(k+1)}$。取$n_{k+1}=\max\{ N_{k},N_{k+1} \}+1$。
	
	递归的,子序列$\{ n_k \}_{k=1}^{\infty}\sub\mathbb{N}^*$,使得对于任意$k\in\mathbb{N}^*$,成立$\| x_{n_{k+1}}-x_{n_k} \|<2^{-k}$,因此$\displaystyle\sum_{k=1}^{\infty}\|x_{n_{k+1}}-x_{n_k}\|<1$,即序列级数$\displaystyle\sum_{k=1}^{\infty}(x_{n_{k+1}}-x_{n_k})$绝对收敛。由必要性假设,序列级数$\displaystyle\sum_{k=1}^{\infty}(x_{n_{k+1}}-x_{n_k})$收敛,即序列$\displaystyle\left\{\sum_{k=1}^{m}(x_{n_{k+1}}-x_{n_k})\right\}_{m=1}^{\infty}$收敛,因此序列$\{x_n\}_{n=1}^{\infty}$的子序列$\{ x_{n_k} \}_{k=1}^{\infty}$收敛。记$x_{n_k}\to x\in X$,那么任取$\varepsilon>0$,存在$K\in\mathbb{N}^*$,使得对于任意$k\ge K$,成立$\| x_{n_k}-x\|<\varepsilon/2$。而序列$\{x_n\}_{n=1}^{\infty}$为Cauchy序列,那么对于此$\varepsilon>0$,存在$N\in\mathbb{N}^*$,使得对于任意$m,n\ge N$,成立$\|x_m-x_n\|<\varepsilon/2$。那么当$n,n_k\ge N$且$k\ge K$,$\| x_n-x \|\le \|x_n-x_{n_k}\|+\|x_{n_k}-x\|<\varepsilon $,因此$x_n\to x\in X$,进而$ X$为完备的赋范线性空间,即Banach空间。
	
	对于充分性,任取绝对收敛序列级数$\displaystyle\sum_{n=1}^{\infty}x_n$,那么数列级数$\displaystyle\sum_{n=1}^{\infty}\|x_n\|$收敛,因此对于任意$\varepsilon>0$,存在$N\in\mathbb{N}^*$,使得对于任意$n\ge N$和$p\in\mathbb{N}^*$,成立$\displaystyle\sum_{k=n+1}^{n+p} \Vert x_k \Vert<\varepsilon$,那么对于此$\varepsilon>0$,成立
	$$
	\left\|\sum_{k=1}^{n+p}x_k-\sum_{k=1}^{n}x_k\right\|
	=\left\|\sum_{k=n+1}^{n+p}x_k\right\|
	\le\sum_{k=n+1}^{n+p} \Vert x_k \Vert
	$$
	因此序列$\displaystyle\left\{\sum_{k=1}^{n}x_k\right\}_{n=1}^{\infty}$为Cauchy序列,由$X$是完备的赋范线性空间,那么序列$\displaystyle\left\{\sum_{k=1}^{n}x_k\right\}_{n=1}^{\infty}$收敛,即序列级数$\displaystyle\sum_{n=1}^{\infty}x_n$收敛。
\end{proof}

\begin{proposition}
	对于线性算子$T:(X,\|\cdot\|_X)\to (Y,\|\cdot\|_Y)$,证明:如果$X$为有限维空间,那么$T$是有界线性算子,且$T(X)$为有限维空间。
\end{proposition}

\begin{proof}
	取$X$的一组基$\{ e_k \}_{k=1}^n$,那么存在$\mu>0$,使得对于任意$\{ \lambda_k \}_{k=1}^n$,成立
	$$
	\sum_{k=1}^{n}|\lambda_n|\le\mu\left\| \sum_{k=1}^n\lambda_k e_k \right\|_X
	$$
	任取$x\in X$,存在$\{ \lambda_k \}_{k=1}^n$,使得成立$\displaystyle x=\sum_{k=1}^{n}\lambda_k e_k$,进而
	\nonumber\begin{align}
		\|T(x)\|_Y
		=&\left\| T\left(\sum_{k=1}^{n}\lambda_k e_k\right) \right\|_Y\\
		=&\left\| \sum_{k=1}^{n}\lambda_kT\left( e_k\right) \right\|_Y\\
		\le&\sum_{k=1}^{n}|\lambda_k|\| T(e_k) \|_Y\\
		\le&\max_{1\le k\le m}\| T(e_k) \|_Y\sum_{k=1}^{n}|\lambda_k|\\
		\le&\mu\max_{1\le k\le m}\| T(e_k) \|_Y\left\| \sum_{k=1}^n\lambda_k e_k \right\|_X\\
		=&\mu\max_{1\le k\le m}\| T(e_k) \|_Y \|x\|_X
	\end{align}
	
	显然$T(X)=\mathrm{span}\{T(e_k)\}_{k=1}^{n}$,因此$\dim T(X)\le n$。
	
\end{proof}

\begin{proposition}
	设$X,Y$是线性赋范空间,$T:X\to Y$为线性算子,证明:如果$T$为单射,那么$\{x_i\}_{i=1}^{n}$在$X$中线性无关,当且仅当$\{Tx_i\}_{i=1}^{n}$在$Y$中线性无关。
\end{proposition}

\begin{proof}
	对于必要性,如果$\{x_i\}_{i=1}^{n}$在$X$中线性无关,那么令
	$$
	\sum_{i=1}^{n}\alpha_i Tx_i=0\implies \sum_{i=1}^{n}T(\alpha_ix_i)=0
	$$
	由于$T$为单射,因此
	$$
	\sum_{i=1}^{n}\alpha_ix_i=0\implies \alpha_1=\cdots=\alpha_n=0
	$$
	于是$\{Tx_i\}_{i=1}^{n}$在$Y$中线性无关。
	
	对于充分性,如果$\{Tx_i\}_{i=1}^{n}$在$Y$中线性无关,那么令
	$$
	\sum_{i=1}^{n}\beta_ix_i=0\implies T\left(\sum_{i=1}^{n}\beta_ix_i\right)=0\implies\sum_{i=1}^{n}\beta_i Tx_i=0 \implies \beta_1=\cdots=\beta_n=0
	$$
	综上所述,原命题得证!
\end{proof}

\begin{proposition}
	对于有界线性算子$T:(X,\|\cdot\|_X)\to (Y,\|\cdot\|_Y)$,证明:
	$$
	\| T \|=\sup_{|x|_X=1}\|T(x)\|_Y=\sup_{|x|_X\le 1}\|T(x)\|_Y
	$$
\end{proposition}

\begin{proof}
	显然!
\end{proof}

\begin{proposition}
	对于Banach空间$X$上的有界线性算子$T:(X,\|\cdot\|)\to (X,\|\cdot\|)$,证明:如果存在有界线性算子$S:(X,\|\cdot\|)\to (X,\|\cdot\|)$,使得成立$TS=ST=I$,那么$T$是有界可逆的,且$T^{-1}=S$;反之,如果$T$是有界可逆的,那么$T^{-1}T=TT^{-1}=I$。
\end{proposition}

\begin{proof}
	如果存在有界线性算子$S:(X,\|\cdot\|)\to (X,\|\cdot\|)$,使得成立$TS=ST=I$,那么由$ST=I$可得$T$为单射,由$TS=I$可得$T$为满射,因此$T$为双射。又因为$T^{-1}=S$,那么$T$为有界可逆线性算子。
	
	反之显然!
\end{proof}

\begin{proposition}
	对于度量空间$(X,d)$,证明:如果$T:X\to X$为压缩映射,那么对于任意$n\in\N^*$,$T^n$为压缩映射;反之不然。
	
\end{proposition}

\begin{proof}
	如果$T:X\to X$为以$0<q<1$为Lipschitz常数的压缩映射,那么对于任意$n\in\N^*$,成立
	$$
	d(T^n(x),T^n(y))\le q^nd(x,y)
	$$
	反之,取$T:[0,1/2]\to\R$为$T(x)=4x^2/3$,那么
	$$
	|T^2(x)-T^2(y)|=\frac{16}{9}|x^4-y^4|=\frac{16}{9}(x^3+x^2y+xy^2+y^3)|x-y|\le \frac{8}{9}|x-y|
	$$
	因此$T^2$为压缩映射,但是对于任意$0<q<1$,取$x=7/16,y=5/16$,那么
	$$
	|T(x)-T(y)|=\frac{4}{3}|x^2-y^2|=|x-y|>q|x-y|
	$$
	因此$T$不为压缩映射。
\end{proof}

\begin{proposition}
	对于$\R^n$中的有界闭子集$K\sub \R^n$,证明:如果映射$T:K\to K$满足对于任意$x,y\in K$,成立$d(T(x),T(y))<d(x,y)$,那么$T$在$K$中存在且存在唯一不动点。
\end{proposition}

\begin{proof}
	我们来证明$T$为压缩映射,反证,如果对于任意$0<q<1$,使得存在$x,y\in K$,使得成立$d(T(x),T(y))>qd(x,y)$,那么对于任意$n\in\N^*$,存在$x_n,y_n\in K$,使得成立$d(T(x_n),T(y_n))>(1-1/n)d(x_n,y_n)$。由于$K$为有界闭集,那么存在$x,y\in K$,不妨设$x_n\to x,y_n\to y$,那么$d(T(x),T(y))\ge d(x,y)$,矛盾!
\end{proof}

\begin{proposition}
	证明:当$|\lambda|$充分小时,积分方程
	$$
	f(x)=\varphi(x)+\lambda \int_0^1K(x,y)f(y)\mathrm{d}y
	$$
	在$L^2[0,1]$中存在且存在唯一解,其中$\varphi\in L^2[0,1]$,$K(x,y)$为$[0,1]^2$上的可测函数,且
	$$
	\int_0^1\int_0^1|K(x,y)|^2\mathrm{d}x\mathrm{d}y<\infty
	$$
\end{proposition}

\chapter{Hilbert空间}

\begin{proposition}
	对于内积空间$X$,以及非零元$x,y\in X$,证明如下命题。
	\begin{enumerate}
		\item 如果$x$与$y$正交,那么$x$与$y$线性无关。
		\item 
		$$
		(x,y)=0\iff \|x+\lambda y \|=\|x-\lambda y\|,\forall\lambda\in\C
		$$
		\item 
		$$
		(x,y)=0\iff \|x+\lambda y \|\ge \|x\|,\forall\lambda\in\C
		$$
	\end{enumerate}
\end{proposition}

\begin{proof}
	对于1,任取$\lambda$和$\mu$,使得成立$\lambda x+\mu y=0$,因此
	$$
	(\lambda x+\mu y,x)=(0,x)\implies \lambda\|x\|^2+\mu(y,x)=0\implies \lambda=0\\
	(\lambda x+\mu y,y)=(0,y)\implies \lambda(x,y)+\mu\|y\|^2=0\implies \mu=0
	$$
	那么$x$与$y$线性无关。
	
	对于2,注意到
	\begin{align*}
		&\|x+\lambda y \|=\|x-\lambda y\|,\forall\lambda\in\C\\
		\iff & (x+\lambda y,x+\lambda y)=(x-\lambda y,x-\lambda y),\forall\lambda\in\C\\
		\iff & \|x\|^2+\overline{\lambda}(x,y)+\lambda\overline{(x,y)}+|\lambda|^2\|y\|^2=
		\|x\|^2-\overline{\lambda}(x,y)-\lambda\overline{(x,y)}+|\lambda|^2\|y\|^2,\forall\lambda\in\C\\
		\iff & \overline{\lambda}(x,y)+\lambda\overline{(x,y)}=0,\forall\lambda\in\C\\
		\implies & (x,y)\overline{(x,y)}=0\\
		\iff & |(x,y)|^2=0\\
		\iff & (x,y)=0
	\end{align*}
	而显然成立$(x,y)=0\implies\overline{\lambda}(x,y)+\lambda\overline{(x,y)}=0,\forall\lambda\in\C$。
	
	对于3,注意到
	\begin{align*}
		&\|x+\lambda y \|\ge \|x\|,\forall\lambda\in\mathbb{C}\\
		\iff & (x+\lambda y,x+\lambda y) \ge \|x\|^2,\forall\lambda\in\mathbb{C}\\
		\iff & \|x\|^2+\overline{\lambda}(x,y)+\lambda\overline{(x,y)}+|\lambda|^2\|y\|^2\ge \|x\|^2,\forall\lambda\in\mathbb{C}\\
		\iff & \overline{\lambda}(x,y)+\lambda\overline{(x,y)}+|\lambda|^2\|y\|^2\ge 0,\forall\lambda\in\mathbb{C}
	\end{align*}
	
	对于必要性,显然成立
	$$
	(x,y)=0\implies\overline{\lambda}(x,y)+\lambda\overline{(x,y)}+|\lambda|^2\|y\|^2\ge 0,\forall\lambda\in\mathbb{C}
	$$
	
	对于充分性,如果对于任意$\lambda\in\mathbb{C}$,成立
	$$
	\overline{\lambda}(x,y)+\lambda\overline{(x,y)}+|\lambda|^2\|y\|^2\ge 0
	$$
	那么若$y=0$,显然成立$(x,y)=0$;若$y\ne0$,取$\lambda=-(x,y)/\|y\|^2$,于是
	$$
	\overline{\lambda}(x,y)+\lambda\overline{(x,y)}+|\lambda|^2\|y\|^2
	=-\frac{|(x,y)|^2}{\|y\|^2}\ge 0\implies (x,y)=0
	$$
\end{proof}

\begin{proposition}{}{2.2}
	如果$\{e_n\}_{n=1}^{\infty}$为内积空间$X$中的正规正交集,那么对于任意$x,y\in X$,成立
	$$
	\sum_{n=1}^{\infty}|(x,e_n)(y,e_n)|\le \|x\| \|y\|
	$$
\end{proposition}

\begin{proof}
	由Bessel不等式,对于任意$n\in\N^*$,成立
	$$
	\sum_{k=1}^{n}|(x,e_k)|^2\le \|x\|^2,\qquad 
	\sum_{k=1}^{n}|(y,e_k)|^2\le \|y\|^2
	$$
	令$n\to\infty$,那么
	$$
	\sum_{n=1}^{\infty}|(x,e_n)|^2\le \|x\|^2,\qquad 
	\sum_{n=1}^{\infty}|(y,e_n)|^2\le \|y\|^2
	$$
	由Hölder不等式
	$$
	\sum_{n=1}^{\infty}|(x,e_n)(y,e_n)|\le
	\left(\sum_{n=1}^{\infty}|(x,e_n)|^2\right)^{1/2}\left(\sum_{n=1}^{\infty}|(y,e_n)|^2\right)^{1/2}\le
	\|x\|\|y\|
	$$
\end{proof}

\begin{proposition}{}{2.3}
	证明:对于Hilbert空间中的正规正交集$\{e_n\}_{n=1}^{\infty}$,如果
	$$
	x=\sum_{n=1}^{\infty}\alpha_n e_n,\qquad
	y=\sum_{n=1}^{\infty}\beta_n e_n
	$$
	那么
	$$
	(x,y)=\sum_{n=1}^{\infty}\alpha_n \overline{\beta_n}
	$$
	且级数绝对收敛。
\end{proposition}

\begin{proof}
	$$
	(x,y)
	= \left(\sum_{n=1}^{\infty}\alpha_n e_n,\sum_{n=1}^{\infty}\beta_n e_n\right)
	= \sum_{i,j=1}^{n}(\alpha_i e_i,\beta_j e_j)
	= \sum_{i,j=1}^{n}\alpha_i\overline{\beta_j}(e_i,e_j)
	= \sum_{n=1}^{\infty}\alpha_n \overline{\beta_n}
	$$
	
	由于
	\begin{align*}
		& (x,e_n)
		= \left(\sum_{k=1}^{\infty}\alpha_k e_k,e_n\right)
		= \sum_{k=1}^{\infty}\alpha_k (e_k,e_n)
		= \alpha_n\\
		& (y,e_n)
		= \left(\sum_{k=1}^{\infty}\beta_k e_k,e_n\right)
		= \sum_{k=1}^{\infty}\beta_k (e_k,e_n)
		= \beta_n
	\end{align*}
	那么由命题\ref{pro:2.2}
	$$
	\sum_{n=1}^{\infty}|\alpha_n\overline{\beta_n}|=
	\sum_{n=1}^{\infty}|(x,e_n)(y,e_n)|\le \|x\| \|y\|
	$$
	因此级数绝对收敛。
\end{proof}

\begin{proposition}
	对于Hilbert空间$\mathcal{H}$中的正规正交基$\{e_n\}_{n=1}^{\infty}$,证明:对于任意$x,y\in \mathcal{H}$,成立
	$$
	(x,y)=\sum_{n=1}^{\infty}(x,e_n)\overline{(y,e_n)}\\
	$$
	且级数绝对收敛。
\end{proposition}

\begin{proof}
	由命题\ref{pro:2.3},该命题显然!
\end{proof}

\begin{proposition}
	对于区域$D\sub \R^n$,定义$\R$上的线性空间
	$$
	L^2(D)=\left\{ f:D\to\C\mid \int_D|f|^2<\infty \right\},\qquad (f,g)=\int_D f\overline{g}
	$$
	证明:$L^2(D)$为Hilbert空间。
\end{proposition}

\begin{proof}
	首先证明$L^2(D)$为内积空间。
	
	正定性:任取$f\in L^2(D)$,显然成立$(f,f)\ge 0$,且
	\nonumber\begin{align}
		&(f,f)=0\\
		\iff &\int_{D}|f|^2=0\\
		\iff &f=0
	\end{align}
	共轭对称性:任取$f,g\in L^2(D)$,那么
	$$
	(f,g)=\iint\limits_{D}f\overline{g}\mathrm{d}x\mathrm{d}y=\overline{\iint\limits_{D}g\overline{f}\mathrm{d}x\mathrm{d}y}=\overline{(g,f)}
	$$
	左线性:任取$\lambda,\mu\in \C$以及$f,g,h\in L^2(D)$,那么显然成立$(\lambda f+\mu g,h)=\lambda(f,h)+\mu(g,h)$。
	
	综合以上三点,$L^2(D)$为内积空间,下面证明$L^2(D)$的完备性。
	
	任取Cauchy序列$\{f_n\}_{n=1}^\infty\sub L^2(D)$,递归寻找子序列$\{ n_k \}_{k=1}^{\infty}\sub\mathbb{N}^*$,使得对于任意$k\in\mathbb{N}^*$,成立$\| f_{n_{k+1}}-f_{n_k} \|_2<2^{-k}$。
	
	1. 取$\varepsilon=2^{-1}$,存在$N_1\in\mathbb{N}^*$,使得对于任意$m,n\ge N_1$,成立$\|f_m-f_n\|_2<2^{-1}$。取$n_1=N_1$。
	2. 如果已取$n_1,\cdots,n_k$,那么取$\varepsilon=2^{-(k+1)}$,于是存在$N_{k+1}\in\mathbb{N}^*$,使得对于任意$m,n\ge N_{k+1}$,成立$\|f_m-f_n\|_2<2^{-(k+1)}$。取$n_{k+1}=\max\{ N_{k},N_{k+1} \}+1$。
	
	递归的,可得子序列$\{ n_k \}_{k=1}^{\infty}\sub\mathbb{N}^*$满足对于任意$k\in\mathbb{N}^*$,成立$\| f_{n_{k+1}}-f_{n_k} \|_2<2^{-k}$。考虑级数
	$$
	f=f_{n_1}+\sum_{k=1}^{\infty}(f_{n_{k+1}}-f_{n_k}),\qquad
	S_m(f)=f_{n_1}+\sum_{k=1}^{m}(f_{n_{k+1}}-f_{n_k})
	$$
	$$
	g=|f_{n_1}|+\sum_{k=1}^{\infty}|f_{n_{k+1}}-f_{n_k}|,\qquad
	S_m(g)=|f_{n_1}|+\sum_{k=1}^{m}|f_{n_{k+1}}-f_{n_k}|
	$$
	
	对于任意$m\in\N^*$,由Minkowski不等式
	$$
	\|S_m(g)\|_2
	\le \|f_{n_1}\|_2+\sum_{k=1}^{m}\| f_{n_{k+1}}-f_{n_k}\|_2
	< \|f_{n_1}\|_2+\sum_{k=1}^{m}2^{-k}
	<1+\|f_{n_1}\|_2
	$$
	由Levi单调收敛定理
	$$
	\|g\|_2
	=\left(\int_X |g|^2\right)^{1/2}
	=\left(\int_X \lim_{m\to\infty} |S_m(g)|^2\right)^{1/2}
	=\lim_{m\to\infty}\left(\int_X |S_m(g)|^2\right)^{1/2}
	=\lim_{m\to\infty}\|S_m(g)\|_2
	\le 1+\|f_{n_1}\|_2
	$$
	因此级数$g$几乎处处收敛,于是级数$f$几乎处处绝对收敛,那么存在零测集$N$,使得级数$f$在$D\setminus N$上绝对收敛。不妨当$x\in N$时,令$f(x)=0$,那么$f$为可测函数。
	
	注意到
	$$
	\|f\|_2
	=\left( \int_{D}|f|^2 \right)^{1/2}
	=\left( \int_{D\setminus N}|f|^2 \right)^{1/2}
	\le \left( \int_{D\setminus N}|g|^2 \right)^{1/2}
	=\left( \int_{D}|g|^2 \right)^{1/2}
	=\|g\|_2<\infty
	$$
	因此$f\in L^p$,同时注意到
	$$
	\| f-f_{n_k} \|_2
	=\left\| \sum_{i=k+1}^{\infty}(f_{n_{i+1}}-f_{n_i}) \right\|_2
	\le \sum_{i=k+1}^{\infty}\| f_{n_{i+1}}-f_{n_i} \|_2
	< \sum_{i=k+1}^{\infty}2^{-i}=\frac{1}{2^k}\to0
	$$
	因此子序列$\{f_{n_k}\}_{k=1}^{\infty}$在$L^2(D)$空间中收敛于$f$。任取$\varepsilon>0$,存在$K\in\N^*$,使得当$n_k\ge k\ge K$时,成立$\|f-f_{n_k}\|_2<\varepsilon/2$且$\|f_k-f_{n_k}\|_2<\varepsilon/2$,于是
	$$
	\|f-f_k\|_2\le\|f-f_{n_k}\|_2+\|f_k-f_{n_k}\|_2<\varepsilon
	$$
	进而序列$\{f_n\}_{n=1}^{\infty}$在$L^2(D)$空间中收敛于$f$。
	
	综上所述,$L^2(D)$为Hilbert空间。
\end{proof}

\begin{proposition}
	$$
	l^p\text{空间为内积空间}\iff p=2
	$$
\end{proposition}

\begin{proof}
	如果$p=2$,那么
	$$
	\|x+y\|_2^2+\|x-y\|_2^2=2(\|x\|_2^2+\|y\|_2^2),\qquad \forall x,y\in l^2
	$$
	因此$L^p$空间为内积空间。
	
	如果$p\ne 2$,那么取$x=(1,1,0,\cdots)$,$y=(1,-1,0,\cdots)$,因此
	$$
	\|x+y\|_p^2=\|x-y\|_p^2=4,\qquad 
	\|x\|_p^2=\|y\|_p^2=2^{2/p}
	$$
	此时
	$$
	\|x+y\|_p^2+\|x-y\|_p^2=2^3,\qquad 
	2(\|x\|_p^2+\|y\|_p^2)=2^{2+2/p}
	$$
	那么
	$$
	\|x+y\|_p^2+\|x-y\|_p^2\ne 
	2(\|x\|_p^2+\|y\|_p^2)
	$$
	进而$l^p$空间不为内积空间。
\end{proof}

\begin{proposition}
	证明:Schmidt正规正交法。
\end{proposition}

\begin{proof}
	易证!
\end{proof}

\begin{proposition}
	证明射影定理:如果$\mathcal{M}$为Hilbert空间$\mathcal{H}$的闭子空间,那么对于任意$x\in \mathcal{H}$,存在且存在唯一$(y,z)\in\mathcal{M}\times\mathcal{M}^\perp$,使得成立$x=y+z$。
\end{proposition}

\begin{proof}
	见课本!
\end{proof}

\begin{proposition}
	\begin{enumerate}
		\item 证明:如果$\mathcal{M}$为HIlbert空间$\mathcal{H}$的子空间,那么$\mathcal{M}^\perp$为$\mathcal{H}$的子空间。
		\item 证明:如果$\mathcal{M}$为HIlbert空间$\mathcal{H}$的子空间,那么$(\overline{\mathcal{M}})^\perp=\mathcal{M}^\perp$。
		\item 证明:如果$\mathcal{M}, \mathcal{N} $为HIlbert空间$\mathcal{H}$的子空间,且$\mathcal{M}\sub \mathcal{N} $,那么$ \mathcal{N} ^\perp\sub \mathcal{M}^\perp$。
	\end{enumerate}
\end{proposition}

\begin{proof}
	对于1,任取$x,y\in\mathcal{M}^\perp$,以及$\lambda\in\C$,那么对于任意$m\in \mathcal{M}$,成立
	$$
	(x,m)=(y,m)=0
	$$
	因此
	$$
	(x+y,m)=(x,m)+(y,m)=0,\qquad
	(\lambda x,m)=\lambda(x,m)=0
	$$
	于是$x+y\in \mathcal{M}^\perp$,且$\lambda x\in \mathcal{M}^\perp$,进而$\mathcal{M}^\perp$为$\mathcal{H}$的子空间。
	
	对于2,一方面,注意到$\mathcal{M}\sub\overline{\mathcal{M}}$,那么$(\overline{\mathcal{M}})^\perp\sub\mathcal{M}^\perp$。
	
	另一方面,任取$x\in \mathcal{M}^\perp$,以及$y\in \overline{\mathcal{M}}$,那么存在$\{y_n\}_{n=1}^{\infty}\sub\mathcal{M}$,使得$y_n\to y$。而$(x,y_n)=0$,因此$(x,y)=0$,那么$x\in (\overline{\mathcal{M}})^\perp$,进而$(\overline{\mathcal{M}})^\perp\supset\mathcal{M}^\perp$。
	
	综合两方面,$(\overline{\mathcal{M}})^\perp=\mathcal{M}^\perp$。
	
	对于3,显然!
\end{proof}

\begin{proposition}
	证明:Hilbert空间$\mathcal{H}$的对偶空间$\mathcal{H}^*$为Banach空间。
\end{proposition}

\begin{proof}
	首先证明$\mathcal{H}^*$为线性空间。
	
	任取$f,g\in\mathcal{H}^*$,任取$\lambda,\mu \in\C$,任取$x,y\in\mathcal{H}$,注意到
	\nonumber\begin{align}
		&(f+g)(x+y)=f(x+y)+g(x+y)=f(x)+f(y)+g(x)+g(y)=(f+g)(x)+(f+g)(y)\\
		&(f+g)(\lambda x)=f(\lambda x)+g(\lambda x)=\lambda f(x)+\lambda g(x)=\lambda (f+g)(x)\\
		&(\lambda f)(x+y)=\lambda f(x+y)=\lambda f(x)+\lambda f(y)=(\lambda f)(x)+(\lambda f)(y)\\
		&(\lambda f)(\mu x)=\lambda f(\mu x)=\lambda \mu f(x)=\mu(\lambda f)(x)\\
		&\sup_{\|x\|\le1}|(f+g)(x)|=\sup_{\|x\|\le1}|f(x)+g(x)|\le \sup_{\|x\|\le1}(|f(x)|+|g(x)|)\le \sup_{\|x\|\le1}|f(x)|+\sup_{\|x\|\le1}|g(x)|=\|f\|+\|g\|\\
		&\sup_{\|x\|\le1}|(\lambda f)(x)|=\sup_{\|x\|\le1}|\lambda f(x)|=\sup_{\|x\|\le1}\lambda |f(x)|=\lambda \sup_{\|x\|\le1}|f(x)|=\lambda \|f\|
	\end{align}
	那么$f+g$与$\lambda f$为有界线性泛函,等价于$f+g$与$\lambda f$为连续线性泛函,因此$f+g,\lambda f\in \mathcal{H}^*$,进而$\mathcal{H}^*$为线性空间。
	
	其次证明$\|\cdot\|$为范数。
	
	对于正定性,显然$\|f\|\ge 0$,且$\|f\|=0\iff \sup\limits_{\|x\|\le1}|f(x)|=0\iff f(x)=0,\forall \|x\|\le1\iff f=0$。事实上,对于任意$x\in\mathcal{H}\setminus\{0\}$,成立$f(x)=\|x\|f(x/\|x\|)$。
	
	对于绝对齐性,注意到$\|\lambda f\|=\sup\limits_{\|x\|\le1}|(\lambda f)(x)|=\sup\limits_{\|x\|\le1}|\lambda f(x)|=|\lambda| \sup\limits_{\|x\|\le1}|f(x)|=|\lambda|\|f\|$。
	
	对于三角不等式,任取$x\in\mathcal{H}$满足$\|x\|\le 1$,注意到$\|f\|+\|g\|=\sup\limits_{\|x\|\le1}|f(x)|+\sup\limits_{\|x\|\le1}|g(x)|\ge |f(x)|+|g(x)|\ge |f(x)+g(x)|$,由$x$的任意性,$\|f\|+\|g\|\ge \sup\limits_{\|x\|\le1}|(f+g)(x)|=\|f+g\|$。
	
	综合这三点,$\|\cdot\|$为范数,进而$\mathcal{H}^*$为赋范线性空间。
	
	最后证明$\mathcal{H}^*$为完备空间。
	
	任取Cauchy序列$\{f_n\}_{n=1}^{\infty}$,那么对于任意$\varepsilon>0$,存在$N\in\N^*$,使得对于任意$m,n\ge N$,成立$\|f_m-f_n\|<\varepsilon$,因此$\sup\limits_{\|x\|\le1}|f_m(x)-f_n(x)|<\varepsilon$,进而当$\|x\|\le 1$时,成立$|f_m(x)-f_n(x)|<\varepsilon$,这表明$\{f_n(x)\}_{n=1}^{\infty}\sub\C$为Cauchy序列。当$\|x\|>1$时,任取$\varepsilon>0$,由于$\{f_n(x/\|x\|)\}_{n=1}^{\infty}\sub\C$为Cauchy序列,那么存在$M\in\N^*$,使得对于任意$m,n\ge N$,成立$|f_m(x/\|x\|)-f_n(x/\|x\|)|<\varepsilon/\|x\|$,因此$|f_m(x)-f_n(x)|=\|x\||f_m(x/\|x\|)-f_n(x/\|x\|)|<\varepsilon$,这表明$\{f_n(x)\}_{n=1}^{\infty}\sub\C$为Cauchy序列。因此对于任意$x\in\mathcal{H}$,序列$\{f_n(x)\}_{n=1}^{\infty}$为Cauchy序列,进而定义$f(x)=\lim\limits_{n\to\infty}f_n(x)$。
	
	第一证明$f\in \mathcal{H}^*$,由于$\{f_n\}_{n=1}^{\infty}$为Cauchy序列,那么对于任意$\varepsilon>0$,存在$N\in\N^*$,使得对于任意$m,n\ge N$,成立$\|f_m-f_n\|<\varepsilon$,因此$| \|f_m\|-\|f_n\| |\le \|f_m-f_n\|\le\varepsilon$,因此$\{\|f_n\|\}_{n=1}^{\infty}\sub \C$为Cauchy序列,因此存在$z\in\C$,使得成立$\lim\limits_{n\to\infty}\|f_n\|=z$。任取$x,y\in\mathcal{H}$,任取$\lambda\in\C$,注意到
	\nonumber\begin{align}
		&f(x+y)=\lim_{n\to\infty}f_n(x+y)=\lim_{n\to\infty}f_n(x)+f_n(y)=\lim_{n\to\infty}f_n(x)+\lim_{n\to\infty}f_n(y)=f(x)+f(y)\\
		&f(\lambda x)=\lim_{n\to\infty}f_n(\lambda x)=\lim_{n\to\infty}\lambda f_n(x)=\lambda \lim_{n\to\infty}f_n(x)=\lambda f(x)\\
		&\sup_{\|x\|\le 1}|f(x)|
		=\sup_{\|x\|\le 1}|\lim_{n\to\infty}f_n(x)|
		=\sup_{\|x\|\le 1}\lim_{n\to\infty}|f_n(x)|
		\le \lim_{n\to\infty}\sup_{\|x\|\le 1}|f_n(x)|
		=\lim_{n\to\infty}\|f_n\|=z
	\end{align}
	因此$f$为有界线性算子,等价于$f$为连续线性泛函,因此$f\in \mathcal{H}^*$。
	
	第二证明$\lim\limits_{n\to\infty}\|f-f_n\|=0$。注意到对于任意$\|x\|\le 1$,
	\nonumber\begin{align}
		&\lim_{n\to\infty}\|f-f_n\|\\
		=&\lim_{n\to\infty}\sup_{\|x\|\le 1}|f(x)-f_n(x)|\\
		=&\lim_{n\to\infty}\sup_{\|x\|\le 1}|\lim_{m\to\infty}f_m(x)-f_n(x)|\\
		=&\lim_{n\to\infty}\sup_{\|x\|\le 1}\lim_{m\to\infty}|f_m(x)-f_n(x)|\\
		\le & \lim_{m,n\to\infty}\sup_{\|x\|\le 1}|f_m(x)-f_n(x)|\\
		=&\lim_{m,n\to\infty}\|f_m-f_n\|\\
		=&0
	\end{align}
	综合这两点,$f_n\to f$,进而$\mathcal{H}^*$为完备空间。
	
	综上所述,$(\mathcal{H}^*,\|\cdot\|)$为完备赋范线性空间,即Banach空间。
\end{proof}

\begin{proposition}{}{2.11}
	对于Hilbert空间$\mathcal{H}$,成立
	$$
	\|x\|=\sup_{\|y\|\le 1}|(x,y)|
	$$
\end{proposition}

\begin{proof}
	记$f:\mathcal{H}\to\C,\quad y\mapsto(y,x)$,注意到$f\in\mathcal{H}$,那么由Frechet-Riesz表现定理
	$$
	\sup_{\|y\|\le 1}|(x,y)|=\sup_{\|y\|\le 1}|f(y)|=\|f\|=\|x\|
	$$
\end{proof}

\begin{proposition}
	对于Hilbert空间$\mathcal{H}$上的有界共轭双线性泛函$f:\mathcal{H}\times \mathcal{H}\to\C$,存在且存在唯一有界线性算子$T:\mathcal{H}\to\mathcal{H}$,使得对于任意$x,y\in \mathcal{H}$,成立$f(x,y)=(T(x),y)$。
\end{proposition}

\begin{proof}
	对于任意$x\in \mathcal{H}$,记线性泛函
	\nonumber\begin{align}
		g_x:\begin{aligned}[t]
			\mathcal{H}&\longrightarrow\C\\
			y&\longmapsto \overline{f(x,y)}
		\end{aligned}
	\end{align}
	那么$g_x$由$x\in\mathcal{H}$唯一确定。
	
	由于$f$有界,那么存在$C>0$,使得对于任意$x,y\in\mathcal{H}$,成立$|f(x,y)|\le C\|x\|\|y\|$,进而
	$$
	\|g_x\|=\sup_{\|y\|\le 1}|g_x(x)|=\sup_{\|y\|\le 1}|f(x,y)|\le C\|x\|
	$$
	那么$g_x\in\mathcal{H}^*$。由Frechet-Riesz表现定理,存在且存在唯一$z_{g_x}\in\mathcal{H}$,使得对于任意$y\in\mathcal{H}$,成立$g_x(y)=(y,z_{g_x})$,且$\|g_x\|=\|z_{g_x}\|$。
	
	注意到$z_{g_x}$由$x$唯一确定,那么记
	\nonumber\begin{align}
		T:\begin{aligned}[t]
			\mathcal{H}&\longrightarrow\mathcal{H}\\
			x&\longmapsto z_{g_x}
		\end{aligned}
	\end{align}
	于是
	\nonumber\begin{align}
		& \|T\|=\sup_{\|x\|\le 1}\|z_{g_x}\|=\sup_{\|x\|\le 1}\|g_x\|\le C\\
		&f(x,y)=\overline{g_x(y)}=\overline{(y,z_{g_x})}=(z_{g_x},y)=(T(x),y)
	\end{align}
\end{proof}

\begin{proposition}
	\begin{enumerate}
		\item 证明:对于Hilbert空间$\mathcal{H}$中的正规正交集$\{ e_n \}_{n=1}^{\infty}$,如果对于任意$x\in\mathcal{H}$,成立$\displaystyle \|x\|^2=\sum_{n=1}^{\infty}|(x,a_n)|^2$,那么$\{ e_n \}_{n=1}^{\infty}$为$\mathcal{H}$的正规正交基。
		\item 证明:对于Hilbert空间$\mathcal{H}$中的正规正交集$\{ a_n \}_{n=1}^{\infty}$与$\{ b_n \}_{n=1}^{\infty}$,如果$\displaystyle \sum_{n=1}^{\infty}\|a_n-b_n\|^2<1$,且$\{ a_n \}_{n=1}^{\infty}$为$\mathcal{H}$的正规正交基,那么$\{ b_n \}_{n=1}^{\infty}$为$\mathcal{H}$的正规正交基。
	\end{enumerate}
\end{proposition}

\begin{proof}
	对于1,将正规正交集$\{e_n\}_{n=1}^{\infty}$扩充为$\mathcal{H}$的正规正交基$ N $,那么对于任意$x\in\mathcal{H}$,成立
	$$
	\|x\|^2=\sum_{e\in N }|(x,e)|^2=\sum_{n=1}^{\infty}|(x,e_n)|^2
	\implies (x,e)=0,\forall e\in N \setminus\{e_n\}_{n=1}^{\infty}
	$$
	因此
	$$
	x=\sum_{e\in N }(x,e)e=\sum_{n=1}^{\infty}(x,e_n)e
	$$
	由$x$的任意性,$\{ e_n \}_{n=1}^{\infty}$为$\mathcal{H}$的正规正交基。
	
	对于2,$\{ b_n \}_{n=1}^{\infty}$不为$\mathcal{H}$的正规正交基,那么存在$x\in\mathcal{H}$,使得成立$x\perp\mathrm{Sp}\{ b_n \}_{n=1}^{\infty}$,进而由Scharz不等式
	$$
	\|x\|^2=\sum_{n=1}^{\infty}|(x,a_n)|^2=\sum_{n=1}^{\infty}|(x,a_n-b_n)|^2
	\le\sum_{n=1}^{\infty}\|x\|^2\|a_n-b_n\|^2<\|x\|^2
	$$
	矛盾!
\end{proof}

\begin{proposition}
	证明:对于Hilbert空间$\mathcal{H}_1$与$\mathcal{H}_2$,有界线性算子$T:\mathcal{H}_1\to \mathcal{H}_2$的Hilbert共轭算子$T^*:\mathcal{H}_2\to \mathcal{H}_1$为有界线性算子。
\end{proposition}

\begin{proof}
	由\ref{pro:2.11},成立
	\nonumber\begin{align}
		\|T^*\| = & \sup_{\|y\|\le 1}\|T^*(y)\|\\
		= & \sup_{\|y\|\le 1}\sup_{\|x\|\le 1}|(x,T^*(y))|\\
		= & \sup_{\|y\|\le 1}\sup_{\|x\|\le 1}|(T(x),y)|\\
		= & \sup_{\|x\|\le 1}\sup_{\|y\|\le 1}|(T(x),y)|\\
		= & \sup_{\|x\|\le 1}\|T(x)\|\\
		= & \|T\|
	\end{align}
\end{proof}

\begin{proposition}
	证明:对于Hilbert空间$\mathcal{H}$,如果有界线性算子$T:\mathcal{H}\to \mathcal{H}$成立对于任意$x\in\mathcal{H}$,成立$\text{Re}(T(x),x)=0$,那么$T+T^*=0$。
\end{proposition}

\begin{proof}
	由于$\text{Re}(T(x),x)=0$,那么
	$$
	(T(x),x)=(x,T^*(x))=\overline{(T^*(x),x)}=-(T^*(x),x)
	$$
	因此对于任意$x\in\mathcal{H}$,成立$((T+T^*)(x),x)=0$。
	
	令$S=T+T^*$,那么$S^*=S$,注意到
	\nonumber\begin{align}
		&(S(x),y)+(S(y),x)=(S(x),x)+(S(y),y)+(S(x),y)+(S(y),x)=(S(x+y),x+y)=0\\
		&i(S(y),x)-i(S(x),y)=(S(x),x)+(S(y),y)-i(S(x),y)+-(S(y),x)=(S(x+iy),x+iy)=0
	\end{align}
	因此
	$$
	(S(x),y)=(S(y),x)=0,\qquad \forall x,y\in\mathcal{H}
	$$
	进而
	$$
	(S(x),S(x))=0\iff S(x)=0 \iff S=0\iff T+T^*=0
	$$
\end{proof}

\begin{proposition}
	对于有界线性算子
	\nonumber\begin{align}
		T:\begin{aligned}[t]
			l^2 &\longrightarrow l^2\\
			(x_n)_{n=1}^{\infty} &\longmapsto \left(\sum_{m=1}^{\infty}a_{n,m}x_m\right)_{n=1}^{\infty}
		\end{aligned}
	\end{align}
	其Hilbert共轭算子为
	\nonumber\begin{align}
		T^*:\begin{aligned}[t]
			l^2 &\longrightarrow l^2\\
			(x_n)_{n=1}^{\infty} &\longmapsto \left(\sum_{m=1}^{\infty}a^*_{n,m}x_m\right)_{n=1}^{\infty}
		\end{aligned}
	\end{align}
	证明:
	$$
	a_{n,m}^*=\overline{a_{m,n}},\qquad \forall m,n\in\N^*
	$$
\end{proposition}

\begin{proof}
	取$l^2$的正规正交基$e_n=(\delta_{n,m})_{m=1}^{\infty}$,那么对于任意$( x_n )_{n=1}^{\infty}\in l^2$,可唯一表示为
	$$
	( x_n )_{n=1}^{\infty}=\sum_{n=1}^{\infty}x_ne_n
	$$
	因此对于有界线性算子$T:l^2\to l^2$,成立
	$$
	T(( x_n )_{n=1}^{\infty})
	=\sum_{n=1}^{\infty}x_nT(e_n)
	=\sum_{n=1}^{\infty}x_n(a_{n,m})_{m=1}^{\infty}
	=\left( \sum_{n=1}^{\infty}x_na_{n,m} \right)_{m=1}^{\infty}
	$$
	进而
	$$
	(T(( x_n )_{n=1}^{\infty}),e_l)
	=\left( \left( \sum_{n=1}^{\infty}x_na_{n,m} \right)_{m=1}^{\infty},e_l\right)
	=\sum_{n=1}^{\infty}x_na_{n,l}
	$$
	
	特别的
	$$
	T(e_n)=(a_{n,m})_{m=1}^{\infty},\qquad (T(e_n),e_m)=a_{n,m}
	$$
	同理可得
	$$
	T^*(e_n)=(a^*_{n,m})_{m=1}^{\infty},\qquad (T^*(e_n),e_m)=a_{n,m}^*
	$$
	由于$T^*$为$T$的Hilbert共轭算子,那么
	$$
	a_{n,m}^*=(T^*(e_n),e_m)=(e_n,T(e_m))=\overline{(T(e_m),e_n)}=\overline{a_{m,n}}
	$$
\end{proof}

\chapter{Banach空间}

\begin{proposition}
	记
	$$
	l^\infty=\{\{x_n\}_{n=1}^{\infty}:\exists M,\forall n\in\N^*,|x_n|\le M\},\qquad 
	\|\{\{x_n\}_{n=1}^{\infty}\|=\sup_{n\in\N^*}|x_n|
	$$
	如果矩阵$(a_{ij})$满足
	$$
	\sup_{i\in\N^*}\sum_{j=1}^{\infty}|a_{ij}|<\infty
	$$
	那么定义线性算子
	\nonumber\begin{align}
		T:\begin{aligned}[t]
			l^\infty&\longrightarrow l^\infty\\
			\{x_n\}_{n=1}^{\infty}&\longmapsto \left\{ \sum_{j=1}^{\infty}a_{ij}x_i \right\}_{i=1}^{\infty}
		\end{aligned}
	\end{align}
	证明:$T$为有界线性算子,且
	$$
	\|T\|=\sup_{i\in\N^*}\sum_{j=1}^{\infty}|a_{ij}|
	$$
\end{proposition}

\begin{proof}
	由题意,存在$M>0$,使得成立$\displaystyle \sup_{i\in\N^*}\sum_{j=1}^{\infty}|a_{ij}|\le M$。
	
	首先证明$T$的定义良好性,任取$\{x_n\}_{n=1}^{\infty}\in l^\infty$,注意到对于任意$i\in\N^*$,成立
	$$
	\left| \sum_{j=1}^{\infty}a_{ij}x_i  \right|\le \sum_{j=1}^{\infty}|a_{ij}||x_i|
	\le \sup_{i\in\N^*}\sum_{j=1}^{\infty}|a_{ij}||x_i|\le M|x_i|
	$$
	因此$\displaystyle \left\{\sum_{j=1}^{\infty}a_{ij}x_i\right\}_{i=1}^{\infty}\in l^\infty$,进而$T$定义良好。
	
	其次证明$T$为有界线性算子,且
	$$
	\|T\|=\sup_{i\in\N^*}\sum_{j=1}^{\infty}|a_{ij}|
	$$
	
	一方面
	\nonumber\begin{align}
		\|T\|=&\sup\frac{\left\|  \left\{ \sum_{j=1}^{\infty}a_{ij}x_i \right\}_{i=1}^{\infty} \right\|}{\| \{x_n\}_{n=1}^{\infty} \|}\\
		=&\sup\frac{\sup\limits_{i\in\N^*}|\sum_{j=1}^{\infty}a_{ij}x_i|}{\sup\limits_{n\in\N^*}|x_n|}\\
		\le & \sup\frac{\sup\limits_{i\in\N^*}\sum_{j=1}^{\infty}|a_{ij}||x_i|}{\sup\limits_{n\in\N^*}|x_n|}\\
		\le & \sup\frac{\left(\sup\limits_{i\in\N^*}\sum_{j=1}^{\infty}|a_{ij}|\right)\left(\sup\limits_{i\in\N^*}|x_i|\right)}{\sup\limits_{n\in\N^*}|x_n|}\\
		=&\sup\limits_{i\in\N^*}\sum_{j=1}^{\infty}|a_{ij}|\\
		\le &M
	\end{align}
	因此$T$为有界线性算子。
	
	另一方面,对于任意$\varepsilon>0$,存在$I\in\N^*$,使得成立$\displaystyle \sum_{j=1}^{\infty}|a_{Ij}|\le \sup_{i\in\N^*}\sum_{j=1}^{\infty}|a_{ij}|-\varepsilon$,因此取$\{ x_n \}_{n=1}^\infty=\{ 0,\cdots,0,\mathop{1}\limits_{I \text{ th}},0,0,\cdots \}$,那么
	$$
	\|T\|
	\ge\frac{\left\|  \left\{ \sum_{j=1}^{\infty}a_{ij}x_i \right\}_{i=1}^{\infty} \right\|}{\| \{x_n\}_{n=1}^{\infty} \|}
	=\frac{\sup\limits_{i\in\N^*}|\sum_{j=1}^{\infty}a_{ij}x_i|}{\sup\limits_{n\in\N^*}|x_n|}
	=\sum_{j=1}^{\infty}|a_{Ij}|
	\le \sup_{i\in\N^*}\sum_{j=1}^{\infty}|a_{ij}|-\varepsilon
	$$
	由$\varepsilon$的任意性
	$$
	\|T\|\ge \sup_{i\in\N^*}\sum_{j=1}^{\infty}|a_{ij}|
	$$
	那么
	$$
	\|T\|=\sup_{i\in\N^*}\sum_{j=1}^{\infty}|a_{ij}|
	$$
\end{proof}

\begin{proposition}
	对于有界数列$\{ a_n \}_{n=1}^{\infty}$,定义线性算子
	\nonumber\begin{align}
		T:\begin{aligned}[t]
			l^1&\longrightarrow l^1\\
			\{ x_n \}_{n=1}^{\infty}&\longmapsto \{ a_nx_n \}_{n=1}^{\infty}
		\end{aligned}
	\end{align}
	证明:$T$为有界线性算子,且
	$$
	\|T\|=\sup_{n\in\N^*}|a_n|
	$$
\end{proposition}

\begin{proof}
	由于$\{ a_n \}_{n=1}^{\infty}$为有界数列,因此存在$M>0$,使得对于任意$n\in\N^*$,成立$|a_n|\le M$,进而$\sup\limits_{n\in\N^*}|a_n|\le M$。
	
	首先证明$T$的定义良好性,任取$\{x_n\}_{n=1}^{\infty}\in l^1$,注意到
	$$
	\sum_{n=1}^{\infty}|a_nx_n|\le
	\sum_{n=1}^{\infty}|a_n||x_n|\le
	\sup_{n\in\N^*}|a_n|\sum_{n=1}^{\infty}|x_n|\le
	M\sum_{n=1}^{\infty}|x_n|
	$$
	因此$\{ a_nx_n \}_{n=1}^{\infty}\in l^1$,进而$T$定义良好。
	
	其次证明$T$为有界线性算子,且
	$$
	\|T\|=\sup_{i\in\N^*}\sum_{j=1}^{\infty}|a_{ij}|
	$$
	
	一方面
	$$
	\|T\|=\sup\frac{\|\{ a_nx_n \}_{n=1}^{\infty}\|}{\|\{ x_n \}_{n=1}^{\infty}\|}
	=\sup\frac{\displaystyle\sum_{n=1}^{\infty}|a_nx_n|}{\displaystyle\sum_{n=1}^{\infty}|x_n|}
	\le \sup\frac{\displaystyle\sup_{n\in\N^*}|a_n|\sum_{n=1}^{\infty}|x_n|}{\displaystyle\sum_{n=1}^{\infty}|x_n|}=\sup_{n\in\N^*}|a_n|\le M
	$$
	因此$T$为有界线性算子。
	
	另一方面,对于任意$\varepsilon>0$,存在$N\in\N^*$,使得成立$|a_N|\ge \sup\limits_{n\in\N^*}|a_n|-\varepsilon$,因此取$\{ x_n \}_{n=1}^\infty=\{ 0,\cdots,0,\mathop{1}\limits_{N \text{ th}},0,0,\cdots \}$,那么
	$$
	\|T\|
	\ge\frac{\|\{ a_nx_n \}_{n=1}^{\infty}\|}{\|\{ x_n \}_{n=1}^{\infty}\|}
	=\frac{\displaystyle\sum_{n=1}^{\infty}|a_nx_n|}{\displaystyle\sum_{n=1}^{\infty}|x_n|}
	=|a_N|\ge \sup\limits_{n\in\N^*}|a_n|-\varepsilon
	$$
	由$\varepsilon$的任意性
	$$
	\|T\|
	\ge \sup\limits_{n\in\N^*}|a_n|
	$$
	综合两方面
	$$
	\|T\|=\sup_{n\in\N^*}|a_n|
	$$
\end{proof}

\begin{proposition}
	对于有界数列$\{ a_n \}_{n=1}^{\infty}$,定义线性算子
	\nonumber\begin{align}
		T:\begin{aligned}[t]
			l^1&\longrightarrow l^1\\
			\{ x_n \}_{n=1}^{\infty}&\longmapsto \{ a_nx_n \}_{n=1}^{\infty}
		\end{aligned}
	\end{align}
	证明:$T$为有界可逆线性算子$\iff \inf\limits_{n\in\N^*}|a_n|>0$。
\end{proposition}

\begin{proof}
	由于$\{ a_n \}_{n=1}^{\infty}$为有界数列,因此存在$M>0$,使得对于任意$n\in\N^*$,成立$|a_n|\le M$,进而$\sup\limits_{n\in\N^*}|a_n|\le M$。
	
	一、如果存在$N\in\N^*$,使得成立$a_N=0$,那么由于
	$$
	T(\{ x_n \}_{n=1}^{\infty})=\{ a_1x_1,\cdots,a_{N-1}x_{N-1},\mathop{0}\limits_{N \text{ th}},a_{N+1}x_{N+1},a_{N+1}x_{N+2},\cdots \}
	$$
	因此不存在$\{ x_n \}_{n=1}^{\infty}\in l^1$,使得成立
	$$
	T(\{ x_n \}_{n=1}^{\infty})=\{ 0,\cdots,0,\mathop{1}\limits_{N \text{ th}},0,0,\cdots \}
	$$
	那么$T$不为满射,进而$T$不为有界可逆线性算子。
	
	二、如果对于任意$n\in\N^*$,成立$a_n\ne0$,那么定义线性算子
	\nonumber\begin{align}
		T^{-1}:\begin{aligned}[t]
			l^1&\longrightarrow l^1\\
			\{ x_n \}_{n=1}^{\infty}&\longmapsto \{ x_n/a_n \}_{n=1}^{\infty}
		\end{aligned}
	\end{align}
	注意到
	$$
	T\circ T^{-1}=T^{-1}\circ T=I
	$$
	因此$T$为可逆算子。
	
	1.如果$\inf\limits_{n\in\N^*}|a_n|>0$,那么存在$a>0$,使得对于任意$n\in\N^*$,成立$|a_n|\ge a>0$。由上题
	$$
	\|T\|=\sup_{n\in\N^*}|a_n|\le M,\qquad \|T^{-1}\|=\sup_{n\in\N^*}1/|a_n|\le 1/a
	$$
	因此$T$为有界可逆线性算子。
	
	2.如果$\inf\limits_{n\in\N^*}|a_n|=0$,那么存在$\{ n_k \}_{k=1}^{\infty}\sub\N^*$,使得成立$\lim\limits_{k\to\infty}|a_{n_k}|=0$。注意到
	$$
	\|T^{-1}\|=\sup_{n\in\N^*}1/|a_n|\ge \sup_{k\in\N^*}1/|a_{n_k}|=\infty
	$$
	因此$T$不为有界可逆线性算子。
\end{proof}

\begin{proposition}
	如果矩阵$(a_{ij})$满足
	$$
	\sum_{i=1}^{\infty}\sum_{j=1}^{\infty}|a_{ij}|^q<\infty
	$$
	那么定义线性算子
	\nonumber\begin{align}
		T:\begin{aligned}[t]
			l^p&\longrightarrow l^q\\
			\{x_n\}_{n=1}^{\infty}&\longmapsto \left\{ \sum_{j=1}^{\infty}a_{ij}x_j \right\}_{i=1}^{\infty}
		\end{aligned}
	\end{align}
	其中$1<p,q<\infty$,且$\frac{1}{p}+\frac{1}{q}=1$。
	
	证明:$T$为有界线性算子。
\end{proposition}

\begin{proof}
	由Hölder不等式
	\nonumber\begin{align}
		\|T(\{x_n\}_{n=1}^{\infty})\|_q=&\left(\sum_{i=1}^{\infty}\left| \sum_{j=1}^{\infty}a_{ij}x_j\right|^q\right)^{1/q}\\
		\le & \left(\sum_{i=1}^{\infty}\left| \sum_{j=1}^{\infty}|a_{ij}||x_j|\right|^q\right)^{1/q}\\
		\le & \left(\sum_{i=1}^{\infty}\left| \left(\sum_{j=1}^{\infty}|a_{ij}|^q\right)^{1/q}\left(\sum_{j=1}^{\infty}|x_j|^p\right)^{1/p}\right|^q\right)^{1/q}\\
		= & \|\{x_n\}_{n=1}^{\infty}\|_p\sum_{i=1}^{\infty}\sum_{j=1}^{\infty}|a_{ij}|^q
	\end{align}
\end{proof}

\begin{proposition}
	证明:对于Banach空间$X$,如果$T,S:X\to X$为有界可逆线性算子,那么$TS$为有界可逆线性算子,且$(TS)^{-1}=S^{-1}T^{-1}$。
\end{proposition}

\begin{proof}
	$$
	\|TS\|=\sup\frac{\|T(S(x))\|}{\|x\|}=\sup\frac{\|T(S(x))\|}{\|S(x)\|}\frac{\|S(x)\|}{\|x\|}\le\sup\frac{\|T(x)\|}{\|x\|}\sup\frac{\|S(x)\|}{\|x\|}=\|T\|\|S\|
	$$
\end{proof}

\begin{proposition}
	证明:对于赋范线性空间$X$与$Y$,如果线性算子$T:X\to Y$有界,那么$\ker T$为$X$的闭子空间。
\end{proposition}

\begin{proof}
	(朴素证明)任取$x\in\overline{\ker T}$,那么存在$\{ x_n \}_{n=1}^{\infty}\sub X$,使得成立$\lim\limits_{n\to\infty}x_n=x$。由于$T$为有界线性算子,那么$T$为连续线性算子,因此
	$$
	T(x)=T\left(\lim_{n\to\infty}x_n\right)=\lim_{n\to\infty}T(x_n)=0
	$$
	进而$x\in\ker T$。由$x$的任意性,$\ker T$为$X$的闭子空间。
	
	(优雅证明)由于$Y$为度量空间,因此$Y$满足$T_1$公理,进而$\{0\}$为$Y$的闭集。而$T$有界$\iff T$连续,因此$\ker T=T^{-1}(0)$为闭集。
\end{proof}

\begin{proposition}{}{3.7}
	证明:对于赋范线性空间$X$,如果$x\in X$满足对于任意连续线性泛函$f:X\to\C$,成立$f(x)=0$,那么$x=0$。
\end{proposition}

\begin{proof}
	如果$x\ne 0$,那么由Hahn-Banach定理的推论,存在连续线性泛函$f:X\to\C$,使得成立
	$$
	f(x)=\|x\|
	$$
	于是
	$$
	0=f(x)=\|x\|\ne 0
	$$
	矛盾!因此$x=0$。
\end{proof}

\begin{proposition}
	证明:如果$X$为Banach空间,那么对于任意$x\in X$,成立
	$$
	\|x\|=\sup_{\substack{f\in X^*\\\|f\|\le 1}}|f(x)|
	$$
\end{proposition}

\begin{proof}
	由典型映射的保范性,命题显然!
\end{proof}

\begin{proposition}
	证明:对于赋范线性空间$X$上的次可加且正齐次泛函$f:X\to \R$,换言之,对于任意$x,y\in X$与$\lambda\in\C$,成立
	$$
	f(x+y)\le f(x)+f(y),\qquad 
	f(\lambda x)=|\lambda|f(x)
	$$
	如下命题等价。
	\begin{enumerate}
		\item $T$在$0$处连续。
		\item $T$在$x_0\in X$处连续。
		\item $T$在$X$上连续。
		\item $T$在$X$上一致连续。
		\item $T$在$X$上Lipschitz连续。
		\item $T$在$X$上有界。
	\end{enumerate}
\end{proposition}

\begin{proof}
	$6\implies 5$:由于
	$$
	f(-x)=f(x),\qquad f(x)-f(y)\le f(x-y)
	$$
	那么
	$$
	|f(x)-f(y)|\le |f(x-y)|
	$$
	由于$f$有界,于是存在$C>0$,使得对于任意$x\in X$,成立$\|f\|\le C\|x\|$。任取$x,y\in X$,由于
	$$
	|f(x)-f(y)|\le |f(x-y)|\le C\|x-y\|
	$$
	那么$T$在$X$上Lipschitz连续。
	
	$5\implies 4\implies 3\implies 2\implies 1$:显然!
	
	$1\implies 6$:由于$f$在$0$处连续,那么存在$\delta>0$,使得当$\|x\|\le \delta$时,成立$|f(x)|\le 1$,因此对于任意$x\in X\setminus\{0\}$,成立
	$$
	|f(x)|=\frac{\|x\|}{\delta}\left\| f\left(\frac{\delta}{\|x\|}x\right) \right\|\le \frac{\|x\|}{\delta}
	$$
	因此$f$在$X$上有界。
\end{proof}

\begin{proposition}
	证明:如果$p(x)$为线性空间$X$上的半范数,那么$p^{-1}(B(r))$为平衡的、吸收的凸集。
\end{proposition}

\begin{proposition}
	证明:凸集的闭包为凸集,平衡集的闭包为平衡集,吸收集的闭包为吸收集。
\end{proposition}

\begin{proposition}
	求解$L^1[0,1]$上有界线性泛函的一般形式。
\end{proposition}

\begin{proposition}
	证明Hellinger-Toeplitz定理:对于Hilbert空间$\mathcal{H}$上的线性算子$T:\mathcal{H}\to\mathcal{H}$,如果对于任意$x,y\in\mathcal{H}$,成立$(T(x),y)=(x,T(y))$,那么$T$为有界线性算子。
\end{proposition}

\begin{proof}
	对于任意$y\in\mathcal{H}$,构造线性泛函
	\begin{align*}
		f_y:\begin{aligned}[t]
			\mathcal{H}&\longrightarrow \C\\
			x&\longmapsto (x,T(y))
		\end{aligned}
	\end{align*}
	由Scharz不等式
	$$
	\|f_y\|=\sup_{\|x\|=1}|f_y(x)|=\sup_{\|x\|=1}|(x,T(y))|\le \sup_{\|x\|=1}\|x\|\|T(y)\|=\|T(y)\|
	$$
	因此$f_y\in\mathcal{H}^*$。由Riesz表现定理,成立$\|f_y\|=\|T(y)\|$。由于对于任意$x\in\mathcal{H}$,由Scharz不等式
	$$
	\sup_{\|y\|=1}|f_y(x)|
	=\sup_{\|y\|=1}|(x,T(y))|
	=\sup_{\|y\|=1}|(T(x),y)|
	\le\sup_{\|y\|=1}\|T(x)\|\|y\|
	=\|T(x)\|<\infty
	$$
	因此由一致有界原理,成立$\displaystyle \sup{\|y\|=1}\|f_y\|<\infty$,因此
	$$
	\|T\|=\sup_{\|y\|=1}\|T(y)\|=\sup_{\|y\|=1}\|f_y\|<\infty
	$$
	进而$T$为有界线性算子。
\end{proof}

\begin{proposition}
	证明:对于Hilbert空间$\mathcal{H}$上的线性算子$T,S:\mathcal{H}\to\mathcal{H}$,如果对于任意$x,y\in\mathcal{H}$,成立$(T(x),y)=(x,S(y))$,那么$T,S$为有界算子,且$T=S^*$。
\end{proposition}

\begin{proof}
	任取$\{x_n\}_{n=1}^\infty\sub\mathcal{H}$,使得成立$x_n\to x$且$T(x_n)\to y$。由于$\mathcal{H}$为Hilbert空间,因此$x\in\mathcal{H}$。由于对于任意$z\in\mathcal{H}$,成立$(T(x_n),z)=(x_n,S(z))$,那么$(y,z)=(x,S(z))$,因此$(y,z)=(T(x),z)$。取$z=T(x)-y$,那么$\|T(x)-y\|=0$,因此$T(x)=y$,进而$T$为闭算子。由闭图形定理,$T$为有界算子。同理可得$S$为有界算子。任取$x,y\in\mathcal{H}$,那么
	$$
	(T(x),y)=(x,S(y))=(S^*(x),y)\implies T=S^*
	$$
\end{proof}

\begin{proposition}
	证明:如果$T:X\to Y$为单的有界线性算子,其中$X,Y$为Banach空间,那么$T^{-1}$为闭算子。
\end{proposition}

\begin{proof}
	由$T$为连续算子,命题得证!
\end{proof}

\begin{proposition}
	对于赋范线性空间$X$与$Y$,$M$为$X$的线性子空间,$T:M\to Y$为线性算子,定义向量空间$X\times Y$上的范数为
	$$
	\|(x,y)\|=\|x\|+\|y\|
	$$
	证明:
	$$
	T\text{为闭算子}\iff G(T)\text{闭集}
	$$
\end{proposition}

\begin{proof}
	对于必要性,如果$T$为闭算子,那么任取$(x,y)\in \overline{G(T)}$,因此存在$\{x_n\}_{n=1}^{\infty}\sub M$,使得成立
	\begin{align*}
		& \lim_{n\to\infty}(x_n,T(x_n))=(x,y)\\
		\iff & \lim_{n\to\infty}\|(x_n-x,T(x_n)-y)\|=0\\
		\iff & \lim_{n\to\infty}\|x_n-x\|+\|T(x_n)-y\|=0\\
		\iff &  \lim_{n\to\infty}\|x_n-x\|=\lim_{n\to\infty}\|T(x_n)-y\|=0\\
		\iff & \lim_{n\to\infty}x_n=x,\qquad 
		\lim_{n\to\infty}T(x_n)=y
	\end{align*}
	由于$T$为闭算子,那么$x\in M$且$T(x)=y$,因此$(x,y)\in G(T)$。由$(x,y)$的任意性,$G(T)$为闭集。
	
	对于充分性,如果$G(T)$为闭集,那么任取$\{x_n\}_{n=1}^{\infty}\sub M$,使得成立
	\begin{align*}
		& \lim_{n\to\infty}x_n=x,\qquad 
		\lim_{n\to\infty}T(x_n)=y\\
		\iff &  \lim_{n\to\infty}\|x_n-x\|=\lim_{n\to\infty}\|T(x_n)-y\|=0\\
		\iff & \lim_{n\to\infty}\|x_n-x\|+\|T(x_n)-y\|=0\\
		\iff & \lim_{n\to\infty}\|(x_n-x,T(x_n)-y)\|=0\\
		\iff & \lim_{n\to\infty}(x_n,T(x_n))=(x,y)\\
		\implies& (x,y)\in \overline{G(T)}
	\end{align*}
	由于$G(T)$为闭集,那么$(x,y)\in G(T)$,因此$x\in M$且$T(x)=y$。由$\{x_n\}_{n=1}^{\infty}$的任意性,$T$为闭算子。
\end{proof}

\begin{proposition}
	对于Banach空间$X$上的点列$\{x_n\}_{n=1}^{\infty}\sub X$,如果对于任意连续线性泛函$f:X\to\C$,成立
	$$
	\sum_{n=1}^{\infty}|f(x_n)|^p<\infty
	$$
	其中$p\ge 1$,那么存在$\mu>0$,使得对于连续线性泛函$f:X\to\C$,成立
	$$
	\sum_{n=1}^{\infty}|f(x_n)|^p<\mu\|f\|^p
	$$
\end{proposition}

\begin{proof}
	对于任意$n\in\mathbb{N}^*$,定义线性算子
	\begin{align*}
		T_n:\begin{aligned}[t]
			X^*&\longrightarrow l^p\\
			f&\longmapsto \{ f(x_1),\cdots,f(x_n),0,0,\cdots \}
		\end{aligned}
	\end{align*}
	由于
	$$
	\|T_n(f)\|_p
	= \left(\sum_{k=1}^{n}|f(x_k)|^p\right)^{1/p}
	\le \left(\sum_{k=1}^{n}\|f\|^p\|x_k\|^p\right)^{1/p}
	= \|f\|\left(\sum_{k=1}^{n}\|x_k\|^p\right)^{1/p}
	$$
	那么对于任意$n\in\N^*$,$T_n$为有界线性算子。由于对于任意连续线性泛函$f:X\to\C$,成立
	$$
	\sup_{n\in\N^*}\|T_n(f)\|_p
	= \sup_{n\in\N^*}\left(\sum_{k=1}^{n}|f(x_k)|^p\right)^{1/p}
	= \left(\sum_{n=1}^{\infty}|f(x_n)|^p\right)^{1/p}
	<\infty
	$$
	那么由一致有界原理,存在$\mu^{1/p}>0$,使得成立$\displaystyle \sup_{n\in\N^*}\|T_n\|<\mu^{1/p}$,因此对于任意连续线性泛函$f:X\to\C$,成立
	$$
	\sum_{n=1}^{\infty}|f(x_n)|^p
	= \sup_{n\in\N^*}\sum_{k=1}^{n}|f(x_k)|^p
	= \sup_{n\in\N^*}\|T_n(f)\|_p^p
	\le \sup_{n\in\N^*}\|T_n\|^p\|f\|^p
	< \mu \|f\|^p
	$$
\end{proof}

\begin{proposition}
	证明:有限维赋范线性空间的对偶空间为有限维赋范线性空间,且维数相同;无穷维赋范线性空间的对偶空间为无穷维赋范线性空间。
\end{proposition}

\begin{proof}
	对于$n$维赋范线性空间$X$,其基为$\{e_k\}_{k=1}^{n}$,由Hahn-Banach定理的推论,存在$\{f_k\}_{k=1}^{n}\sub X^*$,使得成立
	$$
	f_i(e_j)=\begin{cases}
		1,\qquad & i=j\\
		0,\qquad & i\ne j
	\end{cases}
	$$
	容易知道$\{f_k\}_{k=1}^{n}$线性无关,且对于任意
	$$
	x=\sum_{k=1}^{n}x_ke_k\in X
	$$
	成立
	$$
	f_i(x)=\sum_{j=1}^{n}x_jf_i(e_j)
	$$
	因此对于任意$f\in X^*$,成立
	$$
	f(x)=\sum_{k=1}^{n}x_kf(e_k)=\sum_{k=1}^{n}f(e_k)f_k(x)
	=\left(\sum_{k=1}^{n}f(e_k)f_k\right)(x)
	$$
	从而
	$$
	f=\sum_{k=1}^{n}f(e_k)f_k
	$$
	进而$\{f_k\}_{k=1}^{n}$为$X^*$的基,于是$X^*$为$n$维赋范线性空间。
	
	对于无穷维赋范线性空间$X$,如果$X^*$为$n$维赋范线性空间,那么$X^{**}$为$n$维赋范线性空间。由于典型映射$\tau$为单的保范线性空间,那么由同构定理
	$$
	X/\ker\tau\cong\im\tau\iff X\cong \tau(X)
	$$
	因此$\tau(X)$为无穷维赋范线性空间。但是$\tau(X)\sub X^{**}$,矛盾!因此$X^*$为无穷维赋范线性空间。
\end{proof}

\begin{proposition}
	证明:对于Banach空间$X$,成立
	$$
	X\text{为自反空间}\iff X^*\text{为自反空间}
	$$
\end{proposition}

\begin{proof}
	$X$的典型映射为
	\begin{align*}
		\psi:\begin{aligned}[t]
			X&\longrightarrow X^{**}\\
			x&\longmapsto F_x,\text{ 其中 }F_x(f)=f(x)
		\end{aligned}
	\end{align*}
	$X^{*}$的典型映射为
	\begin{align*}
		\Psi:\begin{aligned}[t]
			X^{*}&\longrightarrow X^{***}\\
			f&\longmapsto \mathscr{F}_f,\text{ 其中 }\mathscr{F}_f(F)=F(f)
		\end{aligned}
	\end{align*}
	
	对于必要性,如果$X$为自反空间,那么$\psi$为满射。任取$\mathscr{F}\in X^{***}$,对于任意$F\in X^{**}$,存在$x\in X$,使得成立$\psi(x)=F$,因此对于任意$f\in X^*$,成立$F(f)=f(x)$,于是
	$$
	\mathscr{F}(F)
	=\mathscr{F}(\psi(x))
	=(\mathscr{F}\circ\psi)(x)
	=F(\mathscr{F}\circ\psi)
	$$
	所以$\Psi(\mathscr{F}\circ\psi)=\mathscr{F}$,那么$\Psi$为满射,进而$X^*$为自反空间。
	
	对于充分性,如果$X^*$为自反空间,那么由必要性,$X^{**}$为自反空间。由于$\im \psi$为$X^{**}$的闭子空间,那么$\im\psi$为自反空间。由于$\psi$为单射,那么$X\cong\im\psi$,进而$X$为自反空间。
\end{proof}

\begin{proposition}
	证明:对于Banach空间$X$与$Y$,如果$T:X\to Y$为保范线性双射,那么$T^*:Y^*\to X^*$为保范线性双射。
\end{proposition}

\begin{proof}
	对于线性
	\begin{align*}
		& T^*(f+g)=(f+g)\circ T=f\circ T+g\circ T=T^*(f)+T^*(g)\\
		& T^*(\lambda g)=(\lambda g)\circ T=\lambda(g\circ T)=\lambda T^*(g)
	\end{align*}

	对于双射性,构造算子
	\begin{align*}
		(T^*)^{-1}:\begin{aligned}[t]
			X^*&\longrightarrow Y^*\\
			f&\longmapsto f\circ T^{-1}
		\end{aligned}
	\end{align*}
	由于
	\begin{align*}
		& ((T^*)^{-1}\circ T^*)(g)
		= (T^*)^{-1}(T^*(g))
		= (T^*)^{-1}(g\circ T)
		= g\circ T\circ T^{-1}
		= g
		\implies
		(T^*)^{-1}\circ T^*=\mathbbm{1}_{Y^*}\\
		& (T^*\circ (T^*)^{-1})(f)
		= T^*((T^*)^{-1}(f))
		= T^*(f\circ T^{-1})
		= f\circ T^{-1}\circ T
		= f
		\implies
		T^*\circ (T^*)^{-1} = \mathbbm{1}_{X^*}
	\end{align*}
	那么$T^*$为双射。
	
	对于保范性,一方面
	$$
	|(T^*(g))(x)|
	= |(g\circ T)(x)|
	= |g(T(x))|
	\le \|g\|\|T(x)\|
	= \|g\|\|x\|\implies
	\|T^*(g)\|\le \|g\|
	$$
	另一方面,对于任意$n\in\N^*$,由于$\displaystyle\|g\|=\sup_{\|y\|\le 1}|g(y)|$,那么存在$y_n\in Y$,使得成立
	$$
	\|y_n\|\le 1,\qquad |g(y_n)|\ge \|g\|-\frac{1}{n}
	$$
	由于$T$为双射,那么存在$\{x_n\}_{n=1}^{\infty}\sub X$,使得对于任意$n\in\N^*$,成立
	$$
	T(x_n)=y_n,\qquad \|x_n\|=\|T(x_n)\|=\|y_n\|\le 1
	$$
	因此
	$$
	|(T^*(g))(x_n)|
	= |(g\circ T)(x_n)|
	= |g(T(x_n))|
	= |g(y_n)|
	\ge \|g\|-\frac{1}{n}
	$$
	进而
	$$
	\|T^*(g)\|
	= \sup_{\|x\|\le 1}|(T^*(g))(x)|
	\ge \sup_{n\in\N^*}|(T^*(g))(x_n)|
	\ge \sup_{n\in\N^*}\|g\|-\frac{1}{n}
	= \|g\|
	$$
	综合两方面,$\|T^*(g)\|=\|g\|$,因此$T^*$为保范算子。
	
	综上所述,$T^*$为保范线性双射。
\end{proof}

\begin{proposition}
	证明:对于赋范线性空间$X$与$Y$,如果$\mathcal{L}(X,Y)$为Banach空间,那么$Y$为Banach空间。
\end{proposition}

\begin{proof}
	由于$X$为非零赋范线性空间,那么存在$x_0\in X\setminus\{0\}$。由Hahn-Banach定理的推论,存在有界线性泛函$f_0:X\to \C$,使得成立
	$$
	\|f_0\|=1,\qquad f_0(x_0)=\|x_0\|
	$$
	任取Cauchy序列$\{y_n\}_{n=1}^{\infty}\sub Y$,对于任意$n\in\N^*$,定义映射
	\begin{align*}
		T_n:\begin{aligned}[t]
			X &\longrightarrow Y\\
			x &\longmapsto \frac{f_0(x)}{f_0(x_0)}y_n
		\end{aligned}
	\end{align*}
	由于
	\begin{align*}
		& T_n(x+y)
		=\frac{f_0(x+y)}{f_0(x_0)}y_n
		=\frac{f_0(x)}{f_0(x_0)}y_n+\frac{f_0(y)}{f_0(x_0)}y_n
		=T_n(x)+T_n(y)\\
		& T_n(\lambda x)
		=\frac{f_0(\lambda x)}{f_0(x_0)}y_n
		= \lambda \frac{f_0(x)}{f_0(x_0)}y_n
		=\lambda T_n(x)
	\end{align*}
	因此$T_n$为线性算子。
	
	由于
	$$
	\|T_n(x)\|
	=\left\| \frac{f_0(x)}{f_0(x_0)}y_n \right\|
	=\frac{\|y_n\|}{\|x_0\|}\|f_0(x)\|
	\le \frac{\|y_n\|}{\|x_0\|}\|f_0\|\|x\|
	= \frac{\|y_n\|}{\|x_0\|}\|x\|
	\implies 
	\|T_n\|\le \frac{\|y_n\|}{\|x_0\|}
	$$
	因此$T_n$为有界算子,进而$\{T_n\}_{n=1}^{\infty}\sub \mathcal{L}(X,Y)$。
	
	由于
	\begin{align*}
		\|T_m-T_n\|
		& = \sup_{\|x\|\le 1}\|T_m(x)-T_n(x)\|\\
		& = \sup_{\|x\|\le 1}\left\| \frac{f_0(x)}{f_0(x_0)}(y_m-y_n) \right\|\\
		& = \sup_{\|x\|\le 1}\frac{\|y_m-y_n\|}{\|x_0\|}\|f_0(x)\|\\
		& \le \sup_{\|x\|\le 1}\frac{\|y_m-y_n\|}{\|x_0\|}\|f_0\|\|x\|\\
		& = \frac{\|y_m-y_n\|}{\|x_0\|}
	\end{align*}
	因此$\{T_n\}_{n=1}^{\infty}$为Cauchy序列。由于$\mathcal{L}(X,Y)$为Banach空间,那么存在$T\in \mathcal{L}(X,Y)$,使得成立$T_n\to T$。由于对于任意$x\in X$​,成立
	$$
	\|T_n(x)-T(x)\|\le \|T_n-T\|\|x\|
	$$
	那么对于任意$x\in X$,成立$T_n(x)\to T(x)$。特别的,$T_n(x_0)\to T(x_0)$。记$y=T(x_0)\in Y$,那么$y_n\to y$。
	
	综上所述,$Y$为Banach空间。
\end{proof}

\begin{proposition}
	证明:对于赋范线性空间$X$,如果对于任意线性泛函$f:X\to \C$,成立
	$$
	f\text{ 依范数 }\|\cdot\|_1\text{ 连续}
	\implies
	f\text{ 依范数 }\|\cdot\|_2\text{ 连续}
	$$
	那么存在$M>0$,使得对于任意$x\in X$,成立
	$$
	\|x\|_1\le M\|x\|_2
	$$
\end{proposition}

\begin{proof}
	记$X_1$为$X$依范数$\|\cdot\|_1$的闭包,$X_2$为$X$依范数$\|\cdot\|_2$的闭包,那么$X_1$与$X_2$均为Banach空间。构造恒等算子
	\begin{align*}
		I:\begin{aligned}[t]
			X_2 &\longrightarrow X_1\\
			x &\longmapsto x
		\end{aligned}
	\end{align*}
	任取$\{x_n\}_{n=1}^{\infty}\sub X_2$以及$x\in X_2$与$y\in X_1$,满足
	$$
	\lim_{n\to\infty}\|x_n-x\|_2
	=\lim_{n\to\infty}\|x_n-y\|_1
	=0
	$$
	任取线性泛函$f:X\to\C$,满足$f$依范数$\|\cdot\|_1$连续,那么$f$依范数$\|\cdot\|_2$连续,因此
	$$
	f(x_n)\to f(x),\qquad 
	f(x_n)\to f(y)
	$$
	那么$f(x)=f(y)$。由习题\ref{pro:3.7},$x=y$,进而$I$为闭算子。由闭图形定理,$I$为有界算子,命题得证!
\end{proof}

\end{document}