\documentclass[lang = cn, scheme = chinese]{elegantbook}
% elegantbook      设置elegantbook文档类
% lang = cn        设置中文环境
% scheme = chinese 设置标题为中文


%% 1.封面设置

\title{2021级泛函分析期末试题}                % 文档标题

\author{若水}                        % 作者

\myemail{ethanmxzhou@163.com}       % 邮箱

\homepage{helloethanzhou.github.io} % 主页

\date{\today}                       % 日期

\logo{PiCreatures_happy.pdf}        % 设置Logo

\cover{阿基米德螺旋曲线.pdf}          % 设置封面图片

% 修改标题页的色带
\definecolor{customcolor}{RGB}{135, 206, 250} 
% 定义一个名为customcolor的颜色,RGB颜色值为(135, 206, 250)

\colorlet{coverlinecolor}{customcolor}     % 将coverlinecolor颜色设置为customcolor颜色

%% 2.目录设置
\setcounter{tocdepth}{3}  % 目录深度为3

%% 3.引入宏包
\usepackage[all]{xy}
\usepackage{bbm, svg, graphicx, float, extpfeil, amsmath, amssymb, mathrsfs, mathalpha, hyperref}


%% 4.定义命令
\newcommand{\N}{\mathbb{N}}            % 自然数集合
\newcommand{\R}{\mathbb{R}}            % 实数集合
\newcommand{\C}{\mathbb{C}}  		   % 复数集合
\newcommand{\Q}{\mathbb{Q}}            % 有理数集合
\newcommand{\Z}{\mathbb{Z}}            % 整数集合
\newcommand{\sub}{\subset}             % 包含
\newcommand{\im}{\text{im }}           % 像
\newcommand{\lang}{\langle}            % 左尖括号
\newcommand{\rang}{\rangle}            % 右尖括号
\newcommand{\function}[5]{
	\begin{align*}
		#1:\begin{aligned}[t]
			#2 &\longrightarrow #3\\
			#4 &\longmapsto #5
		\end{aligned}
	\end{align*}
}                                     % 函数

\begin{document}
	
\maketitle       % 创建标题页

\frontmatter     % 开始前言部分

\tableofcontents % 创建目录

\mainmatter      % 开始正文部分

\chapter{}

\begin{proposition}
	证明:存在且存在唯一$f\in C[a,b]$,使得对于任意$a\le x \le b$,成立
	$$
	f(x)=\varphi(x)+\mu\int_a^bK(x,y)f(y)\mathrm{d}y
	$$
	其中$\varphi\in C[a,b]$且$K\in C[a,b]^2$,同时
	$$
	|\mu||a-b|M<1,\qquad 
	M=\max_{a\le x,y\le b}|K(x,y)|
	$$
\end{proposition}

\begin{proof}
	构造映射
	\begin{align*}
		T:\begin{aligned}[t]
			C[a,b]&\longrightarrow C[a,b]\\
			f&\longmapsto F,\text{ 其中 }F(x)=\varphi(x)+\mu\int_a^bK(x,y)f(y)\mathrm{d}y
		\end{aligned}
	\end{align*}
	由于
	\begin{align*}
		\|T(f)-T(g)\|
		=&\sup_{x\in[a,b]}|(T(f))(x)-(T(g))(x)|\\
		=&|\mu|\sup_{x\in[a,b]}\left| \int_a^bK(x,y)(f(y)-g(y))\mathrm{d}y \right|\\
		\le & |\mu|\sup_{x\in[a,b]}\int_a^b|K(x,y)||f(y)-g(y)|\mathrm{d}y\\
		\le & |\mu||a-b|M\sup_{x\in[a,b]}|f(x)-g(x)|\\
		=& |\mu||a-b|M\|f-g\|
	\end{align*}
	而$|\mu||a-b|M<1$,那么$T$为压缩映射,由压缩映像原理,存在且存在唯一$f\in C[a,b]$,使得成立$T(f)=f$​,因此成立Fredholm积分方程
	$$
	f(x)=\varphi(x)+\mu\int_a^bK(x,y)f(y)\mathrm{d}y
	$$
\end{proof}

\chapter{}

\begin{proposition}{}{2.1}
	已知$\left\{ \frac{1}{\sqrt{2\pi}},\frac{1}{\sqrt{\pi}}\sin nx,\frac{1}{\sqrt{\pi}}\cos nx \right\}_{n=1}^{\infty}$为$L^2[-\pi,\pi]$的正规正交基,定义函数
	\begin{align*}
		f:\begin{aligned}[t]
			[-\pi,\pi]&\longrightarrow \R\\
			x&\longmapsto |x|
		\end{aligned}
	\end{align*}
	
	计算
	$$
	\left(f(x),\frac{1}{\sqrt{2\pi}}\right),\qquad 
	\left(f(x),\frac{1}{\sqrt{\pi}}\sin nx\right),\qquad 
	\left(f(x),\frac{1}{\sqrt{\pi}}\cos nx\right)
	$$
	其中$n\in\N^*$。
\end{proposition}

\begin{proof}
	\begin{align*}
		&\left(f(x),\frac{1}{\sqrt{2\pi}}\right)=\frac{\pi\sqrt{2\pi}}{2}\\
		&\left(f(x),\frac{1}{\sqrt{\pi}}\sin nx\right)=0\\
		&\left(f(x),\frac{1}{\sqrt{\pi}}\cos nx\right)=\begin{cases}
			\displaystyle-\frac{4}{\sqrt{\pi}n^2},\qquad & n\text{ 为奇}\\
			0,\qquad & \text{ 为偶}
		\end{cases}
	\end{align*}
\end{proof}

\begin{proposition}{}{2.2}
	证明:
	$$
	\sum_{n=1}^{\infty}\frac{1}{n^2}=\frac{\pi^2}{6}
	$$
\end{proposition}

\begin{proof}
	由试题\ref{pro:2.1}
	$$
	f(x)=\frac{\pi}{2}-\frac{4}{\pi}\sum_{n=1}^{\infty}\frac{\cos(2n-1)x}{(2n-1)^2}
	$$
	取$x=0$,那么
	$$
	\sum_{n=1}^{\infty}\frac{1}{(2n-1)^2}=\frac{\pi^2}{8}
	$$
	因此
	$$
	\sum_{n=1}^{\infty}\frac{1}{n^2}
	=\sum_{n=1}^{\infty}\frac{1}{(2n-1)^2}+\sum_{n=1}^{\infty}\frac{1}{(2n)^2}
	=\frac{\pi^2}{8}+\frac{1}{4}\sum_{n=1}^{\infty}\frac{1}{n^2}
	$$
	进而
	$$
	\sum_{n=1}^{\infty}\frac{1}{n^2}=\frac{\pi^2}{6}
	$$
\end{proof}

\begin{proposition}
	证明:
	$$
	\sum_{n=1}^{\infty}\frac{1}{n^4}=\frac{\pi^4}{90}
	$$
\end{proposition}

\begin{proof}
	(法一)由试题\ref{pro:2.1}
	$$
	f(x)=\frac{\pi\sqrt{2\pi}}{2}\cdot\frac{1}{\sqrt{2\pi}}+\sum_{n=1}^{\infty}\frac{-4}{\sqrt{\pi}(2n-1)^2}\cdot\frac{1}{\sqrt{\pi}}\cos (2n-1)x
	$$
	由Parseval公式
	$$
	\|f\|^2=\frac{\pi^3}{2}+\sum_{n=1}^{\infty}\frac{16}{\pi(2n-1)^4}
	$$
	而$\|f\|^2=2\pi^3/3$​,因此
	$$
	\frac{\pi^3}{2}+\sum_{n=1}^{\infty}\frac{16}{\pi(2n-1)^4}=\frac{2\pi^3}{3}
	$$
	那么
	$$
	\sum_{n=1}^{\infty}\frac{1}{(2n-1)^4}=\frac{\pi^4}{96}
	$$
	因此
	$$
	\sum_{n=1}^{\infty}\frac{1}{n^4}
	=\sum_{n=1}^{\infty}\frac{1}{(2n-1)^4}+\sum_{n=1}^{\infty}\frac{1}{(2n)^4}
	=\frac{\pi^4}{96}+\frac{1}{16}\sum_{n=1}^{\infty}\frac{1}{n^4}
	$$
	进而
	$$
	\sum_{n=1}^{\infty}\frac{1}{n^4}=\frac{\pi^4}{90}\\
	$$
	
	(法二)定义函数
	\begin{align*}
		g:\begin{aligned}[t]
			[-\pi,\pi]&\longrightarrow \R\\
			x&\longmapsto |x|^3
		\end{aligned}
	\end{align*}
	将$g$展开为Fourier级数
	$$
	g(x)=\frac{\pi^3}{4}
	+\sum_{n=1}^{\infty}\frac{3\pi}{2n^2}\cos{2nx}
	+\sum_{n=1}^{\infty}\frac{6\pi}{(2n-1)^2}\cos{(2n-1)x}
	+\sum_{n=1}^{\infty}\frac{24}{\pi(2n-1)^4}\cos(2n-1)x
	$$
	取$x=0$​,那么
	$$
	\frac{\pi^3}{4}
	+\sum_{n=1}^{\infty}\frac{3\pi}{2n^2}
	-\sum_{n=1}^{\infty}\frac{6\pi}{(2n-1)^2}
	+\sum_{n=1}^{\infty}\frac{24}{\pi(2n-1)^4}=0
	$$
	由试题\ref{pro:2.2}
	$$
	\sum_{n=1}^{\infty}\frac{1}{(2n-1)^2}=\frac{\pi^2}{8},\qquad \sum_{n=1}^{\infty}\frac{1}{n^2}=\frac{\pi^2}{6}
	$$
	代入,可得
	$$
	\sum_{n=1}^{\infty}\frac{1}{(2n-1)^4}=\frac{\pi^4}{96}
	$$
	因此
	$$
	\sum_{n=1}^{\infty}\frac{1}{n^4}
	=\sum_{n=1}^{\infty}\frac{1}{(2n-1)^4}+\sum_{n=1}^{\infty}\frac{1}{(2n)^4}
	=\frac{\pi^4}{96}+\frac{1}{16}\sum_{n=1}^{\infty}\frac{1}{n^4}
	$$
	进而
	$$
	\sum_{n=1}^{\infty}\frac{1}{n^4}=\frac{\pi^4}{90}\\
	$$
\end{proof}

\chapter{}

\begin{proposition}
	对于Hilbert空间$\mathcal{H}$上的线性算子$T:\mathcal{H}\to\mathcal{H}$,如果对于任意$x,y\in\mathcal{H}$,成立$(T(x),y)=(x,T(y))$,那么$T$为有界算子。
\end{proposition}

\begin{proof}
	(法一)对于任意$y\in\mathcal{H}$,构造线性泛函
	\begin{align*}
		f_y:\begin{aligned}[t]
			\mathcal{H}&\longrightarrow \C\\
			x&\longmapsto (x,T(y))
		\end{aligned}
	\end{align*}
	由Scharz不等式
	$$
	\|f_y\|=\sup_{\|x\|=1}|f_y(x)|=\sup_{\|x\|=1}|(x,T(y))|\le \sup_{\|x\|=1}\|x\|\|T(y)\|=\|T(y)\|
	$$
	因此$f_y\in\mathcal{H}^*$。由Frechet-Riesz表现定理,成立$\|f_y\|=\|T(y)\|$。由于对于任意$x\in\mathcal{H}$,由Scharz不等式
	$$
	\sup_{\|y\|=1}|f_y(x)|
	=\sup_{\|y\|=1}|(x,T(y))|
	=\sup_{\|y\|=1}|(T(x),y)|
	\le\sup_{\|y\|=1}\|T(x)\|\|y\|
	=\|T(x)\|<\infty
	$$
	因此由一致有界原理,成立$\displaystyle\sup_{\|y\|=1}\|f_y\|<\infty$,因此
	$$
	\|T\|=\sup_{\|y\|=1}\|T(y)\|=\sup_{\|y\|=1}\|f_y\|<\infty
	$$
	进而$T$为有界算子。
	
	(法二)任取$\{x_n\}_{n=1}^{\infty}\sub X$,使得成立
	$$
	\lim_{n\to\infty}x_n=x,\qquad 
	\lim_{n\to\infty}T(x_n)=y
	$$
	那么对于任意$z\in\mathcal{H}$以及$n\in\N^*$,成立
	$$
	(T(x_n),z)=(x_n,T(z))
	$$
	由内积的连续性
	$$
	(y,z)=(x,T(z))=(T(x),z)
	$$
	进而$T(x)=y$,因此$T$为闭算子。由闭图形定理,$T$为有界算子。
\end{proof}

\chapter{}

\begin{definition}
	对于有限测度的可测集合$E$,定义
	$$
	S(E)=\left\{ \text{ 几乎处处有限的可测函数 }f:E\to\R \right\},\qquad 
	d(f,g)=\int_E \frac{|f-g|}{1+|f-g|}
	$$
\end{definition}

\begin{lemma}{}{引理11}
	$$
	\frac{|x+y|}{1+|x+y|}\le\frac{|x|}{1+|x|}+\frac{|y|}{1+|y|},\qquad x,y\in\C
	$$
\end{lemma}

\begin{proof}
	构造函数
	$$
	f(x)=\frac{x}{1+x},\qquad x\in[0,\infty)
	$$
	由于$f$在$[0,\infty)$上单调递增,那么由三角不等式$|x+y|\le |x|+|y|$,可得
	$$
	\frac{|x+y|}{1+|x+y|}
	= f(|x+y|)
	\le f(|x|+|y|)
	= \frac{|x|+|y|}{1+|x|+|y|}
	\le \frac{|x|}{1+|x|}+\frac{|y|}{1+|y|}
	$$
\end{proof}

\begin{proposition}
	$S(E)$空间为度量空间。
\end{proposition}

\begin{proof}
	仅证明三角不等式,由引理\ref{lem:引理11}
	\begin{align*}
		d(f,h)
		& = \int_a^b\frac{|f-h|}{1+|f-h|}\\
		& = \int_a^b\frac{|(f-g)+(g-h)|}{1+|(f-g)+(g-h)|}\\
		& \le \int_a^b \left( \frac{|f-g|}{1+|f-g|}+\frac{|g-h|}{1+|g-h|} \right)\\
		& = \int_a^b\frac{|f-g|}{1+|f-g|}+\int_a^b\frac{|g-h|}{1+|g-h|}\\
		& = d(f,g)+d(g,h)
	\end{align*}
\end{proof}

\begin{proposition}
	在$S(E)$,成立
	$$
	\{f_n\}_{n=1}^{\infty}\text{依度量}d\text{收敛于}f\iff \{f_n\}_{n=1}^{\infty}\text{依测度收敛于}f
	$$
\end{proposition}

\begin{proof}
	对于必要性,任取$\{f_n\}_{n=1}^{\infty}\sub S(E)$,使得成立$\{f_n\}_{n=1}^{\infty}$依度量$d$收敛于$f\in S(E)$。任取$\varepsilon>0$,记
	$$
	E_n=E[|f_n-f|\ge \varepsilon]
	$$
	由引理\ref{lem:引理11}
	\begin{align*}
		d(f_n,f)
		& = \int_a^b\frac{|f_n-f|}{1+|f_n-f|}\\
		& \ge \int_{E_n}\frac{|f_n-f|}{1+|f_n-f|}\\
		& \ge \int_{E_n}\frac{\varepsilon}{1+\varepsilon}\\
		& = m(E_n)\frac{\varepsilon}{1+\varepsilon}
	\end{align*}
	从而$m(E_n)\to 0$,因此$\{f_n\}_{n=1}^{\infty}$依测度收敛于$f$。
	
	对于充分性,任取$\{f_n\}_{n=1}^{\infty}\sub S(E)$,使得成立$\{f_n\}_{n=1}^{\infty}$依测度收敛于$f\in S(E)$,那么$\{ f_n-f \}_{n=1}^{\infty}$依测度收敛于$0$,而
	$$
	\frac{|f_n-f|}{1+|f_n-f|}\le |f_n-f|
	$$
	那么函数序列
	$$
	\left\{ \frac{|f_n-f|}{1+|f_n-f|}\right\}_{n=1}^{\infty}
	$$
	依测度收敛于$0$。又由于
	$$
	\frac{|f_n-f|}{1+|f_n-f|}\le 1
	$$
	那么由Lebesgue控制收敛定理,成立
	$$
	d(f_n,f)=\int_a^b\frac{|f_n-f|}{1+|f_n-f|}\to0
	$$
	因此$\{f_n\}_{n=1}^{\infty}$依度量$d$收敛于$f$。
\end{proof}

\chapter{}

\begin{proposition}
	对于Hilbert空间$\mathcal{H}$上的有界共轭双线性泛函$f:\mathcal{H}\times \mathcal{H}\to\C$,存在且存在唯一有界线性算子$T:\mathcal{H}\to\mathcal{H}$,使得成立$\|T\|=\|f\|$,且对于任意$x,y\in \mathcal{H}$,成立$f(x,y)=(T(x),y)$。
\end{proposition}

\begin{proof}
	由Frechet-Riesz表现定理,存在保范共轭线性双射$\tau:\mathcal{H}^*\to \mathcal{H}$,使得对于任意$x\in\mathcal{H}$与$\varphi\in \mathcal{H}^*$,成立$\varphi(x)=(x,\tau(\varphi))$。
	
	定义映射
	\begin{align*}
		\pi:\begin{aligned}[t]
			\mathcal{H}&\longrightarrow \mathcal{H}^*\\
			x&\longmapsto \varphi_x,\text{ 其中 }\varphi_x(y)=\overline{f(x,y)}
		\end{aligned}
	\end{align*}
	由于
	\begin{align*}
		&(\pi(x+y))(z)=\overline{f(x+y,z)}=\overline{f(x,z)}+\overline{f(x,z)}=\varphi_{x}(z)+\varphi_{y}(z)=(\pi(x))(z)+(\pi(y))(z)\\
		&(\pi(\lambda x))(y)=\varphi_{\lambda x}(y)=\overline{f(\lambda x,y)}=\overline{\lambda f(x,y)}=\overline{\lambda}\varphi_{x}(y)=\overline{\lambda}(\pi(x))(y)
	\end{align*}
	那么
	$$
	\pi(x+y)=\pi(x)+\pi(y),\qquad 
	\pi(\lambda x)=\overline{\lambda}\pi(x)
	$$
	
	定义映射
	$$
	T=\tau\circ \pi:\mathcal{H}\to \mathcal{H}
	$$
	那么
	$$
	f(x,y)=\overline{\varphi_x(y)}=\overline{(y,\tau(\varphi_x))}=\overline{(y,(\tau\circ \pi)(x))}=\overline{(y,T(x))}=(T(x),y)
	$$
	由于
	\begin{align*}
		&T(x+y)=(\tau\circ \pi)(x+y)=\tau(\pi(x))+\tau(\pi(y))=(\tau\circ \pi)(x)+(\tau\circ \pi)(y)=T(x)+T(y)\\
		&T(\lambda x)=(\tau\circ \pi)(\lambda x)=\tau(\pi(\lambda x))=\tau(\overline{\lambda }\pi(x))=\lambda \tau(\pi(x))=\lambda (\tau\circ \pi)(x)=\lambda T(x)
	\end{align*}
	那么$T$为线性算子。
	
	由Frechet-Riesz表现定理的推论
	\begin{align*}
		\|T\| = & \sup_{\|x\|\le 1}\|T(x)\|\\
		= & \sup_{\|x\|\le 1}\sup_{\|y\|\le 1}|(T(x),y)|\\
		= & \sup_{\|x\|\le 1}\sup_{\|y\|\le 1}|f(x,y)|\\
		= & \|f\|
	\end{align*}
	因此$T$为有界线性算子。
	
	如果存在有界线性算子$S:\mathcal{H}\to\mathcal{H}$,使得对于任意$x,y\in \mathcal{H}$,成立$f(x,y)=(S(x),y)$,那么
	$$
	((T-S)(x),y)=(T(x),y)-(S(x),y)=f(x,y)-f(x,y)=0
	$$
	进而$T=S$,进而$T$为唯一的。
\end{proof}

\chapter{}

\begin{proposition}
	证明:存在保范线性双射
	\begin{align*}
		\tau:\begin{aligned}[t]
			(l^p)^*&\longrightarrow l^q\\
			f&\longmapsto \{a_n\}_{n=1}^{\infty}
		\end{aligned}
	\end{align*}
	使得成立
	\begin{align*}
		f:\begin{aligned}[t]
			l^p&\longrightarrow \C\\
			\{x_n\}_{n=1}^{\infty}&\longmapsto \sum_{n=1}^{\infty}a_nx_n
		\end{aligned}
	\end{align*}
	其中$1\le p<\infty$,且$\frac{1}{p}+\frac{1}{q}=1$。
\end{proposition}

\begin{proof}
	任取$f\in (l^1)^*$。考察$l^1$空间的正规正交基$\{e_n\}_{n=1}^{\infty}$,其中$e_n=\{0,\cdots,0,\underset{\text{第}n\text{个}}{1},0,0,\cdots\}$,对于任意$\{x_n\}_{n=1}^{\infty}\in l^1$,成立
	$$
	\{x_n\}_{n=1}^{\infty}=\sum_{n=1}^{\infty}x_ne_n
	$$
	该级数在$l^1$中收敛,因此
	$$
	f(\{x_n\}_{n=1}^{\infty})=\sum_{n=1}^{\infty}x_nf(e_n)
	$$
	由于$\|e_n\|=1$,那么$|f(e_n)|\le \|f\|$。令
	$$
	a_n=f(e_n),\qquad n\in\N^*
	$$
	那么$\{a_n\}_{n=1}^{\infty}$为由$f$决定的有界数列,进而
	$$
	f(\{x_n\}_{n=1}^{\infty})=\sum_{n=1}^{\infty}a_nx_n
	$$
	
	当$p=1$时,一方面
	$$
	\|f\|\le\sup_{n\in\N^*}|a_n|
	$$
	另一方面
	$$
	|f(\{x_n\}_{n=1}^{\infty})|
	\le \sum_{n=1}^{\infty}|a_n||x_n|
	\le \sup_{n\in\N^*}|a_n|\sum_{n=1}^{\infty}|x_n|
	= \sup_{n\in\N^*}|a_n|\| \{x_n\}_{n=1}^{\infty} \|
	\implies 
	\|f\|\le \sup_{n\in\N^*}|a_n|
	$$
	综合两方面
	$$
	\|f\|=\sup_{n\in\N^*}|a_n|
	$$
	
	当$1< p <\infty$时,令
	$$
	y_{k}^{(n)}=\begin{cases}
		|a_k|^{q-1}\text{sgn}(a_k),\qquad & k\le n\\
		0,\qquad & k>n
	\end{cases}
	$$
	那么
	$$
	f(\{y_k^{(n)}\}_{k=1}^{\infty})=\sum_{k=1}^{n}|a_k|^q
	$$
	而
	$$
	f(\{y_k^{(n)}\}_{k=1}^{\infty})
	\le \|f\|\| \{y_k^{(n)}\}_{k=1}^{\infty} \|_p
	= \|f\|\left(\sum_{k=1}^{\infty}|y_k^{(n)}|^p\right)^{1/p}
	= \|f\|\left(\sum_{k=1}^{n}|a_k|^q\right)^{1/p}
	$$
	因此
	$$
	\left(\sum_{k=1}^{n}|a_k|^q\right)^{1/q}\le \|f\|
	$$
	令$n\to\infty$
	$$
	\left(\sum_{n=1}^{\infty}|a_n|^q\right)^{1/q}\le \|f\|
	$$
	因此$\{a_n\}_{n=1}^{\infty}\in l^q$,且
	$$
	\|f\|\ge \| \{a_n\}_{n=1}^{\infty} \|_q
	$$
	由Hölder不等式
	$$
	|f(\{x_n\}_{n=1}^{\infty})|
	\le \sum_{n=1}^{\infty}|a_n||x_n|
	\le \|\{a_n\}_{n=1}^{\infty}\|_q \|\{x_n\}_{n=1}^{\infty}\|_p \implies
	\|f\|\le \| \{a_n\}_{n=1}^{\infty} \|_q
	$$
	从而
	$$
	\|f\|=\| \{a_n\}_{n=1}^{\infty} \|_q
	$$
\end{proof}
	
\end{document}
