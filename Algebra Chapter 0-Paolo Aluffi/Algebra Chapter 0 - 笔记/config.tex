\documentclass[lang = cn, scheme = chinese, thmcnt = section]{elegantbook}
% elegantbook      设置elegantbook文档类
% lang = cn        设置中文环境
% scheme = chinese 设置标题为中文
% thmcnt = section 设置计数器


%% 1.封面设置

\title{Algebra Chapter 0 - Paolo Aluffi - NoteBook}                % 文档标题

\author{若水}               % 作者

\myemail{ethanmxzhou@163.com} % 邮箱

\homepage{helloethanzhou.github.io} % 主页

\date{\today}               % 日期

\extrainfo{上善若水任方圆}   % 箴言

\logo{PiCreatures_happy.pdf}        % 设置Logo

\cover{阿基米德螺旋曲线.pdf}          % 设置封面图片

% 修改标题页的色带
\definecolor{customcolor}{RGB}{135, 206, 250} 
% 定义一个名为customcolor的颜色,RGB颜色值为(135, 206, 250)

\colorlet{coverlinecolor}{customcolor}     % 将coverlinecolor颜色设置为customcolor颜色

%% 2.目录设置
\setcounter{tocdepth}{3}  % 目录深度为3

%% 3.引入宏包
\usepackage[all]{xy}
\usepackage{bbm, svg, graphicx, float, extpfeil, amsmath, amssymb, mathrsfs, mathalpha, boondox-cal, boondox-calo, hyperref, graphicx, romannum, chemarrow, booktabs, fontspec, ctex}

%% 4.定义命令
\newcommand{\N}{\mathbb{N}}            % 自然数集合
\newcommand{\R}{\mathbb{R}}            % 实数集合
\newcommand{\C}{\mathbb{C}}  		   % 复数集合
\newcommand{\Q}{\mathbb{Q}}            % 有理数集合
\newcommand{\Z}{\mathbb{Z}}            % 整数集合
\newcommand{\F}{\mathbb{F}}
\newcommand{\sub}{\subset}             % 包含
\newcommand{\im}{\text{im }}           % 像
\newcommand{\lang}{\langle}            % 左尖括号
\newcommand{\rang}{\rangle}            % 右尖括号
\newcommand{\dis}{\displaystyle}
\newcommand{\cont}{\text{cont}}
\newcommand{\cha}{\text{char}}
\newcommand{\function}[5]{
	\begin{align*}
		#1:\begin{aligned}[t]
			#2 &\longrightarrow #3\\
			#4 &\longmapsto #5
		\end{aligned}
	\end{align*}
}                                     % 函数

\newcommand{\lhdneq}{%
	\mathrel{\ooalign{$\lneq$\cr\raise.22ex\hbox{$\lhd$}\cr}}} % 真正规子群

\newcommand{\rhdneq}{%
	\mathrel{\ooalign{$\gneq$\cr\raise.22ex\hbox{$\rhd$}\cr}}} % 真正规子群

\newcommand{\upiff}{\mathrel{\rotatebox[origin=c]{90}{$\iff$}}} % 竖着的等价

\newcommand{\Rmnum}[1]{\uppercase\expandafter{\romannumeral #1}}  %定义命令输入大写罗马数字
\newcommand{\rmnum}[1]{\romannumeral #1}  %定义命令输入小写罗马数字

