% 导入配置
% \documentclass[lang = cn, scheme = chinese, thmcnt = section]{elegantbook}
% elegantbook      设置elegantbook文档类
% lang = cn        设置中文环境
% scheme = chinese 设置标题为中文
% thmcnt = section 设置计数器


%% 1.封面设置

\title{Algebra Chapter 0 - Paolo Aluffi - NoteBook}                % 文档标题

\author{若水}               % 作者

\myemail{ethanmxzhou@163.com} % 邮箱

\homepage{helloethanzhou.github.io} % 主页

\date{\today}               % 日期

\extrainfo{上善若水任方圆}   % 箴言

\logo{PiCreatures_happy.pdf}        % 设置Logo

\cover{阿基米德螺旋曲线.pdf}          % 设置封面图片

% 修改标题页的色带
\definecolor{customcolor}{RGB}{135, 206, 250} 
% 定义一个名为customcolor的颜色,RGB颜色值为(135, 206, 250)

\colorlet{coverlinecolor}{customcolor}     % 将coverlinecolor颜色设置为customcolor颜色

%% 2.目录设置
\setcounter{tocdepth}{3}  % 目录深度为3

%% 3.引入宏包
\usepackage[all]{xy}
\usepackage{bbm, svg, graphicx, float, extpfeil, amsmath, amssymb, mathrsfs, mathalpha, boondox-cal, boondox-calo, hyperref, graphicx, romannum, chemarrow, booktabs, fontspec, ctex}

%% 4.定义命令
\newcommand{\N}{\mathbb{N}}            % 自然数集合
\newcommand{\R}{\mathbb{R}}            % 实数集合
\newcommand{\C}{\mathbb{C}}  		   % 复数集合
\newcommand{\Q}{\mathbb{Q}}            % 有理数集合
\newcommand{\Z}{\mathbb{Z}}            % 整数集合
\newcommand{\F}{\mathbb{F}}
\newcommand{\sub}{\subset}             % 包含
\newcommand{\im}{\text{im }}           % 像
\newcommand{\lang}{\langle}            % 左尖括号
\newcommand{\rang}{\rangle}            % 右尖括号
\newcommand{\dis}{\displaystyle}
\newcommand{\cont}{\text{cont}}
\newcommand{\cha}{\text{char}}
\newcommand{\function}[5]{
	\begin{align*}
		#1:\begin{aligned}[t]
			#2 &\longrightarrow #3\\
			#4 &\longmapsto #5
		\end{aligned}
	\end{align*}
}                                     % 函数

\newcommand{\lhdneq}{%
	\mathrel{\ooalign{$\lneq$\cr\raise.22ex\hbox{$\lhd$}\cr}}} % 真正规子群

\newcommand{\rhdneq}{%
	\mathrel{\ooalign{$\gneq$\cr\raise.22ex\hbox{$\rhd$}\cr}}} % 真正规子群

\newcommand{\upiff}{\mathrel{\rotatebox[origin=c]{90}{$\iff$}}} % 竖着的等价

\newcommand{\Rmnum}[1]{\uppercase\expandafter{\romannumeral #1}}  %定义命令输入大写罗马数字
\newcommand{\rmnum}[1]{\romannumeral #1}  %定义命令输入小写罗马数字



% \begin{document}

\chapter{群论的MATLAB函数}

定义名为 \lstinline{Group} 的数据结构,包含 \lstinline{element}、\lstinline{group} 、\lstinline{order} 三个属性。

\begin{enumerate}
	\item \lstinline{element}:群元素。每一行代表一个元素,第$k$行元素的指标定义为$k$。
	\item \lstinline{group}:群运算表。$(i,j)$处的数字表示$i$指标对应元素与$j$指标对应元素作运算后的元素对应的指标。
	\item \lstinline{order}:群的阶。
\end{enumerate}

\begin{table}[htbp]
	\centering
	\caption{群论的MATLAB函数}
	\renewcommand{\arraystretch}{1.5}
	\begin{tabular}{|>{\centering\arraybackslash}m{1.5cm}|>{\centering\arraybackslash}m{5cm}|>{\centering\arraybackslash}m{4cm}|}
		\hline
		类别 & 函数 & 名称 \\
		\hline
		\multirow{6}{*}{特殊群} 
		& \hyperref[cyclicGroup]{cyclicGroup} & 循环群 \\
		& \hyperref[moduloNGroup]{moduloNGroup} & 循环群的单位群 \\
		& \hyperref[dihedralGroup]{dihedralGroup} & 二面体群 \\
		& \hyperref[symmetricGroup]{symmetricGroup} & 对称群 \\
		& \hyperref[alternatingGroup]{alternatingGroup} & 交错群 \\
		& \hyperref[quaternionGroup]{quaternionGroup} & 四元数群 \\
		\hline
		\multirow{10}{*}{群结构} 
		& \hyperref[subgroup]{subgroup} & 子群 \\
		& \hyperref[normalSubgroup]{normalSubgroup} & 正规子群 \\
		& \hyperref[generatedSubgroup]{generatedSubgroup} & 生成子群 \\
		& \hyperref[generatedNormalSubgroup]{generatedNormalSubgroup} & 生成正规子群 \\
		& \hyperref[centerGroup]{centerGroup} & 中心子群 \\
		& \hyperref[characteristicSubgroup]{characteristicSubgroup} & 特征子群 \\
		& \hyperref[commutatorGroup]{commutatorGroup} & 换位子群 \\
		& \hyperref[centralizer]{centralizer} & 中心化子 \\
		& \hyperref[conjugacyClass]{conjugacyClass} & 共轭类 \\
		& \hyperref[derivedGroup]{derivedGroup} & 导群 \\
		\hline
		\multirow{10}{*}{群运算} 
		& \hyperref[groupOperate]{groupOperate} & 群运算 \\
		& \hyperref[inverseMatrix]{inverseMatrix} & 逆元素矩阵 \\
		& \hyperref[orderMatrix]{orderMatrix} & 元素阶矩阵 \\
		& \hyperref[groupHomomorphism]{groupHomomorphism} & 群同态态射 \\
		& \hyperref[groupIsomorphism]{groupIsomorphism} & 群同构态射 \\
		& \hyperref[isCommutativeGroup]{isCommutativeGroup} & 是否为交换群 \\
		& \hyperref[isSimpleGroup]{isSimpleGroup} & 是否为单群 \\
		& \hyperref[isSolvableGroup]{isSolvableGroup} & 是否为可解群 \\
		& \hyperref[directProductOfGroups]{directProductOfGroups} & 直积 \\
		\hline
		\multirow{3}{*}{辅助函数} 
		& \hyperref[EulerFunction]{EulerFunction} & Euler函数 \\
		& \hyperref[fullPermutation]{fullPermutation} & 全排列 \\
		& \hyperref[generateCombinations]{generateCombinations} & 组合矩阵 \\
		\hline
	\end{tabular}
\end{table}

\section{特殊群}

\begin{lstlisting}[language=Matlab, caption={循环群}, label={cyclicGroup}]
	function [element, group, realGroup, order] = cyclicGroup(n)
	
	    % 名称:模n循环群Z/nZ
	    % 输入:循环群阶数n
	    % 输出: 1.群元素element;2.群运算表group和realGroup;3.群阶order
	    % 说明:以element表中元素的位置建立与元素间的双射
	    %       如下出现的数字均代表元素的位置,不代表真实元素
	    %       realGroup为真实元素
	    %       group(i,j)=element(i)+element(j)
	
	    %% 1.群元素
	    element = 0: n-1;
	
	    %% 2.群阶
	    order = n;
	
	    %% 3.群运算表
	    [i, j] = meshgrid(1: n, 1: n);
	    realGroup = mod(i + j - 2, n);
	    group = realGroup + 1;
	
	end
	
\end{lstlisting}

\begin{lstlisting}[language=Matlab, caption={循环群的单位群}, label={moduloNGroup}]
	function [element, group, realGroup, order] = moduloNGroup(n)
	
	    % 名称:模n单位群(Z/nZ)*
	    % 输入:模n单位群阶数n
	    % 输出: 1.群元素element;2.群运算表group和realGroup;3.群阶order
	    % 说明:以element表中元素的位置建立与元素间的双射
	    %       如下出现的数字均代表元素的位置,不代表真实元素
	    %       realGroup为真实元素
	    %       group(i,j)=element(i)*element(j)
	
	    %% 1.群元素和群阶
	    [order, element] = EulerFunction(n);
	
	    %% 3.群运算表
	    realGroup = mod(element' * element, n);
	    group = zeros(order, order);
	    for i = 1: order
	        for j = 1: order
	            for k = 1: order
	                if element(k) == realGroup(i, j)
	                    indice = k;
	                    break
	                end
	            end
	            % 赋值
	            group(i, j) = indice;
	        end
	    end
	
	end
	
\end{lstlisting}

\begin{lstlisting}[language=Matlab, caption={二面体群}, label={dihedralGroup}]
	function [element, group, order] = dihedralGroup(n)
	
	    % 名称:n阶二面体群D_2n
	    % 输入:二面体群阶数n
	    % 输出: 1.群元素element;2.群运算表group;3.群阶order
	    % 说明:以element表中元素的位置建立与元素间的双射
	    %       如下出现的数字均代表元素的位置,不代表真实元素
	    %       element为2n行2列矩阵
	    %       其中element(k)代表第k个元素
	    %       element的行向量(i,j)代表映射sigma^i tau^j
	    %       sigma^n = tau^2 = sigma tau sigma tau = 1
	    %       group(i,j)=element(i) \circ element(j)
	
	    %% 1.群元素
	    element = [(0: n-1)', zeros(n, 1); (0: n-1)', ones(n, 1)];
	
	    %% 2.群阶
	    order = 2 * n;
	    %% 3.群运算表
	
	    % 初始化
	    group = zeros(order, order);
	    for i = 1: order
	        for j = 1: order
	            
	            % 利用sigma^n = tau^2 = sigma tau sigma tau = 1降次
	            temp = [element(i, :), element(j, :)];
	
	            % tau^2 = sigma^n = 1降次
	            if temp(2) == 0
	                map = [mod(temp(1)+temp(3), n), temp(4)];
	            elseif temp(3) == 0
	                map = [temp(1), mod(temp(2)+temp(4), 2)];
	            else
	                map = [mod(n+temp(1)-temp(3), n), mod(1+temp(4), 2)];
	            end
	
	            % 找到复合映射的编号
	            for k = 1: order
	                if element(k, :) == map
	                    indice = k;
	                    break
	                end
	            end
	            % 赋值
	            group(i, j) = indice;
	
	        end
	    end
	
	end
	
\end{lstlisting}

\begin{lstlisting}[language=Matlab, caption={对称群}, label={symmetricGroup}]
	function [element, group, order] = symmetricGroup(n)
	
	    % 名称:n阶对称群S_n
	    % 输入:对称群阶数n
	    % 输出: 1.群元素element;2.群运算表group;3.群阶order
	    % 说明:以element表中元素的位置建立与元素间的双射
	    %       如下出现的数字均代表元素的位置,不代表真实元素
	    %       element为n!行n列矩阵
	    %       其中element(k)代表第k个元素,即第k个双射
	    %       第j列代表j在该行代表映射下的像
	    %       group(i,j)=element(j) \circ element(i)
	
	    %% 1.群元素
	    element = fullPermutation(n);
	    element = element(end: -1: 1, :);
	
	    %% 2.群阶
	    order = factorial(n);
	
	    %% 3.群运算表
	    
	    % 初始化
	    group = zeros(order, order);
	    for i = 1: order
	        for j = 1: order
	            % 初始化
	            permutation = zeros(1, n);
	            permutation_i = element(i, :);
	            permutation_j = element(j, :);
	            % 计算复合映射
	            for k = 1: n
	                 permutation(k) = permutation_j(permutation_i(k));
	            end
	            % 找到复合映射的编号
	            for k = 1: order
	                if element(k, :) == permutation
	                    indice = k;
	                    break
	                end
	            end
	            % 赋值
	            group(i, j) = indice;
	        end
	    end
	
	end
	
\end{lstlisting}

\begin{lstlisting}[language=Matlab, caption={交错群}, label={alternatingGroup}]
	function [element, group, order] = alternatingGroup(n)
	
	    % 名称:n阶交错群A_n
	    % 输入:交错群阶数n
	    % 输出: 1.群元素element;2.群运算表group;3.群阶order
	    % 说明:以element表中元素的位置建立与元素间的双射
	    %       如下出现的数字均代表元素的位置,不代表真实元素
	    %       element为n!行n列矩阵
	    %       其中element(k)代表第k个元素,即第k个双射
	    %       第j列代表j在该行代表映射下的像
	    %       group(i,j)=element(j) \circ element(i)
	
	    %% 1.群元素
	    permutations = perms(1: n); % 生成所有n元置换的排列
	
	    element = [];
	    for i = 1:size(permutations, 1)
	        perm = permutations(i, :);
	        sign = 1; % 初始化符号为正号
	        
	        for j = 1:n
	            for k = (j+1):n
	                if perm(j) > perm(k)
	                    sign = -sign; % 交换时改变符号
	                end
	            end
	        end
	        
	        if sign == 1 % 符号为正号表示是偶置换
	            element = [element; perm];
	        end
	    end
	    element = element(end: -1: 1, :);
	
	    %% 2.群阶
	    order = factorial(n) / 2;
	
	    %% 3.群运算表
	    
	    % 初始化
	    group = zeros(order, order);
	    for i = 1: order
	        for j = 1: order
	            % 初始化
	            permutation = zeros(1, n);
	            permutation_i = element(i, :);
	            permutation_j = element(j, :);
	            % 计算复合映射
	            for k = 1: n
	                 permutation(k) = permutation_j(permutation_i(k));
	            end
	            % 找到复合映射的编号
	            for k = 1: order
	                if element(k, :) == permutation
	                    indice = k;
	                    break
	                end
	            end
	            % 赋值
	            group(i, j) = indice;
	        end
	    end
	
	end
	
\end{lstlisting}

\begin{lstlisting}[language=Matlab, caption={四元数群}, label={quaternionGroup}]
	function [element, group, realGroup, order] = quaternionGroup
	
	    % 名称:四元数群
	    % 输出: 1.群元素element;2.群运算表group和realGroup;3.群阶order
	    % 说明:以element表中元素的位置建立与元素间的双射
	    %       如下出现的数字均代表元素的位置,不代表真实元素
	    %       realGroup为真实元素
	    %       group(i,j)=element(i)*element(j)
	
	    %% 函数
	
	    order = 8;
	
	    element = {
	        '1', '-1', ...
	        'i', '-i', ...
	        'j', '-j', ...
	        'k', '-k'};
	
	    realGroup = {
	        '1', '-1', 'i', '-i', 'j', '-j', 'k', '-k'; ...
	        '-1', '1', '-i', 'i', '-j', 'j', '-k', 'k'; ...
	        'i', '-i', '-1', '1', 'k', '-k', '-j', 'j'; ...
	        '-i', 'i', '1', '-1', '-k', 'k', 'j', '-j'; ...
	        'j', '-j', '-k', 'k', '-1', '1', 'i', '-i'; ...
	        '-j', 'j', 'k', '-k', '1', '-1', '-i', 'i'; ...
	        'k', '-k', 'j', '-j', '-i', 'i', '-1', '1'; ...
	        '-k', 'k', '-j', 'j', 'i', '-i', '1', '-1'};
	
	    group = zeros(order, order);
	    for i = 1: order
	        for j = 1: order
	            for n = 1: order
	                if isequal(realGroup(i, j), element(n))
	                    group(i, j) = n;
	                end
	            end
	        end
	    end
	
	end
\end{lstlisting}

\section{群结构}

\begin{lstlisting}[language=Matlab, caption={子群}, label={subgroup}]
	function subgroupMatrix = subgroup(group)
	
	    % 名称:子群
	    % 输入:群运算表group
	    % 输出:所有子群
	    % 说明:subgroupMatrix每一行代表一个子群的元素的编号
	
	    %% 求解子群的阶
	    order = size(group, 1);
	    factors = [];
	    loc = 1;
	    for fac = 1: order
		    if mod(order, fac) == 0
	            factors(loc) = fac;
			    loc = loc + 1;
		    end
	    end
	
	    %% 求解子群
	    invMatrix = inverseMatrix(group);
	    subgroupMatrix = [];
	    for n = factors
	        combinations = generateCombinations(n, order);
	        for com = combinations'
	            judge = 1;
	            for i = com'
	                for j = com'
	                    invj = invMatrix(j);
	                    if ~ismember(group(i, invj), com)
	                        judge = 0;
	                        break
	                    end
	                end
	                if judge == 0
	                    break
	                end
	            end
	            if judge == 1 && i == com(end) && j == com(end)
	                subgroupMatrix = [subgroupMatrix; com', zeros(1, order - size(com', 2))];
	            end
	        end
	    end
	
	end
	
\end{lstlisting}

\begin{lstlisting}[language=Matlab, caption={正规子群}, label={normalSubgroup}]
	function normalSubgroupMatrix = normalSubgroup(group)
	
	    % 名称:正规子群
	    % 输入:群运算表group
	    % 输出:所有正规子群
	    % 说明:normalSubgroupMatrix每一行代表一个正规子群的元素的编号
	
	    %% 函数1
	    order = size(group, 1);
	    subgroupMatrix = subgroup(group);
	    subgroupNumber = size(subgroupMatrix, 1);
	    invmatrix = inverseMatrix(group);
	    
	    normalSubgroupMatrix = [];
	    for N = 1: subgroupNumber
	
	        judge = 1;
	        normalSubgroup = subgroupMatrix(N, :);
	
	        for n = 1: order
	
	            if normalSubgroup(n) == 0
	                break
	            end
	
	            for g = 1: order
	                temp = group(group(g, normalSubgroup(n)), invmatrix(g));
	                if ~ismember(temp, normalSubgroup)
	                    judge = 0;
	                    break
	                end
	            end
	
	            if judge == 0
	                break
	            end
	        end
	        
	        if judge == 1
	            normalSubgroupMatrix = [normalSubgroupMatrix; normalSubgroup];
	        end
	    end
	
	    %% 函数2
	    % order = size(group, 1);
	    % subgroupMatrix = subgroup(group);
	    % subgroupNumber = size(subgroupMatrix, 1);
	    % 
	    % normalSubgroupMatrix = [];
	    % for N = 1: subgroupNumber
	    % 
	    %     judge = 1;
	    %     normalSubgroup = subgroupMatrix(N, :);
	    %     normalSubgroup(normalSubgroup == 0) = [];
	    %     normalSubgroupNumber = size(normalSubgroup, 2);
	    % 
	    %     for g = 1: order
	    %         gN = zeros(normalSubgroupNumber, 1);
	    %         Ng = zeros(normalSubgroupNumber, 1);
	    %         for n = 1: normalSubgroupNumber
	    %             gN(n) = group(g, normalSubgroup(n));
	    %             Ng(n) = group(normalSubgroup(n), g);
	    %         end
	    %         gN = unique(sort(gN));
	    %         Ng = unique(sort(Ng));
	    %         if ~isequal(gN, Ng)
	    %             judge = 0;
	    %             break
	    %         end
	    %     end
	    %     if judge == 1
	    %         normalSubgroupMatrix = [normalSubgroupMatrix; normalSubgroup, zeros(1, order-size(normalSubgroup, 2))];
	    %     end
	    % 
	    % end
	
	end
	
\end{lstlisting}

\begin{lstlisting}[language=Matlab, caption={生成子群}, label={generatedSubgroup}]
	function generatedSubgroupMatrix = generatedSubgroup(group, subset)
	
	    % 名称:生成子群
	    % 输入:群运算表group,子集subset
	    % 输出:生成子群元素的编号
	
	    %% 函数
	    subgroupMatrix = subgroup(group);
	    % 初始化一个空矩阵来存储符合条件的子集
	    supsetgroupMatrix = [];
	    % 逐行比较subset和subgroupMatrix的每一行
	    for n = 1: size(subgroupMatrix, 1)
	        % 如果subset的所有元素都在subgroupMatrix的这一行中
	        if all(ismember(subset, subgroupMatrix(n, :)))
	            % 将这一行添加到supsetgroupMatrix中
	            supsetgroupMatrix = [supsetgroupMatrix; subgroupMatrix(n, :)];
	        end
	    end
	    
	    % 找出supsetgroupMatrix每一行的公共元素
	    generatedSubgroupMatrix = supsetgroupMatrix(1, :);
	    for n = 1: size(supsetgroupMatrix, 1)
	        generatedSubgroupMatrix = intersect(generatedSubgroupMatrix, supsetgroupMatrix(n, :));
	    end
	    generatedSubgroupMatrix = transpose(nonzeros(generatedSubgroupMatrix));
	
	end
\end{lstlisting}

\begin{lstlisting}[language=Matlab, caption={生成正规子群}, label={generatedNormalSubgroup}]
	function generatedNormalSubgroupMatrix = generatedNormalSubgroup(group, subset)
	
	    % 名称:生成正规子群
	    % 输入:群运算表group,子集subset
	    % 输出:生成子群元素的编号
	
	    %% 函数
	    normalSubgroupMatrix = normalSubgroup(group);
	    % 初始化一个空矩阵来存储符合条件的子集
	    supsetnormalGroupMatrix = [];
	    % 逐行比较subset和subgroupMatrix的每一行
	    for n = 1: size(normalSubgroupMatrix, 1)
	        % 如果subset的所有元素都在subgroupMatrix的这一行中
	        if all(ismember(subset, normalSubgroupMatrix(n, :)))
	            % 将这一行添加到supsetgroupMatrix中
	            supsetnormalGroupMatrix = [supsetnormalGroupMatrix; normalSubgroupMatrix(n, :)];
	        end
	    end
	    
	    % 找出supsetgroupMatrix每一行的公共元素
	    generatedNormalSubgroupMatrix = supsetnormalGroupMatrix(1, :);
	    for n = 1: size(supsetnormalGroupMatrix, 1)
	        generatedNormalSubgroupMatrix = intersect(generatedNormalSubgroupMatrix, supsetnormalGroupMatrix(n, :));
	    end
	    generatedNormalSubgroupMatrix = transpose(nonzeros(generatedNormalSubgroupMatrix));
	
	end
\end{lstlisting}

\begin{lstlisting}[language=Matlab, caption={中心子群}, label={centerGroup}]
	function centreGroupMatrix = centerGroup(group)
	
	    % 名称:中心子群
	    % 输入:群运算表group
	    % 输出:中心子群元素的编号
	
	    %% 函数
	    centreGroupMatrix = [];
	    order = size(group, 1);
	    for n = 1: order
	        judge = 1;
	        for k = 1: order
	            if group(n, k) ~= group(k, n)
	                judge = 0;
	            end
	        end
	        if judge == 1
	            centreGroupMatrix = [centreGroupMatrix, n];
	        end
	    end
	
	end
	
\end{lstlisting}

\begin{lstlisting}[language=Matlab, caption={特征子群}, label={characteristicSubgroup}]
	function characteristicSubgroupMatrix = characteristicSubgroup(group)
	
	    % 名称:特征子群
	    % 输入:群运算表group
	    % 输出:所有特征子群
	    % 说明:characteristicSubgroupMatrix每一行代表特征正规子群的元素的编号
	
	    %% 函数
	    subgroupMatrix = normalSubgroup(group);
	    mapMatrix = groupIsomorphism(group, group);
	    order = size(group, 1);
	    subgroupNumber = size(subgroupMatrix, 1);
	    mapNumber = size(mapMatrix, 1);
	    characteristicSubgroupMatrix = [];
	    for i = 1: subgroupNumber
	        judge = 1;
	        subgroups = subgroupMatrix(i, :);
	        for j = 1: mapNumber
	            im = [];
	            map = mapMatrix(j, :);
	            for n = 1: order
	                if subgroups(n) ~= 0
	                    im = [im map(subgroups(n))];
	                end
	            end
	            if ~all(ismember(im, subgroups))
	                judge = 0;
	                break
	            end
	        end
	        if judge == 1
	            characteristicSubgroupMatrix = [characteristicSubgroupMatrix; subgroups];
	        end
	    end
	
	end
	
\end{lstlisting}

\begin{lstlisting}[language=Matlab, caption={换位子群}, label={commutatorGroup}]
	function commutatorGroupMatrix = commutatorGroup(group, elementMatrix)
	
	    % 名称:换位子群
	    % 输入:群运算表group,群元素编号(可选)
	    % 输出:换位子群元素的编号
	
	    %% 函数
	    matrix = inverseMatrix(group);
	    narginchk(1, 2); % 检查输入参数数量,允许1到2个参数
	    if nargin == 1
	        order = size(group, 1);
	        commutatorGroupMatrix = zeros(1, order^2);
	        for i = 1: order
	            for j = 1: order
	                commutatorGroupMatrix((i-1) * order + j) = group(group(group(i, j), matrix(i)), matrix(j));
	            end
	        end
	    else
	        commutatorGroupMatrix = [];
	        for i = elementMatrix
	            for j = elementMatrix
	                commutatorGroupMatrix = [commutatorGroupMatrix, group(group(group(i, j), matrix(i)), matrix(j))];
	            end
	        end
	    end
	    commutatorGroupMatrix = sort(unique(commutatorGroupMatrix));
	    % commutatorGroupMatrix = generatedNormalSubgroup(group, commutatorGroupMatrix);
	
	end
	
\end{lstlisting}

\begin{lstlisting}[language=Matlab, caption={中心化子}, label={centralizer}]
	function matrix = centralizer(g, group)
	
	    % 名称:中心化子
	    % 输入:元素编号g,群运算表group
	    % 输出:编号为g的元素的共轭类的编号
	
	    %% 函数
	    order = size(group, 1);
	    matrix = [];
	    for n = 1: order
	        if group(g, n) == group(n, g)
	            matrix = [matrix, n];
	        end
	    end
	
	end
	
\end{lstlisting}

\begin{lstlisting}[language=Matlab, caption={共轭类}, label={conjugacyClass}]
	function class = conjugacyClass(g, group)
	
	    % 名称:共轭类
	    % 输入:元素编号g,群运算表group
	    % 输出:编号为g的元素的共轭类的编号
	
	    %% 函数
	    order = size(group, 1);
	    matrix = inverseMatrix(group);
	    class = [];
	    for n = 1: order
	        element = group(group(n, g), matrix(n));
	        class = [class, element];
	    end
	    class = sort(unique(class));
	
	end
	
\end{lstlisting}

\begin{lstlisting}[language=Matlab, caption={导群}, label={derivedGroup}]
	function derivedGroupMatrix = derivedGroup(group, degree)
	
	    % 名称:导群
	    % 输入:群运算表group和导群阶
	    % 输出:导群元素的编号
	
	    %% 函数
	    order = size(group, 1);
	    derivedGroupMatrix = 1: order; % 初始化
	    for n = 1: degree % 迭代次数
	        derivedGroupMatrix = commutatorGroup(group, derivedGroupMatrix);
	    end
	    
	end
	
\end{lstlisting}

\section{群运算}

\begin{lstlisting}[language=Matlab, caption={群运算}, label={groupOperate}]
	function result = groupOperate(Group, g, h)
	    
	    % 名称:群运算
	    % 输入:群Group,以及群元素编号g和h
	    % 输出:群元素Group(g)*Group(h)的编号
	    % 关于群Group:
	    % 给定群Group中元素的编号:1, ..., |Group|
	    % Group(n)表示群G中编号为n的元素
	    % Group为|G|行|G|列矩阵,其中Group(i,j)=Group(i)*Group(j)
	
	    %% 计算群运算
	    result = Group(g, h);
	
	end
	
\end{lstlisting}

\begin{lstlisting}[language=Matlab, caption={逆元素矩阵}, label={inverseMatrix}]
	function matrix = inverseMatrix(group)
	
	    % 名称:逆元素矩阵
	    % 输入:群运算表group
	    % 输出:逆元素编号矩阵matrix
	    % 说明:matrix(k)表示编号为k的元素的逆元素的编号
	
	    %% 函数
	    order = size(group, 1);
	    matrix = zeros(1, order);
	    for i = 1: order
	        for j = 1: order
	            if group(i, j) == 1
	                matrix(i) = j;
	                break
	            end
	        end
	    end
	
	end
	
\end{lstlisting}

\begin{lstlisting}[language=Matlab, caption={元素阶矩阵}, label={orderMatrix}]
	function matrix = orderMatrix(group)
	
	    % 名称:元素阶矩阵
	    % 输入:群运算表group
	    % 输出:元素阶矩阵matrix
	    % 说明:matrix(k)表示编号为k的元素的阶
	
	    %% 函数
	    order = size(group, 1);
	    matrix = zeros(1, order);
	    for loc = 1: order
	        if loc == 1
	            matrix(loc) = 1;
	        else
	            element = loc;
	            for n = 2: order
	                element = group(element, loc);
	                if element == 1
	                    matrix(loc) = n;
	                    break
	                end
	            end
	        end
	    end
	
	end
	
\end{lstlisting}

\begin{lstlisting}[language=Matlab, caption={群同态态射}, label={groupHomomorphism}]
	function mapMatrix = groupHomomorphism(G, H)
	
	    % 名称:群同态态射
	    % 输入:群G和H
	    % 输出:G-->H的同态态射
	    % 其中每一行代表一个态射,第j列代表j在该态射下的像
	
	    %% 函数
	    
	    % 定义群的阶
	    m = size(G, 1);
	    n = size(H, 1);
	
	    % 初始化映射矩阵
	    mapMatrix = zeros(n^m, m);
	    
	    % 生成所有可能的映射
	    for i = 1: n^m
	        temp = i - 1;
	        for j = m:-1:1
	            quotient = floor(temp / n);
	            remainder = mod(temp, n);
	            mapMatrix(i, j) = remainder + 1;
	            temp = quotient;
	        end
	    end
	
	    % 初始化元素矩阵
	    elementMatrix = [];
	
	    % 生成元素矩阵
	    for k = 1: m
	        elementMatrix = [elementMatrix, [k*ones(1, m); 1: m]];
	    end
	
	    % 筛选同态态射
	    for k = 1: n^m
	        for element = elementMatrix
	            if mapMatrix(k, groupOperate(G, element(1), element(2))) ~= groupOperate(H, mapMatrix(k, element(1)), mapMatrix(k, element(2)))
	                mapMatrix(k, :) = zeros(1, m);
	                break
	            end
	        end
	    end
	
	    % 删除非同态态射
	    mapMatrix = mapMatrix(any(mapMatrix, 2), :);
	
	end
	
\end{lstlisting}

\begin{lstlisting}[language=Matlab, caption={群同构态射}, label={groupIsomorphism}]
	function mapMatrix = groupIsomorphism(G, H)
	
	    % 名称:群同构态射
	    % 输入:群G和H
	    % 输出:G-->H的同构态射
	    % 其中每一行代表一个态射,第j列代表j在该态射下的像
	
	    %% 函数
	
	    % 定义群的阶
	    m = size(G, 1);
	    n = size(H, 1);
	
	    if m == n
	
	        % 计算G-->H的同态态射
	        mapMatrix = groupHomomorphism(G, H);
	        
	        % 定义目标排列
	        permutation = 1: m;
	        
	        % 初始化一个逻辑向量,用于标记符合条件的行
	        is_permutation = false(size(mapMatrix, 1), 1);
	        
	        % 遍历每一行
	        for k = 1: size(mapMatrix, 1)
	            % 判断是否是目标排列
	            if isequal(sort(mapMatrix(k, :)), permutation)
	                is_permutation(k) = true;
	            end
	        end
	        
	        % 从矩阵中选择符合条件的行
	        mapMatrix = mapMatrix(is_permutation, :);
	
	    else
	        mapMatrix = [];
	    end
	
	end
	
\end{lstlisting}

\begin{lstlisting}[language=Matlab, caption={是否为交换群}, label={isCommutativeGroup}]
	function judge = isCommutativeGroup(group)
	
	    % 名称:判断是否为交换群
	    % 输入:群运算表
	    % 输出:若可交换,则输出1;否则,输出0
	
	    judge = all(all(group == group'));
	
	end
	
\end{lstlisting}

\begin{lstlisting}[language=Matlab, caption={是否为单群}, label={isSimpleGroup}]
	function judge = isSimpleGroup(group)
	
	    % 名称:判断是否为单群
	    % 输入:群运算表
	    % 输出:若为单群,则输出1;否则,输出0
	    
	    %% 函数
	    if size(group, 1) == 1
	        judge = 1;
	    else
	        normalSubgroupMatrix = normalSubgroup(group);
	        judge = size(normalSubgroupMatrix, 1) == 2;
	    end
	
	end
	
\end{lstlisting}

\begin{lstlisting}[language=Matlab, caption={是否为可解群}, label={isSolvableGroup}]
	function judge = isSolvableGroup(group)
	
	    % 名称:判断是否为可解群
	    % 输入:群运算表
	    % 输出:若为可解群,则输出1;否则,输出0
	    
	    %% 函数
	    order = size(group, 1);
	    elementMatrix = 1: order;
	    m = 1;
	    n = 0;
	    while m ~= n
	        m = size(elementMatrix, 2);
	        elementMatrix = commutatorGroup(group, elementMatrix);
	        n = size(elementMatrix, 2);
	    end
	    if m == 1
	        judge = 1;
	    else
	        judge = 0;
	    end
	
	end
	
\end{lstlisting}

\begin{lstlisting}[language=Matlab, caption={群的直积}, label={directProductOfGroups}]
	function [element, group, order] = directProductOfGroups(G, H)
	
	    % 名称:群的直积
	    % 输入:群G和H
	    % 输出:1.群元素element;2.群运算表group;3.群阶order
	    % 说明:以element表中元素的位置建立与元素间的双射
	    %       如下出现的数字均代表元素的位置,不代表真实元素
	    %       element为order行2列矩阵
	    %       其中element(k)代表第k个元素
	    %       element的行向量(i,j)代表G的第i个元素与H的第j个元素
	    %       group(i,j)=element(i) * element(j)
	
	
	    %% 1.群阶
	    m = size(G, 1);
	    n = size(H, 1);
	    order = m * n;
	
	    %% 2.群元素
	    element = zeros(order, 2);
	    for k = 1: order
	        element(k, 1) = ceil(k / n);
	        element(k, 2) = k - n * (ceil(k / n)-1);
	    end
	
	    %% 3.群运算表
	    group = zeros(order, order);
	
	    for i = 1: order
	        for j = 1: order
	            group(i, j) = ...
	                n * (G(element(i, 1), element(j, 1)) - 1) ...
	                + H(element(i, 2), element(j, 2));
	        end
	    end
	
	end
	
\end{lstlisting}

\section{辅助函数}

\begin{lstlisting}[language=Matlab, caption={Euler函数}, label={EulerFunction}]
	function [N, Matrix] = EulerFunction(n)
	
	    % 生成 1 到 n-1 的矩阵
	    numbers = 1: n-1;
	    
	    % 计算每个数与 n 的最大公约数
	    gcdMatrix = gcd(numbers, n);
	    
	    % 使用逻辑索引找到与 n 互素的数
	    coprime = numbers(gcdMatrix == 1);
	    
	    % 将结果转换为矩阵
	    Matrix = reshape(coprime, 1, []);
	
	    % 计算元素个数
	    N = size(Matrix, 2);
	
	end
	
\end{lstlisting}

\begin{lstlisting}[language=Matlab, caption={全排列}, label={fullPermutation}]
	function permutation = fullPermutation(n)
	
	    % 名称:全排列
	    % 输入:n
	    % 输出:n!行n列矩阵,每一行为一个全排列
	
	    %%
	
	    if n == 1
	        permutation = 1;
	    else
	        sub_permutations = fullPermutation(n - 1);
	        m = size(sub_permutations, 1);
	        permutation = zeros(m * n, n);
	        for i = 1: m
	            for j = 1: n
	                permutation((i - 1) * n + j, :) = [sub_permutations(i, 1:j-1), n, sub_permutations(i, j:end)];
	            end
	        end
	    end
	
	end
	
\end{lstlisting}

\begin{lstlisting}[language=Matlab, caption={组合矩阵}, label={generateCombinations}]
	function combinations = generateCombinations(k, n)
	    % 生成组合矩阵的函数
	    % 输入:组合大小k,元素总数n
	    % 输出:一个C_n^k行,k列矩阵,包含所有可能的组合
	    
	    %% 函数1
	    % 生成所有可能的组合
	    combinations = cell2mat(arrayfun(@(x) nchoosek(1:n, x), k, 'UniformOutput', false)');
	
	    %% 函数2
	
	    % 初始化组合矩阵为空
	    combinations = [];
	    % 初始化当前组合为空
	    currentCombination = [];
	    % 调用递归函数来生成组合
	    generate(1, k, n, currentCombination);
	    
	    % 递归函数,生成所有可能的组合
	    % 输入参数:
	    %   - start: 当前迭代的起始元素
	    %   - k: 剩余要选择的元素数量
	    %   - n: 候选元素的总数
	    %   - currentCombination: 当前的组合
	    function generate(start, k, n, currentCombination)
	        % 当剩余要选择的元素数量为0时,表示已经生成一个组合
	        if k == 0
	            % 将当前组合添加到组合矩阵中
	            combinations = [combinations; currentCombination];
	            return;
	        end
	        
	        % 遍历候选元素,生成组合
	        for i = start:n
	            % 递归调用generate函数,继续生成组合
	            generate(i+1, k-1, n, [currentCombination, i]);
	        end
	    end
	    
	end
	
\end{lstlisting}

% \end{document}