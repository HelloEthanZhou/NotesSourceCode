\documentclass[lang = cn, % 设置中文环境
scheme = chinese          % 设置标题为中文
% thmcnt = section,         % 设置计数器
]
{elegantbook}             % 设置elegantbook文档类


%% 1.封面设置

\title{Lectures in Harmonic Analysis-Thomas H. Wolf-Notebook}                % 文档标题

\author{若水}               % 作者

\myemail{ethanmxzhou@163.com} % 邮箱

\homepage{helloethanzhou.github.io} % 主页

\date{\today}               % 日期

\extrainfo{上善若水任方圆}   % 箴言

\logo{PiCreatures_happy.pdf}        % 设置Logo

\cover{阿基米德螺旋曲线.pdf}          % 设置封面图片

% 修改标题页的色带
\definecolor{customcolor}{RGB}{135, 206, 250} 
% 定义一个名为customcolor的颜色,RGB颜色值为(135, 206, 250)

\colorlet{coverlinecolor}{customcolor}     % 将coverlinecolor颜色设置为customcolor颜色

%% 2.目录设置
\setcounter{tocdepth}{3}  % 目录深度为3

%% 3.引入宏包
\usepackage[all]{xy}
\usepackage{bbm, svg, graphicx, float, extpfeil, amsmath, amssymb, mathrsfs, mathalpha, hyperref, centernot, physics}


%% 4.定义命令
\newcommand{\N}{\mathbb{N}}            % 自然数集合
\newcommand{\R}{\mathbb{R}}            % 实数集合
\newcommand{\C}{\mathbb{C}}  		   % 复数集合
\newcommand{\Q}{\mathbb{Q}}            % 有理数集合
\newcommand{\Z}{\mathbb{Z}}            % 整数集合
\newcommand{\sub}{\subset}             % 包含
\newcommand{\im}{\text{im }}           % 像
\newcommand{\lang}{\langle}            % 左尖括号
\newcommand{\rang}{\rangle}            % 右尖括号
\newcommand{\bs}{\boldsymbol}          % 向量加黑
\newcommand{\ee}[1]{\mathrm{e}^{#1}}           % 微分d
\newcommand{\dis}{\displaystyle}
\newcommand{\pll}{\kern 0.56em/\kern -0.8em /\kern 0.56em} % 平行
\newcommand{\function}[5]{
\begin{align*}
	#1:\begin{aligned}[t]
		#2 &\longrightarrow #3\\
		#4 &\longmapsto #5
	\end{aligned}
\end{align*}
}                                     % 函数

\newcommand{\lhdneq}{%
\mathrel{\ooalign{$\lneq$\cr\raise.22ex\hbox{$\lhd$}\cr}}} % 真正规子群

\newcommand{\rhdneq}{%
\mathrel{\ooalign{$\gneq$\cr\raise.22ex\hbox{$\rhd$}\cr}}} % 真正规子群

%% 5.参考文献

\addbibresource[location=local]{reference.bib} % 添加本地的参考文献文件reference.bib

\begin{document}

\maketitle       % 创建标题页

\frontmatter     % 开始前言部分

\chapter*{致谢}

\markboth{致谢}{致谢}

\vspace*{\fill}
	\begin{center}
		
		\large{感谢 \textbf{ 勇敢的 } 自己}
		
	\end{center}
\vspace*{\fill}

\tableofcontents % 创建目录

\mainmatter      % 开始正文部分

\chapter{$L^1$ Fourier 变换}

\begin{definition}{$L^1$ Fourier 变换 $L^1$ Fourier transform}
	\begin{enumerate}
		\item 对于$f\in L^1(\R^n)$,定义其 Fourier 变换为
		\begin{align*}
			\widehat{f}:\begin{aligned}[t]
				\R^n &\longrightarrow \C\\
				\xi &\longmapsto \int\ee{-2\pi i x \cdot \xi}f(x)\dd x
			\end{aligned}
		\end{align*}
		其中$x \cdot \xi$表示内积。
		\item 更一般的,对于$\R^n$上的赋有范数
		\[
		\|\mu\|=|\mu|(\R^n)
		\]
		的有限复值测度空间$M(\R^n)$,其中$|\mu|$为总变差,通过$f\to\mu,\dd\mu=f\dd x$,$L^1(\R^n)$包含在$M(\R^n)$中,此时推广 Fourier 变换为
		\[
		\widehat{\mu}(\xi)=\int\ee{-2\pi i x \cdot \xi}\dd\mu(x)
		\]
	\end{enumerate}
\end{definition}

\begin{example}
	对于$a\in\R^n$与$E\sub\R^n$,定义 Dirac 测度
	\[
	\delta_a(E)=\begin{cases}
		1,\qquad & a\in E\\
		0,\qquad & a\notin E
	\end{cases}
	\]
	那么
	\[
	\widehat{\delta_a}(\xi)=\ee{-2\pi i a\cdot \xi}
	\]
\end{example}

\begin{proof}
	\[
	\widehat{\delta_a}(\xi)
	=\int\ee{-2\pi i x \cdot \xi}\dd \delta_a(x)
	=\ee{-2\pi i a\cdot \xi}
	\]
\end{proof}

\begin{example}
	令$\Gamma(x)=\ee{-\pi|x|^2}$,则
	\[
	\widehat{\Gamma}(\xi)=\ee{-\pi|\xi|^2}
	\]
\end{example}

\begin{proof}
	由于
	\[
	\widehat{\Gamma}(\xi)
	=\int\ee{-2\pi i x \cdot \xi}\Gamma(x)\dd x
	=\int\ee{-2\pi i x \cdot \xi}\ee{-\pi|x|^2}\dd x
	\]
	注意到该积分的变量仅为一维变量,因此不妨考虑$n=1$。由 Gauss 积分$\dis \int_{-\infty}^{+\infty}\ee{-\pi x^2}\dd x=1$,则
	\[
	\int_{-\infty}^{+\infty}\ee{-2\pi i x \cdot \xi}\ee{-\pi|x|^2}\dd x
	=\ee{-\pi|\xi|^2}
	\]
\end{proof}

$L^1$ Fourier 变换存在一些基本的估计,我们罗列如下。

\begin{proposition}
	如果$\mu\in M(\R^n)$,那么$\widehat{\mu}$为有界函数,且
	\[
	\|\widehat{\mu}\|_{\infty}
	\le\|\mu\|_{M(\R^n)}
	\]
\end{proposition}

\begin{proof}
	对于任意$\xi\in\R^n$,由于
	\[
	|\widehat{\mu}(\xi)|
	=\left| \int\ee{-2\pi i x \cdot \xi}\dd\mu(x) \right|
	\le\int|\ee{-2\pi i x \cdot \xi}|\dd|\mu|(x)
	=\|\mu\|_{M(\R^n)}
	\]
	因此
	\[
	\|\widehat{\mu}\|_{\infty}
	=\sup_{\xi\in\R^n}|\widehat{\mu}(\xi)|
	\le\|\mu\|_{M(\R^n)}
	\]
\end{proof}

\begin{proposition}
	如果$\mu\in M(\R^n)$,那么$\widehat{\mu}$为连续函数。
\end{proposition}

\begin{proof}
	固定$\xi\in\R^n$,考虑
	\[
	\widehat{\mu}(\xi+h)
	=\int\ee{-2\pi i x \cdot (\xi+h)}\dd\mu(x)
	\]
	由于$|\ee{-2\pi i x \cdot (\xi+h)}|=1$且$|\mu|(\R^n)<\infty$,那么由控制收敛定理
	\[
	\lim_{h\to 0}\widehat{\mu}(\xi+h)
	=\lim_{h\to 0}\int\ee{-2\pi i x \cdot (\xi+h)}\dd\mu(x)
	=\int\lim_{h\to 0}\ee{-2\pi i x \cdot (\xi+h)}\dd\mu(x)
	=\int\ee{-2\pi i x \cdot \xi}\dd\mu(x)
	=\widehat{\mu}(\xi)
	\]
\end{proof}

现在我们列出 Fourier 变换的一些基本性质,这些性质并不涉及微分和积分。

\begin{proposition}{Fourier 变换的基本性质}{Fourier 变换的基本性质}
	令$f\in L^1(\R^n),\tau\in\R^n$,且$T:\R^n\to\R^n$为可逆线性变换。
	\begin{enumerate}
		\item 令$f_{\tau}(x)=f(x-\tau)$,则
		\[
		\widehat{f_\tau}(\xi)
		=\ee{-2\pi i \tau \cdot \xi}\widehat{f}(\xi)
		\]
		\item 令$e_\tau(x)=\ee{2\pi i x\cdot \tau}$,则
		\[
		\widehat{e_\tau f}(\xi)=\widehat{f}(\xi-\tau)
		\]
		\item 令$T^{-t}$表示$T$的逆转置,则
		\[
		\widehat{f\circ T}=|\det(T)|^{-1}\widehat{f}\circ T^{-t}
		\]
		\item 令$\tilde{f}(x)=\overline{f(-x)}$,则
		\[
		\widehat{\tilde{f}}=\overline{\widehat{f}}
		\]
	\end{enumerate}
\end{proposition}

\begin{proof}
	\begin{enumerate}
		\item 
		\[
		\widehat{f_\tau}(\xi)
		=\int\ee{-2\pi i x \cdot \xi}f_\tau(x)\dd x
		=\int\ee{-2\pi i x \cdot \xi}f(x-\tau)\dd x
		=\ee{-2\pi i \tau}\int\ee{-2\pi i x \cdot \xi}f(x)\dd x
		=\ee{-2\pi i \tau \cdot \xi}\widehat{f}(\xi)
		\]
		\item 
		\[
		\widehat{e_\tau f}(\xi)
		=\int\ee{-2\pi i x \cdot \xi}e_\tau(x)f(x)\dd x
		=\int\ee{-2\pi i x \cdot (\xi+\tau)}f(x)\dd x
		=\int\ee{-2\pi i x \cdot \xi}f(x-\tau)\dd x
		=\widehat{f}(\xi-\tau)
		\]
		\item 
		\begin{align*}
			\widehat{f\circ T}(\xi)
			& = \int\ee{-2\pi i x \cdot \xi}f(Tx)\dd x \\
			& = |\det(T)|^{-1}\int\ee{-2\pi i (T^{-1}x \cdot \xi)}f(x)\dd x \\
			& = |\det(T)|^{-1}\int\ee{-2\pi i (x \cdot T^{-t}\xi)}f(x)\dd x \\
			& = |\det(T)|^{-1}\widehat{f}\circ T^{-t}(\xi)
		\end{align*}
		\item 
		\[
		\widehat{\tilde{f}}(\xi)
		= \int\ee{-2\pi i x \cdot \xi}\tilde{f}(x)\dd x
		= \int\ee{-2\pi i x \cdot \xi}\overline{f(-x)}\dd x 
		= \int\ee{-2\pi i x \cdot \xi}\overline{f(x)}\dd x 
		= \overline{\int\ee{-2\pi i x \cdot \xi}f(x)\dd x}
		= \overline{\widehat{f}(\xi)}
		\]
	\end{enumerate}
\end{proof}

进一步考察情况3。如果$T$为正交变换,那么$\widehat{f\circ T}=\widehat{f}\circ T$。特别的,如果$f$为径向函数(radial function),那么$\widehat{f}$亦为径向函数。所谓径向函数$f:\R^n\to\R^n$,就是成立如下性质之一的函数:
\begin{enumerate}
	\item 存在函数$\varphi:\R\to\R^n$,使得对于任意$x\in\R^n$,成立$f(x)=\varphi(\|x\|)$。
	\item 对于任意旋转$\rho\in\text{SO}_n(\R)$,成立$f\circ\rho=f$。
\end{enumerate}
如果$T$为膨胀(dilation),即存在$r>0$,使得成立$Tx=rx$,那么$f(rx)$的 Fourier 变换为$r^{-n}\widehat{f}(r^{-1}\xi)$。反之,$r^{-n}f(r^{-1}x)$的 Fourier 变换为$\widehat{f}(r\xi)$。





























\end{document}