\documentclass[lang = cn, % 设置中文环境
scheme = chinese          % 设置标题为中文
% thmcnt = section,         % 设置计数器
]
{elegantbook}             % 设置elegantbook文档类


%% 1.封面设置

\title{Lectures in Harmonic Analysis-Thomas H. Wolf-Notebook}                % 文档标题

\author{若水}               % 作者

\myemail{ethanmxzhou@163.com} % 邮箱

\homepage{helloethanzhou.github.io} % 主页

\date{\today}               % 日期

\extrainfo{上善若水任方圆}   % 箴言

\logo{PiCreatures_happy.pdf}        % 设置Logo

\cover{阿基米德螺旋曲线.pdf}          % 设置封面图片

% 修改标题页的色带
\definecolor{customcolor}{RGB}{135, 206, 250} 
% 定义一个名为customcolor的颜色,RGB颜色值为(135, 206, 250)

\colorlet{coverlinecolor}{customcolor}     % 将coverlinecolor颜色设置为customcolor颜色

%% 2.目录设置
\setcounter{tocdepth}{3}  % 目录深度为3

%% 3.引入宏包
\usepackage[all]{xy}
\usepackage{bbm, svg, graphicx, float, extpfeil, amsmath, amssymb, mathrsfs, mathalpha, hyperref, centernot, physics}


%% 4.定义命令
\newcommand{\N}{\mathbb{N}}            % 自然数集合
\newcommand{\R}{\mathbb{R}}            % 实数集合
\newcommand{\C}{\mathbb{C}}  		   % 复数集合
\newcommand{\Q}{\mathbb{Q}}            % 有理数集合
\newcommand{\Z}{\mathbb{Z}}            % 整数集合
\newcommand{\sub}{\subset}             % 包含
\newcommand{\im}{\text{im }}           % 像
\newcommand{\lang}{\langle}            % 左尖括号
\newcommand{\rang}{\rangle}            % 右尖括号
\newcommand{\bs}{\boldsymbol}          % 向量加黑
\newcommand{\ee}[1]{\mathrm{e}^{#1}}           % 微分d
\newcommand{\dis}{\displaystyle}
\newcommand{\supp}{\text{supp }}
\newcommand{\pll}{\kern 0.56em/\kern -0.8em /\kern 0.56em} % 平行
\newcommand{\function}[5]{
\begin{align*}
	#1:\begin{aligned}[t]
		#2 &\longrightarrow #3\\
		#4 &\longmapsto #5
	\end{aligned}
\end{align*}
}                                     % 函数

\newcommand{\lhdneq}{%
\mathrel{\ooalign{$\lneq$\cr\raise.22ex\hbox{$\lhd$}\cr}}} % 真正规子群

\newcommand{\rhdneq}{%
\mathrel{\ooalign{$\gneq$\cr\raise.22ex\hbox{$\rhd$}\cr}}} % 真正规子群

%% 5.参考文献

\addbibresource[location=local]{reference.bib} % 添加本地的参考文献文件reference.bib

\begin{document}

\maketitle       % 创建标题页

\frontmatter     % 开始前言部分

\chapter*{致谢}

\markboth{致谢}{致谢}

\vspace*{\fill}
	\begin{center}
		
		\large{感谢 \textbf{ 勇敢的 } 自己}
		
	\end{center}
\vspace*{\fill}

\tableofcontents % 创建目录

\mainmatter      % 开始正文部分

\chapter{$L^1$ Fourier 变换}

\begin{definition}{$L^1$ Fourier 变换 $L^1$ Fourier transform}
	\begin{enumerate}
		\item 对于$f\in L^1(\R^n)$,定义其 Fourier 变换为
		\begin{align*}
			\widehat{f}:\begin{aligned}[t]
				\R^n &\longrightarrow \C\\
				\xi &\longmapsto \int\ee{-2\pi i x \cdot \xi}f(x)\dd x
			\end{aligned}
		\end{align*}
		其中$x \cdot \xi$表示内积。
		\item 更一般的,对于$\R^n$上的赋有范数
		\[
		\|\mu\|=|\mu|(\R^n)
		\]
		的有限复值测度空间$M(\R^n)$,其中$|\mu|$为总变差,通过$f\to\mu,\dd\mu=f\dd x$,$L^1(\R^n)$包含在$M(\R^n)$中,此时推广 Fourier 变换为
		\[
		\widehat{\mu}(\xi)=\int\ee{-2\pi i x \cdot \xi}\dd\mu(x)
		\]
	\end{enumerate}
\end{definition}

\begin{example}
	对于$a\in\R^n$与$E\sub\R^n$,定义 Dirac 测度
	\[
	\delta_a(E)=\begin{cases}
		1,\qquad & a\in E\\
		0,\qquad & a\notin E
	\end{cases}
	\]
	那么
	\[
	\widehat{\delta_a}(\xi)=\ee{-2\pi i a\cdot \xi}
	\]
\end{example}

\begin{proof}
	\[
	\widehat{\delta_a}(\xi)
	=\int\ee{-2\pi i x \cdot \xi}\dd \delta_a(x)
	=\ee{-2\pi i a\cdot \xi}
	\]
\end{proof}

\begin{example}
	令$\Gamma(x)=\ee{-\pi|x|^2}$,则
	\[
	\widehat{\Gamma}(\xi)=\ee{-\pi|\xi|^2}
	\]
\end{example}

\begin{proof}
	由于
	\[
	\widehat{\Gamma}(\xi)
	=\int\ee{-2\pi i x \cdot \xi}\Gamma(x)\dd x
	=\int\ee{-2\pi i x \cdot \xi}\ee{-\pi|x|^2}\dd x
	\]
	注意到该积分的变量仅为一维变量,因此不妨考虑$n=1$。由 Gauss 积分$\dis \int_{-\infty}^{+\infty}\ee{-\pi x^2}\dd x=1$,则
	\[
	\int_{-\infty}^{+\infty}\ee{-2\pi i x \cdot \xi}\ee{-\pi|x|^2}\dd x
	=\ee{-\pi|\xi|^2}
	\]
\end{proof}

$L^1$ Fourier 变换存在一些基本的估计,我们罗列如下。

\begin{proposition}{}{Fourier变换为有界变换}
	如果$\mu\in M(\R^n)$,那么$\widehat{\mu}$为有界函数,且
	\begin{gather}
		\label{2式}
		\|\widehat{\mu}\|_{\infty}
		\le\|\mu\|_{M(\R^n)}
	\end{gather}
\end{proposition}

\begin{proof}
	对于任意$\xi\in\R^n$,由于
	\[
	|\widehat{\mu}(\xi)|
	=\left| \int\ee{-2\pi i x \cdot \xi}\dd\mu(x) \right|
	\le\int|\ee{-2\pi i x \cdot \xi}|\dd|\mu|(x)
	=\|\mu\|_{M(\R^n)}
	\]
	因此
	\[
	\|\widehat{\mu}\|_{\infty}
	=\sup_{\xi\in\R^n}|\widehat{\mu}(\xi)|
	\le\|\mu\|_{M(\R^n)}
	\]
\end{proof}

\begin{proposition}{}{Fourier变换为光滑变换}
	如果$\mu\in M(\R^n)$,那么$\widehat{\mu}$为连续函数。
\end{proposition}

\begin{proof}
	固定$\xi\in\R^n$,考虑
	\[
	\widehat{\mu}(\xi+h)
	=\int\ee{-2\pi i x \cdot (\xi+h)}\dd\mu(x)
	\]
	由于$|\ee{-2\pi i x \cdot (\xi+h)}|=1$且$|\mu|(\R^n)<\infty$,那么由控制收敛定理
	\[
	\lim_{h\to 0}\widehat{\mu}(\xi+h)
	=\lim_{h\to 0}\int\ee{-2\pi i x \cdot (\xi+h)}\dd\mu(x)
	=\int\lim_{h\to 0}\ee{-2\pi i x \cdot (\xi+h)}\dd\mu(x)
	=\int\ee{-2\pi i x \cdot \xi}\dd\mu(x)
	=\widehat{\mu}(\xi)
	\]
\end{proof}

现在我们列出 Fourier 变换的一些基本性质,这些性质并不涉及微分和积分。

\begin{proposition}{Fourier 变换的基本性质}{Fourier 变换的基本性质}
	令$f\in L^1(\R^n),\tau\in\R^n$,且$T:\R^n\to\R^n$为可逆线性变换。
	\begin{enumerate}
		\item 令$f_{\tau}(x)=f(x-\tau)$,则
		\[
		\widehat{f_\tau}(\xi)
		=\ee{-2\pi i \tau \cdot \xi}\widehat{f}(\xi)
		\]
		\item 令$e_\tau(x)=\ee{2\pi i x\cdot \tau}$,则
		\[
		\widehat{e_\tau f}(\xi)=\widehat{f}(\xi-\tau)
		\]
		\item 令$T^{-t}$表示$T$的逆转置,则
		\[
		\widehat{f\circ T}=|\det(T)|^{-1}\widehat{f}\circ T^{-t}
		\]
		\item 令$\tilde{f}(x)=\overline{f(-x)}$,则
		\[
		\widehat{\tilde{f}}=\overline{\widehat{f}}
		\]
	\end{enumerate}
\end{proposition}

\begin{proof}
	\begin{enumerate}
		\item 
		\[
		\widehat{f_\tau}(\xi)
		=\int\ee{-2\pi i x \cdot \xi}f_\tau(x)\dd x
		=\int\ee{-2\pi i x \cdot \xi}f(x-\tau)\dd x
		=\ee{-2\pi i \tau}\int\ee{-2\pi i x \cdot \xi}f(x)\dd x
		=\ee{-2\pi i \tau \cdot \xi}\widehat{f}(\xi)
		\]
		\item 
		\[
		\widehat{e_\tau f}(\xi)
		=\int\ee{-2\pi i x \cdot \xi}e_\tau(x)f(x)\dd x
		=\int\ee{-2\pi i x \cdot (\xi+\tau)}f(x)\dd x
		=\int\ee{-2\pi i x \cdot \xi}f(x-\tau)\dd x
		=\widehat{f}(\xi-\tau)
		\]
		\item 
		\begin{align*}
			\widehat{f\circ T}(\xi)
			& = \int\ee{-2\pi i x \cdot \xi}f(Tx)\dd x \\
			& = |\det(T)|^{-1}\int\ee{-2\pi i (T^{-1}x \cdot \xi)}f(x)\dd x \\
			& = |\det(T)|^{-1}\int\ee{-2\pi i (x \cdot T^{-t}\xi)}f(x)\dd x \\
			& = |\det(T)|^{-1}\widehat{f}\circ T^{-t}(\xi)
		\end{align*}
		\item 
		\[
		\widehat{\tilde{f}}(\xi)
		= \int\ee{-2\pi i x \cdot \xi}\tilde{f}(x)\dd x
		= \int\ee{-2\pi i x \cdot \xi}\overline{f(-x)}\dd x 
		= \int\ee{-2\pi i x \cdot \xi}\overline{f(x)}\dd x 
		= \overline{\int\ee{-2\pi i x \cdot \xi}f(x)\dd x}
		= \overline{\widehat{f}(\xi)}
		\]
	\end{enumerate}
\end{proof}

进一步考察情况3。如果$T$为正交变换,那么$\widehat{f\circ T}=\widehat{f}\circ T$。特别的,如果$f$为径向函数(radial function),那么$\widehat{f}$亦为径向函数。所谓径向函数$f:\R^n\to\R^n$,就是成立如下性质之一的函数:
\begin{enumerate}
	\item 存在函数$\varphi:\R\to\R^n$,使得对于任意$x\in\R^n$,成立$f(x)=\varphi(|x|)$。
	\item 对于任意旋转$\rho\in\text{SO}_n(\R)$,成立$f\circ\rho=f$。
\end{enumerate}
如果$T$为膨胀(dilation),即存在$r>0$,使得成立$Tx=rx$,那么$f(rx)$的 Fourier 变换为$r^{-n}\widehat{f}(r^{-1}\xi)$。反之,$r^{-n}f(r^{-1}x)$的 Fourier 变换为$\widehat{f}(r\xi)$。

Fourier 变换有一个性质:如果$f$在空间中是局域的(localized),那么$\widehat{f}$是光滑的;如果$f$在空间中是光滑的,那么$\widehat{f}$是局域的。下面我们讨论这方面的一些简单表现。

令$D(x_0,r)=\{ x\in\R^n:|x-x_0|<r \}$。对于多指标$\alpha\in\Z_{\ge 0}^n$,定义
\[
D^\alpha=\frac{\partial^{\alpha_1}}{\partial x_1^{\alpha_1}}\cdots \frac{\partial^{\alpha_n}}{\partial x_n^{\alpha_n}},\qquad
x^\alpha=\prod_{j=1}^{n}x_j^{\alpha_j}
\]
$\alpha$的长度定义为
\[
|\alpha|=\sum_{j=1}^{n}\alpha_j
\]
在$\Z_{\ge 0}^n$上定义偏序
\begin{align*}
	& \alpha\le\beta \iff \alpha_j\le \beta_j,\forall j \\
	& \alpha<\beta \iff \alpha\le\beta\text{且}\alpha\ne\beta
\end{align*}

\begin{proposition}{}{命题1.3}
	对于$\mu\in M(\R^n)$,如果$\supp \mu$为紧集,那么$\widehat{\mu}\in C^{\infty}(\R^n)$,且
	\begin{gather}
		\label{7式}
		D^{\alpha}(\widehat{\mu})
		=\widehat{(-2\pi i x)^{\alpha}\mu}
	\end{gather}
	进一步,若$\supp \mu\sub D(0,R)$,则
	\begin{gather}
		\label{8式}
		\|D^{\alpha}(\widehat{\mu})\|_{\infty}
		\le (2\pi R)^{|\alpha|}\|\mu\|
	\end{gather}
\end{proposition}

\begin{proof}
	注意到$(\ref{7式})\implies(\ref{8式})$。事实上,由命题\ref{pro:Fourier变换为有界变换}
	\[
	\|D^{\alpha}(\widehat{\mu})\|_{\infty}
	=\|\widehat{(-2\pi i x)^{\alpha}\mu}\|_{\infty}
	\le \|(-2\pi i x)^{\alpha}\mu\|
	\le (2\pi R)^{|\alpha|}\|\mu\|
	\]
	
	进一步,对于任意多指标$\alpha\in\Z_{\ge 0}^n$,$(2\pi i x)^\alpha\mu$为有限测度且具有紧支集,因此如果我们能证明当$|\alpha|=1$时,$\widehat{\mu}\in C^1$且成立$(\ref{7式})$,那么该命题可由此归纳证明。
	
	固定$1\le  j \le n$与$\xi\in\R^n$,令$e_j$为第$j$个标准基向量,考虑差商
	\[
	\Delta(h)
	=\frac{\widehat{\mu}(\xi+he_j)-\widehat{\mu}(\xi)}{h}
	=\int \frac{\ee{-2\pi i h x_j}-1}{h}\ee{-2\pi i \xi\cdot x}\dd \mu(x)
	\]
	由于
	\[
	\left| \frac{\ee{-2\pi i h x_j}-1}{h} \right|
	\le 2\pi |x_j|
	\]
	因此由控制收敛定理
	\begin{gather}
		\label{77式}
		\lim_{h\to 0}\Delta(h)
		=\int \lim_{h\to 0}\frac{\ee{-2\pi i h x_j}-1}{h}\ee{-2\pi i \xi\cdot x}\dd \mu(x)
		=\int -2\pi i x_j\ee{-2\pi i \xi\cdot x}\dd \mu(x)
	\end{gather}
	当$|\alpha|=1$时,$(\ref{77式})\implies(\ref{7式})$,且$(\ref{7式})$与命题\ref{pro:Fourier变换为光滑变换}$\implies \widehat{\mu}\in C^1$。
\end{proof}

估计$\ref{2式}$表明如果$\mu$在空间中是局域的,那么$\widehat{\mu}$是光滑的。现在我们考虑反问题,$\mu$光滑意味着$\widehat{\mu}$局域。

我们首先考虑一个引理。令$\phi:\R^n\to\R$为$C^\infty$函数,满足如下性质:
\begin{enumerate}
	\item 若$|x|\le 1$,则$\phi(x)=1$。
	\item 若$|x|\ge 2$,则$\phi(x)=0$。
	\item $0\le \phi \le 1$
	\item $\phi$为径向的。
\end{enumerate}

定义$\phi_k(x)=\phi(x/k)$。假设对于任意多指标$\alpha\in\Z_{\ge 0}^n$,存在不依赖于$k$的常数$C_\alpha$,使得成立$|D^\alpha(\phi_k)|\le C_\alpha/k^{|\alpha|}$。进一步假设若$\alpha\ne 0$,则$\supp D^\alpha(\phi)\sub \{ x\in\R^n:k\le |x|\le 2k \}$。

\begin{lemma}{}{光滑则局域的引理}
	对于$f\in C^N(\R^n)$,如果对于任意$0\le |\alpha| \le N$,成立$D^\alpha(f)\in L^1(\R^n)$,那么令$f_k=\phi_kf$,则当$0\le |\alpha| \le N$时,成立
	\[
	\lim_{k\to\infty}\|D^\alpha(f_k)-D^\alpha(f)\|_1=0
	\]
\end{lemma}

\begin{proof}
	注意到
	\[
	\lim_{k\to\infty}\|\phi_kD^\alpha(f)-D^\alpha(f)\|_1=0
	\]
	因此由 Minkowski 不等式
	\[
	\lim_{k\to\infty}\|D^\alpha(\phi_kf)-\phi_kD^\alpha(f)\|_1=0
	\]
	然而,由 Leibniz 法则
	\[
	D^\alpha(\phi_kf)-\phi_kD^\alpha(f)
	=\sum_{0<\beta\le \alpha}c_{\beta}D^{\alpha-\beta}(f)D^\beta(\phi_k)
	\]
	其中$c_\beta$为常数。因此由 Hölder 不等式
	\[
	\|D^\alpha(\phi_kf)-\phi_kD^\alpha(f)\|_1
	\le C\sum_{0<\beta\le \alpha}\|D^\beta(\phi_k)\|_{\infty}\|D^{\alpha-\beta}(f)\|_{L^1(\{ x:|x|\ge k \})}
	\le \frac{C}{k}\sum_{0<\beta\le \alpha}\|D^{\alpha-\beta}(f)\|_{L^1(\{ x:|x|\ge k \})}
	\]
	从而
	\[
	\lim_{k\to\infty}\|D^\alpha(f_k)-D^\alpha(f)\|_1=0
	\]
\end{proof}

我们回到反问题,证明的关键在于分部积分。

\begin{proposition}{}{光滑则局域}
	对于$f\in C^N$,如果对于任意$0\le |\alpha| \le N$,成立$D^\alpha(f)\in L^1$,那么当$0\le |\alpha| \le N$时,成立
	\begin{gather}
		\label{11式}
		\widehat{D^\alpha(f)}(\xi)
		=(2\pi i \xi)^{\alpha}\widehat{f}(\xi)
	\end{gather}
	且存在常数$C$,使得对于任意$\xi\in\R^n$,成立
	\begin{gather}
		\label{12式}
		|\widehat{f}(\xi)|
		\le C(1+|\xi|)^{-N}
	\end{gather}
\end{proposition}

\begin{proof}
	如果$f\in C^1$且具有紧支集,那么由分部积分
	\[
	\int \frac{\partial f(x)}{\partial x_j}\ee{-2\pi i x\cdot \xi}\dd x
	=2\pi i \xi_j \int \ee{-2\pi i x\cdot \xi}f(x)\dd x
	\]
	当$|\alpha|=1$时,成立$(\ref{11式})$。由数学归纳,只要$f\in C^N(\R^n)$且具有紧支集,那么对于任意$0\le |\alpha| \le N$,成立$(\ref{11式})$。
	
	为了去掉紧支集的条件,由引理\ref{lem:光滑则局域的引理},$f_k$成立$(\ref{11式})$。由引理\ref{lem:光滑则局域的引理}与命题\ref{pro:Fourier变换为有界变换},当$k\to\infty$时,$\widehat{D^\alpha(f_k)}\rightrightarrows D^\alpha(f)$,$(2\pi i \xi)^{\alpha}\widehat{f_k}(\xi)\rightrightarrows (2\pi i \xi)^{\alpha}\widehat{f}(\xi)$,因此成立$(\ref{11式})$。
	
	为了证明$(\ref{12式})$,$(\ref{11式})$与命题\ref{pro:Fourier变换为有界变换}$\implies \xi^\alpha\widehat{f}\in L^\infty(\R^n)$。另一方面很容易估计
	\begin{gather}
		\label{14式}
		C_N^{-1}(1+|\xi|)^N
		\le \sum_{|\alpha|\le N}|\xi^\alpha|
		\le C_N(1+|\xi|)^N
	\end{gather}
	因此成立$(\ref{12式})$。
\end{proof}

\chapter{Schwartz 空间}

\begin{definition}{Schwartz 空间}
	记 Schwartz 空间为$\mathcal{S}$,称$f:\R^n\to\C\in\mathcal{S}$,如果$f\in C^\infty$且成立如下命题之一。
	\begin{enumerate}
		\item 对于任意$\alpha,\beta$,成立$x^\alpha D^\beta(f)\in L^\infty$。
		\item 对于任意$N,\beta$,成立$(1+|x|)^N D^\beta(f)\in L^\infty$。
		\item 对于任意$\alpha,\beta$,成立$\dis\lim_{x\to\infty}x^\alpha D^\beta(f)=0$。
	\end{enumerate}
	引入范数
	\[
	\|f\|_{\alpha\beta}=\|x^\alpha D^\beta(f)\|_{\infty}
	\]
\end{definition}

\begin{proof}
	$1\iff 3$:由$(\ref{14式})$可得。

	$1\iff 4$:平凡!
\end{proof}

\begin{definition}{收敛}
	称序列$\{ f_k \}\sub\mathcal{S}$在$\mathcal{S}$中收敛于$f\in\mathcal{S}$,并记作$f_k\to f$,如果对于任意$\alpha,\beta$,成立
	\[
	\lim_{k\to\infty}\|f_k-f\|_{\alpha\beta}=0
	\]
\end{definition}

\begin{example}
	令$C_0^\infty$表示$C^\infty$中具有紧支集的函数全体,则$C_0^\infty\sub\mathcal{S}$。
\end{example}

\begin{proof}
	等价于证明$x^\alpha D^\beta(f)$为有界函数。事实上,如果$f\in C_0^\infty$,那么$D^\beta(f)$为具有紧支集的连续函数,因此为有界函数。而$x^\alpha$在$D^\beta(f)$的支集上为有界的,进而$x^\alpha D^\beta(f)$为有界函数。
\end{proof}

\begin{example}
	令$f(x)=\ee{-\pi |x|^2}$,则$f\in \mathcal{S}$。
\end{example}

\begin{note}
	我们先在$\R^1$中感受一下。当$f(x)=\ee{-\pi x^2}$时
	\[
	x^nf(x)=\frac{x^n}{\ee{\pi x^2}},\qquad
	x^nf'(x)=\frac{-2 \pi x^{n+1}}{\ee{\pi x^2}},\qquad
	x^nf''(x)=\frac{(4 \pi ^2 x^2-2 \pi) x^n}{\ee{\pi x^2}},\qquad
	x^nf'''(x)=\frac{(12 \pi^2x-8 \pi^3 x^3) x^n}{\ee{\pi x^2}}
	\]
	因此存在多项式$p_m(x)$,使得成立
	\[
	x^nf^{(m)}(x)=\frac{p_m(x)}{\ee{\pi x^2}}
	\]
	这样$x^nf^{(m)}(x)$就有界了。
	
	下面回到$\R^n$中。
\end{note}

\begin{proof}
	注意到对于多项式$p(x)$,任意偏导数
	\[
	\frac{\partial}{\partial x_j}(p(x)f(x))
	\]
	均为$q(x)f(x)$的形式,其中$q(x)$为多项式。由归纳,对于任意$\beta$,$D^\beta(f)$为多项式与$f$的积的形式,进而对于任意$\alpha,\beta$,$x^\alpha D^\beta(f)$为多项式与$f$的积的形式。由 L'Hospital 法则,$x^\alpha D^\beta(f)$为有界函数。
\end{proof}

\begin{example}
	\begin{enumerate}
		\item $f_N(x)=(1+|x|^2)^{-N}\notin \mathcal{S}$
		\item $g(x)=\ee{-\pi|x|^2}\sin(\ee{\pi |x|^2})\notin \mathcal{S}$
	\end{enumerate}
\end{example}

\begin{proof}
	\begin{enumerate}
		\item 不妨在$\R^1$与$N=1$中考虑,此时
		\[
		f(x)=\frac{1}{1+x^2}
		\]
		那么
		\[
		x^3f(x)=\frac{x^3}{1+x^2}
		\]
		无界。
		\item 不妨在$\R^1$与$N=1$中考虑,此时
		\[
		g(x)=\ee{-\pi x^2}\sin(\ee{\pi x^2})
		\]
		那么
		\[
		g'(x)=2 \pi  x \cos \left(e^{\pi  x^2}\right)-2 \pi  e^{-\pi  x^2} x
		\sin \left(e^{\pi  x^2}\right)
		\]
		无界。
	\end{enumerate}
\end{proof}

\begin{note}
	通过这些例子,我们大概能感受到$\mathcal{S}$中的函数长什么样子,即其任意阶导数的衰减速度比任意阶多项式的增长速度快。
\end{note}

我们现在来讨论$\mathcal{S}$的一些简单性质,然后是一些稍微不那么简单的性质。

\begin{proposition}
	$\mathcal{S}$对于微分与多项式乘法封闭,且该运算连续。同时$\mathcal{S}$对于乘法封闭。
	\begin{enumerate}
		\item 如果$f\in\mathcal{S}$,那么对于任意多指标$\alpha\in\Z_{\ge 0}^n$,成立$D^\alpha(f)\in\mathcal{S}$。
		\item 如果$f\in\mathcal{S}$,那么对于任意多指标$\alpha\in\Z_{\ge 0}^n$,成立$x^\alpha f\in\mathcal{S}$。
		\item 如果$\{ f_k,f \}\sub\mathcal{S}$,且$f_k\to f$,那么对于任意多指标$\alpha\in\Z_{\ge 0}^n$,成立$D^\alpha(f_k)\to D^\alpha(f)$。
		\item 如果$\{ f_k,f \}\sub\mathcal{S}$,且$f_k\to f$,那么对于任意多指标$\alpha\in\Z_{\ge 0}^n$,成立$x^\alpha f_k\to x^\alpha f$。
		\item 如果$f,g\in\mathcal{S}$,那么$fg\in\mathcal{S}$。
	\end{enumerate}
\end{proposition}

\begin{proof}
	\begin{enumerate}
		\item 注意到$x^\alpha D^\beta (D^\gamma (f))=x^\alpha D^{\beta+\gamma}(f)$,则$D^\gamma (f)\in \mathcal{S}$。
		\item 由 Leibniz 法则
		\[
		x^\alpha D^\beta(x^\gamma f)
		=\sum_{0\le\delta\le\beta}c_\delta x^\alpha D^\delta(x^\gamma)D^{\beta-\delta}(f)
		\]
		因此$x^\alpha D^\beta(x^\gamma f)$为$x$的单项式与$f$的导数的线性组合,因此为有界的,进而$x^\gamma f\in\mathcal{S}$。
		\item 由于$f_n\to f$,那么
		\[
		\lim_{k\to\infty}\|x^\alpha D^{\beta+\gamma}(f_k-f)\|_\infty=0
		\]
		等价于
		\[
		\lim_{k\to\infty}\|x^\alpha D^{\beta}(D^{\gamma}(f_k)-D^{\gamma}(f))\|_\infty=0
		\]
		因此$D^\gamma(f_k)\to D^\gamma(f)$。
		\item 由 Leibniz 法则与 Minkowski 不等式
		\begin{align*}
			\| x^\alpha D^\beta(x^\gamma f_k-x^\gamma f) \|_\infty
			& = \left\|\sum_{0\le\delta\le\beta}c_\delta x^\alpha D^\delta(x^\gamma)D^{\beta-\delta}(f_k-f) \right\|_\infty \\
			& = \left\|\sum_{\delta}c'_\delta x^{\alpha+\gamma_\delta}D^{\beta_\delta}(f_k-f) \right\|_\infty \\
			& \le \sum_{\delta}\left\|c'_\delta x^{\alpha+\gamma_\delta}D^{\beta_\delta}(f_k-f) \right\|_\infty \\
		\end{align*}
		而$f_k\to f$,则$x^\gamma f_k\to x^\gamma f$。
		\item 显然$\mathcal{S}$对于乘法封闭。
	\end{enumerate}
\end{proof}

\begin{proposition}{}{支集在S中稠密}
	$C_0^\infty$在$\mathcal{S}$中稠密;换言之,对于任意$f\in\mathcal{S}$,存在$\{ f_k \}\sub C_0^\infty$,使得成立$f_k\to f$。
\end{proposition}

\begin{proof}
	这和引理\ref{lem:光滑则局域的引理}的证明是类似的。令$f_k=\phi_k f$,显然$f_k\in C_0^\infty$。若要证明$f_k\to f$,即要证明
	\[
	\|  x^\alpha D^\beta(\phi_k f)- x^\alpha D^\beta(f) \|_\infty\to 0
	\]
	由 Minkowski 不等式
	\[
	\|  x^\alpha D^\beta(\phi_k f)- x^\alpha D^\beta(f) \|_\infty
	\le \|  \phi_kx^\alpha D^\beta(f)- x^\alpha D^\beta(f) \|_\infty
	+\|  x^\alpha(D^\beta(\phi_k f)- \phi_k D^\beta(f)) \|_\infty
	\]
	对于第一项
	\[
	\lim_{k\to\infty}\phi_kx^\alpha D^\beta(f)- x^\alpha D^\beta(f)=0
	\]
	对于第二项,由 Leibniz 法则
	\[
	\|  x^\alpha(D^\beta(\phi_k f)- \phi_k D^\beta(f)) \|_\infty
	\le C\sum_{\gamma <\beta}\|x^\alpha D^\gamma (f) \|_\infty \|D^{\beta-\gamma}(\phi_k) \|_\infty
	\]
	由于$f\in\mathcal{S}$且$\|D^{\beta-\gamma}(\phi_k) \|_\infty\le C/k$,则第二项$\infty0$。
\end{proof}

我们可以加强命题\ref{pro:支集在S中稠密}。定义$C_0^\infty$张量函数
\function{f}{\R^n}{\C}{x}{\prod_{j}\phi_j(x_j)}
其中$\phi_j\in C_0^\infty(\R)$。

\begin{proposition}{}{支集在S中稠密2}
	$C_0^\infty$的张量函数的线性组合在$\mathcal{S}$中稠密。
\end{proposition}

\begin{proof}
	由命题\ref{pro:支集在S中稠密},可说明如果$f\in C_0^\infty$,那么存在序列$\{ g_k \}$,使得成立
	\begin{enumerate}
		\item 每个$g_k$为$C_0^\infty$张量函数。
		\item $g_k$的支集包含在一个不依赖于$k$的固定的紧集$E$中。
		\item 对于任意$\alpha$,$D^\alpha(g_k)\rightrightarrows D^\alpha(f)$。
	\end{enumerate}
	
	为了构造$\{ g_k \}$,我们使用一个关于 Fourier 级数的基本事实:对于以$2\pi$为周期的$C^\infty(\R^n)$函数$f$,其可展开为
	\[
	f(\theta)=\sum_{\nu\in\Z^n}a_{\nu}\ee{i\nu\cdot\theta}
	\]
	其中$\{ a_\nu \}$成立
	\[
	\sum_{\nu\in\Z^n}(1+|\nu|)^N|a_\nu|<\infty,\qquad \forall N\in\N^*
	\]
	考虑 Fourier 级数的部分和,我们因此得到一个三角函数多项式序列$p_k$,使得对于任意$\alpha$,$D^\alpha(p_k)\rightrightarrows D^\alpha(f)$。
	
	在构造$\{g_k\}$时,我们可以不妨假设$x\in\supp f$意味着$|x_j|\le 1$,例如我们可以使用$f(Rx)$来替换$f(x)$。令$C_0^\infty$为$C_0^\infty$单变量函数,其中在$[-1,1]$上为$1$,在$[-2,2]^c$上为$0$。令$\tilde{f}$为在$[-\pi,\pi]^n$上等于$f$的$2\pi $周期的函数。然后我们有一个三角函数多项式序列$p_k$,使得对于任意$\alpha$,$D^\alpha(p_k)\rightrightarrows D^\alpha(\tilde{f})$。令$\dis g_k(x)=\prod_j \phi(x_j)p_k(x)$,则$g_k$满足1和2。由引理\ref{lem:光滑则局域的引理}与命题\ref{pro:支集在S中稠密},3成立。
\end{proof}

在本章的最后,我们证明可以修改$\mathcal{S}$的定义为:称$f\in\mathcal{S}$,如果对于任意$\alpha,\beta$,成立$x^\alpha D^\beta(f)\in L^1$,此时依$L^\infty$收敛$\iff$依$L^1$收敛。证明的必要性比较容易,为证明充分性,我们引入新的记号,并不加证明的引入一个引理。

如果$f:\R^n\to\infty$为$C^k$函数且若$x\in \R^n$,则定义
\[
\Delta_k^f(x)=\sum_{|\alpha|=k}|D^\alpha(f(x))|
\]
引入记号$D(x_0,r)=\{ x:|x-x_0|\le r \}$。同时记$x\lesssim y\iff \exists C,x\le Cy$。

\begin{lemma}{}{S等价定义引理}
	假设$f$为$C^\infty$函数,则对于任意$x$,成立
	\[
	|f(x)| \lesssim \sum_{0\le j \le n+1}\|\Delta_j^f\|_{L^1(D(x,1))}
	\]
\end{lemma}

\begin{theorem}{}{S等价定义}
	称$f:\R^n\to\C\in\mathcal{S}$,如果$f\in C^\infty$且成立如下命题之一。
	\begin{enumerate}
		\item 对于任意$\alpha,\beta$,成立$x^\alpha D^\beta(f)\in L^\infty$。
		\item 对于任意$\alpha,\beta$,成立$x^\alpha D^\beta(f)\in L^1$。
	\end{enumerate}
\end{theorem}

\begin{proof}
	$1\implies 2$:假设$f$对于任意$\alpha,\beta$,成立$x^\alpha D^\beta(f)\in L^\infty$。令$N=|\alpha|+n+1$,则由$(1+|x|)^{-n-1}$的可积性,结合 Hölder 不等式
	\[
	\|x^\alpha D^\beta(f) \|_1
	\le \|(1+|x|)^ND^\beta(f) \|_\infty
	\|x^\alpha(1+|x|)^{-N} \|_1
	<\infty
	\]
	
	$2\implies 1$:假设$f$对于任意$\alpha,\beta$,成立$x^\alpha D^\beta(f)\in L^1$。由引理\ref{lem:S等价定义引理}
	\[
	|D^\beta(f(x))| \lesssim \sum_{0\le j \le |\beta|+n+1}\int_{D(x,1)}|D^\gamma(f(y))|\dd y
	\]
	因此
	\begin{align*}
		(1+|x|)^N|D^\beta(f(x))|
		& \lesssim (1+|x|)^N\sum_{0\le j \le |\beta|+n+1}\int_{D(x,1)}|D^\gamma(f(y))|\dd y \\
		& \lesssim \sum_{0\le j \le |\beta|+n+1}\int_{D(x,1)}(1+|y|)^N|D^\gamma(f(y))|\dd y
	\end{align*}
	进而
	\[
	\|(1+|x|)^ND^\beta(f)\|_\infty
	\lesssim \sum_{0\le j \le |\beta|+n+1}\|(1+|x|)^ND^\gamma(f)\|_1
	\]
\end{proof}

下面我们给出 Fourier 变换实际上是$\mathcal{S}$上的连续变换。

\begin{theorem}
	若$f\in\mathcal{S}$,则$\widehat{f}\in\mathcal{S}$。进一步,映射$f\mapsto \widehat{f}$为$\mathcal{S}\to\mathcal{S}$的光滑映射。
\end{theorem}

\begin{proof}
	我们仅给出第一个证明。
	
	如果$f\in\mathcal{S}$,那么由定理\ref{thm:S等价定义},$f\in L^1$,进而$\widehat{f}\in L^\infty$。因此,若$f\in\mathcal{S}$,则$\widehat{D^\alpha (x^\beta f)}\in L^\infty$。然而,由命题\ref{pro:命题1.3}与\ref{pro:光滑则局域}
	\[
	\widehat{D^\alpha (x^\beta f)}(\xi)
	=(2\pi i)^{|\alpha|}(-2\pi i)^{-|\beta|}\xi^\alpha D^\beta\widehat{f}(\xi)
	\]
	因此$\xi^\alpha D^\beta(f)\in L^\infty$,这说明$\widehat{f}\in S$。
\end{proof}

\chapter{Fourier 逆变换与 Plancherel 定理}

\begin{definition}{卷积 convolution}
	$\phi$与$f$的卷积为
	\begin{gather}
		\label{23式}
		\phi*f(x)=\int \phi(y)f(x-y)\dd y
	\end{gather}
\end{definition}

卷积有一些良好的性质,我们一并回顾,但并不给出证明。

\begin{proposition}
	\begin{enumerate}
		\item 如果$\phi\in L^1$且$f\in L^p$,其中$1\le p \le\infty$,那么积分$(\ref{23式})$几乎处处绝对收敛,且
		\[
		\| \phi*f \|_p
		\le \|\phi\|_1 \|f\|_p
		\]
		\item 如果$\phi$为具有紧支集的连续函数,且$f\in L^1_{\text{loc}}$,那么积分$(\ref{23式})$处处绝对收敛,且$\phi*f$为连续函数。
		\item 如果$\phi\in L^p$,且$f\in L^q$,其中$1/p+1/q=1$,那么积分$(\ref{23式})$处处绝对收敛,且$\phi*f$为连续函数,同时
		\[
		\| \phi*f \|_\infty
		\le \|\phi\|_p \|f\|_q 
		\]
	\end{enumerate}
\end{proposition}

除此之外,我们还应注意到卷积是可交换的:$f*\phi=\phi*f$。同时注意到
\[
\supp(\phi*f)\sub \supp(\phi)+\supp(f)
\]
其中$E+F=\{ x+y:x\in E,y\in F \}$为集合的加法。

更多时候$\phi$是固定的,并且为性质非常好的函数。卷积更多作为一种算子
\[
f\longmapsto \phi*f
\]

\begin{lemma}
	如果$\phi\in C_0^\infty$且$f\in L^1_{\text{loc}}$,那么$\phi*f\in C^\infty$且
	\[
	D^\alpha(\phi*f)=(D^\alpha (\phi))*f
	\]
\end{lemma}


























\end{document}