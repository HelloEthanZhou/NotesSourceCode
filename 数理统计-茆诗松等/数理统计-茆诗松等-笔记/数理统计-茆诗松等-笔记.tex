\documentclass[lang = cn, scheme = chinese, thmcnt = section]{elegantbook}
% elegantbook      设置elegantbook文档类
% lang = cn        设置中文环境
% scheme = chinese 设置标题为中文
% thmcnt = section 设置计数器


%% 1.封面设置

\title{数理统计 - 茆诗松等 - 笔记}                % 文档标题

\author{若水}                        % 作者

\myemail{ethanmxzhou@163.com}       % 邮箱

\homepage{helloethanzhou.github.io} % 主页

\date{\today}                       % 日期

\logo{PiCreatures_happy.pdf}        % 设置Logo

\cover{阿基米德螺旋曲线.pdf}          % 设置封面图片

% 修改标题页的色带
\definecolor{customcolor}{RGB}{135, 206, 250} 
% 定义一个名为customcolor的颜色,RGB颜色值为(135, 206, 250)

\colorlet{coverlinecolor}{customcolor}     % 将coverlinecolor颜色设置为customcolor颜色

%% 2.目录设置
\setcounter{tocdepth}{3}  % 目录深度为3

%% 3.引入宏包
\usepackage[all]{xy}
\usepackage{bbm, svg, graphicx, float, extpfeil, amsmath, amssymb, mathrsfs, mathalpha, hyperref}


%% 4.定义命令
\newcommand{\N}{\mathbb{N}}            % 自然数集合
\newcommand{\R}{\mathbb{R}}            % 实数集合
\newcommand{\C}{\mathbb{C}}  		   % 复数集合
\newcommand{\Q}{\mathbb{Q}}            % 有理数集合
\newcommand{\Z}{\mathbb{Z}}            % 整数集合
\newcommand{\sub}{\subset}             % 包含
\newcommand{\im}{\text{im }}           % 像
\newcommand{\lang}{\langle}            % 左尖括号
\newcommand{\rang}{\rangle}            % 右尖括号
\newcommand{\bs}{\boldsymbol}          % 向量加黑
\newcommand{\dd}{\mathrm{d}}           % 微分d
\newcommand{\dis}{\displaystyle}
\newcommand{\pll}{\kern 0.56em/\kern -0.8em /\kern 0.56em} % 平行
\newcommand{\function}[5]{
	\begin{align*}
		#1:\begin{aligned}[t]
			#2 &\longrightarrow #3\\
			#4 &\longmapsto #5
		\end{aligned}
	\end{align*}
}                                     % 函数

\newcommand{\lhdneq}{%
	\mathrel{\ooalign{$\lneq$\cr\raise.22ex\hbox{$\lhd$}\cr}}} % 真正规子群

\newcommand{\rhdneq}{%
	\mathrel{\ooalign{$\gneq$\cr\raise.22ex\hbox{$\rhd$}\cr}}} % 真正规子群

%% 5.参考文献

\addbibresource[location=local]{reference.bib} % 添加本地的参考文献文件reference.bib

\begin{document}
	
\maketitle       % 创建标题页

\frontmatter     % 开始前言部分

\chapter*{致谢}

\markboth{致谢}{致谢}

\vspace*{\fill}
	\begin{center}
		
		\large{感谢 \textbf{ 勇敢的 } 自己}
		
	\end{center}
\vspace*{\fill}

\tableofcontents % 创建目录

\mainmatter      % 开始正文部分

\chapter{统计量及其分布}

\section{总体与概率}

\subsection{总体与个体}

\textbf{总体}:研究对象的全体

\textbf{个体}:构成总体的每个成员

\subsection{样本}

\textbf{样本}:从总体中随机的抽取$n$个个体,记其指标值为
$$
x_1,\cdots,x_n
$$
那么此称为总体的一个样本,$n$称为\textbf{样本容量},或简称样本量,样本中的个体称为\textbf{样品}。

\textbf{简单随机抽样原则}:随机性,独立性

\begin{definition}{联合分布函数}
	总体$X$具有分布函数$F(x)$,$x_1,\cdots,x_n$为取自该总体的容量为$n$的样本,那么样本联合分布函数为
	$$
	F(x_1,\cdots,x_n)=\prod_{k=1}^{n}{F(x_k)}
	$$
\end{definition}

\section{样本数据的整理与表示}

\subsection{经验分布函数}

\begin{definition}{经验分布函数}
	对于取自总体分布函数为$F(x)$的样本$x_1,\cdots,x_n$,记其对应的次序统计量为$x_{(1)},\cdots,x_{(n)}$,定义该样本的经验分布函数为
	$$
	F_n(x)=\begin{cases}
		0,&\quad x<x_{(1)}\\
		k/n,&\quad x_{(k)}\le x<x_{(k+1)},k=1,\cdots,n-1\\
		1,&\quad x\ge x_{(n)}
	\end{cases}
	$$
\end{definition}

\begin{proposition}{经验分布函数的性质}
	\begin{enumerate}
		\item $F_n(x)$非减且右连续。
		\item $F_n(-\infty)=0,\qquad F_n(+\infty)=1$
	\end{enumerate}
\end{proposition}

\begin{theorem}{Glivenko定理}
	对于取自总体分布函数为$F(x)$的样本$x_1,\cdots,x_n$,记其经验分布函数为$F_n(x)$,那么
	$$
	P\left(\lim_{n\to\infty}\sup_{x\in\R}{|F_n(x)-F(x)|}=0\right)=1
	$$
\end{theorem}

\subsection{频数频率分布表}

\begin{definition}{频数频率分布表}
	\begin{enumerate}
		\item 对样本进行分组:通常为$5\sim 20$个。
		\item 确定每组组距:
		$$
		组距d=\frac{\text{样本最大观测值}-\text{样本最小观测值}}{\text{组数}}
		$$
		\item 确定每组组限:
		$$
		(a_0,a_1],\cdots,(a_{n-1},a_n]
		$$
		\item 统计样本数据落入每个区间的个数——频数
	\end{enumerate}
\end{definition}

\begin{table}[H]
	\centering
	\renewcommand{\arraystretch}{1} % 调整行间距
	\caption{频数频率分布表}
	\begin{tabular}{@{\extracolsep{3pt}}ccc} % 调整列间距
		\toprule
		分组区间 & 频数 & 频率 \\
		\midrule
		$(a_0,a_1]$ & $f_1$ & $\frac{f_1}{\dis\sum_{k=1}^{n}{f_k}}$ \\
		$\vdots$ & $\vdots$ & $\vdots$ \\
		$(a_{n-1},a_n]$ & $f_n$ & $\frac{f_n}{\dis\sum_{k=1}^{n}{f_k}}$\\
		\bottomrule
	\end{tabular}
\end{table}

\section{统计量及其分布}

\subsection{统计量与抽样分布}

\begin{definition}{统计量}
	对于取自总体的样本$x_1,\cdots,x_n$,若$T=T(x_1,\cdots,x_n)$中不含有任何位置参数,那么称$T$为统计量。
\end{definition}

\begin{definition}{抽样分布}
	统计量的分布称为抽样分布。
\end{definition}

\subsection{样本均值极其抽样分布}

\begin{definition}{样本均值}
	对于取自总体的样本$x_1,\cdots,x_n$,其算术平均值称为样本均值,记为$\overline{x}$,即
	$$
	\overline{x}=\frac{1}{n}\sum_{k=1}^{n}{x_k}
	$$
	特别的,在分组样本中,样本均值的近似公式为
	$$
	\overline{x}=\frac{1}{n}\sum_{k=1}^{m}{x_kf_k}
	$$
	其中$m$为组数,$x_k$为第$k$组的组中值,$f_k$为第$k$组的频数,同时
	$$
	n=\sum_{k=1}^{n}{f_k}
	$$
\end{definition}

\begin{proposition}{样本均值的性质}
	\begin{enumerate}
		\item 样本的所有偏差之和为$0$,即
		$$
		\sum_{k=1}^{n}{(x_k-\overline{x})}=0
		$$
		\item 对于任意$c\in\R$,成立
		$$
		\sum_{k=1}^{n}{(x_k-\overline{x})^2}\le\sum_{k=1}^{n}{(x_k-c)^2}
		$$
		当且仅当$c=\overline x$时等号成立。
	\end{enumerate}
\end{proposition}

\begin{theorem}
	对于取自总体的样本$x_1,\cdots,x_n$,记其样本均值为$\overline{x}$。
	\begin{enumerate}
		\item 如果总体分布为$N(\mu,\sigma^2)$,那么$\overline{x}$满足分布$N(\mu,\frac{\sigma^2}{n})$。
		\item 对于一般的总体的分布,记$E(x)=\mu,\mathrm{Var}(x)=\sigma^2$,那么当$n\to\infty$时,$\overline{x}$满足近似分布$N(\mu,\frac{\sigma^2}{n})$,记作
		$$
		\overline{x}\dot{\sim}N(\mu,\frac{\sigma^2}{n})
		$$
	\end{enumerate}
\end{theorem}

\subsection{样本方差}

\begin{definition}{样本方差}
	对于取自总体的样本$x_1,\cdots,x_n$,定义其样本方差为
	$$
	s^2=\frac{1}{n-1}\sum_{k=1}^{n}(x_k-\overline{x})^2=\frac{1}{n-1}\left( \sum_{k=1}^{n}{x_k^2}-n\overline{x}^2 \right)
	$$
\end{definition}

\begin{theorem}
	对于具有二阶矩的总体$X$,即$E(X)=\mu$,$\mathrm{Var}(X)=\sigma^2<+\infty$,取自总体的样本$x_1,\cdots,x_n$,记$\overline{x}$和$s^2$分别为样本均值和样本方差,那么
	\begin{align*}
		&E(\overline{x})=\mu,\qquad \mathrm{Var}(\overline{x})=\frac{\sigma^2}{n}\\
		&E(s^2)=\sigma^2
	\end{align*}
\end{theorem}

\subsection{样本矩及其函数}

\begin{definition}{样本原点矩}
	对于取自总体的样本$x_1,\cdots,x_n$,定义其样本$k$阶原点矩为
	$$
	a_k=\frac{1}{n}\sum_{i=1}^{n}{x_i^k}
	$$
\end{definition}

\begin{definition}{样本中心矩}
	对于取自总体的样本$x_1,\cdots,x_n$,定义其样本$k$阶中心矩为
	$$
	b_k=\frac{1}{n}\sum_{i=1}^{n}{(x_i-\overline{x})^k}
	$$
\end{definition}

\begin{definition}{样本偏差}
	对于取自总体的样本$x_1,\cdots,x_n$,定义其样本偏差为
	$$
	\hat{\beta_s}=\frac{b_3}{b_2^{3/2}}
	$$
\end{definition}

\begin{proposition}{样本偏差的性质}
	样本偏差反应总体分布密度曲线的对称性。
	\begin{enumerate}
		\item $\hat{\beta_s}=0$:完全对称
		\item $\hat{\beta_s}>0$:存在右长尾
		\item $\hat{\beta_s}<0$:存在左长尾
	\end{enumerate}
\end{proposition}

\begin{definition}{样本峰度}
	对于取自总体的样本$x_1,\cdots,x_n$,定义其样本峰度为
	$$
	\hat{\beta_k}=\frac{b_3}{b_2^{\frac{3}{2}}}-3
	$$
\end{definition}

\begin{proposition}{样本峰度的性质}
	样本峰度反应总体分布密度曲线在其峰值附近的陡峭程度。
	\begin{enumerate}
		\item $\hat{\beta_k}>0$:比正态分布陡峭,称为尖顶型
		\item $\hat{\beta_k}<0$:比正态分布平缓,称为平顶型
	\end{enumerate}
\end{proposition}

\subsection{次序统计量及其分布}

\begin{definition}{次序统计量}
	对于取自总体的样本$x_1,\cdots,x_n$,称其次序统计量为
	$$
	x_{(1)},\cdots,x_{(n)}
	$$
	其中$x_{(1)}\le \cdots\le x_{(n)}$。
\end{definition}

\begin{theorem}{单个次序统计量的分布}
	对于取自总体的样本$x_1,\cdots,x_n$,如果$X$的密度函数为$p(x)$,分布函数为$F(x)$,那么第$k$个次序统计量$x_{(k)}$的密度函数为
	$$
	p_k(x)=\frac{n!}{(k-1)!(n-k)!}\left( F(x) \right)^{k-1}\left( 1-F(x) \right)^{n-k}p(x)
	$$
\end{theorem}

\begin{corollary}
	\begin{enumerate}
		\item $x_{(1)}$的密度函数为
		$$
		p_1(x)=np(x)(1-F(x))^{n-1}
		$$
		分布函数为
		$$
		F_1(x)=1-(1-F(x))^n
		$$
		\item $x_{(n)}$​的密度函数为
		$$
		p_n(x)=np(x)(F(x))^{n-1}
		$$
		分布函数为
		$$
		F_n(x)=(F(x))^n
		$$
	\end{enumerate}
\end{corollary}

\begin{theorem}{两个次序统计量的联合分布}
	对于取自总体的样本$x_1,\cdots,x_n$,如果$X$的密度函数为$p(x)$,分布函数为$F(x)$,那么第$i$个次序统计量$x_{(i)}$和第$j$个次序统计量$x_{(j)}$的联合分布密度函数为
	$$
	\small{p_{ij}(x,y)=\frac{n!}{(i-1)!(j-i-1)!(n-j)!}(F(x))^{i-1}(F(x)-F(y))^{j-i-1}(1-F(y))^{n-j}p(x)p(y)}
	$$
	其中$i<j$。
\end{theorem}

\subsection{样本分位数与分位数}

\begin{definition}{样本$p$分位数}
	对于取自总体的样本$x_1,\cdots,x_n$,定义其样本$p$分位数为
	$$
	m_p=\begin{cases}
		x_{([np+1])},\quad & np\notin\Z\\
		\frac{1}{2}\left(x_{(np)}+x_{(np+1)}\right),\quad & np\in\Z
	\end{cases}
	$$
	其中$p\in(0,1)$。
\end{definition}

\begin{definition}{$\alpha$分位数}
	对于随机变量$X$,称$x_\alpha$为其$\alpha$分位数,如果
	$$
	P(X\le x_\alpha)=\alpha
	$$
\end{definition}

\begin{theorem}
	如果总体密度函数为$p(x)$,$x_p$为其$p$分位数,$p(x)$在$x_p$处连续且$p(x_p)>0$,那么当$n\to+\infty$时,样本$p$分位数$m_p$的渐进分布为
	$$
	m_p\dot{\sim}N\left( x_p,\frac{p(1-p)}{np^2(x_p)} \right)
	$$
\end{theorem}

\subsection{五数概括与箱线图}

\begin{definition}{五数概括}
	$$
	x_{\min},\quad Q_1=m_{0.25},\quad m_{0.5},\quad Q_3=m_{0.75},\quad x_{\max}
	$$
\end{definition}

\begin{definition}{箱线图}
	\begin{enumerate}
		\item 画一个箱子,其两侧恰为第一$4$分位数和第三$4$分位数,在中位数位置上画一条竖线,其在箱子内,这个箱子包含了样本中$50\%$的数据。
		\item 在箱子左右两侧各引出一条水平线,分别至最小值和最大值为止。每条线段中包含了样本$25\%$的数据。
	\end{enumerate}
\end{definition}

\section{三大抽样分布}

\begin{table}[htbp]
	\centering
	\caption{三大抽样分布}
	\renewcommand{\arraystretch}{3}
	\resizebox{\linewidth}{!}{\begin{tabular}{|c|c|c|c|c|c|c|}
			\hline
			分布名称 & 表示 & 统计量的构造 & 抽样分布密度函数 & 期望 & 方差 & 特征函数 \\
			\hline
			正态分布 & $N(\mu,\sigma^2)$ &  & $p(x)=\frac{1}{\sqrt{2\pi}\sigma}\mathrm{e}^{-\frac{(x-\mu)^2}{2\sigma^2}},x\in\R$ & $\mu$ & $\sigma^2$ & $\mathrm{e}^{i\mu t-\frac{1}{2}\sigma^2 t^2}$ \\
			\hline
			$\Gamma$分布 & $\Gamma(\alpha,\lambda)$ &  & $p(x)=\frac{\lambda^\alpha}{\Gamma(\alpha)}x^{\alpha-1}\mathrm{e}^{-\lambda x},x>0$ & $\frac{\alpha}{\lambda}$ & $\frac{\alpha}{\lambda^2}$ & $\left( 1-\frac{it}{\lambda} \right)^{-\alpha}$ \\
			\hline
			$\chi^2$分布 & $\chi^2(n)$ & $\chi^2(n)=\sum_{k=1}^{n}{(N(0,1))^2}$ & $p(x)=\frac{1}{\Gamma\left(\frac{n}{2}\right)2^{\frac{n}{2}}}x^{\frac{n}{2}-1}\mathrm{e}^{-\frac{x}{2}},x>0$ & $n$ & $2n$ & $(1-2it)^{-\frac{n}{2}}$ \\
			\hline
			$F$分布 & $F(m,n)$ & $F(m,n)=\frac{\chi^2(m)/m}{\chi^2(n)/n}$ & $p(x)=\frac{\Gamma\left( \frac{m+n}{2} \right)\left( \frac{m}{n} \right)^{\frac{m}{2}}}{\Gamma\left( \frac{m}{2} \right)\Gamma\left( \frac{n}{2} \right)}x^{\frac{m}{2}-1}\left( 1+\frac{m}{n}x \right)^{-\frac{m+n}{2}},x>0$ & $\frac{n}{n-2}$ & $\frac{2n^2(m+n-2)}{m(n-2)^2(n-4)}$ &  \\
			\hline
			$t$分布 & $t(n)$ & $t(n)=\frac{N(0,1)}{\sqrt{\chi^2(n)/n}}$ & $p(x)=\frac{\Gamma\left( \frac{n+1}{2} \right)}{\sqrt{n\pi}\Gamma\left( \frac{n}{2} \right)}\left( 1+\frac{x^2}{n} \right)^{-\frac{n+1}{2}},x\in\R$ & $0$ & $\frac{n}{n-2}$ &  \\
			\hline
		\end{tabular}}
\end{table}

\begin{definition}{$\Gamma$函数}
	$$
	\Gamma(x)=\int_0^{\infty}{t^{x-1}\mathrm{e}^{-t}\mathrm{d}t}
	$$
\end{definition}

\begin{proposition}{分布间的联系}
	\begin{enumerate}
		\item 若$X\sim N(0,1)$,那么
		$$
		X^2\sim \Gamma\left( \frac{1}{2},\frac{1}{2} \right)
		$$
		因此
		$$
		\chi^2(n)=\Gamma\left(\frac{n}{2},\frac{1}{2}\right)
		$$
		\item 如果$X\sim N(\mu,\sigma^2)$,那么
		$$
		aX+b\sim N(a\mu+b,a^2\sigma^2)
		$$
		\item 如果独立分布$X\sim N(\mu_1,\sigma_1^2)$和$Y\sim N(\mu_2,\sigma^2_2)$,那么
		$$
		X+Y\sim N(\mu_1+\mu_2,\sigma_1^2+\sigma_2^2)
		$$
		\item 如果$X\sim\Gamma(\alpha,\lambda)$,那么
		$$
		aX\sim \Gamma(\alpha,\frac{\lambda}{a})
		$$
		\item 如果独立分布$X\sim\Gamma(\alpha_1,\lambda)$和$Y\sim\Gamma(\alpha_2,\lambda)$,那么
		$$
		X+Y\sim \Gamma(\alpha_1+\alpha_2,\lambda)
		$$
	\end{enumerate}
\end{proposition}

\subsection{$\chi^2$分布}

\begin{definition}{$\chi^2$分布}
	对于独立同分布于标准正态分布的$N(0,1)$的随机变量$X_1,\cdots,X_n$,称随机变量$X=X_1^2+\cdots+X_n^2$的分布为自由度为$n$的$\chi^2$分布,记作$X\sim \chi^2(n)$,其密度函数为
	$$
	p(x)=\frac{1}{\Gamma\left(\frac{n}{2}\right)2^{\frac{n}{2}}}x^{\frac{n}{2}-1}\mathrm{e}^{-\frac{x}{2}},\qquad x>0
	$$
\end{definition}

\begin{proposition}{$\chi^2$分布的性质}
	对于来自正态总体$N(\mu,\sigma^2)$的样本$x_1,\cdots,x_n$,记其样本均值和样本方差分别为$\overline{x}$和$s^2$,那么
	\begin{enumerate}
		\item $\overline{x}$和$s^2$相互独立。
		\item 
		$$
		\overline{x}\sim N(\mu,\frac{\sigma^2}{n})
		$$
		\item 
		$$
		\frac{(n-1)s^2}{\sigma^2}\sim \chi^2(n-1)=\Gamma\left(\frac{n-1}{2},\frac{1}{2}\right)
		$$
		即
		$$
		s^2\sim \Gamma\left(\frac{n-1}{2},\frac{n-1}{2\sigma^2}\right)
		$$
	\end{enumerate}
\end{proposition}

\subsection{$F$分布}

\begin{definition}{$F$分布}
	对于独立的随机变量$X\sim\chi^2(m)$和$Y\sim\chi^2(n)$,称随机变量$F=\frac{X/m}{Y/n}$的分布为自由度为$m$和$n$的$F$分布,记作$F\sim F(m,n)$​,其密度函数为
	$$
	p(x)=\frac{\Gamma\left( \frac{m+n}{2} \right)\left( \frac{m}{n} \right)^{\frac{m}{2}}}{\Gamma\left( \frac{m}{2} \right)\Gamma\left( \frac{n}{2} \right)}x^{\frac{m}{2}-1}\left( 1+\frac{m}{n}x \right)^{-\frac{m+n}{2}},x>0
	$$
\end{definition}

\begin{corollary}
	对于独立的分别来自$N(\mu_1,\sigma^2_1)$和$N(\mu_2,\sigma^2_2)$的样本$x_1,\cdots,x_m$和$y_1,\cdots,y_m$,那么记其样本方差分别为$s_x^2$和$s_y^2$,那么
	$$
	\frac{s_x^2/\sigma^2_1}{s_y^2/\sigma^2_2}\sim F(m-1,n-1)
	$$
\end{corollary}

\subsection{$T$分布}

\begin{definition}{$T$分布}
	对于独立的随机变量$X\sim N(0,1)$和$Y\sim \chi^2(n)$,称随机变量$t=\frac{X}{\sqrt{\frac{Y}{n}}}$的分布为自由度为$n$的$t$分布,记作$t\sim t(n)$​,其密度函数为
	$$
	p(x)=\frac{\Gamma\left( \frac{n+1}{2} \right)}{\sqrt{n\pi}\Gamma\left( \frac{n}{2} \right)}\left( 1+\frac{x^2}{n} \right)^{-\frac{n+1}{2}},x\in\R
	$$
\end{definition}

\begin{corollary}
	对于来自正态分布$N(\mu,\sigma^2)$的样本$x_1,\cdots,x_n$,记其样本均值和样本方差分别为$\overline{x}$和$s^2$,那么
	$$
	\frac{\sqrt{n}(\overline{x}-\mu)}{s}\sim T(n-1)
	$$
\end{corollary}

\begin{corollary}
	对于独立的分别来自$N(\mu_1,\sigma^2)$和$N(\mu_2,\sigma^2)$的样本$x_1,\cdots,x_m$和$y_1,\cdots,y_m$,那么记其样本方差分别为$s_x^2$和$s_y^2$,且
	$$
	s_w^2=\frac{(m-1)s_x^2+(n-1)s_y^2}{m+n-2}
	$$
	那么
	$$
	\frac{(\overline{x}-\overline{y})-(\mu_1-\mu_2)}{s_w\sqrt{\frac{1}{m}+\frac{1}{n}}}\sim T(m+n-2)
	$$
\end{corollary}

\section{充分统计量}

\subsection{充分性}

\begin{definition}{充分统计量}
	对于来自总体分布函数为$F(x;\theta)$的样本$x_1,\cdots,x_n$,称统计量$T=T(x_1,\cdots,x_n)$为$\theta$的充分统计量,如果给定$T$的取值后,样本$x_1,\cdots,x_n$的条件分布与$\theta$无关。
\end{definition}

\subsection{因子分解定理}

\begin{theorem}{Fischer-Neyman因子分解定理}
	对于来自总体概率函数为$f(x;\theta)$的样本$x_1,\cdots,x_n$,那么$T=T(x_1,\cdots,x_n)$为充分统计量的充分必要条件为,存在函数$g(t,\theta)$和$h(x_1,\cdots,x_n)$,使得对于任意的$\theta$和$x_1,\cdots,x_n$,成立
	$$
	f(x_1,\cdots,x_n;\theta)=g(T(x_1,\cdots,x_n);\theta)h(x_1,\cdots,x_n)
	$$
\end{theorem}

\begin{theorem}
	对于充分统计量$T$,如果存在函数$h$,使得$T=h(S)$,那么统计量$S$也为充分统计量。
\end{theorem}

\chapter{参数估计}

\section{点估计的概念}

\subsection{点估计及无偏性}

\begin{definition}{点估计}
	对于来自总体的样本$x_1,\cdots,x_n$,用于估计未知参数$\theta$的统计量$\hat{\theta}=\hat{\theta}(x_1,\cdots,x_n)$称为$\theta$的点估计。
\end{definition}

\begin{definition}{无偏估计}
	对于$\theta$的点估计$\hat{\theta}=\hat{\theta}(x_1,\cdots,x_n)$,$\theta$的参数空间为$\Theta$,称$\hat{\theta}$为$\theta$的无偏估计,如果对于任意$\theta\in\Theta$,成立
	$$
	E_{\theta}(\hat{\theta})=\theta
	$$
\end{definition}

\begin{definition}{可估参数}
	称参数$\theta$为可估参数,如果存在无偏估计$\hat{\theta}=\hat{\theta}(x_1,\cdots,x_n)$。
\end{definition}

\subsection{有效性}

\begin{definition}{有效性}
	对于$\theta$的两个无偏估计$\hat{\theta}_1$和$\hat{\theta}_2$,称$\hat{\theta}_1$比$\hat{\theta}_2$有效,如果对于任意$\theta\in\Theta$,成立
	$$
	\mathrm{Var}(\hat{\theta}_1)\le\mathrm{Var}(\hat{\theta}_2)
	$$
	且存在$\theta_0\in\Theta$,使得成立
	$$
	\mathrm{Var}(\hat{\theta}_1)<\mathrm{Var}(\hat{\theta}_2)
	$$
\end{definition}

\section{矩估计及相关性}

\subsection{替换原理和矩法估计}

\begin{definition}{替换原理}
	\begin{enumerate}
		\item 用样本矩替换总体矩。
		\item 用样本矩的函数替换总体矩的函数。
	\end{enumerate}	
	根据替换原理,在总体分布形式未知场合对参数作出估计:
	\begin{enumerate}
		\item 用样本均值$\overline{x}$估计总体均值$E(X)$。
		\item 用样本方差$s^2$估计总体方差$\mathrm{Var}(X)$。
		\item 用事件$A$出现的频率估计事件$A$发生的概率。
		\item 用样本$p$分位数估计总体的$p$分位数。
	\end{enumerate}
\end{definition}

\begin{theorem}{Khinchin大数定律}{Khinchin大数定律}
	对于独立同分布的随机变量序列$X_1,\cdots,X_n$,如果对于任意$i=1,\cdots,n$,总体$X$的$k$阶原点矩$E(X^k)$存在,那么对于任意$\varepsilon>0$,成立
	$$
	\lim_{n\to\infty}P\left\{ \left| \frac{1}{n}\sum_{i=1}^{n}{X_i^k}-E(X^k) \right|\ge\varepsilon \right\}=0
	$$
\end{theorem}

\subsection{概率函数已知时未知参数的矩估计}

\begin{definition}{矩估计}
	对于具有概率函数$p(x;\theta_1,\cdots,\theta_k)$的总体,以及样本$x_1,\cdots,x_n$,其中$(\theta_1,\cdots,\theta_k)\in\Theta$是未知参数或参数向量,如果总体的$i$阶原点矩$\mu_i$存在,而且$\theta_i=\theta_i(\mu_1,\cdots,\mu_k)$,其中$1\le i\le k$,那么$\theta_i$的矩估计为
	$$
	\hat{\theta}_i=\theta_i(a_1,\cdots,a_k),\quad i=1,\cdots,k
	$$
	其中$a_i$为样本$i$阶原点矩
	$$
	a_i=\frac{1}{n}\sum_{j=1}^{n}{x_j^i},\quad i=1,\cdots,k
	$$
	进一步,对于$\theta_1,\cdots,\theta_k$的函数$\eta=g(\theta_1,\cdots,\theta_k)$的矩估计为
	$$
	\hat{\eta}=g(\hat{\theta}_1,\cdots,\hat{\theta}_k)
	$$
\end{definition}

\subsection{相合性}

\begin{definition}{相合性}
	对于未知参数$\theta$,以及$\theta$的一个估计量$\hat{\theta}_n=\hat{\theta}_n(x_1,\cdots,x_n)$,称$\hat{\theta}_n$为参数$\theta$的相合估计,如果对于任意$\varepsilon>0$,成立
	$$
	\lim_{n\to\infty}{P\left( \left|\hat{\theta}_n-\theta\right|\ge\varepsilon \right)}=0
	$$
\end{definition}

\begin{theorem}{相合估计的充分条件}
	对于$\theta$的一个估计量$\hat{\theta}_n=\hat{\theta}_n(x_1,\cdots,x_n)$,如果
	$$
	\lim_{n\to\infty}{E( \hat{\theta}_n)}=\theta,\qquad
	\lim_{n\to\infty}{\mathrm{Var}( \hat{\theta}_n)}=0
	$$
	那么$\hat{\theta}_n$为参数$\theta$的相合估计。
\end{theorem}

\begin{theorem}{相合估计在连续函数下的像为相合估计}
	如果$\hat{\theta}_{n_1},\cdots,\hat{\theta}_{n_k}$分别是$\theta_1,\cdots,\theta_k$的相合估计,$\eta=g(\theta_1,\cdots,\theta_k)$是连续函数,那么$\hat{\eta}=g(\hat{\theta}_{n_1},\cdots,\hat{\theta}_{n_k})$是$\eta$的相合估计。
\end{theorem}

\section{最大似然估计}

\subsection{最大似然估计}

\begin{definition}{似然函数}
	对于概率函数为$p(x;\theta)$的总体,其中$\theta\in\Theta$为一个或多个未知参数组成的参数向量,$\Theta$为参数空间,$x_1,\cdots,x_n$是来自该总体的样本,称样本的联合概率函数
	$$
	L(\theta)=L(\theta;x_1,\cdots,x_n)=\prod_{k=1}^{n}{p(x_k;\theta)}
	$$
	为样本的似然函数。
\end{definition}

\begin{definition}{最大似然估计 MLE}
	对于概率函数为$p(x;\theta)$的总体,其中$\theta\in\Theta$为一个或多个未知参数组成的参数向量,$\Theta$为参数空间,$x_1,\cdots,x_n$是来自该总体的样本,统计量$\hat{\theta}=\hat{\theta}(x_1,\cdots,x_n)$为$\theta$的最大似然估计,如果对于任意$\theta\in\Theta$,成立
	$$
	L(\hat{\theta})\ge L(\theta)
	$$
\end{definition}

\begin{theorem}{最大似然估计的不变性}
	如果$\hat{\theta}$为$\theta$的最大似然估计,那么对于任意函数$g$,$g(\hat{\theta})$是$g(\theta)$的最大似然估计。
\end{theorem}

\begin{theorem}{正态分布参数的最大似然估计}
	对于来自正态分布$N(\mu,\sigma^2)$的样本$x_1,\cdots,x_n$,记样本均值为$\overline{x}$,样本方差为$s^2$,那么$\mu$和$\sigma^2$的最大似然估计分别为
	$$
	\hat{\mu}=\overline{x},\qquad
	\hat{\sigma}^2=\frac{n-1}{n}s^2
	$$
\end{theorem}

\subsection{渐进正态性}

\begin{definition}{渐进正态分布}
	参数$\theta$的相合估计$\hat{\theta}_n$称为渐进正态的,如果存在趋于$0$的非负常数序列$\sigma_n(\theta)$,使得成立$\frac{\hat{\theta}_n-\theta}{\sigma_n(\theta)}$依分布收敛于标准正态分布。此时也称$\hat{\theta}_n$服从渐进正态分布$N(\theta,\sigma^2_n(\theta))$,记为$\hat{\theta}_n\sim AN(\theta,\sigma^2_n(\theta))$。$\sigma^2_n(\theta)$称为$\hat{\theta}_n$的渐近方差。
\end{definition}

\begin{theorem}
	对于密度函数为$p(x;\theta)$的总体$X$,其中$\theta\in\Theta$,如果
	\begin{enumerate}
		\item 对于任意$x$,以及任意$\theta\in\Theta$,偏导数$\frac{\partial \ln{p}}{\partial \theta}$,$\frac{\partial^2 \ln{p}}{\partial \theta^2}$和$\frac{\partial^3 \ln{p}}{\partial \theta^3}$都存在。
		\item 对于任意$\theta\in\Theta$,成立
		$$
		\left| \frac{\partial p}{\partial \theta} \right|<F_1(x),\qquad 
		\left| \frac{\partial^2 p}{\partial \theta^2} \right|<F_2(x),\qquad 
		\left| \frac{\partial^3 p}{\partial \theta^3} \right|<F_3(x)
		$$
		其中函数$F_1(x),F_2(x),F_3(x)$满足
		\begin{align*}
			& \int_{-\infty}^{\infty}{F_1(x)\mathrm{d}x}<\infty,\qquad
			\int_{-\infty}^{\infty}{F_2(x)\mathrm{d}x}<\infty\\
			& \sup_{\theta\in\Theta}{\int_{-\infty}^{\infty}{F_3(x)p(x;\theta)\mathrm{d}x}}<\infty
		\end{align*}
		\item 对于任意$\theta\in\Theta$,成立
		$$
		0<I(\theta)=\int_{-\infty}^{\infty}{\left(\frac{\partial \ln{p}}{\partial \theta}\right)^2p(x;\theta)\mathrm{d}x}<\infty
		$$
	\end{enumerate}
	那么对于来自该总体的样本$x_1,\cdots,x_n$,存在未知参数$\theta$的最大似然估计$\hat{\theta}_n=\hat{\theta}_n(x_1,\cdots,x_n)$,且$\hat{\theta}_n$具有相合性和渐近正态性,同时
	$$
	\hat{\theta}_n\sim AN\left(\theta,\frac{1}{n I(\theta)}\right)
	$$
\end{theorem}

\section{最小方差无偏估计}

\subsection{均方误差}

\begin{definition}{均方误差}
	对于$\theta$的点估计$\hat{\theta}=\hat{\theta}(x_1,\cdots,x_n)$,称$E( \hat{\theta}-\theta)^2$为$\hat{\theta}$关于$\theta$的均方误差,记为$\mathrm{MSE}(\hat{\theta},\theta)$,或$\mathrm{M}_{\theta}(\hat{\theta})$。
\end{definition}

\begin{proposition}{均方误差的性质}
	\begin{enumerate}
		\item - 对于$\theta$的任意估计$\hat{\theta}$而言,成立
		$$
		\mathrm{MSE}(\hat{\theta},\theta)=\mathrm{Var}(\hat{\theta})+( E(\hat{\theta})-\theta )^2
		$$
		\item 对于$\theta$的无偏估计$\hat{\theta}$而言,成立
		$$
		\mathrm{MSE}(\hat{\theta},\theta)=\mathrm{Var}(\hat{\theta})
		$$
	\end{enumerate}
\end{proposition}

\begin{definition}{一致最小均方误差估计}
	对于样本$x_1,\cdots,x_n$,以及待估参数$\theta$的一个估计类,称$\hat{\theta}(x_1,\cdots,x_n)$是该估计类中$\theta$中的一致最小均方误差估计,如果对于该估计类中另外任意一个$\theta$的估计$\tilde{\theta}$,在参数空间$\Theta$上均成立
	$$
	\mathrm{MSE}_{\theta}(\hat{\theta})\le\mathrm{MSE}_{\theta}(\tilde{\theta})
	$$
\end{definition}

\subsection{一致最小方差无偏估计}

\begin{definition}{一致最小方差无偏估计 UMVUE}
	对于$\theta$的一个无偏估计$\hat{\theta}$,称$\hat{\theta}$是$\theta$的一致最小方差无偏估计,如果对于$\theta$的任意无偏估计$\tilde{\theta}$,在参数空间$\Theta$上均成立
	$$
	\mathrm{Var}_{\theta}(\hat{\theta})\le\mathrm{Var}_{\theta}(\tilde{\theta})
	$$
\end{definition}

\begin{theorem}
	对于来自某总体的样本$X=(x_1,\cdots,x_n)$,如果$\hat{\theta}=\hat{\theta}(X)$是$\theta$的一个无偏估计,$\mathrm{Var}(\hat{\theta})<\infty$,那么$\hat{\theta}$是$\theta$的一致最小方差无偏估计的充分必要条件是,对于任意满足$E(\varphi(X))=0$和$\mathrm{Var}(\varphi(X))<\infty$的$\varphi(X)$,以及任意$\theta\in\Theta$,成立
	$$
	\mathrm{Cov}_{\theta}(\hat{\theta},\varphi)=0
	$$
	即
	$$
	E(\hat{\theta}\varphi)=0
	$$
\end{theorem}

\subsection{充分性原则}

\begin{theorem}
	对于来自总体概率密度函数为$p(x;\theta)$的样本$x_1,\cdots,x_n$,如果$T=T(x_1,\cdots,x_n)$是$\theta$的充分统计量,那么对于$\theta$的任意无偏估计$\hat{\theta}=\hat{\theta}(x_1,\cdots,x_n)$,成立$\tilde{\theta}=E(\hat{\theta}|T)$是$\theta$的无偏估计,且
	$$
	\mathrm{Var}(\tilde{\theta})\le\mathrm{Var}(\hat{\theta})
	$$
\end{theorem}

\subsection{Cramer-Rao不等式}

\begin{definition}{Fisher信息量}
	对于满足如下条件的概率函数为$p(x;\theta),\theta\in\Theta$的总体
	\begin{enumerate}
		\item 参数空间$\Theta$是直线上的一个开区间。
		\item 支撑$S=\{ x:p(x;\theta)>0 \}$与$\theta$无关。
		\item 导数$\frac{\partial}{\partial \theta}p(x;\theta)$对任意$\theta\in\Theta$均存在。
		\item 对于$p(x;\theta)$,积分与微分运算可交换次序,即
		$$
		\int_{-\infty}^{\infty}{\frac{\partial}{\partial \theta}p(x;\theta)\mathrm{d}x}=
		\frac{\partial}{\partial \theta}\int_{-\infty}^{\infty}{p(x;\theta)\mathrm{d}x}=
		0
		$$
		\item 期望$E\left( \frac{\partial}{\partial \theta}\ln{p(x;\theta)} \right)^2$存在。
	\end{enumerate}
	称
	$$
	I(\theta)=E\left( \frac{\partial}{\partial \theta}\ln{p(x;\theta)} \right)^2
	$$
	为总体分布的Fisher信息量。如果二阶导数$\frac{\partial ^2}{\partial \theta^2}p(x;\theta)$对于任意$\theta\in\Theta$存在,那么
	$$
	I(\theta)=-E\left( \frac{\partial^2}{\partial \theta^2}\ln{p(x;\theta)} \right)
	$$
\end{definition}

\begin{theorem}{Cramer-Rao不等式}
	对于满足Fisher信息量定义的总体分布$p(x;\theta)$,$X=(x_1,\cdots,x_n)$是来自该总体的样本,如果$T=T(X)$是$g(\theta)$的任意无偏估计,即
	$$
	g(\theta)=\int_{\R^n}{T(X)L(X;\theta)\mathrm{d}X}
	$$
	其中$L(x_1,\cdots,x_n;\theta)$为$X=(x_1,\cdots,x_n)$的总体概率密度函数
	$$
	L(X;\theta)=\prod_{k=1}^{n}{p(x_k;\theta)}
	$$
	并且$g'(\theta)=\frac{\partial g(\theta)}{\partial \theta}$存在,同时对于任意$\theta\in\Theta$,$g(\theta)$的微商可在积分号下进行,即
	$$
	g'(\theta)=\int_{\R^n}{T(X)\frac{\partial}{\partial\theta}L(X;\theta)\mathrm{d}X}
	$$
	(对于离散总体,将上述积分号改为求和符号)那么
	$$
	\mathrm{Var}(T)\ge\frac{(g'(\theta))^2}{nI(\theta)}
	$$
	其中$I(\theta)$为总体分布的Fisher信息量,$\frac{(g'(\theta))^2}{nI(\theta)}$称为$g(\theta)$的无偏估计的方差的C-R下界。当等号成立时,称$T=T(X)$为$g(\theta)$的{\bf 有效估计},有效估计一定是一致最小方差无偏估计。
\end{theorem}

\section{Bayes估计}

\section{统计判断的基础}

\textbf{Bayes学派}基本观点:任意未知量都可看作随机变量,可用一个概率分布去描述,这个分布称为先验分布。

\subsection{Bayes公式的密度函数形式}

\begin{theorem}{Bayes公式的密度函数形式}
	\begin{enumerate}
		\item $p(x\mid\theta)$表示随机变量$\theta$取给定值时总体的条件概率函数。
		\item 根据参数$\theta$的先验信息确定先验分布$\pi(\theta)$。
		\item 样本$X=(x_1,\cdots,x_n)$的产生分两步进行,首先设想从先验分布$\pi(\theta)$产生一个个体$\theta_0$,其次从$p(X\mid\theta)$中产生一组样本,此时样本$X$的联合条件概率函数为
		$$
		P(X\mid\theta_0)=\prod_{k=1}^{n}{p(x_k\mid\theta)}
		$$
		\item 由于$\theta_0$是设想出来的,因此需要考虑$\pi(\theta)$,那么样本$X$和参数$\theta$的联合分布为
		$$
		h(X,\theta)=P(X\mid\theta)\pi(\theta)
		$$
		\item 将$h(X,\theta)$分解为
		$$
		h(X,\theta)=\pi(\theta\mid X)m(X)
		$$
		其中$m(X)$为$X$的边际概率函数
		$$
		m(X)=\int_{\Theta}{h(X,\theta)\mathrm{d}\theta}=
		\int_{\Theta}{P(X\mid\theta)\pi(\theta)\mathrm{d}\theta}
		$$
		进而$\theta$的后验分布为
		$$
		\pi(\theta\mid X)=\frac{h(X,\theta)}{m(X)}=\frac{P(X\mid\theta)\pi(\theta)}{\int_{\Theta}{P(X\mid\theta)\pi(\theta)\mathrm{d}\theta}}
		$$
	\end{enumerate}
\end{theorem}

\subsection{Bayes估计}

\begin{definition}{Bayes估计}
	由后验分布$\pi(\theta\mid X)$估计$\theta$有三种常用的方法:
	\begin{enumerate}
		\item 最大后验估计:后验分布的密度函数的最大值点。
		\item 后验中位数估计:后验分布的中位数。
		\item 后验期望估计:后验分布的均值。
	\end{enumerate}
	称后验期望估计为Bayes估计,记为$\hat{\theta}$。
\end{definition}

\subsection{共轭先验分布}

\begin{definition}{共轭先验分布}
	对于总体分布$p(x;\theta)$中的参数$\theta$,$\pi(\theta)$是其先验分布,如果对于任意来自该总体的样本观测值得到的后验分布$\pi(\theta\mid X)$与$\pi(\theta)$属于同一个分布族,那么称该分布族为$\theta$的共轭先验分布(族)。
\end{definition}

\section{区间估计}

\subsection{区间估计的概念}

\begin{definition}{置信区间}
	对于总体的参数$\theta\in\Theta$,以及来自该总体的样本$x_1,\cdots,x_n$,给定$\alpha\in(0,1)$,如果两个统计量$\hat{\theta}_{L}=\hat{\theta}_{L}(x_1,\cdots,x_n)$和$\hat{\theta}_{U}=\hat{\theta}_{U}(x_1,\cdots,x_n)$,满足对于任意$\theta\in\Theta$,成立
	$$
	P_{\theta}(\hat{\theta}_{L}\le\theta\le\hat{\theta}_{U})\ge 1-\alpha
	$$
	那么称随机区间$[\hat{\theta}_{L},\hat{\theta}_{U}]$为$\theta$的置信水平为$1-\alpha$的置信区间,或简称$[\hat{\theta}_{L},\hat{\theta}_{U}]$为$\theta$的$1-\alpha$置信区间。其中$\hat{\theta}_{L}$和$\hat{\theta}_{U}$分别称为$\theta$的(双侧)置信下限和置信上限。
\end{definition}

\begin{definition}{同等置信区间}
	对于总体的参数$\theta\in\Theta$,以及来自该总体的样本$x_1,\cdots,x_n$,给定$\alpha\in(0,1)$,如果两个统计量$\hat{\theta}_{L}=\hat{\theta}_{L}(x_1,\cdots,x_n)$和$\hat{\theta}_{U}=\hat{\theta}_{U}(x_1,\cdots,x_n)$,满足对于任意$\theta\in\Theta$,成立
	$$
	P_{\theta}(\hat{\theta}_{L}\le\theta\le\hat{\theta}_{U})=1-\alpha
	$$
	那么称随机区间$[\hat{\theta}_{L},\hat{\theta}_{U}]$为$\theta$的置信水平为$1-\alpha$的同等置信区间。
\end{definition}

\begin{definition}{单侧置信下限}
	对于总体的参数$\theta\in\Theta$,以及来自该总体的样本$x_1,\cdots,x_n$,给定$\alpha\in(0,1)$,如果统计量$\hat{\theta}_{L}=\hat{\theta}_{L}(x_1,\cdots,x_n)$满足对于任意$\theta\in\Theta$,成立
	$$
	P_{\theta}(\hat{\theta}_{L}\le\theta)\ge 1-\alpha
	$$
	那么称$\hat{\theta}_{L}$为$\theta$的(单侧)置信下限。
\end{definition}

\begin{definition}{单侧置信上限}
	对于总体的参数$\theta\in\Theta$,以及来自该总体的样本$x_1,\cdots,x_n$,给定$\alpha\in(0,1)$,如果统计量$\hat{\theta}_{U}=\hat{\theta}_{U}(x_1,\cdots,x_n)$满足对于任意$\theta\in\Theta$,成立
	$$
	P_{\theta}(\hat{\theta}_{U}\ge\theta)\ge 1-\alpha
	$$
	那么称$\hat{\theta}_{U}$为$\theta$的(单侧)置信上限。
\end{definition}

\subsection{枢轴量法}

\begin{theorem}{构造枢轴量的方法}
	\begin{enumerate}
		\item 构造函数$G=G(x_1,\cdots,x_n,\theta)$,使得$G$的分布不依赖于$\theta$,此函数$G$称为{\bf 枢轴量}。
		\item 选择常数$a,b$,使得对于给定$\alpha\in(0,1)$,使得成立
		$$
		P(a\le G\le b)=1-\alpha
		$$
		\item 将不等式$a\le G\le b$等价变形为$\hat{\theta}_{L}\le\theta\le\hat{\theta}_{U}$​,即
		$$
		P_{\theta}(\hat{\theta}_{L}\le\theta\le\hat{\theta}_{U})=1-\alpha
		$$
		那么区间$[\hat{\theta}_{L},\hat{\theta}_{U}]$为$\theta$的置信水平为$1-\alpha$同等置信区间。
		\item 其中常数$a,b$的选择应该使得区间$[\hat{\theta}_{L},\hat{\theta}_{U}]$​的长度最短,否则使得成立
		$$
		P(G<a)=P(G>b)=\frac{\alpha}{2}
		$$
		称这样得到的置信区间$[\hat{\theta}_{L},\hat{\theta}_{U}]$为{\bf 等尾置信区间}。
	\end{enumerate}
\end{theorem}

\subsection{单个正态总体参数的置信区间}

\begin{table}[htbp]
	\centering
	\caption{单个正态总体参数的置信区间}
	\renewcommand{\arraystretch}{2}
	\resizebox{\linewidth}{!}{
	\begin{tabular}{|c|c|c|c|c|}
		\hline
		目标 & 条件 & 枢轴量 & 分布 & 置信区间 \\ \hline
		$\mu$ & $\sigma$已知 & $\frac{\sqrt{n}(\overline{x}-\mu)}{\sigma}$ & $N(0,1)$ & $\left[\overline{x}-\frac{\sigma}{\sqrt{n}}n_{1-\frac{\alpha}{2}},\quad\overline{x}+\frac{\sigma}{\sqrt{n}}n_{1-\frac{\alpha}{2}}\right]$ \\ \hline
		$\mu$ & $\sigma$未知 & $\frac{\sqrt{n}(\overline{x}-\mu)}{s}$ & $T(n-1)$ & $\left[\overline{x}-\frac{s}{\sqrt{n}}t_{1-\frac{\alpha}{2}},\quad\overline{x}+\frac{s}{\sqrt{n}}t_{1-\frac{\alpha}{2}}\right]$ \\ \hline
		$\sigma^2$ & $\mu$未知 & $\frac{(n-1)s^2}{\sigma^2}$ & $\chi^2(n-1)$ & $\left[\frac{(n-1)s^2}{\chi^2_{1-\frac{\alpha}{2}}},\quad\frac{(n-1)s^2}{\chi^2_{\frac{\alpha}{2}}}\right]$ \\ \hline
	\end{tabular}
	}
\end{table}

\subsection{大样本置信区间}

在有些场合,寻找枢轴量及其分布比较困难。在样本量充分大时,可用渐进分布来构造近似的置信区间。以下为二点分布关于比例$p$的置信区间。

对于来自二点分布$b(1,p)$的样本$x_1,\cdots,x_n$,由中心极限定理
$$
N=\frac{\sqrt{n}(\overline{x}-p)}{\sqrt{p(1-p)}}\dot{\sim}N(0,1)
$$
因此置信水平为$1-\alpha$的同等置信区间为
{\small $$
	\left[\frac{1}{1+\frac{n^2_{1-\frac{\alpha}{2}}}{n}}\left( \overline{x}+\frac{n^2_{1-\frac{\alpha}{2}}}{2n}-\sqrt{\frac{\overline{x}(1-\overline{x})}{n}n^2_{1-\frac{\alpha}{2}}+\left( \frac{n^2_{1-\frac{\alpha}{2}}}{2n} \right)^2} \right),\qquad
	\frac{1}{1+\frac{n^2_{1-\frac{\alpha}{2}}}{n}}\left( \overline{x}+\frac{n^2_{1-\frac{\alpha}{2}}}{2n}+\sqrt{\frac{\overline{x}(1-\overline{x})}{n}n^2_{1-\frac{\alpha}{2}}+\left( \frac{n^2_{1-\frac{\alpha}{2}}}{2n} \right)^2} \right)\right]
	$$}
其中$n_{1-\frac{\alpha}{2}}$为$N(0,1)$的$1-\frac{\alpha}{2}$分位数。由于$n$充分大,略去$\frac{n^2_{1-\frac{\alpha}{2}}}{n}$项,因此置信水平为$1-\alpha$的同等置信区间近似为
$$
\left[\overline{x}-n_{1-\frac{\alpha}{2}}\sqrt{\frac{\overline{x}(1-\overline{x})}{n}},\quad
\overline{x}+n_{1-\frac{\alpha}{2}}\sqrt{\frac{\overline{x}(1-\overline{x})}{n}}\right]
$$

\subsection{样本量的确定}

\begin{definition}{保证概率}
	称置信水平$1-\alpha$为保证概率。
\end{definition}

\begin{definition}{绝对误差}
	称置信区间的半径(即长度的一半)为绝对误差。
\end{definition}

\subsection{两个正态总体下的置信区间}

$x_1,\cdots,x_{m}$是取自$N(\mu_1,\sigma^2_1)$的样本,$y_1,\cdots,y_{n}$是取自$N(\mu_2,\sigma^2_2)$的样本,两个样本相互独立,记$\overline{x}$和$\overline{y}$分别记为两者的样本均值,$s^2_x$和$s^2_y$分别记为两者的样本方差。

\begin{table}[htbp]
	\centering
	\caption{两个正态总体下的置信区间}
	\renewcommand{\arraystretch}{2}
	\resizebox{\linewidth}{!}{
		\begin{tabular}{|c|c|c|c|c|}
			\hline
			目标 & 条件 & 枢轴量 & 分布 & 置信区间 \\ \hline
			$\mu_1-\mu_2$ & $\sigma^2_1$和$\sigma^2_2$已知 & $\frac{(\overline{x}-\overline{y})-(\mu_1-\mu_2)}{\sqrt{\frac{\sigma_1^2}{m}+\frac{\sigma_2^2}{n}}}$ & $N(0,1)$ & $\left[(\overline{x}-\overline{y})-n_{1-\frac{\alpha}{2}}\sqrt{\frac{\sigma_1^2}{m}+\frac{\sigma_2^2}{n}},\quad(\overline{x}-\overline{y})+n_{1-\frac{\alpha}{2}}\sqrt{\frac{\sigma_1^2}{m}+\frac{\sigma_2^2}{n}}\right]$ \\ \hline
			$\mu_1-\mu_2$ & $\sigma^2_1=\sigma^2_2$未知 & $\frac{(\overline{x}-\overline{y})-(\mu_1-\mu_2)}{s_{w}\sqrt{\frac{1}{m}+\frac{1}{n}}}$ & $T(m+n-2)$ & $\left[(\overline{x}-\overline{y})-s_w t_{1-\frac{\alpha}{2}}\sqrt{\frac{1}{m}+\frac{1}{n}},\quad (\overline{x}-\overline{y})+s_w t_{1-\frac{\alpha}{2}}\sqrt{\frac{1}{m}+\frac{1}{n}}\right]$ \\ \hline
			$\mu_1-\mu_2$ & $\frac{\sigma^2_2}{\sigma^2_1}=c$已知 & $\frac{(\overline{x}-\overline{y})-(\mu_1-\mu_2)}{s_{w_c}\sqrt{\frac{1}{m}+\frac{c}{n}}}$ & $T(m+n-2)$ & $\left[(\overline{x}-\overline{y})-s_{w_c} t_{1-\frac{\alpha}{2}}\sqrt{\frac{1}{m}+\frac{c}{n}},\quad (\overline{x}-\overline{y})+s_{w_c} t_{1-\frac{\alpha}{2}}\sqrt{\frac{1}{m}+\frac{c}{n}}\right]$ \\ \hline
			$\mu_1-\mu_2$ & $n_1$和$n_2$充分大 & $\frac{(\overline{x}-\overline{y})-(\mu_1-\mu_2)}{\sqrt{\frac{s_x^2}{m}+\frac{s_y^2}{n}}}$ & $N(0,1)$ & $\left[(\overline{x}-\overline{y})-n_{1-\frac{\alpha}{2}}\sqrt{\frac{s_x^2}{m}+\frac{s_y^2}{n}},\quad (\overline{x}-\overline{y})+n_{1-\frac{\alpha}{2}}\sqrt{\frac{s_x^2}{m}+\frac{s_y^2}{n}}\right]$ \\ \hline
			$\mu_1-\mu_2$ & 一般情况 & $\frac{(\overline{x}-\overline{y})-(\mu_1-\mu_2)} {s_0}$ & $T(l)$ & $\left[ (\overline{x}-\overline{y})-s_0t_{1-\frac{\alpha}{2}},\quad (\overline{x}-\overline{y})+s_0t_{1-\frac{\alpha}{2}} \right]$ \\ \hline
			$\sigma^2_2/\sigma^2_1$ & 一般情况 & $\frac{s_x^2/\sigma^2_1}{s_y^2/\sigma^2_2}$ & $F(m-1,n-1)$ & $\left[\frac{s_x^2}{s_y^2}\frac{1}{f_{1-\frac{\alpha}{2}}},\quad \frac{s_x^2}{s_y^2}\frac{1}{f_{\frac{\alpha}{2}}}\right]$ \\ \hline
		\end{tabular}
	}
\end{table}

其中
\begin{align*}
	& s_w^2=\frac{(m-1)s_x^2+(n-1)s_y^2}{m+n-2}\\
	& s_{w_c}^2=\frac{(m-1)s_x^2+(n-1)\frac{s_y^2}{c}}{m+n-2}\\
	& s_0^2=\frac{s_x^2}{m}+\frac{s_y^2}{n}\\
	& l=\left[ \frac{s_0^4}{\frac{s_x^4}{m^2(m-1)}+\frac{s_y^4}{n^2(n-1)}} \right]
\end{align*}

\chapter{假设检验}

\section{假设检验的基本思想与概念}

\subsection{假设检验问题}

\textbf{基本思想}:如果试验结果与假设$H$发生矛盾,那么拒绝原假设$H$,否则接受原假设$H$。

\textbf{假设检验问题}:

\begin{enumerate}
	\item \textbf{假设}:两个非空不交参数集合。
	\item \textbf{检验}:通过样本对一个假设作出“对”或“不对”的具体判断规则。
	\item \textbf{参数假设检验问题}:假设可用一个参数的集合表示的检验问题。
\end{enumerate}

\subsection{假设检验的基本步骤}

\textbf{一、建立假设}

对于来自参数分布族$\{ F(x,\theta):\theta\in\Theta \}$的样本$x_1,\cdots,x_n$,其中$\Theta$为参数空间,如果非空集合$\Theta_0\sub\Theta$,那么命题$H_0:\theta\in\Theta_0$称为\textbf{原假设}或\textbf{零假设},命题$H_a:\theta\in\Theta\setminus\Theta_0$称为\textbf{对立假设}或\textbf{备择假设},那么$H_0$对$H_a$的假设检验问题记为
$$
H_0:\theta\in\Theta_0 \qquad
\mathrm{vs}
\qquad H_a:\theta\in\Theta\setminus\Theta_0
$$
如果$\Theta_0$仅含有一个点,那么称$H_0$为\textbf{简单原假}设,否则称为\textbf{复杂原假设}或\textbf{复合原假设}。当$H_0$为简单假设时,其形式可写为$H_0:\theta=\theta_0$,此时备择假设通常有如下三种可能:
$$
H_1:\theta\ne\theta_0,\qquad
H_2:\theta<\theta_0,\qquad
H_3:\theta>\theta_0
$$
称$H_0\quad\mathrm{vs}\quad H_1$为\textbf{双侧假设}或\textbf{双边假设},$H_0\quad\mathrm{vs}\quad H_2$以及$H_0\quad\mathrm{vs}\quad H_3$为\textbf{单侧假设}或\textbf{单边假设}。

在假设检验中,通常将不宜轻易否定的假设作为原假设。

\textbf{二、选择检验统计量,给出拒绝域形式}

当有了具体的样本后,将样本空间划分为两个互不相交的部分$W$和$\overline{W}$,当样本属于$W$时,拒绝$H_0$,否则接受$H_0$。称$W$为该检验的\textbf{拒绝域},$\overline{W}$为该检验的\textbf{接受域}。事实上,在拒绝域和接受域外,还有\textbf{保留域},但通常将保留域合并于接受域内。

选择分布已知的\textbf{检验统计量}$T(X)$,确定拒绝域$W$的形式。

\textbf{三、选择显著性水平}

当$\theta\in\Theta_0$时,样本由于随机性却落入了拒绝域$W$,于是采取了拒绝$H_0$的错误决策,称之为\textbf{第一类错误}或\textbf{拒真错误},记第一类错误概率为
$$
\alpha(\theta)=P\{X\in W\mid H_0\},\quad \theta\in\Theta_0
$$

当$\theta\in\Theta\setminus\Theta_0$时,样本由于随机性却落入了接受域$\overline{W}$,于是采取了接受$H_0$的错误决策,称之为\textbf{第二类错误}或\textbf{取伪错误},记第二类错误概率为
$$
\beta(\theta)=P\{X\in \overline{W}\mid H_a\},\quad \theta\in\Theta\setminus\Theta_0
$$

\begin{definition}{势函数}
	对于检验问题
	$$
	H_0:\theta\in\Theta_0 \qquad
	\mathrm{vs}
	\qquad H_a:\theta\in\Theta\setminus\Theta_0
	$$
	其拒绝域为$W$,那么定义势函数为
	$$
	\rho(\theta)=P_{\theta}(X\in W),\quad \theta\in\Theta
	$$
	即
	$$
	\rho(\theta)=\begin{cases}
		\alpha(\theta),\quad & \theta\in\Theta_0\\
		1-\beta(\theta),\quad & \theta\in\Theta\setminus\Theta_0\\
	\end{cases}
	$$
\end{definition}

\begin{definition}{显著性检验}
	对于检验问题
	$$
	H_0:\theta\in\Theta_0 \qquad
	\mathrm{vs}
	\qquad H_a:\theta\in\Theta\setminus\Theta_0
	$$
	其势函数为$\rho(\theta)$,如果一个检验满足对于任意$\theta\in\Theta_0$,成立
	$$
	\rho(\theta)\le\alpha
	$$
	那么称该检验为显著性水平为$\alpha$的显著性检验,简称水平为$\alpha$的检验。
\end{definition}

\textbf{四、给出拒绝域}

依据显著性水平$\alpha$以及拒绝域$W$的形式,确定具体的拒绝域。

\textbf{五、做出判断}

由拒绝域$W$唯一相互确定的\textbf{判断准则}为

\begin{enumerate}
	\item 如果$(x_1,\cdots,x_n)\in W$,那么拒绝$H_0$。
	\item 如果$(x_1,\cdots,x_n)\in \overline{W}$,那么接受$H_0$。
\end{enumerate}

\subsection{检验的$p$值}

\begin{definition}{检验的$p$值}
	在假设检验问题中,利用样本观测值能够作出拒绝原假设的最小显著性水平称为检验的$p$值。
	\begin{enumerate}
		\item 如果$p\le\alpha$,那么在显著性水平$\alpha$下拒绝$H_0$。
		\item 如果$p>\alpha$,那么在显著性水平$\alpha$下接受$H_0$。
	\end{enumerate}
\end{definition}

\section{正态总体参数假设检验}

\subsection{单个正态总体均值的检验}

\begin{table}[H]
	\centering
	\caption{单个正态总体均值的检验}
	\renewcommand{\arraystretch}{1.5}
	\resizebox{\linewidth}{!}{
		\begin{tabular}{|c|c|c|c|c|c|c|c|}
			\hline
			检验 & 条件 & $H_0$ & $H_a$ & 统计检验量 & 分布 & 拒绝域 & $p$值 \\ \hline
			&  & $\mu\le \mu_0$ & $\mu>\mu_0$ &  &  & $\{ u\ge u_{1-\alpha} \}$ & $1-\Phi(u_0)$ \\ \hline
			$u$检验 & $\sigma$已知 & $\mu\ge \mu_0$ & $\mu<\mu_0$ & $u=\frac{\sqrt{n}(\overline{x}-\mu_0)}{\sigma}$ & $N(0,1)$ & $\{ u\le u_\alpha \}$ & $\Phi(u_0)$ \\ \hline
			&  & $\mu=\mu_0$ & $\mu\ne \mu_0$ &  &  & $\{ |u|\ge u_{1-\frac{\alpha}{2}} \}$ & $2(1-\Phi(|u_0|))$ \\ \hline
			&  & $\mu\le \mu_0$ & $\mu>\mu_0$ &  &  & $\{ t\ge t_{1-\alpha} \}$ & $P(T\ge t_0)$ \\ \hline
			$t$检验 & $\sigma$未知 & $\mu\ge \mu_0$ & $\mu<\mu_0$ & $t=\frac{\sqrt{n}(\overline{x}-\mu_0)}{s}$ & $T(n-1)$ & $\{ t\le t_{\alpha} \}$ & $P(T\le t_0)$ \\ \hline
			&  & $\mu=\mu_0$ & $\mu\ne \mu_0$ &  &  & $\{|t|\ge t_{1-\frac{\alpha}{2}}\}$ & $P(|T|\ge |t_0|)$ \\ \hline
		\end{tabular}
	}
\end{table}

\subsection{两个正态总体均值差的检验}

\begin{table}[H]
	\centering
	\caption{两个正态总体均值差的检验}
	\renewcommand{\arraystretch}{2}
	\resizebox{\linewidth}{!}{
		\begin{tabular}{|c|c|c|c|c|c|c|c|}
			\hline
			检验 & 条件 & $H_0$ & $H_a$ & 检验统计量 & 分布 & 拒绝域 & $p$值 \\ \hline
			&  & $\mu_1\le \mu_2$ & $\mu_1>\mu_2$ &  &  & $\{ u\ge u_{1-\alpha} \}$ & $1-\Phi(u_0)$ \\ \hline
			$u$检验 & $\sigma_1,\sigma_2$已知 & $\mu_1\ge \mu_2$ & $\mu_1<\mu_2$ & $u=\frac{\overline{x}-\overline{y}}{\sqrt{\frac{\sigma_1^2}{m}+\frac{\sigma_2^2}{n}}}$ & $N(0,1)$ & $\{ u\le u_\alpha \}$ & $\Phi(u_0)$ \\ \hline
			&  & $\mu_1=\mu_2$ & $\mu_1\ne\mu_2$ &  &  & $\{ |u|\ge u_{1-\frac{\alpha}{2}} \}$ & $2(1-\Phi(|u_0|))$ \\ \hline
			&  & $\mu_1\le \mu_2$ & $\mu_1>\mu_2$ &  &  & $\{ t\ge t_{1-\alpha} \}$ & $P(T\ge t_0)$ \\ \hline
			$t$检验 & $\sigma_1=\sigma_2$未知 & $\mu_1\ge \mu_2$ & $\mu_1<\mu_2$ & $t=\frac{\overline{x}-\overline{y}}{s_w\sqrt{\frac{1}{m}+\frac{1}{n}}}$ & $T(m+n-2)$ & $\{t\le t_\alpha\}$ & $P(T\le t_0)$ \\ \hline
			&  & $\mu_1= \mu_2$ & $\mu_1\ne\mu_2$ &  &  & $\{|t|\ge t_{1-\frac{\alpha}{2}}\}$ & $P(|T|\ge |t_0|)$ \\ \hline
			&  & $\mu_1\le \mu_2$ & $\mu_1>\mu_2$ &  &  & $\{ u\ge u_{1-\alpha} \}$ & $1-\Phi(u_0)$ \\ \hline
			$u$检验 & $m,n$充分大 & $\mu_1\ge \mu_2$ & $\mu_1<\mu_2$ & $u=\frac{\overline{x}-\overline{y}}{\sqrt{\frac{s_x^2}{m}+\frac{s_y^2}{n}}}$ & $N(0,1)$ & $\{ u\le u_\alpha \}$ & $\Phi(u_0)$ \\ \hline
			&  & $\mu_1=\mu_2$ & $\mu_1\ne\mu_2$ &  &  & $\{ |u|\ge u_{1-\frac{\alpha}{2}} \}$ & $2(1-\Phi(|u_0|))$ \\ \hline
			&  & $\mu_1\le \mu_2$ & $\mu_1>\mu_2$ &  &  & $\{ t\ge t_{1-\alpha} \}$ & $P(T\ge t_0)$ \\ \hline
			$t$检验 & 一般情况 & $\mu_1\ge \mu_2$ & $\mu_1<\mu_2$ & $t=\frac{\overline{x}-\overline{y}}{\sqrt{\frac{s_x^2}{m}+\frac{s_y^2}{n}}}$ & $T(l)$ & $\{t\le t_\alpha\}$ & $P(T\le t_0)$ \\ \hline
			&  & $\mu_1=\mu_2$ & $\mu_1\ne\mu_2$ &  &  & $\{|t|\ge t_{1-\frac{\alpha}{2}}\}$ & $P(|T|\ge|t_0|)$ \\ \hline
		\end{tabular}
	}
\end{table}

其中
\begin{align*}
	& s_w^2=\frac{(m-1)s_x^2+(n-1)s_y^2}{m+n-2}\\
	& l=\left[ \frac{\left(\frac{s_x^2}{m}+\frac{s_y^2}{n}\right)^2}{\frac{s_x^4}{m^2(m-1)}+\frac{s_y^4}{n^2(n-1)}} \right]
\end{align*}

\subsection{成对数据检验}

\begin{table}[H]
	\centering
	\caption{成对数据检验}
	\renewcommand{\arraystretch}{1.5}
	\begin{tabular}{|c|c|c|c|c|c|}
		\hline
		$H_0$ & $H_a$ & 统计检验量 & 分布 & 拒绝域 & $p$值 \\ \hline
		$\mu\le 0$ & $\mu>0$ &  &  & $\{ t\ge t_{1-\alpha} \}$ & $P(T\ge t_0)$ \\ \hline
		$\mu\ge 0$ & $\mu<0$ & $t=\frac{\sqrt{n}\overline{d}}{s_d}$ & $T(n-1)$ & $\{ t\le t_{\alpha} \}$ & $P(T\le t_0)$ \\ \hline
		$\mu=0$ & $\mu\ne 0$ &  &  & $\{|t|\ge t_{1-\frac{\alpha}{2}}\}$ & $P(|T|\ge |t_0|)$ \\ \hline
	\end{tabular}
\end{table}

\subsection{正态总体方差的检验}

\begin{table}[H]
	\centering
	\caption{正态总体方差的检验}
	\renewcommand{\arraystretch}{2}
	\resizebox{\linewidth}{!}{
		\begin{tabular}{|c|c|c|c|c|c|c|c|}
			\hline
			检验 & 条件 & $H_0$ & $H_a$ & 统计检验量 & 分布 & 拒绝域 & $p$值 \\ \hline
			&  & $\sigma^2\le\sigma_0^2$ & $\sigma^2>\sigma_0^2$ &  &  & $\{ \chi^2\ge\chi^2_{1-\alpha} \}$ & $P(\chi^2\ge\chi^2_0)$ \\ \hline
			$\chi^2$检验 & 一个 & $\sigma^2\ge\sigma_0^2$ & $\sigma^2<\sigma_0^2$ & $\chi^2=\frac{(n-1)s^2}{\sigma_0^2}$ & $\chi^2(n-1)$ & $\{ \chi^2\le\chi^2_{\alpha} \}$ & $P(\chi^2\le\chi^2_0)$ \\ \hline
			&  & $\sigma^2=\sigma_0^2$ & $\sigma^2\ne\sigma_0^2$ &  &  & $\{ \chi^2\le\chi^2_{\frac{\alpha}{2}} \}\cup\{ \chi^2\ge\chi^2_{1-\frac{\alpha}{2}} \}$ & $2\min\{ P(\chi^2\le\chi^2_0),P(\chi^2\ge\chi^2_0) \}$ \\ \hline
			&  & $\sigma_1^2\le\sigma_2^2$ & $\sigma_1^2>\sigma_2^2$ &  &  & $\{ F\ge F_{1-\alpha} \}$ & $P(F\ge F_0)$ \\ \hline
			$F$检验 & 两个 & $\sigma_1^2\ge\sigma_2^2$ & $\sigma_1^2<\sigma_2^2$ & $F=\frac{s_x^2}{s_y^2}$ & $F(m-1,n-1)$ & $\{ F\le F_{\alpha} \}$ & $P(F\le F_0)$ \\ \hline
			&  & $\sigma_1^2=\sigma_2^2$ & $\sigma_1^2\ne\sigma_2^2$ &  &  & $\{ F\le F_{\frac{\alpha}{2}} \}\cup\{ F\ge F_{1-\frac{\alpha}{2}} \}$ & $2\min\{ P(F\le F_0),P(F\ge F_0) \}$ \\ \hline
		\end{tabular}
	}
\end{table}

\section{其他分布参数的假设检验}

\begin{table}[H]
	\centering
	\caption{其他分布参数的假设检验}
	\renewcommand{\arraystretch}{2}
	\resizebox{\linewidth}{!}{
		\begin{tabular}{|c|c|c|c|c|c|c|c|}
			\hline
			检验 & 条件 & $H_0$ & $H_a$ & 统计检验量 & 分布 & 拒绝域 & $p$值 \\ \hline
			&  & $\lambda\le\lambda_0$ & $\lambda>\lambda_0$ &  &  & $\{ \chi^2\ge\chi^2_{1-\alpha} \}$ & $P(\chi^2\ge\chi^2_0)$ \\ \hline
			$\chi^2$分布 & $\mathrm{Exp}(\frac{1}{\lambda})$ & $\lambda\ge\lambda_0$ & $\lambda<\lambda_0$ & $\frac{2n\overline{x}}{\lambda_0}$ & $\chi^2(2n)$ & $\{ \chi^2\le\chi^2_{\alpha} \}$ & $P(\chi^2\le\chi^2_0)$ \\ \hline
			&  & $\lambda=\lambda_0$ & $\lambda\ne\lambda_0$ &  &  & $\{ \chi^2\le\chi^2_{\frac{\alpha}{2}} \}\cup\{ \chi^2\ge\chi^2_{1-\frac{\alpha}{2}} \}$ & $2\min\{ P(\chi^2\le\chi^2_0),P(\chi^2\ge\chi^2_0) \}$ \\ \hline
			&  & $p\le p_0$ & $p> p_0$ &  &  &  & $P(x\ge x_0)$ \\ \hline
			$B$检验 & $B(1,p)$ & $p\ge p_0$ & $p< p_0$ & $x$ & $B(n,p)$ &  & $P(x\le x_0)$ \\ \hline
			&  & $p= p_0$ & $p\ne p_0$ &  &  &  & $2\min\{ P(x\le x_0),P(x\ge x_0) \}$ \\ \hline
			&  & $\theta\le\theta_0$ & $\theta>\theta_0$ &  &  & $\{ u\ge u_{1-\alpha} \}$ & $1-\Phi(u_0)$ \\ \hline
			$u$检验 & 大样本分布$F(x;\theta)$ & $\theta\ge\theta_0$ & $\theta<\theta_0$ & $\frac{\sqrt{n}(\overline{x}-\theta_0)}{\sqrt{\sigma^2(\hat{\theta})}}$ & $N(0,1)$ & $\{ u\le u_{\alpha} \}$ & $\Phi(u_0)$ \\ \hline
			&  & $\theta=\theta_0$ & $\theta\ne\theta_0$ &  &  & $\{ |u|\ge u_{1-\frac{\alpha}{2}} \}$ & $2(1-\Phi(|u_0|))$ \\ \hline
		\end{tabular}
	}
\end{table}

其中分布$F(x;\theta)$的均值为$\theta$,方差为$\sigma(\theta)$,$\hat{\theta}$为$\theta$的最大似然估计。

\section{似然比检验与分布拟合检验}

\subsection{似然比检验的思想}

\begin{definition}{似然比}
	对于来自密度函数为$p(x;\theta),\theta\in\Theta$的总体的样本$x_1,\cdots,x_n$,对于如下检验问题
	$$
	H_0:\theta\in\Theta_0\quad \mathrm{vs}\quad H_a:\theta\in\Theta\setminus\Theta_0
	$$
	定义改假设检验问题的似然比统计量为
	$$
	\Lambda(x_1,\cdots,x_n)=\frac{\sup_{\theta\in\Theta}{p(x_1,\cdots,x_n;\theta)}}{\sup_{\theta\in\Theta_0}{p(x_1,\cdots,x_n;\theta)}}
	$$
	即
	$$
	\Lambda(x_1,\cdots,x_n)=\frac{p(x_1,\cdots,x_n;\hat{\theta})}{p(x_1,\cdots,x_n;\hat{\theta}_0)}
	$$
	其中$\hat{\theta}$和$\hat{\theta}_0$分别为参数空间$\Theta$和$\Theta_0$上的最大似然估计。
\end{definition}

\begin{definition}{似然比检验 LRT}
	对于来自密度函数为$p(x;\theta),\theta\in\Theta$的总体的样本$x_1,\cdots,x_n$,对于如下检验问题
	$$
	H_0:\theta\in\Theta_0\quad \mathrm{vs}\quad H_a:\theta\in\Theta\setminus\Theta_0
	$$
	其似然比统计量
	$$
	\Lambda(x_1,\cdots,x_n)=\frac{p(x_1,\cdots,x_n;\hat{\theta})}{p(x_1,\cdots,x_n;\hat{\theta}_0)}
	$$
	作为检验问题的检验统计量,且取拒绝域为$W=\{ \Lambda(x_1,\cdots,x_n)\ge \lambda_0 \}$,其中临界值$\lambda_0$满足对于任意$\theta\in\Theta_0$,成立
	$$
	P_\theta(\Lambda(x_1,\cdots,x_n)\ge \lambda_0)\le \alpha
	$$
	那么称此检验为显著性水平为$\alpha$的似然比检验。
\end{definition}

\subsection{分布数据的$\chi^2$拟合优度检验}

\begin{theorem}
	总体被分为$r$类$A_1,\cdots,A_r$,考虑假设检验
	$$
	H_0:A_k\text{所占的比率为}p_k,\quad k=1,\cdots,r
	$$
	其中$p_k$已知且$\sum_{k=1}^{r}{p_k}=1$。从该总体抽出$n$个样本,$n_k$为样本中属于$A_k$的样本个数,记检验统计量为
	$$
	\chi^2=\sum_{k=1}^{r}\frac{(n_k-np_k)^2}{np_k}
	$$
	那么当$H_0$成立时,成立
	$$
	\chi^2\xrightarrow{\mathrm{L}}\chi^2(r-1)
	$$
	因此对于显著性水平$\alpha$,拒绝域为$W=\{ \chi^2\ge\chi^2_{1-\alpha} \}$,检验的$p$值为$p=P(\chi^2\ge\chi^2_0)$。
	
	如果$A_k$出现的概率含有$s$个参数,那么可用最大似然估计方法估计出该$s$个参数,然后再算出$p_k$的估计值$\hat{p}_k$,于是统计检验量
	$$
	\chi^2=\sum_{k=1}^{r}\frac{(n_k-n\hat{p}_k)^2}{n\hat{p}_k}\xrightarrow{\mathrm{L}}\chi^2(r-s-1)
	$$
\end{theorem}

\subsection{分布的$\chi^2$拟合优度检验}

对于来自分布函数为$F(x)$的总体的样本$x_1,\cdots,x_n$,考虑假设检验问题
$$
H_0:F(x)=F_0(x)
$$
其中$F_0(x)$为可含参的理论分布。

\textbf{一、总体$X$为离散分布}

如果总体$X$为至多可数个值$a_1,a_2,\cdots$,将其分为$r$类$A_1,\cdots,A_r$,使得每一个$A_k$中的样本个数$n_k$不小于$5$,记$P(X\in A_k)=p_k$,那么原假设检验转化为
$$
H_0:A_k所占的比率为p_k,\quad k=1,\cdots,r
$$

\textbf{二、总体$X$为连续分布}

如果总体$X$的分布为$F_0$,选取$-\infty=a_0<a_1<\cdots<a_{r-1}<a_r=\infty$,记$A_k=(a_{k-1},a_k]$,那么
$$
p_k=P(X\in A_k)=F_0(a_k)-F_0(a_{k-1}),\quad k=1,\cdots,r
$$
于是原假设转化为
$$
H_0:A_k所占的比率为p_k,\quad k=1,\cdots,r
$$

\subsection{列联表的独立性检验}

将总体分为两个属性$A$和$B$,其中$A$有$r$个类$A_1,\cdots,A_r$,$B$有$s$个类$B_1,\cdots,B_s$,从总体中抽取$n$个样本,设其中有$n_{ij}$个个体属于$A_i$和$B_j$,构造列联表$\{n_{ij}\}_{r\times s}$。

记总体中的个体仅属于$A_i$和仅属于$B_j$的概率分别为$p_{i\cdot}$和$p_{\cdot j}$,总体中的个体同时属于$A_i$和$B_j$的概率为$p_{ij}$,那么得到二维离散分布表$\{p_{ij}\}_{r\times s}$,$A$和$B$两属性度量的假设可表述为
$$
H_0:p_{ij}=p_{i\cdot}p_{\cdot j},\quad
i=1,\cdots,r;j=1,\cdots,s
$$
$H_0$成立时$p_{ij}$的最大似然估计为
$$
\hat{p}_{ij}=\frac{1}{n^2}\sum_{k=1}^{r}{n_{kj}}\sum_{k=1}^{s}{n_{ik}}
$$
那么检验统计量为
$$
\chi^2=\sum_{i,j}{\frac{(n_{ij}-n\hat{p}_{ij})^2}{n\hat{p}_{ij}}}\xrightarrow{\mathrm{L}}\chi^2((r-1)(s-1))
$$
因此对于显著性水平$\alpha$,拒绝域为$W=\{ \chi^2\ge\chi^2_{1-\alpha} \}$,检验的$p$值为$p=P(\chi^2\ge\chi^2_0)$。

\section{正态性检验}

\subsection{正态概率纸}

对于给定的样本观测值$x_1,\cdots,x_n$,做点
$$
\left( x_{(k)},\frac{k-0.375}{n+0.25} \right),\quad k=1,\cdots,n
$$

如果诸点在一条直线附近,那么认为该批数据来自正态总体;否则不认为该批数据来自正态总体。

\subsection{W检验}

对于来自正态分布总体$N(\mu,\sigma^2)$的样本$x_1,\cdots,x_n$,其中$8\le n\le 50$,定义W统计量为
$$
W=\frac{\dis\sum_{k=1}^{n}{(w_k-\overline{w})^2(x_{(k)}-\overline{x})^2}}
{\dis\sum_{k=1}^{n}{(w_k-\overline{w})^2}\sum_{k=1}^{n}(x_{(k)}-\overline{x})^2}=
\frac{\dis\sum_{k=1}^{[\frac{n}{2}]}{w_k^2(x_{(k)}-x_{(n+1-k)})^2}}{\dis\sum_{k=1}^{n}{(x_{(k)}-\overline{x})^2}}
$$
其中
$$
\boldsymbol{e}=\begin{pmatrix}
	E\left( \frac{x_{(1)}-\mu}{\sigma} \right)\\
	\vdots\\
	E\left( \frac{x_{(n)}-\mu}{\sigma} \right)
\end{pmatrix},\qquad
\boldsymbol{C}=\begin{pmatrix}
	\mathrm{Cov}\left( \frac{x_{(1)}-\mu}{\sigma},\frac{x_{(1)}-\mu}{\sigma} \right)&\cdots&\mathrm{Cov}\left( \frac{x_{(1)}-\mu}{\sigma},\frac{x_{(n)}-\mu}{\sigma} \right)\\
	\vdots&\ddots&\vdots\\
	\mathrm{Cov}\left( \frac{x_{(n)}-\mu}{\sigma},\frac{x_{(1)}-\mu}{\sigma} \right)&\cdots&\mathrm{Cov}\left( \frac{x_{(n)}-\mu}{\sigma},\frac{x_{(n)}-\mu}{\sigma} \right)
\end{pmatrix}
$$

$$
\begin{pmatrix}
	w_1\\\vdots\\w_n
\end{pmatrix}=\frac{\boldsymbol{C}^{-1}\boldsymbol{e}}{\sqrt{\boldsymbol{e}^T (\boldsymbol{C}^{-1})^2 \boldsymbol{e}}}
$$

拒绝域为$\{ W\le W_{\alpha} \}$,其中$W_\alpha$为$\alpha$分位数。

\subsection{EP检验}

对于来自正态分布总体$N(\mu,\sigma^2)$的样本$x_1,\cdots,x_n$,其中$n\ge 8$,定义EP检验统计量为
$$
T_{\mathrm{EP}}=1+\frac{n}{\sqrt{3}}+\frac{2}{n}\sum_{i=2}^{n}\sum_{j=1}^{i-1}\exp\left(-\frac{(x_j-x_i)^2}{2\frac{n-1}{n}s^2}\right)-\sqrt{2}\sum_{k=1}^{n}\exp\left(-\frac{(x_k-\overline{x})^2}{4\frac{n-1}{n}s^2}\right)
$$
拒绝域为$\{ T_{\mathrm{EP}}\ge T_{{\mathrm{EP}}_{1-\alpha}} \}$,其中$T_{{\mathrm{EP}}_{1-\alpha}}$为$1-\alpha$分位数。

\section{非参数检验}

\subsection{游程检验}

对于依时间顺序连续得到的样本观测值$x_1,\cdots,x_n$,记样本中位数为$m_e$,对于$k=1,\cdots,n$,记
$$
y_k=\begin{cases}
	1,\quad & x_k\ge m_e\\
	0,\quad & x_k< m_e
\end{cases}
$$
$y_1,\cdots,y_n$构成$0-1$序列。

记$0-1$序列中$0$和$1$的个数分别为$n_1$和$n_2$,游程总数为$R$,那么$1<n_1,n_2<n$且$2\le R\le n$。同时$|n_1-n_2|$为$0$或$1$。原假设为
$$
H_0:\text{样本序列符合随机抽取的原则}
$$
$R$的分布如下
$$
P(R=2k)=\frac{2{{n_1-1}\choose{k-1}}{{n_2-1}\choose{k-1}}}
{{{n_1+n_2}\choose{n_1}}},\quad k=1,\cdots,\scriptsize{\left[\frac{n_1+n_2}{2}\right]}
$$

$$
P(R=2k+1)=\frac{{{n_1-1}\choose{k-1}}{{n_2-1}\choose{k}}+{{n_1-1}\choose{k}}{{n_2-1}\choose{k-1}}}
{{{n_1+n_2}\choose{n_1}}},\quad k=1,\cdots,\scriptsize{\left[\frac{n_1+n_2-1}{2}\right]}
$$

拒绝域为$\{R\le R_{\frac{\alpha}{2}}\}\cup\{R\ge R_{1-\frac{\alpha}{2}}\}$,检验的$p$值为$2\min\{ P(R\le R_0),P(R\ge R_0) \}$。

\subsection{符号检验}

\begin{table}[H]
	\centering
	\caption{符号检验}
	\renewcommand{\arraystretch}{2}
	\resizebox{\linewidth}{!}{
		\begin{tabular}{|c|c|c|c|}
			\hline
			$H_0$ & $H_a$ & 拒绝域 & 检验的$p$值 \\ \hline
			$x_p\le x_0$ & $x_p>x_0$ & $\{S^+\ge c\}$ & $\dis\sum_{k=S_0^+}^{n}{{{n}\choose{k}}(1-p)^k p^{n-k}}$ \\ \hline
			$x_p\ge x_0$ & $x_p<x_0$ & $\{S^+\le c\}$ & $\dis\sum_{k=0}^{S_0^+}{{{n}\choose{k}}(1-p)^k p^{n-k}}$ \\ \hline
			$x_p=x_0$ & $x_p\ne x_0$ & $\{S^+\le c_1\}\cup\{S^+\ge c_2\}$ & $2\min\left\{ \dis\sum_{k=0}^{S_0^+}{{{n}\choose{k}}(1-p)^k p^{n-k}},\sum_{k=S_0^+}^{n}{{{n}\choose{k}}(1-p)^k p^{n-k}} \right\}$ \\ \hline
		\end{tabular}
	}
\end{table}

其中$S^+$为$x_1-x_0,\cdots,x_n-x_0$​中正数的个数,即
$$
S^+=\sum_{k=1}^{n}{I_{x_k> x_0}}
$$

\subsection{秩和检验}

\begin{definition}{秩}
	对于来自连续分布$F(x)$的简单随机样本$x_1,\cdots,x_n$,次序样本为$x_{(1)},\cdots,x_{(n)}$,称$x_k$秩为$r_k$,如果$x_k=x_{(r_k)}$,记作$R_k=r_k$。
\end{definition}

\begin{definition}{秩统计量}
	对于来自连续分布$F(x)$的简单随机样本$x_1,\cdots,x_n$,$R_k$为$x_k$的秩,那么称$R=(R_1,\cdots,R_n)$为$x_1,\cdots,x_n$的秩统计量。
\end{definition}

对于来自连续分布$F(x-\theta)$的简单随机样本$x_1,\cdots,x_n$,其中$\theta$为总体的中位数,记$R_k$为$|x_k|$在$|x_1|,\cdots,|x_n|$中的秩,定义符号秩和统计量为
$$
W^+=\sum_{k=1}^{n}{R_k I_{x_k>0}}\sim W^+(n)
$$

\begin{table}[H]
	\centering
	\renewcommand{\arraystretch}{1.5}
	\begin{tabular}{|c|c|c|}
		\hline
		$H_0$ & $H_a$ & 拒绝域 \\ \hline
		$\theta\le 0$ & $\theta>0$ & $\{W^+\le W^+_\alpha\}$ \\ \hline
		$\theta\ge 0$ & $\theta<0$ & $\{W^+\ge W^+_\alpha\}$ \\ \hline
		$\theta=0$ & $\theta\ne0$ & $\{ W^+\le W^+_{\frac{\alpha}{2}} \}\cup\{ W^+\ge W^+_{1-\frac{\alpha}{2}} \}$ \\ \hline
	\end{tabular}
\end{table}

其中$W^+_{\alpha}+W^+_{1-\alpha}=\frac{1}{2}n(n-1)$。

对于来自连续分布$F(x-\theta_1)$的简单随机样本$x_1,\cdots,x_m$和对于来自连续分布$F(x-\theta_2)$的简单随机样本$y_1,\cdots,y_n$,产生的秩为
$$
R=(Q_1,\cdots,Q_m,R_1,\cdots,R_n)
$$
那么秩和统计量为
$$
W=\sum_{k=1}^{n}{R_k}\sim W(m,n)
$$

\begin{table}[htbp]
	\centering
	\renewcommand{\arraystretch}{1.5}
	\begin{tabular}{|c|c|c|}
		\hline
		$H_0$ & $H_a$ & 拒绝域 \\ \hline
		$\theta_1\le \theta_2$ & $\theta_1>\theta_2$ & $\{W\le W_\alpha\}$ \\ \hline
		$\theta_1\ge \theta_2$ & $\theta_1<\theta_2$ & $\{W\ge W_\alpha\}$ \\ \hline
		$\theta_1=\theta_2$ & $\theta_1\ne\theta_2$ & $\{ W\le W_{\frac{\alpha}{2}} \}\cup\{ W\ge W_{1-\frac{\alpha}{2}} \}$ \\ \hline
	\end{tabular}
\end{table}

其中$W_\alpha+W_{1-\alpha}=n(m+n-1)$。

\chapter{方差分析与回归分析}

\section{方差分析}

\subsection{问题的提出}

因子:$A$

水平:$A_1,\cdots,A_r$

结果:$y_{ij}$,其中$i=1,\cdots,r$

\subsection{单因子方差分析的统计模型}

在单因子试验中,记因子为$A$,设其由$r$个水平,记为$A_1,\cdots,A_r$,在每一个水平下考察的指标可以看成一个总体,现有$r$个水平,故有$r$个总体,假定:

\begin{enumerate}
	\item 每一个总体均为正态分布,记为$N(\mu_k,\sigma^2_k)$,其中$k=1,\cdots,r$。
	\item 各总体的方差相同,记为$\sigma_1^2=\cdots=\sigma_r^2=\sigma^2$。
	\item 从每一总体中抽取的样本是互相独立的,即所有的试验结果$y_{ij}$都相互独立。
\end{enumerate}

作假设检验:
$$
H_0:\mu_1=\cdots=\mu_r
\quad \mathrm{vs} \quad 
H_a:\mu_1,\cdots,\mu_r不全相等
$$

如果$H_0$成立,称因子$A$的$r$个水平没有显著差异,简称因此$A$不显著。

对$r$个总体每个作$m$次重复实现,得到试验结果$\{y_{ij}\}_{r\times m}$,定义随机误差为
$$
\varepsilon_{ij}=y_{ij}-\mu_{i}
$$
那么试验结果$y_{ij}$的数据结构式为
$$
y_{ij}=\mu_{i}+\varepsilon_{ij}
$$

单因子方差分析的统计模型为
$$
\begin{cases}
	y_{ij}=\mu_{i}+\varepsilon_{ij}\\
	\varepsilon_{ij}\text{相互独立}\\
	\varepsilon_{ij}\sim N(0,\sigma^2)
\end{cases}
$$

总均值
$$
\mu=\frac{1}{r}\sum_{k=1}^{r}{\mu_k}
$$

因子$A$的第$k$个水平的主效应
$$
a_i=\mu_i-\mu
$$
容易知道
$$
\sum_{k=1}^{r}{a_k}=0
$$

$$
\mu_k=\mu+a_k
$$

于是统计模型改写为
$$
\begin{cases}
	y_{ij}=\mu+a_i+\varepsilon_{ij}\\
	\dis\sum_{i=1}^{r}{a_i}=0\\
	\varepsilon_{ij}\text{相互独立}\\
	\varepsilon_{ij}\sim N(0,\sigma^2)
\end{cases}
$$
统计假设改写为
$$
H_0:a_1=\cdots=a_r=0
\quad \mathrm{vs} \quad 
H_a:a_1,\cdots,a_r\text{不全为}0
$$

\subsection{平方和分解}

\textbf{一、实验数据}

\begin{table}[H]
	\centering
	\caption{符号检验}
	% \renewcommand{\arraystretch}{2}
	\begin{tabular}{|c|c|c|c|}
		\hline
		因子水平 & 试验数据 & 和 & 均值 \\ \hline
		$A_1$ & $y_{11},\cdots,y_{1m}$ & $\dis T_1=\sum_{j=1}^{m}{y_{1j}}$ & $\overline{y}_1=\frac{T_1}{m}$ \\ \hline
		$\vdots$ & $\vdots$ & $\vdots$ & $\vdots$ \\ \hline
		$A_r$ & $y_{r1},\cdots,y_{rm}$ & $\dis T_r=\sum_{j=1}^{m}{y_{rj}}$ & $\overline{y}_r=\frac{T_r}{m}$ \\ \hline
		& & $\dis T=\sum_{i=1}^{r}{T_i}$ & $\overline{y}=\frac{T}{rm}=\frac{T}{n}$ \\ \hline
	\end{tabular}
\end{table}

\textbf{二、组内偏差与组间方差}

记
\begin{align*}
	& y_{ij}-\overline{y}=(y_{ij}-\overline{y}_{i})+(\overline{y}_{i}-\overline{y})\\
	& \overline{\varepsilon}_i=\frac{1}{m}\sum_{j=1}^{m}{\varepsilon_{ij}}\\
	& \overline{\varepsilon}=\frac{1}{r}\sum_{j=1}^{m}{\overline{\varepsilon_{i}}}=\frac{1}{n}\sum_{i,j}{\varepsilon_{ij}}
\end{align*}
组内偏差为
$$
y_{ij}-\overline{y}_i=\varepsilon_{ij}-\overline{\varepsilon}_i
$$
组间偏差为
$$
\overline{y}_i-\overline{y}=a_i+\overline{\varepsilon}_i-\overline{\varepsilon}
$$

\textbf{三、偏差平方和及其自由度}

\textbf{偏差平方和}
$$
Q=\sum_{k=1}^{n}(y_k-\overline{y})^2
$$

\textbf{自由度}
$$
f_Q=n-1
$$

\textbf{四、总平方和分解公式}

\textbf{总偏差平方和}
$$
S_T=\sum_{i,j}{(y_{ij}-\overline{y})^2}=\sum_{i,j}{y_{ij}^2}-\frac{T^2}{n},\quad f_T=n-1
$$

\textbf{组内偏差平方和(因子$A$的偏差平方和)}
$$
S_A=m\sum_{i=1}^{r}{(\overline{y}_{i}-\overline{y})^2}=\frac{1}{m}\sum_{i=1}^{r}{T_i^2}-\frac{T^2}{n},\quad f_A=r-1
$$

\textbf{组内偏差平方和(误差偏差平方和)}
$$
S_e=\sum_{i,j}{(y_{ij}-\overline{y}_i)^2}=\sum_{i,j}{y_{ij}^2}-\frac{1}{m}\sum_{i=1}^{r}{T_i^2},\quad f_e=n-r
$$

\textbf{总平方和分解式}
$$
S_T=S_A+S_e
$$

\subsection{检验方法}

\textbf{均方}
$$
\mathrm{MS}=\frac{Q}{f_Q}
$$

\textbf{因子均方和误差均方}
$$
\mathrm{MS}_A=\frac{S_A}{f_A},\qquad \mathrm{MS}_e=\frac{S_e}{f_e}
$$

\begin{theorem}
	\begin{enumerate}
		\item 
		$$
		\frac{S_e}{\sigma^2}\sim\chi^2(n-r),\quad E(S_e)=(n-r)\sigma^2
		$$
		\item $$
		E(S_A)=(r-1)\sigma^2+m\sum_{i=1}^{r}{a_i^2}
		$$
		\item 若$H_0$成立,那么
		$$
		\frac{S_A}{\sigma^2}\sim\chi^2(r-1),\quad E(S_e)=(r-1)\sigma^2
		$$
		\item $S_A$与$S_e$相互独立。
	\end{enumerate}
\end{theorem}

\textbf{检验统计量}
$$
F=\frac{\mathrm{MS}_A}{\mathrm{MS}_e}\sim F(r-1,n-r)
$$
拒绝域
$$
W=\{F\ge F_{1-\alpha}\}
$$
\begin{enumerate}
	\item $F\ge F_{1-\alpha}$:拒绝原假设,认为因子$A$显著。
	\item $F\le F_{1-\alpha}$:接受原假设,认为因子$A$不显著。
\end{enumerate}
检验的$p$值为
$$
p=P(F\ge F_0)
$$

\begin{table}[H]
	\centering
	\caption{单因子方差分析表}
	% \renewcommand{\arraystretch}{2}
	\begin{tabular}{|c|c|c|c|c|c|}
		\hline
		来源 & 平方和 & 自由度 & 均方 & $F$比 & $p$值 \\ \hline
		因子$A$ & $S_A$ & $f_A=r-1$ & $\mathrm{MS}_A=\frac{S_A}{f_A}$ & $F=\frac{\mathrm{MS}_A}{\mathrm{MS}_e}$ & $p=P(F\ge F_0)$ \\ \hline
		误差$e$ & $S_e$ & $f_e=n-r$ & $\mathrm{MS}_e=\frac{S_e}{f_e}$ & & \\ \hline
		总和$T$ & $S_T$ & $f_T=n-1$ & & & \\ \hline
	\end{tabular}
\end{table}

\subsection{参数估计}

\textbf{一、点估计}

\begin{enumerate}
	\item 
	$$
	y_{ij}\sim N(\mu+a_i,\sigma^2)
	$$
	\item $\mu$的最大似然估计为
	$$
	\hat{\mu}=\overline{y}
	$$
	\item $a_i$的最大似然估计为
	$$
	\hat{a}_i=\overline{y}_i-\overline{y}
	$$
	\item $\sigma^2$的最大似然估计为
	$$
	\hat{\sigma}^2=\mathrm{MS}_e
	$$
\end{enumerate}

\textbf{二、置信区间}

由于
$$
\overline{y}_i\sim N(\mu_i,\frac{\sigma^2}{m}),\qquad
\frac{S_e}{\sigma^2}\sim\chi^2(n-r)
$$
因此
$$
\frac{\sqrt{m}(\overline{y}_i-\mu_i)}{\sqrt{\hat{\sigma}^2}}\sim T(f_e)
$$
进而$\mu_i$的$1-\alpha$的置信区间为
$$
\left[
\overline{y}_i-t_{1-\frac{\alpha}{2}}\frac{\hat{\sigma}}{\sqrt{m}},\quad
\overline{y}_i+t_{1-\frac{\alpha}{2}}\frac{\hat{\sigma}}{\sqrt{m}}
\right]
$$

\subsection{重复数不等情形}

\textbf{一、数据}

记从第$i$个水平下的总体获得$m_i$个试验结果,记为$y_{i1},\cdots,y_{im_i}$,其中$i=1,\cdots,r$,实验总次数为$n=m_1+\cdots+m_r$,统计模型为
$$
\begin{cases}
	y_{ij}=\mu_i+\varepsilon_{ij}\\
	\varepsilon_{ij}\textbf{相互独立}\\
	\varepsilon_{ij}\sim N(0,\sigma^2)
\end{cases}
$$

\textbf{二、总均值}

\textbf{加权均值}
$$
\mu=\frac{1}{n}\sum_{i=1}^{r}{m_i\mu_i}
$$
\textbf{水平效应}:
$$
a_i=\mu_i-\mu
$$
统计模型为
$$
\begin{cases}
	y_{ij}=\mu+a_i+\varepsilon_{ij}\\
	\sum_{i=1}^{r}{m_ia_i}=0\\
	\varepsilon_{ij}相互独立\\
	\varepsilon_{ij}\sim N(0,\sigma^2)
\end{cases}
$$

\textbf{四、各平方和的计算}

\begin{align*}
	& T_i=\sum_{j=1}^{m_i}{y_{ij}},\qquad 
	\overline{y}_i=\frac{T_i}{m_i}\\
	& T=\sum_{i,j}{y_{ij}}=\sum_{i=1}^{r}T_i,\qquad \overline{y}=\frac{T}{n}
\end{align*}

\begin{align*}
	&S_T=\sum_{i,j}{(y_{ij}-\overline{y})^2}=\sum_{i,j}{y_{ij}^2}-\frac{T^2}{n}
	,\quad & f_T=n-1\\
	&S_A=\sum_{i=1}^{r}{m_i(\overline{y}_{i}-\overline{y})^2}=\sum_{i=1}^{r}{\frac{T_i^2}{m_i}}-\frac{T^2}{n}
	,\quad & f_A=r-1\\
	&S_e=\sum_{i,j}{(y_{ij}-\overline{y}_i)^2}=\sum_{i,j}{y_{ij}^2}-\sum_{i=1}^{r}{\frac{T_i^2}{m_i}}
	,\quad&  f_e=n-r
\end{align*}

\section{多重比较}

\subsection{水平均值差的置信区间}

检验问题
$$
H_0:\mu_i-\mu_j=0\quad \mathrm{vs} \quad H_a:\mu_i-\mu_j\ne0
$$
由于
$$
\overline{y}_i-\overline{y}_j\sim N\left( \mu_i-\mu_j,\left( \frac{1}{m_i}+\frac{1}{m_j} \right)\sigma^2 \right)
$$
而$\frac{S_e}{\sigma^2}\sim\chi^2(n-r)$,因此
$$
\frac{(\overline{y}_i-\overline{y}_j)-(\mu_i-\mu_j)}{\sqrt{\left( \frac{1}{m_i}+\frac{1}{m_j} \right)\hat{\sigma}^2}}\sim T(n-r)
$$
那么置信水平为$1-\alpha$的置信区间为
$$
\left[
\overline{y}_i-\overline{y}_j-t_{1-\frac{\alpha}{2}}\sqrt{\left( \frac{1}{m_i}+\frac{1}{m_j} \right)\hat{\sigma}^2},\quad
\overline{y}_i-\overline{y}_j+t_{1-\frac{\alpha}{2}}\sqrt{\left( \frac{1}{m_i}+\frac{1}{m_j} \right)\hat{\sigma}^2}
\right]
$$
这也是检验问题的接受域$\overline{W}$。如果包含$0$,那么接受原假设,认为$\mu_i$和$\mu_j$无显著差异;反之拒绝原假设,认为$\mu_i$和$\mu_j$存在显著差异。

\subsection{多重比较问题}

首先经过方差检验,表明因子$A$是显著的,即$r$个水平均值不全相等,那么考虑如下多重比较问题检验
$$
H_0^{ij}:\mu_i=\mu_j,\quad 1\le i< j\le r
$$
拒绝域
$$
W=\bigcup_{1\le i< j\le r}\{ |\overline{y}_i-\overline{y}_j|\ge c_{ij} \}
$$

\subsection{重复数相等的T法}

当$m_1=\cdots=m_r=m$时,记$c_{ij}=c$,于是检验统计量为
$$
\frac{\sqrt{m}(\overline{y}_i-\mu_i)}{\hat{\sigma}}\sim T(n-r)
$$
当原假设成立时,$\mu_1=\cdots=\mu_r=\mu$,此时
$$
P(W)=
P\left(q(r,n-r)\ge\frac{c\sqrt{m}}{\hat{\sigma}} \right)
$$
其中$t$化极差统计量为
$$
q(r,n-r)= \max_{1\le i\le r}\frac{\sqrt{m}(\overline{y}_i-\mu_i)}{\hat{\sigma}}-\min_{1\le j\le r}\frac{\sqrt{m}(\overline{y}_j-\mu_j)}{\hat{\sigma}}
$$
仅与$n$和$r$有关。由$P(W)=\alpha$,可知
$$
c=q_{1-\alpha}\frac{\hat{\sigma}}{\sqrt{m}}
$$
因此,如果
$$
|\overline{y}_i-\overline{y}_j|\ge q_{1-\alpha}\frac{\hat{\sigma}}{\sqrt{m}}
$$
那么认为水平$A_i$和$A_j$存在显著差异;反之认为水平$A_i$和$A_j$无显著差异。

\subsection{重复数不等场合的S法}

由于
$$
\frac{(\overline{y}_i-\overline{y}_j)-(\mu_i-\mu_j)}{\sqrt{\left( \frac{1}{m_i}+\frac{1}{m_j} \right)\hat{\sigma}^2}}\sim T(n-r)
$$
当原假设成立时,$\mu_1=\cdots=\mu_r=\mu$,此时
$$
\frac{(\overline{y}_i-\overline{y}_j)^2}{\left( \frac{1}{m_i}+\frac{1}{m_j} \right)\hat{\sigma}^2}\sim F(1,n-r)
$$
令$c_{ij}=c\sqrt{\frac{1}{m_i}+\frac{1}{m_j}}$,那么
$$
P(W)=P\left( \max_{1\le i<j \le r}\frac{(\overline{y}_i-\overline{y}_j)^2}{\left( \frac{1}{m_i}+\frac{1}{m_j} \right)\hat{\sigma}^2}\ge\frac{c^2}{\hat{\sigma}^2} \right)
$$
其中
$$
\frac{\max_{1\le i<j \le r}\frac{(\overline{y}_i-\overline{y}_j)^2}{\left( \frac{1}{m_i}+\frac{1}{m_j} \right)\hat{\sigma}^2}}{r-1}\sim F(r-1,n-r)
$$
由$P(W)=\alpha$,可知
$$
\frac{c^2}{\hat{\sigma}^2}=(r-1)f_{1-\alpha}
$$
即
$$
c_{ij}=\sqrt{(r-1)f_{1-\alpha}\hat{\sigma}^2\left(\frac{1}{m_i}+\frac{1}{m_j}\right)}
$$
其中$f_{1-\alpha}$为$F(r-1,n-r)$的$1-\alpha$​分位数。因此,如果
$$
|\overline{y}_i-\overline{y}_j|\ge \sqrt{(r-1)f_{1-\alpha}\hat{\sigma}^2\left(\frac{1}{m_i}+\frac{1}{m_j}\right)}
$$
那么认为水平$A_i$和$A_j$存在显著差异;反之认为水平$A_i$和$A_j$无显著差异。

\section{方差齐性检验}

\textbf{方差齐性检验}
$$
H_0:\sigma_1^2=\cdots=\sigma_r^2
$$

\subsection{Hartley检验}

对于单因子方差分析中含有$r$个样本,当$m_1=\cdots=m_r=m$时,设第$i$个样本方差为
$$
s_i^2=\frac{1}{m-1}\sum_{k=1}^{m}(y_{ij}-\overline{y}_i)^2
$$
检验统计量为
$$
H=\frac{\max\{ s_i^2 \}}{\min\{s_i^2\}}\sim H(r,m-1)
$$
拒绝域为
$$
W=\{ H\ge H_{1-\alpha} \}
$$

\subsection{Bartlett检验}

对于单因子方差分析中含有$r$个样本,设第$i$个样本方差为
$$
s_i^2=\frac{1}{m_i-1}\sum_{k=1}^{m_i}(y_{ij}-\overline{y}_i)^2=\frac{Q_i}{f_i}
$$
其中$m_i$为第$i$个样本的容量且$m_i\ge 5$,$Q_i=\sum_{k=1}^{m_i}(y_{ij}-\overline{y}_i)^2$和$f_i=m_i-1$为该样本的偏差平方和自由度。$s_i^2$的算术加权平均即为均方误差
$$
\mathrm{MS}_e=\frac{1}{f_e}\sum_{i=1}^{r}Q_i=\sum_{i=1}^{r}\frac{f_i}{f_e}s_i^2
$$
其加权几何平均为
$$
\mathrm{GMS}_e=\left( \prod_{i=1}^{r}(s_i^2)^{f_i} \right)^{\frac{1}{f_e}}
$$
其中$f_e=\sum_{i=1}^{r}f_i=n-r$。由算术-几何平均不等式
$$
\mathrm{MS}_e\ge \mathrm{GMS}_e
$$
当且仅当$s_1^2=\cdots=s_r^2$时等号成立。而
$$
B=\frac{f_e}{C}\ln\frac{\mathrm{MS}_e}{\mathrm{GMS}_e}\dot{\sim} \chi^2(r-1)
$$
其中
$$
C=1+\frac{1}{3(r-1)}\left( \sum_{i=1}^{r}\frac{1}{f_i}-\frac{1}{f_e} \right)
$$
因此拒绝域为
$$
W=\left\{ B\ge\chi^2_{1-\alpha} \right\}
$$

\subsection{修正的Bartlett检验}

修正的检验统计量
$$
B'=\frac{f_2BC}{f_1(A-BC)}\dot{\sim}F(f_1,f_2)
$$
其中
$$
f_1=r-1,\qquad f_2=\frac{r+1}{(C-1)^2},\qquad A=\frac{f_2}{2-C+\frac{2}{f_20}}
$$
拒绝域为
$$
W=\{ B'\ge F_{1-\alpha} \}
$$

\section{一元线性回归}

\subsection{变量间的两类关系}

确定性关系,相关关系

\subsection{一元线性回归模型}

第一类回归问题
$$
f(x)=E(Y\mid x)=\int_{-\infty}^{\infty}yp(y\mid x)\mathrm{d}x
$$

第二类回归问题
$$
y=f(x)+\varepsilon
$$
其中$\varepsilon\sim N(0,\sigma^2)$。

一元回归模型:
$$
\begin{cases}
	y_i=\beta_0+\beta x_i+\varepsilon_i\\
	\varepsilon_i\text{相互独立}\\
	\varepsilon_i\sim N(0,\sigma^2)
\end{cases}
$$
由数据$(x_i,y_i)$得到的$\beta_0$和$\beta$的估计$\hat{\beta}_0$和$\hat{\beta}$,称
$$
\hat{y}=\hat{\beta}_0+\hat{\beta}x
$$
为$y$关于$x$的回归函数。给定$x=x_0$,称$\hat{y}_0=\hat{\beta}_0+\hat{\beta}x_0$为回归值。

\subsection{回归系数的最小二乘估计}

\begin{align*}
	&l_{xy}=\sum_{i=1}^{n}(x_i-\overline{x})(y-\overline{y})=\sum_{i=1}^{n} x_iy_i-\frac{1}{n}\sum_{i=1}^{n} x_i\sum_{i=1}^{n} y_i=\sum_{i=1}^{n} x_iy_i-n\overline{x}\overline{y}\\
	&l_{xx}=\sum_{i=1}^{n}(x_i-\overline{x})^2=\sum_{i=1}^{n} x_i^2-\frac{1}{n}\left(\sum_{i=1}^{n} x_i\right)^2=\sum_{i=1}^{n} x_i^2-n\overline{x}^2\\
	&l_{yy}=\sum_{i=1}^{n}(y-\overline{y})^2=\sum_{i=1}^{n} x_iy_i-\frac{1}{n}\left(\sum_{i=1}^{n} y_i\right)^2=\sum_{i=1}^{n} y_i^2-n\overline{y}^2
\end{align*}

$\beta_0$和$\beta$的最小二乘估计(LSE)$\hat{\beta}_0$和$\hat{\beta}$为
\begin{align*}
	&\hat{\beta}=\frac{l_{xy}}{l_{xx}}\\
	& \hat{\beta}_0=\overline{y}-\hat{\beta}\overline{x}
\end{align*}

\begin{theorem}
	在如下模型下,成立
	$$
	\begin{cases}
		y_i=\beta_0+\beta x_i+\varepsilon_i\\
		\varepsilon_i相互独立\\
		\varepsilon_i\sim N(0,\sigma^2)
	\end{cases}
	$$
	\begin{enumerate}
		\item 
		$$
		\hat{\beta}_0\sim N\left( \beta_0,\left(\frac{1}{n}+\frac{\overline{x}^2}{l_{xx}}\right)\sigma^2 \right),\qquad 
		\hat{\beta}\sim N\left( \beta,\frac{\sigma^2}{l_{xx}} \right)
		$$
		\item $$
		\mathrm{Cov}(\hat{\beta}_0,\hat{\beta})=-\frac{\overline{x}}{l_{xx}}\sigma^2
		$$
		\item 给定$x_0$
		$$
		\hat{y}_0=\hat{\beta}_0+\hat{\beta}x_0\sim N\left( \beta_0+\beta x_0,\left(\frac{1}{n}+\frac{(\overline{x}-x_0)^2}{l_{xx}}\right)\sigma^2 \right)
		$$
	\end{enumerate}
\end{theorem}

\subsection{回归模型的显著性检验}

\textbf{显著性}:$\beta\ne0$称为显著,否则称为不显著。

显著性假设检验:
$$
H_0:\beta=0\quad \mathrm{vs} \quad H_a:\beta\ne0
$$

\begin{table}[H]
	\centering
	\caption{方差分析表}
	% \renewcommand{\arraystretch}{2}
	\begin{tabular}{|c|c|c|c|c|}
		\hline
		来源 & 平方和 & 自由度 & 均方 & $F$比 \\ \hline
		回归 & $S_R$ & $f_R=1$ & $\mathrm{MS}_R=\frac{S_R}{f_R}$ & $F=\frac{\mathrm{MS}_A}{\mathrm{MS}_e}$ \\ \hline
		残差 & $S_e$ & $f_e=n-2$ & $\mathrm{MS}_e=\frac{S_e}{f_e}$ & \\ \hline
		总和 & $S_T$ & $f_T=n-1$ & & \\ \hline
	\end{tabular}
\end{table}

\textbf{一、$F$检验}

总偏差平方和:
$$
S_T=\sum_{i=1}^{n}(y_i-\overline{y})^2=l_{yy}
$$
回归平方和:
$$
S_R=\sum_{i=1}^{n}(\hat{y}_i-\overline{y})^2=\frac{l_{xy}^2}{l_{xx}}
$$
残差平方和:
$$
S_e=\sum_{i=1}^{n}(y_i-\hat{y}_i)^2
$$
平方和分解:
$$
S_T=S_R+S_e
$$

\begin{theorem}
	在如下模型下,成立
	$$
	\begin{cases}
		y_i=\beta_0+\beta x_i+\varepsilon_i\\
		\varepsilon_i\text{相互独立}\\
		\varepsilon_i\sim N(0,\sigma^2)
	\end{cases}
	$$
	\begin{enumerate}
		\item 
		$$
		E(S_R)=\sigma^2+\hat{\beta}l_{xx},\qquad E(S_e)=(n-2)\sigma^2
		$$
		\item 
		$$
		\frac{S_e}{\sigma^2}\sim \chi^2(n-2)
		$$
		\item 如果$H_0$​成立,那么
		$$
		\frac{S_R}{\sigma^2}\sim \chi^2(1)
		$$
		\item $S_R$与$S_e$、$\overline{y}$独立。
	\end{enumerate}
\end{theorem}

\textbf{统计检验量}
$$
F=\frac{(n-2)S_R}{S_e}\sim F(1,n-2)
$$
拒绝域为$W=\{ F\ge F_{1-\alpha} \}$。

\textbf{二、$T$检验}

\textbf{检验统计量}
$$
T=\frac{\sqrt{(n-2)l_{xx}}\hat{\beta}}{\sqrt{S_e}}\sim T(n-2)
$$
拒绝域为$W=\{ |t|\ge t_{1-\frac{\alpha}{2}}\}$。

\textbf{三、相关系数检验}

相关系数假设检验:
$$
H_0:\rho=0\quad \mathrm{vs} \quad H_a:\rho\ne0
$$
\textbf{检验统计量}:相关系数
$$
r=\frac{l_{xy}}{\sqrt{l_{xx}l_{yy}}}=\sqrt{\frac{F}{F+(n-2)}}\sim r(n-2)= \sqrt{\frac{F(1,n-2)}{F(1,n-2)+(n-2)}}
$$
\begin{enumerate}
	\item $|r|=1$:$(x_i,y_i)$共线。
	\item $r>0$:$(x_i,y_i)$正相关。
	\item $r<0$:$(x_i,y_i)$负相关。
	\item $r=0$:$(x_i,y_i)$不相关。
\end{enumerate}

拒绝域为$W=\{ |r|\ge r_{1-\alpha} \}$,其中
$$
r_{1-\alpha}=\sqrt{\frac{F_{1-\alpha}}{F_{1-\alpha}+(n-2)}}
$$

\subsection{估计与预测}

\textbf{一、$E(y_0)$的置信区间}

枢轴量为
$$
\frac{\hat{y}_0-E(y_0)}{\sqrt{\frac{S_e}{n-2}}\sqrt{\frac{1}{n}+\frac{(x_0-\overline{x})^2}{l_{xx}}}}\sim T(n-2)
$$
$1-\alpha$的置信区间为
$$
\left[
\hat{y}_0-t_{1-\frac{\alpha}{2}}\sqrt{\frac{S_e}{n-2}}\sqrt{\frac{1}{n}+\frac{(x_0-\overline{x})^2}{l_{xx}}},\quad 
\hat{y}_0+t_{1-\frac{\alpha}{2}}\sqrt{\frac{S_e}{n-2}}\sqrt{\frac{1}{n}+\frac{(x_0-\overline{x})^2}{l_{xx}}}
\right]
$$

\textbf{二、$y_0$的预测区间}

枢轴量为
$$
\frac{y_0-\hat{y}_0}{\sqrt{\frac{S_e}{n-2}}\sqrt{1+\frac{1}{n}+\frac{(x_0-\overline{x})^2}{l_{xx}}}}\sim T(n-2)
$$
预测区间为
$$
\left[
\hat{y}_0-t_{1-\frac{\alpha}{2}}\sqrt{\frac{S_e}{n-2}}\sqrt{1+\frac{1}{n}+\frac{(x_0-\overline{x})^2}{l_{xx}}},\quad 
\hat{y}_0+t_{1-\frac{\alpha}{2}}\sqrt{\frac{S_e}{n-2}}\sqrt{1+\frac{1}{n}+\frac{(x_0-\overline{x})^2}{l_{xx}}}
\right]
$$

\subsection{曲线回归方程的比较}

\textbf{决定系数}:越大说明残差越小,回归曲线拟合越好。
$$
R^2=1-\frac{\sum(y_i-\hat{y}_i)^2}{\sum(y_i-\overline{y})^2}
$$

\textbf{剩余标准差}:越小,回归曲线拟合越好。
$$
s=\sqrt{\frac{\sum(y_i-\hat{y}_i)^2}{n-2}}
$$

\appendix

\chapter{概率模型}

\begin{table}[htbp]
	\centering
	\renewcommand{\arraystretch}{3}
	\resizebox{\linewidth}{!}{\begin{tabular}{|c|c|c|c|c|c|}
			\hline
			概率模型 & 密度函数$p(x)$ & 参数范围 & 数学期望$E\xi$ & 方差$D\xi$ & 特征函数$f(t)$ \\
			\hline
			退化分布$I_c(x)$ & $p(x)=\begin{cases}1,&x=c\\0,&x\ne c\end{cases}$ &  & $c$ & $0$ & $\mathrm{e}^{ict}$ \\
			\hline
			Bernoulli分布 & $p(x)=\begin{cases}1-p,&x=0\\p,&x=1\end{cases}$ & $0<p<1$ & $p$ & $p(1-p)$ & $p\mathrm{e}^{it}+1-p$ \\
			\hline
			二项分布$B(n,p)$ & $b(k;n,p)={{n}\choose{k}}p^k(1-p)^{n-k}$ & $0\le k \le n;0<p<1$ & $np$ & $np(1-p)$ & $(p\mathrm{e}^{it}+1-p)^n$ \\
			\hline
			Poisson分布$P(\lambda)$ & $p(k;\lambda)=\frac{\lambda^k}{k!}\mathrm{e}^{-\lambda}$ & $k\in\N;\lambda>0$ & $\lambda$ & $\lambda$ & $\mathrm{e}^{\lambda(\mathrm{e}^{it}-1)}$ \\
			\hline
			几何分布 & $g(k;p)=p(1-p)^{k-1}$ & $k\in\N^*,0<p<1$ & $\frac{1}{p}$ & $\frac{q}{p^2}$ & $\frac{p\mathrm{e}^{it}}{1-(1-p)\mathrm{e}^{it}}$ \\
			\hline
			超几何分布 & $p_k = \frac{{\binom{M}{k} \binom{N-M}{n-k}}}{{\binom{N}{n}}}$ & $M,n\le N;0\le  k\le \min\{M,n\}$ & $\frac{nM}{N}$ & $\frac{nM(N-M)(N-n)}{N^2(N-1)}$ & $\dis\sum_{k=0}^{n} \frac{{\binom{M}{k} \binom{N-M}{n-k}}}{{\binom{N}{n}}} e^{ikt}
			$ \\
			\hline
			Pascal分布 & $p_k={{k-1}\choose{r-1}}p^r(1-p)^{k-r}$ & $k\ge r,0<p<1$ & $\frac{r}{p}$ & $\frac{r(1-p)}{p^2}$ & $(\frac{(1-p)\mathrm{e}^{it}}{1-(1-p)\mathrm{e}^{it}})^r$ \\
			\hline
			负二项分布 & $p_k={{-r}\choose{k}}p^r(p-1)^k$ & $k\in\N,0<p<1,r>0$ & $\frac{r(1-p)}{p}$ & $\frac{r(1-p)}{p^2}$ & $(\frac{p}{1-(1-p)\mathrm{e}^{it}})^r$ \\
			\hline
			正态分布$N(\mu,\sigma^2)$ & $p(x)=\frac{1}{\sqrt{2\pi}\sigma}\mathrm{e}^{-\frac{(x-\mu)^2}{2\sigma^2}}$ &  & $\mu$ & $\sigma$ & $\mathrm{e}^{i\mu t-\frac{1}{2}\sigma^2t^2}$ \\
			\hline
			均匀分布$U[a,b]$ & $p(x)=\begin{cases}\frac{1}{b-a},&a\le x\le b\\0,&\text{其他}\end{cases}$ & $a<b$ & $\frac{a+b}{2}$ & $\frac{(b-a)^2}{12}$ & $\frac{\mathrm{e}^{ibt}-\mathrm{e}^{iat}}{i(b-a)t}$ \\
			\hline
			指数分布$\mathrm{Exp}(\lambda)$ & $p(x)=\begin{cases}\lambda\mathrm{e}^{-\lambda x},&x\ge 0\\0,&x<0\end{cases}$ & $\lambda>0$ & $\lambda^{-1}$ & $\lambda^{-2}$ & $(1-\frac{it}{\lambda})^{-1}$ \\
			\hline
			$\chi^2$分布 & $p(x)=\begin{cases}\frac{1}{2^{\frac{n}{2}}\Gamma(\frac{n}{2})}x^{\frac{n}{2}-1}\mathrm{e}^{-\frac{x}{2}},&x\ge 0\\0,&x<0\end{cases}$ & $n\in \N^*$ & $n$ & $2n$ & $(1-2it)^{-\frac{n}{2}}$ \\
			\hline
			$\Gamma$分布$\Gamma(r,\lambda)$ & $p(x)=\begin{cases}\frac{\lambda^r}{\Gamma(r)}x^{r-1}\mathrm{e}^{-\lambda x},&x\ge 0\\0,&x<0\end{cases}$ & $r,\lambda>0$ & $\frac{r}{\lambda}$ & $\frac{r}{\lambda^2}$ & $(1-\frac{it}{\lambda})^{-r}$ \\
			\hline
			Cauchy分布 & $p(x)=\frac{1}{\pi}\frac{\lambda}{\lambda^2+(x-\mu)^2}$ & $\mu\in\R,\lambda>0$ & 不存在 & 不存在 & $\mathrm{e}^{i\mu t-\lambda|t|}$ \\
			\hline
			$t$分布 & $p(x)=\frac{\Gamma(\frac{n+1}{2})}{\sqrt{n\pi}\Gamma(\frac{n}{2})}(1+\frac{x^2}{n})^{-\frac{n+1}{2}}$ & $n\in\N^*$ & $0(n>1)$ & $\frac{n}{n-2}(n>2)$ &  \\
			\hline
			Pareto分布 & $p(x)=\begin{cases}rA^r\frac{1}{x^{r+1}},&x\ge A\\0,&x<A\end{cases}$ & $r,A>0$ & ($r>1$时存在) & ($r>2$时存在) &  \\
			\hline
			$F$分布 & $p(x)=\begin{cases}\frac{\Gamma(\frac{m+n}{2})}{\Gamma(\frac{m}{2})\Gamma(\frac{n}{2})}m^{\frac{m}{2}}n^{\frac{n}{2}}\frac{x^{\frac{m}{2}-1}}{(n+mx)^{\frac{m+n}{2}}},&x\ge 0\\0,&x<0\end{cases}$ & $m,n\in \N^*$ & $\frac{n}{n-2}(n>2)$ & $\frac{2n^2(m+n-2)}{m(n-2)^2(n-4)}(n>4)$ &  \\
			\hline
			$\beta$分布 & $p(x)=\begin{cases}\frac{\Gamma(p+q)}{\Gamma(p)\Gamma(q)}x^{p-1}(1-x)^{q-1},&0<x<1\\0,&\text{其他}\end{cases}$ & $p,q>0$ & $\frac{p}{p+q}$ & $\frac{pq}{(p+q)^2(p+q+1)}$ & $\dis\frac{\Gamma(p+q)}{\Gamma(p)}\sum_{k=0}^{\infty}{\frac{\Gamma(p+k)(it)^k}{\Gamma(p+q+k)\Gamma(k+1)}}$ \\
			\hline
			对数正态分布 & $p(x)=\begin{cases}\frac{1}{\sqrt{2\pi}\sigma x}\mathrm{e}^{\frac{(\ln x-\alpha)^2}{2\sigma^2}},&x>0\\0,&x\le0\end{cases}$ & $\alpha,\sigma>0$ & $\mathrm{e}^{\alpha+\frac{\sigma^2}{2}}$ & $\mathrm{e}^{2\alpha+\sigma^2}(\mathrm{e}^{\sigma^2}-1)$ &  \\
			\hline
			Weibull分布 & $p(x)=\begin{cases}\alpha\lambda x^{\alpha-1}\mathrm{e}^{-\lambda x^{\alpha}},&x>0\\0,&x\le0\end{cases}$ & $\lambda,\alpha>0$ & $\Gamma(\frac{1}{\alpha}+1)\lambda^{-\frac{1}{\alpha}}$ & $\lambda^{-\frac{2}{\alpha}}(\Gamma(\frac{2}{\alpha}+1)-(\Gamma(\frac{1}{\alpha}+1))^2)$ &  \\
			\hline
			Rayleigh分布 & $p(x)=\begin{cases}x\mathrm{e}^{-\frac{x^2}{2}},&x\ge 0\\0,&x<0\end{cases}$ &  & $\sqrt{\frac{\pi}{2}}$ & $2-\frac{\pi}{2}$ &  \\
			\hline
	\end{tabular}}
\end{table}

\end{document}