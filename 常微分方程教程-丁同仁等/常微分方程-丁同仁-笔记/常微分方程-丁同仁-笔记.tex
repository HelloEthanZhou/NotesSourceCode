\documentclass[lang = cn, scheme = chinese, thmcnt = section]{elegantbook}
% elegantbook      设置elegantbook文档类
% lang = cn        设置中文环境
% scheme = chinese 设置标题为中文
% thmcnt = section 设置计数器


%% 1.封面设置

\title{常微分方程 - 丁同仁 - 笔记}                % 文档标题

\author{若水}                        % 作者

\myemail{ethanmxzhou@163.com}       % 邮箱

\homepage{helloethanzhou.github.io} % 主页

\date{\today}                       % 日期

\logo{PiCreatures_happy.pdf}        % 设置Logo

\cover{阿基米德螺旋曲线.pdf}          % 设置封面图片

% 修改标题页的色带
\definecolor{customcolor}{RGB}{135, 206, 250} 
% 定义一个名为customcolor的颜色,RGB颜色值为(135, 206, 250)

\colorlet{coverlinecolor}{customcolor}     % 将coverlinecolor颜色设置为customcolor颜色

%% 2.目录设置
\setcounter{tocdepth}{3}  % 目录深度为3

%% 3.引入宏包
\usepackage[all]{xy}
\usepackage{bbm, svg, graphicx, float, extpfeil, amsmath, amssymb, mathrsfs, mathalpha, hyperref}


%% 4.定义命令
\newcommand{\N}{\mathbb{N}}            % 自然数集合
\newcommand{\R}{\mathbb{R}}            % 实数集合
\newcommand{\C}{\mathbb{C}}  		   % 复数集合
\newcommand{\Q}{\mathbb{Q}}            % 有理数集合
\newcommand{\Z}{\mathbb{Z}}            % 整数集合
\newcommand{\sub}{\subset}             % 包含
\newcommand{\im}{\text{im }}           % 像
\newcommand{\lang}{\langle}            % 左尖括号
\newcommand{\rang}{\rangle}            % 右尖括号
\newcommand{\bs}{\boldsymbol}          % 向量加黑
\newcommand{\dd}{\mathrm{d}}           % 微分d
\newcommand{\ee}[1]{\mathrm{e}^{#1}}           % 微分d
\newcommand{\rank}{\text{rank}}        % 秩
\newcommand{\tr}{\text{tr}}            % 迹
\newcommand{\function}[5]{
	\begin{align*}
		#1:\begin{aligned}[t]
			#2 &\longrightarrow #3\\
			#4 &\longmapsto #5
		\end{aligned}
	\end{align*}
}                                     % 函数

\newcommand{\lhdneq}{%
	\mathrel{\ooalign{$\lneq$\cr\raise.22ex\hbox{$\lhd$}\cr}}} % 真正规子群

\newcommand{\rhdneq}{%
	\mathrel{\ooalign{$\gneq$\cr\raise.22ex\hbox{$\rhd$}\cr}}} % 真正规子群

\begin{document}

\maketitle       % 创建标题页

\frontmatter     % 开始前言部分

\chapter*{致谢}

\markboth{致谢}{致谢}

\vspace*{\fill}
\begin{center}
	
	\large{感谢 \textbf{ 勇敢的 } 自己}
	
\end{center}
\vspace*{\fill}

\tableofcontents % 创建目录

\mainmatter      % 开始正文部分

\chapter{微分方程的解}

\section{初等积分法}

\subsection{恰当方程}

\begin{definition}{恰当方程}
	称一阶微分方程
	$$
	P(x,y)\mathrm{d}{x} +Q(x,y)\mathrm{d}{y} =0
	$$
	为恰当方程或全微分方程,如果存在一个可微函数$\Phi(x,y)$,使得成立
	$$
	\mathrm{d}{\Phi(x,y)}=P(x,y)\mathrm{d}{x} +Q(x,y)\mathrm{d}{y}
	$$
\end{definition}

\begin{theorem}{Green定理}
	设函数$P(x,y)$和$Q(x,y)$在区域
	$$
	R:\qquad 
	|x-x_0|<r_x,\qquad |y-y_0|<r_y
	$$
	上连续,且具有连续的一阶偏导数,则一阶微分方程
	$$
	P(x,y)\mathrm{d}{x} +Q(x,y)\mathrm{d}{y} =0
	$$
	为恰当方程的充分必要条件为恒等式
	$$
	\frac{\partial P}{\partial y}(x,y)\equiv\frac{\partial Q}{\partial x}(x,y)
	$$
	在$R$内成立。
\end{theorem}

\begin{theorem}{恰当方程的通解}
	恰当方程
	$$
	P(x,y)\mathrm{d}{x} +Q(x,y)\mathrm{d}{y} =0
	$$
	的通解为
	$$
	\Phi(x,y)=\int_{x_0}^{x}P(x,y_0)\mathrm{d}x+\int_{y_0}^{y}Q(x,y)\mathrm{d} y
	$$
	或
	$$
	\Phi(x,y)=\int_{y_0}^{y}P(x_0,y)\mathrm{d}y+\int_{x_0}^{x}Q(x,y)\mathrm{d}x
	$$
	其中$(x_0,y_0)$为任意取定的一点。
\end{theorem}

\subsection{变量分离方程}

\begin{theorem}{变量分离方程的通解}
	微分方程
	$$
	P(x)\dd x+Q(y)\dd y=0
	$$
	的通解为
	$$
	\int P(x)\mathrm{d} x +\int Q(y)\mathrm{d} y =C,\qquad C\in \R
	$$
\end{theorem}

\subsection{一阶线性方程}

\begin{definition}{一阶线性方程}
	称微分方程
	$$
	\frac{\mathrm{d} y}{\mathrm{d} x}+p(x)y=q(x)
	$$
	为一阶线性方程,其中$p(x)$和$q(x)$在$(a,b)$上连续。
\end{definition}

\begin{theorem}{一阶线性方程的通解}
	一阶线性方程
	$$
	\frac{\mathrm{d} y}{\mathrm{d} x}+p(x)y=q(x)
	$$
	的通解为
	$$
	y=\exp\left(-\int p(x)\mathrm{d} x\right)\left(C+\int q(x)\exp\left(\int p(x)\mathrm{d} x\right)\mathrm{d} x\right),\qquad C\in \R
	$$
	特别的,初值问题%
	$$
	\frac{\mathrm{d}{y}}{\mathrm{d}{x}}+p(x)y=q(x),\qquad
	y(x_0)=y_0
	$$
	的解为
	$$
	y=\exp\left(-\int_{x_0}^{x}{p(t)\mathrm{d}{t}}\right)\left(y_0+\int_{x_0}^{x}{q(s)\exp\left(\int_{x_0}^{s}{p(t)\mathrm{d}{t}}\right)\mathrm{d}{s}}\right)
	$$
	或
	$$
	y=y_{0}\exp\left(-\int_{x_0}^{x}{p(t)\mathrm{d}{t}}\right)+\int_{x_0}^{x}{q(s)\exp\left(-\int_{s}^{x}{p(t)\mathrm{d}{t}}\right)\mathrm{d}{s}}
	$$
\end{theorem}

\begin{theorem}{积分因子法}
	方程
	$$
	\frac{\mathrm{d} y}{\mathrm{d} x}+p(x)y=q(x)
	$$
	改写为对称形式
	\[ \mathrm{d} y+p(x)y\mathrm{d} x=q(x)\mathrm{d} x \]
	两侧与因子$\displaystyle \exp\left(\int p(x)\mathrm{d} x\right)$作积,整理得
	\[ \dd\left(y\exp\left(\int{p(x)\mathrm{d}{x}}\right)\right)=\mathrm{d}\left({\int{q(x)\exp\left(\int{p(x)\mathrm{d}{x}}\right)\mathrm{d}{x}}}\right)
	 \]
	两侧积分,便可得到解
	$$
	y=\exp\left(-\int p(x)\mathrm{d} x\right)\left(C+\int q(x)\exp\left(\int p(x)\mathrm{d} x\right)\mathrm{d} x\right),\qquad C\in \R
	$$
\end{theorem}

\begin{theorem}{常数变易法}
	设方程
	$$
	\frac{\mathrm{d}{y}}{\mathrm{d}{x}}+p(x)y=q(x)
	$$
	的解为%
	$$
	y=C(x)\exp\left(-\int{p(x)\mathrm{d}{x}}\right)
	$$
	代入原方程得%
	$$
	C'(x)=q(x)\exp\left(\int{p(x)\mathrm{d}{x}}\right)
	$$
	因此%
	$$
	C(x)=\int q(x)\exp\left(\int{p(x)\mathrm{d}{x}}\right)\dd x+C,\qquad C\in\R
	$$
	进而可得到解
	$$
	y=\exp\left(-\int p(x)\mathrm{d} x\right)
	\left(C+\int q(x)\exp\left(\int p(x)\mathrm{d} x\right)\mathrm{d} x\right),\qquad C\in \R
	$$
\end{theorem}

\subsection{齐次方程}

\begin{theorem}
	对于齐次方程
	$$
	\frac{\dd y}{\dd x}=\varphi\left(\frac{y}{x}\right)
	$$
	令$z=y/x$,可求得通解。
\end{theorem}

\subsection{Bernoulli方程}

\begin{theorem}
	对于Bernoulli方程
	$$
	\frac{\mathrm{d}{y}}{\mathrm{d}{x}}+p(x)y=q(x)y^n,\qquad n\neq0,1
	$$
	令$z=y^{1-n}$,可求得通解。
\end{theorem}

\subsection{Riccati方程}

\begin{theorem}
	对于Riccati方程
	$$
	\frac{\mathrm{d}{y}}{\mathrm{d}{x}}=p(x)y^2+q(x)y+r(x)
	$$
	若已知特解为$y=\varphi(x)$,则令$z=y-\varphi$,可求得通解。
\end{theorem}

\section{一阶隐式微分方程}

\subsection{微分法}

若微分方程
$$
F\left(x,y,\frac{\mathrm{d}y}{\mathrm{d}x}\right)=0
$$
可解出
$$
y=f(x,p)
$$
这里$p=\dd y/\dd x$。设$f$连续可微,则方程$y=f(x,p)$对$x$进行微分,可得到
$$
(f_x'(x,p)-p)\mathrm{d}x+f_p'(x,p)\mathrm{d}p=0
$$
此为关于$x$和$p$的显式方程。

\begin{enumerate}
	\item 若得到方程
	$$
	(f_x^{\prime}(x,p)-p)\mathrm{d}x+f_p^{\prime}(x,p)\mathrm{d}p=0
	$$
	的通解$p=u(x,C)$,那么方程$y=f(x,p)$的通解为$y=f(x,u(x,C))$。
	\item 若方程
	$$
	(f_x^{\prime}(x,p)-p)\mathrm{d}x+f_p^{\prime}(x,p)\mathrm{d}p=0
	$$
	含有特解$p=w(x)$,则方程$y=f(x,p)$含有特解$y=f(x,w(x))$。
	\item 若得到方程
	$$
	(f_x^{\prime}(x,p)-p)\mathrm{d}x+f_p^{\prime}(x,p)\mathrm{d}p=0
	$$
	的通解$x=v(p,C)$,那么方程$y=f(x,p)$的通解为%
	$$
	\begin{cases}
		x=v(p,C)\\
		y=f(v(p,C),p)
	\end{cases}
	\qquad p\text{为参数}
	$$
	\item 若方程
	$$
	(f_x^{\prime}(x,p)-p)\mathrm{d}x+f_p^{\prime}(x,p)\mathrm{d}p=0
	$$
	含有特解$x=z(p)$,则方程$y=f(x,p)$含有特解%
	$$
	\begin{cases}
		x=z(p)\\
		y=f(z(p),p)
	\end{cases}
	\qquad p\text{为参数}
	$$
\end{enumerate}

\subsection{参数法}

对于一阶隐式微分方程
$$
F\left(x,y,\frac{\mathrm{d}y}{\mathrm{d}x}\right)=0
$$
若其参数表达式为
$$
x=f(u,v),\qquad y=g(u,v),\qquad \frac{\mathrm{d}y}{\mathrm{d}x}=h(u,v)
$$
其中$u$和$v$为参数,则有
$$
(h(u,v)f_u^{\prime}(u,v)-g_u^{\prime}(u,v))\mathrm{d}u+(h(u,v)f_v^{\prime}(u,v)-g_v^{\prime}(u,v))\mathrm{d}v=0
$$
若求得方程
$$
(h(u,v)f_u^{\prime}(u,v)-g_u^{\prime}(u,v))\mathrm{d}u+(h(u,v)f_v^{\prime}(u,v)-g_v^{\prime}(u,v))\mathrm{d}v=0
$$
的通解$v=w(u,C)$,则可得到原方程的通解
$$
\begin{cases}x=f(u,w(u,C))\\y=g(u,w(u,C))\end{cases}
$$

\section{线性微分方程}

\subsection{一般理论}

\begin{theorem}{存在唯一性定理}
	对于$n$阶线性微分方程初值问题
	$$
	\frac{\mathrm{d}\bs{y}}{\mathrm{d}x}=\bs{A}(x)\bs{y}+\bs{f}(x),\qquad 
	\bs{y}(x_0)=\bs{y}_0
	$$
	如果$n$阶系数矩阵函数$\bs{A}(x)$和右端函数$\bs{f}(x)$在开区间$(a,b)$上连续,那么其解$\bs{y}=\bs{y}(x)$在开区间$(a,b)$上存在且存在唯一。
\end{theorem}

\begin{definition}{基本解矩阵}
	$n$阶线性微分方程
	$$
	\frac{\mathrm{d}\bs{y}}{\mathrm{d}x}=\bs{A}(x)\bs{y}+\bs{f}(x)
	$$
	存在$n$个线性无关的解%
	$$
	\bs{y}_1(x)=\left(\begin{matrix}y_{11}(x)\\\vdots\\y_{n1}(x)\end{matrix}\right)\qquad 
	\cdots\qquad 
	\bs{y}_n(x)=\left(\begin{matrix}y_{1n}(x)\\\vdots\\y_{nn}(x)\end{matrix}\right)
	$$
	构成微分方程的基本解矩阵
	$$
	\bs{\Phi}(x)=\begin{pmatrix}
		y_{11}(x)&\cdots&y_{1n}(x)\\
		\vdots&\ddots&\vdots\\
		y_{n1}(x)&\cdots&y_{nn}(x)
	\end{pmatrix}
	$$
	由此原方程的通解为
	$$
	\bs{y}(x)=\bs{\Phi}(x)\bs{c},\qquad \bs{c}\in\R^n
	$$
\end{definition}

\begin{definition}{Wronsky行列式}
	函数
	$$
	\bs{y}_1(x)=\left(\begin{matrix}y_{11}(x)\\\vdots\\y_{n1}(x)\end{matrix}\right)\qquad 
	\cdots\qquad 
	\bs{y}_n(x)=\left(\begin{matrix}y_{1n}(x)\\\vdots\\y_{nn}(x)\end{matrix}\right)
	$$
	的Wronsky行列式为%
	$$
	W(x)=\begin{vmatrix}
		y_{11}(x)&\cdots&y_{1n}(x)\\
		\vdots&\ddots&\vdots\\
		y_{n1}(x)&\cdots&y_{nn}(x)
	\end{vmatrix}
	$$
\end{definition}

\begin{proposition}{Wronsky行列式的性质}
	如果函数
	$$
	\bs{y}_1(x)=\left(\begin{matrix}y_{11}(x)\\\vdots\\y_{n1}(x)\end{matrix}\right)\qquad 
	\cdots\qquad 
	\bs{y}_n(x)=\left(\begin{matrix}y_{1n}(x)\\\vdots\\y_{nn}(x)\end{matrix}\right)
	$$
	线性相关,那么其Wronsky行列式恒为$0$,即
	$$
	W(x)=\begin{vmatrix}
		y_{11}(x)&\cdots&y_{1n}(x)\\
		\vdots&\ddots&\vdots\\
		y_{n1}(x)&\cdots&y_{nn}(x)
	\end{vmatrix}\equiv0
	$$
\end{proposition}

\begin{theorem}{Liouville公式}
	如果
	$$
	\bs{\Phi}(x)=\left(\begin{matrix}
		y_{11}(x)&\cdots&y_{1n}(x)\\
		\vdots&\ddots&\vdots\\
		y_{n1}(x)&\cdots&y_{nn}(x)
	\end{matrix}\right)
	$$
	为齐次线性微分方程
	$$
	\frac{\mathrm{d}\bs{y}}{\mathrm{d}x}=\bs{A}(x)\bs{y}
	$$
	的基本解矩阵,那么其Wronsky行列式$W(x)=|\bs{\Phi}(x)|$成立
	$$
	W(x)=W(x_0)\exp\left(\int_{x_0}^{x}{\text{tr}(\bs{A}(x))\mathrm{d}x}\right),\qquad a<x<b
	$$
	其中$a<x_0<b$。
\end{theorem}

\begin{theorem}{常数变易法}
	如果$\bs{\Phi}(x)$是齐次线性微分方程
	$$
	\frac{\mathrm{d}\bs{y}}{\mathrm{d}x}=\bs{A}(x)\bs{y}
	$$
	的基本解矩阵,那么微分方程
	$$
	\frac{\mathrm{d}\bs{y}}{\mathrm{d}x}=\bs{A}(x)\bs{y}+\bs{f}(x)
	$$
	在区间$a<x<b$上的通解可表示为
	$$
	\bs{y}=\bs{\Phi}(x)\left(\bs{c}+\int_{x_0}^{x}{\Phi^{-1}(s)\bs{f}(s)\mathrm{d}s}\right),\qquad \bs{c}\in\R^n
	$$
	且微分方程满足初值条件$\bs{y}(x_0)=\bs{y}_0$的解为
	$$
	\bs{y}=\bs{\Phi}(x)\Phi^{-1}(x_0)\bs{y}_0+\bs{\Phi}(x)\int_{x_0}^{x}{\Phi^{-1}(s)\bs{f}(s)\mathrm{d}s}
	$$
\end{theorem}

\subsection{常系数线性微分方程}

\begin{theorem}{常系数线性微分方程}
	常系数线性微分方程%
	$$
	\frac{\dd \bs{y}}{\dd x}=\bs{A}\bs{y}
	$$
	的基本解矩阵为%
	$$
	\bs{\Phi}(x)=\ee{\bs{A}x}
	$$
	若$\bs{A}$可对角化为$\bs{A}=\bs{Q\Lambda Q}^{-1}$,则
	$$
	\bs{\Phi}(x)=\bs{Q}\ee{\bs{\Lambda}x}\bs{Q}^{-1}
	$$
\end{theorem}

\begin{theorem}
	如果$y=\varphi(x)$是二阶齐次线性方程
	$$
	y''+p(x)y'+q(x)y=0
	$$
	的非零解,其中$p(x)$和$q(x)$是区间$a<x<b$上的连续函数,那么微分方程的通解为
	$$
	y=\varphi(x)\left(C_1+C_2\int_{x_0}^{x}{\frac{1}{\varphi^2(x)}\exp\left(-\int_{x_0}^{s}{p(t)\mathrm{d}t}\right)\mathrm{d}s}\right),\qquad C_1,C_2\in\R
	$$
\end{theorem}

\begin{theorem}{常系数齐次线性方程}
	如果常系数齐次线性方程
	$$
	y^{(n)}+a_1y^{(n-1)}+\cdots+a_{n-1}y^{\prime}+a_ny=0
	$$
	的特征方程
	$$
	\lambda^n+a_1\lambda^{n-1}+\cdots+a_{n-1}\lambda+a_n=0
	$$
	在复数域$\C$中共有$r$个互不相同的根$\lambda_1,\cdots,\lambda_r$,且对应的重数分别为$n_1,\cdots,n_r$,满足$n_1+\cdots+n_r=n$,那么函数组
	$$
	\begin{cases}
		\mathrm{e}^{\lambda_1 x},
		\qquad 
		x\mathrm{e}^{\lambda_1 x},
		\qquad
		\cdots,
		\qquad 
		x^{n_1-1}\mathrm{e}^{\lambda_1 x}\\
		\cdots\\
		\mathrm{e}^{\lambda_r x},\qquad
		x\mathrm{e}^{\lambda_r x},\qquad
		\cdots,\qquad
		x^{n_r-1}\mathrm{e}^{\lambda_r x}
	\end{cases}
	$$
	是齐次微分方程的一个基本解组。
\end{theorem}

\begin{theorem}{常系数非齐次线性方程}
	对于常系数非齐次线性方程
	$$
	y^{(n)}+a_1y^{(n-1)}+\cdots+a_{n-1}y^{\prime}+a_ny=f(x)
	$$
	其特解$\varphi^*(x)$如下。
	\begin{enumerate}
		\item $f(x)=P_n(x)\mathrm{e}^{\lambda x}$:
		$$
		\varphi^*(x)=x^k Q_n(x)\mathrm{e}^{\lambda x}
		$$
		其中$P_n(x)$和$Q_n(x)$为$n$次多项式,$k\in \N$为特征方程
		$$
		\lambda^n+a_1\lambda^{n-1}+\cdots+a_{n-1}\lambda+a_n=0
		$$
		的特征根中$\lambda$的重数。
		\item $f(x)=(A_m(x)\cos(\omega x)+B_n(x)\sin(\omega x))\mathrm{e}^{\lambda x}$:
		$$
		\varphi^*(x)=x^k(P_l(x)\cos(\omega x)+Q_l\sin(\omega x))\mathrm{e}^{\lambda x}
		$$
		其中$A_m(x)$和$B_n(x)$分别为关于$x$的$m$和$n$次多项式,$P_l(x)$和$Q_l(x)$为的$l$次多项式且$l=\max(m,n)$,$k\in \N$为特征方程
		$$
		\lambda^n+a_1\lambda^{n-1}+\cdots+a_{n-1}\lambda+a_n=0
		$$
		的特征根中$\lambda+i\omega$的重数。
	\end{enumerate}
\end{theorem}

\begin{theorem}{Euler方程}
	对于Euler方程%
	$$
	x^ny^{(n)}+a_1x^{n-1}y^{(n-1)}+\cdots+a_ny=f(x)
	$$
	令$x=\ee{t}$,那么%
	$$
	x^n\frac{\dd^n y}{\dd t^n}
	=\prod_{k=0}^{n-1}\left(\frac{\dd}{\dd t}-k\right)y
	$$
\end{theorem}

\section{幂级数解法}

\subsection{Cauchy定理}

\begin{definition}{解析函数}
	称函数$f(x,y)$在区域$G\in{\R}^2$内为解析的,如果对于$G$内的任意一点$(x_0,y_0)$,存在正常数$a$和$b$,使得函数$f(x,y)$在区域
	$$
	|x-x_0|\le a,\qquad
	|y-y_0|\le b
	$$
	内可以展成$(x-x_0)$和$(y-y_0)$的收敛幂级数
	$$
	f(x,y)=\sum_{i,j=0}^{\infty}{a_{ij}(x-x_0)^i(y-y_0)^j}
	$$
\end{definition}

\begin{theorem}{Cauchy定理}
	如果函数$f(x,y)$在矩形区域$R:|x-x_0|\le\alpha,|y-y_0|\le\beta$上可以展开成$(x-x_0)$和$(y-y_0)$的收敛幂级数
	$$
	f(x,y)=\sum_{i,j=0}^{\infty}{a_{ij}(x-x_0)^i(y-y_0)^j}
	$$
	则初值问题
	$$
	\frac{\mathrm{d}y}{\mathrm{d}x}=f(x,y),\qquad 
	y(x_0)=y_0
	$$
	在$x_0$点的邻域$|x-x_0|\le\rho$内存在且存在唯一解析解$y=y(x)$,其中
	$$
	\rho=a(1-\mathrm{e}^{-\frac{b}{2aM}}),\qquad 
	M\le|a_{ij}|a^ib^j
	$$
\end{theorem}

\subsection{幂级数解法}

\begin{theorem}{幂级数解法}
	微分方程
	$$
	y^{\prime\prime}+p(x)y^{\prime}+q(x)y=0
	$$
	中的系数函数$p(x)$和$q(x)$在区间$|x-x_0|<r$可以展成$(x-x_0)$的收敛幂级数,则微分方程在区间$|x-x_0|<r$存在收敛的幂级数解
	$$
	y=\sum_{n=0}^{\infty}{C_n(x-x_0)^n}
	$$
	其中$C_0$和$C_1$为任意常数,而当$n\ge2$时$C_n$可以通过递推公式确定。
\end{theorem}

\chapter{常微分方程四大定理}

\section{存在与存在唯一性定理}

\subsection{Peano存在定理}

\begin{theorem}{Asscoli引理}
	对于$[a,b]$上的函数族$\mathscr{F}$,如果$\mathscr{F}$在$[a,b]$上一致有界且等度连续,那么存在函数序列$\{f_n(x)\}_{n=1}^{\infty}\sub\mathscr{F}$,使得$f_n(x)$在$[a,b]$上一致收敛。
\end{theorem}

\begin{theorem}{Peano存在定理}
	对于初值问题
	\begin{equation*}
		\frac{\dd y}{\dd x}=f(x,y),\qquad 
		y(x_0)=y_0
		\label{Peano存在定理的初值问题}
		\tag{*}
	\end{equation*}
	如果$f(x,y)$在矩形区域
	$$
	R:\qquad 
	|x-x_0|\le a,\qquad 
	|y-y_0|\le b
	$$
	内连续,那么原初值问题在区间$[x_0-h,x_0+h]$内存在解,其中
	$$
	h=\min{\left\{a,b/M\right\}},\qquad 
	M>\max_{(x,y)\in{R}}{|f(x,y)|}
	$$
\end{theorem}

\begin{proof}
	\begin{enumerate}
		\item 构造Euler折线%
		$$
		\varphi_n(x)=y_0+\sum_{k=0}^{-s+1}f(x_k,y_k)(x_{k-1}-x_k)+f(x_{-s},y_{-s})(x-x_{-s})
		$$
		\item 证明Euler序列$y=\varphi_n(x)$存在一致收敛子列,不妨仍记为$y=\varphi_n(x)$。
		\item 证明Euler序列$y=\varphi_n(x)$成立%
		$$
		\varphi_n(x)=y_0+\int_{x_0}^{x}f(x,\varphi_n(x))\dd x+\delta_n(x)
		$$
		其中$\delta_n(x)\to 0$。
		\item Euler序列$y=\varphi_n(x)$的极限$y=\varphi(x)$为初值问题(\ref{Peano存在定理的初值问题})的解。
	\end{enumerate}
\end{proof}

\begin{note}
	不可去掉“$f(x,y)$连续”的条件,例如:对于
	$$
	f(x,y)=\begin{cases}
		0,\qquad & x\ne 0\\
		1,\qquad & x=0
	\end{cases}
	$$
	初值问题
	$$
	\frac{\dd y}{\dd x}=f(x,y),\qquad 
	y(0)=0
	$$
	不存在解。事实上,如果$y(x)$可导,那么其导函数不存在第一类间断点。
\end{note}

\subsection{Picard存在唯一性定理}

\begin{theorem}{Picard存在唯一性定理}
	对于初值问题
	\begin{equation*}
		\frac{\dd y}{\dd x}=f(x,y),\qquad 
		y(x_0)=y_0
		\label{Picard存在唯一性定理的初值问题}
		\tag{1}
	\end{equation*}
	如果$f(x,y)$在矩形区域
	$$
	R:\qquad 
	|x-x_0|\le a,\qquad 
	|y-y_0|\le b
	$$
	内连续,且对$y$成立以$L$为Lipschitz系数的Lipschitz条件,那么初值问题(\ref{Picard存在唯一性定理的初值问题})在区间$I=[x_0-h,x_0+h]$内存在且存在唯一解,其中
	$$
	h=\min{\left\{a,\frac{b}{M}\right\}},\qquad 
	M>\max_{(x,y)\in{R}}{|f(x,y)|}
	$$
\end{theorem}

\begin{proof}
	{\bf 经典证明:}
	\begin{enumerate}
		\item 初值问题(\ref{Picard存在唯一性定理的初值问题})$\iff$积分方程
		\begin{equation*}
			y=y_0+\int_{x_0}^{x}f(x,y)\dd x
			\label{Picard存在唯一性定理的初值问题的积分形式}
			\tag{2}
		\end{equation*}
		事实上,一方面,若$y=y(x),x\in I$为初值问题(\ref{Picard存在唯一性定理的初值问题})的解,则%
		$$
		y'(x)=f(x,y(x)),\qquad y(x_0)=y_0,\qquad x\in I
		$$
		由此积分%
		$$
		y(x)=y_0+\int_{x_0}^{x}f(x,y(x))\dd x,\qquad x\in I
		$$
		因此$y=y(x),x\in I$为积分方程(\ref{Picard存在唯一性定理的初值问题的积分形式})的解。
		
		另一方面,若$y=y(x),x\in I$为积分方程(\ref{Picard存在唯一性定理的初值问题的积分形式})的解,则
		$$
		y(x)=y_0+\int_{x_0}^{x}f(x,y(x))\dd x,\qquad x\in I
		$$
		由此微分
		$$
		y'(x)=f(x,y(x)),\qquad y(x_0)=y_0,\qquad x\in I
		$$
		因此$y=y(x),x\in I$为初值问题(\ref{Picard存在唯一性定理的初值问题})的解。
		\item 迭代构造Picard序列
		\begin{equation*}
			y_{n+1}(x)=y_0+\int_{x_0}^{x}f(x,y_n(x))\dd x,\qquad y_0(x)=y_0,\qquad x\in I
			\label{Picard序列}
			\tag{3}
		\end{equation*}
		由归纳法证明:Picard序列$y=y_n(x)$在$I$上连续,且成立不等式%
		$$
		|y_n(x)-y_0|\le M|x-x_0|
		$$
		\begin{enumerate}
			\item 当$n=0$时,由于$f(x,y_0(x))=f(x,y_0)$为$I$上的连续函数,因此%
			$$
			y_{1}(x)=y_0+\int_{x_0}^{x}f(x,y_0(x))\dd x
			$$
			在$I$上连续可微,且成立
			\begin{equation*}
				|y_1(x)-y_0|=
				\left| \int_{x_0}^{x}f(x,y_0(x))\dd x \right|
				\le \int_{x_0}^{x}|f(x,y_0(x))|\dd x
				\le M|x-x_0|
				\label{Picard序列的第一式}
				\tag{4}
			\end{equation*}
			从而在$I$上成立$|y_1(x)-y_0|\le Mh\le b$。
			\item 假设$y=y_k(x)$在$I$上连续,且成立不等式%
			$$
			|y_k(x)-y_0|\le M|x-x_0|
			$$
			则%
			$$
			y_{k+1}(x)=y_0+\int_{x_0}^{x}f(x,y_k(x))\dd x
			$$
			在$I$上连续可微,且成立
			$$
			|y_{k+1}(x)-y_0|=
			\left| \int_{x_0}^{x}f(x,y_k(x))\dd x \right|
			\le \int_{x_0}^{x}|f(x,y_k(x))|\dd x
			\le M|x-x_0|
			$$
			从而在$I$上成立$|y_{k+1}(x)-y_0|\le Mh\le b$。
		\end{enumerate}
		\item 证明:Picard序列$y=y_n(x)$在$I$一致收敛于积分方程(\ref{Picard存在唯一性定理的初值问题的积分形式})的解。
		
		注意到,序列$y_n(x)$的收敛性$\iff$级数
		\begin{equation*}
			\sum_{n=1}^{\infty}(y_{n+1}(x)-y_n(x))
			\label{Picard序列的级数}
			\tag{5}
		\end{equation*}
		的收敛性。下面证明:级数(\ref{Picard序列的级数})在$I$上一致收敛。为此,我们归纳证明不等式:
		\begin{equation*}
			|y_{n+1}(x)-y_n(x)|\le\frac{M}{L}\frac{(L|x-x_0|)^{n+1}}{(n+1)!},\qquad x\in I
			\label{Picard序列的不等式}
			\tag{6}
		\end{equation*}
		\begin{enumerate}
			\item 当$n=0$时,(\ref{Picard序列的第一式})$\implies$(\ref{Picard序列的不等式})。
			\item 假设当$n=k$时成立(\ref{Picard序列的不等式}),则先由(\ref{Picard序列})推出%
			$$
			|y_{k+2}(x)-y_{k+1}(x)|
			=\left|\int_{x_0}^{x}(f(x,y_{k+1}(x))-f(x,y_k(x)))\dd x\right|
			$$
			然后利用Lipschitz条件与归纳假设,可得
			\begin{align*}
				|y_{k+2}(x)-y_{k+1}(x)|
				& \le \left|\int_{x_0}^{x}L|y_{k+1}(x)-y_k(x)|\dd x\right| \\
				& \le M \left| \int_{x_0}^{x}\frac{(L|x-x_0|)^{k+1}}{(k+1)!}\dd x \right| \\
				& = \frac{M}{L}\frac{(L|x-x_0|)^{k+2}}{(k+2)!}
			\end{align*}
			因此当$n=k+1$时,成立(\ref{Picard序列的不等式})。
		\end{enumerate}
		由归纳假设,成立(\ref{Picard序列的不等式})。
		
		显然,不等式(\ref{Picard序列的不等式})蕴含级数(\ref{Picard序列的级数})在$I$上一致收敛,因此Picard序列$y=y_n(x)$在$I$上一致收敛,因此极限函数%
		$$
		\varphi(x)=\lim_{n\to\infty}y_n(x),\qquad x\in I
		$$
		在区间$I$上连续。由$f(x,y)$的连续性与Picard序列$y=y_n(x)$的一致连续性,在(\ref{Picard序列})中令$n\to\infty$可得%
		$$
		\varphi(x)=y_0+\int_{x_0}^{x}f(x,\varphi(x))\dd x,\qquad x\in I
		$$
		因此$y=\varphi(x)$为积分方程(\ref{Picard存在唯一性定理的初值问题的积分形式})在$I$上的解。
		\item 最后证明唯一性。假设积分方程(\ref{Picard存在唯一性定理的初值问题的积分形式})在$I$上存在两个解$y=\varphi(x)$与$y=\psi(x)$,则由积分方程(\ref{Picard存在唯一性定理的初值问题的积分形式}),可得%
		$$
		\varphi(x)-\psi(x)
		=\int_{x_0}^{x}(f(x,\varphi(x))-f(x,\psi(x)))\dd x
		$$
		由Lipschitz条件
		\begin{equation*}
			|\varphi(x)-\psi(x)|
			\le L\int_{x_0}^{x}|u(x)-v(x)|\dd x
			\label{存在唯一性的唯一性}
			\tag{7}
		\end{equation*}
		取$|u(x)-v(x)|$在$I$上的上界$K$,则由(\ref{存在唯一性的唯一性})
		$$
		|\varphi(x)-\psi(x)|
		\le LK|x-x_0|
		$$
		代入(\ref{存在唯一性的唯一性})
		$$
		|\varphi(x)-\psi(x)|
		\le K\frac{(L|x-x_0|)^2}{2}
		$$
		由归纳法%
		$$
		|\varphi(x)-\psi(x)|
		\le K\frac{(L|x-x_0|)^n}{n!}
		$$
		令$n\to\infty$,则$\varphi(x)=\psi(x)$。进而说明积分方程(\ref{Picard存在唯一性定理的初值问题的积分形式})在$I$上的解唯一。
	\end{enumerate}
\end{proof}

\begin{theorem}{Picard存在唯一性定理}
	对于初值问题
	\begin{equation*}
		\frac{\dd y}{\dd x}=f(x,y),\qquad 
		y(x_0)=y_0
		\label{Picard存在唯一性定理的初值问题问题}
		\tag{*}
	\end{equation*}
	如果$f(x,y)$在带形区域
	$$
	R:\qquad 
	|x-x_0|\le \delta,\qquad 
	y\in\R
	$$
	内连续,且对$y$成立以$L$为Lipschitz系数的Lipschitz条件,那么初值问题(\ref{Picard存在唯一性定理的初值问题问题})在区间$I=[x_0-h,x_0+h]$内存在且存在唯一连续解,其中$0<h<\min\{ \delta,1/L \}$。
\end{theorem}

\begin{proof}
	{\bf 压缩映像原理:}考虑连续函数空间$C(I)$,构造算子
	\begin{align*}
		T:\begin{aligned}[t]
			C(I) &\longrightarrow C(I)\\
			\varphi &\longmapsto T_\varphi,\text{其中}T_\varphi(x)=y_0+\int_{x_0}^{x}f(t,\varphi(t))\dd t
		\end{aligned}
	\end{align*}
	容易知道$y=\varphi(x)$为初值问题(\ref{Picard存在唯一性定理的初值问题问题})的连续解$\iff \varphi$为$T$的不动点。由于$f(x,y)$对$y$成立以$L$为Lipschitz系数的Lipschitz条件,那么
	\begin{align*}
		\| T(\varphi)-T(\psi) \|
		& = \sup_{x\in I}|(T(\varphi)-T(\psi))(x)| \\
		& = \sup_{|x-x_0|\le h}\left| \int_{x_0}^{x}(f(t,\varphi(t))-f(t,\psi(t)))\dd t \right| \\
		& \le \sup_{|x-x_0|\le h} \int_{x_0}^{x}|f(t,\varphi(t))-f(t,\psi(t))|\dd t \\
		& \le \sup_{|x-x_0|\le h} \int_{x_0}^{x}L|\varphi(t)-\psi(t)|\dd t \\
		& \le \sup_{|x-x_0|\le h} \int_{x_0}^{x}L\|\varphi-\psi\|\dd t \\
		& = \sup_{|x-x_0|\le h} (x-x_0)L\|\varphi-\psi\| \\
		& = Lh\|\varphi-\psi\|
	\end{align*}
	而$Lh<1$,从而$T$为压缩映射,由压缩映像原理,$T$存在且存在唯一不动点,进而值问题(\ref{Picard存在唯一性定理的初值问题问题})在区间$I$内存在且存在唯一连续解。
\end{proof}

\begin{note}
	“$f(x,y)$对$y$成立Lipschitz条件”仅为充分条件,例如对于初值问题
	$$
	\frac{\dd y}{\dd x}=\sqrt{|y|},\qquad 
	y(0)=0
	$$
	函数$f(x,y)=\sqrt{|y|}$在$y$的任意邻域内不成立Lipschitz条件,但原初值问题存在且存在唯一解
	$$
	y=\begin{cases}
		x^2/4,\qquad & x\ge 0\\
		-x^2/4,\qquad & x<0
	\end{cases}
	$$
\end{note}

\begin{definition}{Osgood条件}
	称连续函数$f(x)$成立Osgood条件,如果存在函数$F(r)$,使得成立
	$$
	F(r)>0,(r>0),\qquad 
	F(0)=0,\qquad 
	\int_{0}^{1}\frac{\dd r}{F(r)}=+\infty
	$$
	且对于任意$x,y$,成立
	$$
	|f(x)-f(y)|\le F(|x-y|)
	$$
\end{definition}

\begin{theorem}{Osgood定理}
	对于初值问题
	$$
	\frac{\dd y}{\dd x}=f(x,y),\qquad 
	y(x_0)=y_0
	$$
	如果$f(x,y)$在矩形区域
	$$
	R:\qquad 
	|x-x_0|\le a,\qquad 
	|y-y_0|\le b
	$$
	内连续,且对$y$成立Osgood条件,那么原初值问题在区间$[x_0-h,x_0+h]$内存在且存在唯一解,其中
	$$
	h=\min{\left\{a,\frac{b}{M}\right\}},\qquad 
	M>\max_{(x,y)\in{R}}{|f(x,y)|}
	$$
\end{theorem}

\section{解的延伸定理}

\subsection{解的延伸定理}

\begin{theorem}{解的延伸定理}
	如果$f(x,y)$在区域$G$内连续,那么对于任意$(x_0,y_0)\in G$,初值问题
	$$
	\frac{\dd y}{\dd x}=f(x,y),\qquad 
	y(x_0)=y_0
	$$
	的积分曲线$\Gamma$延伸至$G$的边界。
\end{theorem}

\begin{theorem}
	如果函数$f(x,y)$在连通开集
	$$
	S:\qquad \alpha <x< \beta,\qquad -\infty <y< +\infty
	$$
	内连续,而且满足不等式
	$$
	|f(x,y)|\le A(x)|y|+B(x)
	$$
	其中$A(x)\ge 0$和$B(x)\ge 0$在区间$\alpha <x< \beta$连续,那么对于任意$(x_0,y_0)\in S$,初值问题
	$$
	\frac{\mathrm{d}y}{\mathrm{d}x}=f(x,y),\qquad 
	y(x_0)=y_0
	$$
	的解区间$\alpha <x< \beta$为最大存在区间。
\end{theorem}

\subsection{比较定理}

\begin{theorem}{第一比较定理}
	设函数$f(x,y)$与$F(x,y)$都在连通开集$G$内连续且满足不等式
	$$
	f(x,y)<F(x,y),\qquad (x,y)\in G
	$$
	又设函数$y=\varphi(x)$与$y=\Phi(x)$在区间$a<x<b$上分别是初值问题
	$$
	\frac{\mathrm{d}y}{\mathrm{d}x}=f(x,y),\qquad 
	y(x_0)=y_0
	$$
	与
	$$
	\frac{\mathrm{d}y}{\mathrm{d}x}=F(x,y),\qquad 
	y(x_0)=y_0
	$$
	的解,其中$(x_0,y_0)\in G$,则有
	$$
	\begin{cases}
		\varphi(x)<\Phi(x),x_0<x<b\\
		\varphi(x)>\Phi(x),a<x<x_0
	\end{cases}
	$$
\end{theorem}

\begin{theorem}{第二比较定理}
	设函数$f(x,y)$与$F(x,y)$都在平面区域$G$内连续且满足不等式
	$$
	f(x,y)<F(x,y),\qquad (x,y)\in G
	$$
	又设函数$y=\varphi(x)$与$y=\Phi(x)$在区间$a<x<b$上分别是初值问题
	$$
	\frac{\mathrm{d}y}{\mathrm{d}x}=f(x,y),\qquad 
	y(x_0)=y_0
	$$
	与
	$$
	\frac{\mathrm{d}y}{\mathrm{d}x}=F(x,y),\qquad 
	y(x_0)=y_0
	$$
	的解,其中$(x_0,y_0)\in G$,并且$y=\varphi(x)$是初值问题
	$$
	\begin{cases}
		\frac{\mathrm{d}y}{\mathrm{d}x}=f(x,y)\\
		y(x_0)=y_0
	\end{cases}
	$$
	的右行最小解和左行最大解,或者$y=\Phi(x)$是初值问题
	$$
	\begin{cases}
		\frac{\mathrm{d}y}{\mathrm{d}x}=F(x,y)\\
		y(x_0)=y_0
	\end{cases}
	$$
	的右行最小解和左行最大解,则有如下比较关系
	$$
	\begin{cases}
		\varphi(x)\le\Phi(x),x_0\le x<b\\
		\varphi(x)\ge\Phi(x),a<x\le x_0
	\end{cases}
	$$
\end{theorem}

\section{解对初值和参数的连续依赖性}

对于一般的微分方程初值问题
$$
\frac{\dd y}{\dd x}=f(x,y,\lambda),\qquad 
y(x_0)=y_0
$$
可作线性变换$x'=x-x_0$与$y'=y-y_0$,那么上述初值问题化为
$$
\frac{\dd y'}{\dd x'}=f(x',y',\lambda),\qquad 
y'(0)=0
$$
因此本节对于探究解对初值和参数的连续依赖性,仅考虑初值问题
$$
\frac{\dd y}{\dd x}=f(x,y,\lambda),\qquad 
y(0)=0
$$

\begin{definition}{解对初值和参数的连续依赖性}
	考虑初值问题%
	$$
	\frac{\dd y}{\dd x}=f(x,y,\lambda),\qquad 
	y(x_0)=y_0
	$$
	记其解为$y=\varphi(x,x_0,y_0,\lambda)$。所谓解对初值和参数的连续依赖性,就是$\varphi$对于$x_0,y_0,\lambda$的连续性。
\end{definition}

\begin{theorem}{解对初值和参数的连续依赖性}
	如果函数$f(x,y,\lambda)$在矩形区域%
	$$
	R:\qquad 
	|x|\le a,\qquad 
	|y|\le b,\qquad 
	|\lambda-\lambda_0|\le c
	$$
	上连续,且对于$y$满足Lipschitz条件,那么微分方程初值问题
	$$
	\frac{\dd y}{\dd x}=f(x,y,\lambda),\qquad 
	y(0)=0
	$$
	的解$y=\varphi(x,\lambda)$在矩形区域%
	$$
	D:\qquad 
	|x|\le h,\qquad 
	|\lambda-\lambda_0|\le c
	$$
	上是连续的,其中%
	$$
	h=\min\{ a,b/M \},\qquad 
	M\ge \max_{(x,y,\lambda)\in G}|f(x,y,\lambda)|
	$$
\end{theorem}

\section{解对初值和参数的连续可微性}

对于一般的微分方程初值问题
$$
\frac{\dd y}{\dd x}=f(x,y,\lambda),\qquad 
y(x_0)=y_0
$$
可作线性变换$x'=x-x_0$与$y'=y-y_0$,那么上述初值问题化为
$$
\frac{\dd y'}{\dd x'}=f(x',y',\lambda),\qquad 
y'(0)=0
$$
因此本节对于探究解对初值和参数的连续可微性,仅考虑初值问题
$$
\frac{\dd y}{\dd x}=f(x,y,\lambda),\qquad 
y(0)=0
$$

\begin{definition}{解对初值和参数的连续可微性}
	考虑初值问题%
	$$
	\frac{\dd y}{\dd x}=f(x,y,\lambda),\qquad 
	y(x_0)=y_0
	$$
	记其解为$y=\varphi(x,x_0,y_0,\lambda)$。所谓解对初值和参数的连续可微性,就是$\varphi$对于$x_0,y_0,\lambda$的连续可微性。
\end{definition}

\begin{theorem}{解对初值和参数的连续可微性}
	如果函数$f(x,y,\lambda)$在矩形区域%
	$$
	R:\qquad 
	|x|\le a,\qquad 
	|y|\le b,\qquad 
	|\lambda-\lambda_0|\le c
	$$
	上连续,且对于$y$与$\lambda$存在连续偏微商,那么微分方程初值问题
	$$
	\frac{\dd y}{\dd x}=f(x,y,\lambda),\qquad 
	y(0)=0
	$$
	的解$y=\varphi(x,\lambda)$在矩形区域%
	$$
	D:\qquad 
	|x|\le h,\qquad 
	|\lambda-\lambda_0|\le c
	$$
	上是连续可微的,其中%
	$$
	h=\min\{ a,b/M \},\qquad 
	M\ge \max_{(x,y,\lambda)\in G}|f(x,y,\lambda)|
	$$
\end{theorem}

\section{四大定理总结}

\begin{theorem}{Picard存在唯一性定理}
	对于初值问题
	$$
	\frac{\dd y}{\dd x}=f(x,y),\qquad 
	y(x_0)=y_0
	$$
	如果$f(x,y)$在矩形区域
	$$
	R:\qquad 
	|x-x_0|\le a,\qquad 
	|y-y_0|\le b
	$$
	内连续,且对$y$成立Lipschitz条件,那么原初值问题在区间$[x_0-h,x_0+h]$内存在且存在唯一解,其中
	$$
	h=\min{\left\{a,b/M\right\}},\qquad 
	M>\max_{(x,y)\in{R}}{|f(x,y)|}
	$$
\end{theorem}

\begin{theorem}{解的延伸定理}
	如果$f(x,y)$在区域$G$内连续,那么对于任意$(x_0,y_0)\in G$,初值问题
	$$
	\frac{\dd y}{\dd x}=f(x,y),\qquad 
	y(x_0)=y_0
	$$
	的积分曲线$\Gamma$延伸至$G$的边界。
\end{theorem}

\begin{theorem}{解对初值和参数的连续依赖性}
	如果函数$f(x,y,\lambda)$在矩形区域%
	$$
	R:\qquad 
	|x|\le a,\qquad 
	|y|\le b,\qquad 
	|\lambda-\lambda_0|\le c
	$$
	上连续,且对于$y$满足Lipschitz条件,那么微分方程初值问题
	$$
	\frac{\dd y}{\dd x}=f(x,y,\lambda),\qquad 
	y(0)=0
	$$
	的解$y=\varphi(x,\lambda)$在矩形区域%
	$$
	D:\qquad 
	|x|\le h,\qquad 
	|\lambda-\lambda_0|\le c
	$$
	上是连续的,其中%
	$$
	h=\min\{ a,b/M \},\qquad 
	M\ge \max_{(x,y,\lambda)\in G}|f(x,y,\lambda)|
	$$
\end{theorem}

\begin{theorem}{解对初值和参数的连续可微性}
	如果函数$f(x,y,\lambda)$在矩形区域%
	$$
	R:\qquad 
	|x|\le a,\qquad 
	|y|\le b,\qquad 
	|\lambda-\lambda_0|\le c
	$$
	上连续,且对于$y$与$\lambda$存在连续偏微商,那么微分方程初值问题
	$$
	\frac{\dd y}{\dd x}=f(x,y,\lambda),\qquad 
	y(0)=0
	$$
	的解$y=\varphi(x,\lambda)$在矩形区域%
	$$
	D:\qquad 
	|x|\le h,\qquad 
	|\lambda-\lambda_0|\le c
	$$
	上是连续可微的,其中%
	$$
	h=\min\{ a,b/M \},\qquad 
	M\ge \max_{(x,y,\lambda)\in G}|f(x,y,\lambda)|
	$$
\end{theorem}

\begin{table}[H]
	\centering
	\begin{tabular}{cc}
		\toprule
		\textbf{条件} & \textbf{结论} \\
		\midrule
		连续性 & 存在解 \\
		连续性 & 解延伸至边界 \\
		连续性+Lipschitz条件 & 存在且存在唯一解 \\
		连续性+Lipschitz条件 & 解对初值与参数连续 \\
		连续性+连续可偏微商 & 解对初值与参数连续可微 \\
		\bottomrule
	\end{tabular}
\end{table}

\section{Lyapunov稳定性}

\begin{definition}{Lyapunov稳定性}
	对于微分方程%
	$$
	\frac{\dd \bs{x}}{\dd t}=f(\bs{x},t)
	$$
	其中$f(\bs{x},t)$对于$\bs{x}\in \Omega\sub\R^n$与$t\in\R$连续,且对于$\bs{x}$成立Lipschitz条件,称其在$[t_0,\infty)$上存在定义的解$\bs{x}=\bs{\varphi}(t)$成立Lyapunov稳定性,如果对于任意$\varepsilon>0$,存在$\delta>0$,使得对于任意成立%
	$$
	|\bs{x}_0-\bs{\varphi}(t_0)|<\delta
	$$
	的$\bs{x}_0\in\R^n$,原微分方程以$\bs{x}(t_0)=\bs{x}_0$为初值的解$\bs{x}=\bs{\varphi}(t,t_0,\bs{x}_0)$在$[t_0,\infty)$上存在定义,且对于任意$t\ge t_0$,成立
	$$
	|\bs{\varphi}(t,t_0,\bs{x}_0)-\bs{\varphi}(t)|<\varepsilon
	$$
\end{definition}

\begin{definition}{Lyapunov渐进稳定性}
	对于微分方程%
	$$
	\frac{\dd \bs{x}}{\dd t}=f(\bs{x},t)
	$$
	其中$f(\bs{x},t)$对于$\bs{x}\in \Omega\sub\R^n$与$t\in\R$连续,且对于$\bs{x}$成立Lipschitz条件,称其存在Lyapunov稳定性的解$\bs{x}=\bs{\varphi}(t)$成立Lyapunov渐进稳定性,如果存在$\delta>0$,使得对于任意成立%
	$$
	|\bs{x}_0-\bs{\varphi}(t_0)|<\delta
	$$
	的$\bs{x}_0\in\R^n$,原微分方程以$\bs{x}(t_0)=\bs{x}_0$为初值的解$\bs{x}=\bs{\varphi}(t,t_0,\bs{x}_0)$在$[t_0,\infty)$上存在定义,且成立%
	$$
	\lim_{t\to+\infty}|\bs{\varphi}(t,t_0,\bs{x}_0)-\bs{\varphi}(t)|=0
	$$
\end{definition}

\section{奇解与包络}

\subsection{奇解}

\begin{definition}{奇解}
	称一阶微分方程
	$$
	F\left(x,y,\frac{\mathrm{d}y}{\mathrm{d}x}\right)=0
	$$
	的特解
	$$
	\Gamma:\qquad y=\varphi(x),\qquad x\in I
	$$
	为奇解,如果对于任意$(x_0,y_0)\in \Gamma$,以及$(x_0,y_0)$的任意邻域内微分方程
	$$
	F\left(x,y,\frac{\mathrm{d}y}{\mathrm{d}x}\right)=0
	$$
	存在不同于$\Gamma$的解,使得其在$(x_0,y_0)$处与$\Gamma$相切。
\end{definition}

\begin{theorem}{奇解存在的必要条件}
	设函数$F(x,y,p)$对$(x,y,p)\in G$是连续的,而且对$y$和$p$有连续的偏微商$F_y^{\prime}$和$F_p^{\prime}$。若函数$y=\varphi(x),x\in J$是微分方程
	$$
	F(x,y,\frac{\mathrm{d}y}{\mathrm{d}x})=0
	$$
	的一个奇解,并且$(x,\varphi(x),\varphi^{\prime}(x))\in G$,则奇解$y=\varphi(x)$满足$p$-判别式
	$$
	\begin{cases}F(x,y,p)=0\\F_p^{\prime}(x,y,p)=0\end{cases}
	$$
	若从方程组中消去$p$,可得到方程
	$$
	\Delta(x,y)=0
	$$
	由此决定的曲线为微分方程
	$$
	F(x,y,\frac{\mathrm{d}y}{\mathrm{d}x})=0
	$$
	的$p$-判别曲线。因此,原微分方程的奇解是一条$p$-判别曲线。
\end{theorem}

\begin{note}
	需要注意的是,由$p$-判别式
	$$
	\begin{cases}F(x,y,p)=0\\F_p^{\prime}(x,y,p)=0\end{cases}
	$$
	所确定的函数$y=\psi(x)$不一定是相应微分方程的解;即使是解,也不一定是奇解。这是因为,在联立方程组时,参数$p$丧失了与$x$和$y$的关系,而成为了一个独立的变量。事实上由$p$-判别式求得的$y=\psi(x)$和$p=p(x)$,一定要满足$\frac{\mathrm{d}y}{\mathrm{d}x}=p$,只有这样,函数$y=\psi(x)$才是微分方程的解,但未必是奇解。
\end{note}

\begin{theorem}{奇解存在的充分条件}
	设函数$F(x,y,p)$对$(x,y,p)\in G$是二阶连续可微的,且由微分方程
	$$
	F\left(x,y,\frac{\mathrm{d}y}{\mathrm{d}x}\right)=0
	$$
	的$p$-判别式
	$$
	\begin{cases}F(x,y,p)=0\\F_p^{\prime}(x,y,p)=0\end{cases}
	$$
	所确定的函数$y=\psi(x),x\in J$为微分方程的解。若满足条件
	$$
	\begin{cases}F_y^{\prime}(x,\psi(x),\psi^{\prime}(x))\ne 0\\F_{pp}^{\prime\prime}(x,\psi(x),\psi^{\prime}(x))\ne 0\\F_p^{\prime}(x,\psi(x),\psi^{\prime}(x))=0\end{cases}
	$$
	对于任意$x\in J$成立,则$y=\psi(x)$是微分方程的奇解。
\end{theorem}

\begin{note}
	奇解存在的充分条件中的
	$$
	\begin{cases}F_y^{\prime}(x,\psi(x),\psi^{\prime}(x))\ne 0\\F_{pp}^{\prime\prime}(x,\psi(x),\psi^{\prime}(x))\ne 0\\F_p^{\prime}(x,\psi(x),\psi^{\prime}(x))=0\end{cases}
	$$
	中的三个条件缺一不可,如
	$$
	(\frac{\mathrm{d}y}{\mathrm{d}x})^2=y^2,\qquad 
	\sin(y\frac{\mathrm{d}y}{\mathrm{d}x})=y,\qquad 
	y=2x+\frac{\mathrm{d}y}{\mathrm{d}x}-\frac{1}{3}(\frac{\mathrm{d}y}{\mathrm{d}x})^3
	$$
\end{note}

\subsection{包络}

\begin{definition}{包络}
	对于单参数$C$的曲线族
	$$
	K(C):\qquad V(x,y,C)=0
	$$
	其中函数$V(x,y,C)$对于$(x,y,C)\in D$是连续可微的。称连续可微的曲线$\Gamma$为曲线族$K(C):V(x,y,C)=0$的包络,如果对于任一点$P\in\Gamma$,在曲线族$K(C):V(x,y,C)=0$中存在曲线$K(C_0)$经过$P$点并在该点与$\Gamma$相切,同时$K(C_0)$在$P$点的某一邻域内不同于$\Gamma$。
\end{definition}

\begin{theorem}{奇解是通解的包络}
	设微分方程
	$$
	F(x,y,\frac{\mathrm{d}y}{\mathrm{d}x})=0
	$$
	有通积分为
	$$
	U(x,y,C)=0
	$$
	又设积分曲线族$U(x,y,C)=0$有包络为
	$$
	\Gamma:y=\varphi(x),x\in J
	$$
	则包络$\Gamma:y=\varphi(x),x\in J$是微分方程的奇解。
\end{theorem}

\begin{theorem}{包络存在的必要条件}
	设$\Gamma$是曲线族
	$$
	K(C):V(x,y,C)=0
	$$
	的一支包络,则其满足如下的$C-$判别式
	$$
	\begin{cases}V(x,y,C)=0\\V_C^{\prime}(x,y,C)=0\end{cases}
	$$
	或消去$C$,得到关系式
	$$
	\Omega(x,y)=0
	$$
\end{theorem}

\begin{theorem}{包络存在的充分条件}
	设由曲线族
	$$
	K(C):V(x,y,C)=0
	$$
	的$C-$判别式
	$$
	\begin{cases}V(x,y,C)=0\\V_C^{\prime}(x,y,C)=0\end{cases}
	$$
	确定一支连续可微且不含于族$K(C):V(x,y,C)=0$的曲线
	$$
	\Lambda:\begin{cases}x=\varphi(C)\\y=\psi(C)\end{cases},C\in J
	$$
	且满足非蜕化性条件
	$$
	(\phi^{\prime}(C),\psi^{\prime}(C))\ne(0,0)\\(V_x^{\prime}(\varphi(C),\psi(C),C),V_y^{\prime}(\varphi(C),\psi(C),C))\ne(0,0)
	$$
	则曲线$\Lambda$是曲线族$K(C):V(x,y,C)=0$的一支包络。
\end{theorem}





























	
\end{document}