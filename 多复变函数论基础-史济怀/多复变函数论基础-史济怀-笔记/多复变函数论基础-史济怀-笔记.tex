\documentclass[lang = cn, scheme = chinese, thmcnt = section]{elegantbook}
% elegantbook      设置elegantbook文档类
% lang = cn        设置中文环境
% scheme = chinese 设置标题为中文
% thmcnt = section 设置计数器


%% 1.封面设置

\title{多复变函数论基础 - 史济怀 - 笔记}                % 文档标题

\author{若水}                        % 作者

\myemail{ethanmxzhou@163.com}       % 邮箱

\homepage{helloethanzhou.github.io} % 主页

\date{\today}                       % 日期

\logo{PiCreatures_happy.pdf}        % 设置Logo

\cover{阿基米德螺旋曲线.pdf}          % 设置封面图片

% 修改标题页的色带
\definecolor{customcolor}{RGB}{135, 206, 250} 
% 定义一个名为customcolor的颜色,RGB颜色值为(135, 206, 250)

\colorlet{coverlinecolor}{customcolor}     % 将coverlinecolor颜色设置为customcolor颜色

%% 2.目录设置
\setcounter{tocdepth}{3}  % 目录深度为3

%% 3.引入宏包
\usepackage[all]{xy}
\usepackage{bbm, svg, graphicx, float, extpfeil, amsmath, amssymb, mathrsfs, mathalpha, hyperref}


%% 4.定义命令
\newcommand{\N}{\mathbb{N}}            % 自然数集合
\newcommand{\R}{\mathbb{R}}            % 实数集合
\newcommand{\C}{\mathbb{C}}  		   % 复数集合
\newcommand{\Q}{\mathbb{Q}}            % 有理数集合
\newcommand{\Z}{\mathbb{Z}}            % 整数集合
\newcommand{\sub}{\subset}             % 包含
\newcommand{\im}{\text{im }}           % 像
\newcommand{\lang}{\langle}            % 左尖括号
\newcommand{\rang}{\rangle}            % 右尖括号
\newcommand{\bs}{\boldsymbol}          % 向量加黑
\newcommand{\dd}{\mathrm{d}}           % 微分d
\newcommand{\pll}{\kern 0.56em/\kern -0.8em /\kern 0.56em} % 平行
\newcommand{\function}[5]{
	\begin{align*}
		#1:\begin{aligned}[t]
			#2 &\longrightarrow #3\\
			#4 &\longmapsto #5
		\end{aligned}
	\end{align*}
}                                     % 函数

\newcommand{\lhdneq}{%
	\mathrel{\ooalign{$\lneq$\cr\raise.22ex\hbox{$\lhd$}\cr}}} % 真正规子群

\newcommand{\rhdneq}{%
	\mathrel{\ooalign{$\gneq$\cr\raise.22ex\hbox{$\rhd$}\cr}}} % 真正规子群

\begin{document}

\maketitle       % 创建标题页

\frontmatter     % 开始前言部分

\chapter*{致谢}

\markboth{致谢}{致谢}

\vspace*{\fill}
	\begin{center}
		
		\large{感谢 \textbf{ 勇敢的 } 自己}
		
	\end{center}
\vspace*{\fill}

\tableofcontents % 创建目录

\mainmatter      % 开始正文部分

\chapter{多复变全纯函数}

\section{全纯函数}

\subsection{全纯函数的定义}

\begin{definition}{域}
	称$\Omega\sub\C^n$为域,如果$\Omega$为连通开集。
\end{definition}

\begin{definition}{多圆柱}
	对于$\bs{a}=(a_1,\cdots,a_n)\in\C^n$,与$\bs{r}=(r_1,\cdots,r_n)\in\R_{> 0}^n$,定义多圆柱为
	$$
	P_{\bs{r}}(\bs{a})=\{ (z_1,\cdots,z_n)\in\C^n:|z_k-a_k|<r_k,1\le k\le n \}
	$$
	特别的,定义单位多圆柱为
	$$
	U^n=P_{\bs{1}}(\bs{0})=\{ (z_1,\cdots,z_n)\in\C^n:|z_k|<1,1\le k\le n \}
	$$
\end{definition}

\begin{definition}{球}
	对于$\bs{z}_0\in\C^n$,与$r>0$,定义球为
	$$
	B_{r}(\bs{z}_0)=\{ \bs{z}\in\C^n:|\bs{z}-\bs{z}_0|<r \}
	$$
	特别的,定义单位球为
	$$
	B_n=B_1(\bs{0})=\{ \bs{z}\in\C^n:|\bs{z}|<1 \}
	$$
\end{definition}

\begin{definition}{全纯函数}
	对于域$\Omega\sub\C^n$,称函数$f:\Omega\to\C$为全纯函数,如果对于任意$\bs{a}\in\Omega$,存在多圆柱$P_{\bs{r}}(\bs{a})$,使得对于任意$\bs{z}\in P_{\bs{r}}(\bs{a})$,存在幂级数
	$$
	f(z)=\sum_{\bs{\alpha}\in\N^n}C_{\bs{\alpha}}(\bs{z}-\bs{a})^{\bs{\alpha}}
	$$
\end{definition}

\begin{theorem}{Taylor展式}{Taylor展式}
	对于域$\Omega\sub\C^n$上的全纯函数$f:\Omega\to\C$,成立
	$$
	f(\bs{z})
	=\sum_{\bs{\alpha}\in\N^n}\frac{f^{(\bs{\alpha})}(\bs{a})}{\bs{\alpha}!}(\bs{z}-\bs{a})^{\bs{\alpha}}
	$$
	其中
	$$
	f^{(\bs{\alpha})}=\frac{\partial^{\alpha_1+\cdots+\alpha_n}}{\partial z_1^{\alpha_1}\cdots \partial z_n^{\alpha_n}}f,\qquad 
	\bs{\alpha}!=\alpha_1!\cdots\alpha_n!
	$$
\end{theorem}

\subsection{唯一性定理}

\begin{theorem}{唯一性定理}{唯一性定理}
	对于域$\Omega\sub\C^n$上的全纯函数$f:\Omega\to\C$,如果存在非空开集$G\sub\Omega$,使得成立$f|_G=0$,那么$f=0$。
\end{theorem}

\begin{proof}
	对于$\bs{\alpha}\in\N^n$,定义
	\begin{align*}
		& K = \{ \bs{z}\in \Omega:\forall\bs{\alpha}\in\N^n,f^{(\bs{\alpha})}(\bs{z})=0 \}\\
		& K_{\bs{\alpha}} = \{ \bs{z}\in \Omega:f^{(\bs{\alpha})}(\bs{z})=0 \}
	\end{align*}
	
	由于$f^{(\bs{\alpha})}$为连续函数,且$\{0\}$为闭集,那么$K_{\bs{\alpha}}$为闭集。由于
	$$
	K=\bigcap_{\bs{\alpha}\in\N^n}K_{\bs{\alpha}}\\
	$$
	那么$K$为闭集。
	
	对于任意$\bs{a}\in K$,由于$f$在$\Omega$上全纯,那么存在$\bs{r}\in\R_{>0}^n$,使得对于任意$\bs{z}\in P_{\bs{r}}(\bs{a})$,成立幂级数展开
	$$
	f(z)
	=\sum_{|\bs{\alpha}|=0}^{n}\frac{f^{(\bs{\alpha})}(\bs{a})}{\bs{\alpha}!}(\bs{z}-\bs{a})^{\bs{\alpha}}=0
	$$
	因此$P_{\bs{r}}(\bs{a})\sub K$。由$\bs{a}$的任意性,$K$为开集。
	
	由于$G$为非空开集,那么$G\sub K$,因此$K$非空。由于
	$$
	\Omega=K\cup (\Omega\setminus K)
	$$
	且$\Omega$连通,那么$\Omega=K$,进而在$\Omega$上成立$f=0$。
\end{proof}

\begin{theorem}{开映射定理}{开映射定理}
	对于域$\Omega\sub\C^n$上的全纯函数$f:\Omega\to\C$,或$f$为常函数,或$f$为开映射。
\end{theorem}

\begin{proof}
	如果$f$不为常函数,对于任意$\bs{a}\in\Omega$,取$\bs{r}\in\R_{>0}^n$,使得成立$P_{\bs{r}}(\bs{a})\sub\Omega$。由唯一性定理\ref{thm:唯一性定理},$f$在$P_{\bs{r}}(\bs{a})$不恒为$f(\bs{a})$,于是存在$\bs{b}\in P_{\bs{r}}(\bs{a})$,使得成立$f(\bs{a})\ne f(\bs{b})$。
	
	构造开集
	$$
	D=\{ \lambda\in\C:\bs{a}+\lambda(\bs{b}-\bs{a}\in P_{\bs{r}}(\bs{a}) \}
	$$
	构造全纯函数$g(\lambda)=f(\bs{a}+\lambda(\bs{b}-\bs{a})$,其中$\lambda\in D$。由于
	$$
	g(0)=f(\bs{a})\ne f(\bs{b})=g(1)
	$$
	那么$g$为$D$上的不为常函数,由开映射定理\ref{thm:开映射定理},$g(D)$为$\C$中的开集,而$g(D)\sub f(P_{\bs{r}}(\bs{a}))$,因此$f(P_{\bs{r}}(\bs{a}))$为$\C$中的开集,进而$f$为开映射。
\end{proof}

\begin{theorem}{最大模原理}{最大模原理}
	对于域$\Omega\sub\C^n$上的全纯非常函数$f:\Omega\to\C$,$|f|$不在$\Omega$的内部取到最大值。
\end{theorem}

\begin{proof}
	如果$|f|$在$\Omega$内部取到最大值,那么存在$\bs{a}\in\Omega^\circ$,使得对于任意$\bs{z}\in\Omega$,成立$|f(\bs{z})|\le |f(\bs{a})|$,因此
	$$
	f(\Omega)\sub \{ z\in\C:|z|\le |f(\bs{a})| \}
	$$
	由开映射定理\ref{thm:开映射定理},$f(\Omega)$为开集,因此
	$$
	f(\Omega)\sub \{ z\in\C:|z|< |f(\bs{a})| \}
	$$
	但是$|f(\bs{a})|\in f(\Omega)$,导出矛盾$|f(\bs{a})|<|f(\bs{a})|$!
\end{proof}

\section{多圆柱的Cauchy积分公式及其应用}

\subsection{多圆柱的Cauchy积分公式}

\begin{definition}{多圆柱的特征边界}
	定义多圆柱
	$$
	P_{\bs{r}}(\bs{a})=\{ (z_1,\cdots,z_n)\in\C^n:|z_k-a_k|<r_k,1\le k\le n \}
	$$
	的特征边界为
	$$
	\partial_DP_{\bs{r}}(\bs{a})
	=\{ (z_1,\cdots,z_n)\in\C^n:|z_k-a_k|=r_k,1\le k\le n \}
	$$
\end{definition}

\begin{theorem}{多圆柱的Cauchy积分公式}{多圆柱的Cauchy积分公式}
	对于域$\Omega\sub\C^n$上的全纯函数$f:\Omega\to\C$,如果$\overline{P}_{\bs{r}}(\bs{a})\sub\Omega$,那么对于任意$\bs{z}\in P_{\bs{r}}(\bs{a})$,成立
	$$
	f(\bs{z})=
	\frac{1}{(2\pi i)^n}\int_{\partial_DP_{\bs{r}}(\bs{a})}\frac{f(\bs{\zeta})}{\prod\limits_{k=1}^{n}(\zeta_k-z_k)}\dd\bs{\zeta}
	$$
\end{theorem}

\begin{theorem}{多圆柱的Taylor展式}{多圆柱的Taylor展式}
	如果函数$f$为多圆柱$P_{\bs{r}}(\bs{a})$上的全纯函数,那么对于任意$\bs{z}\in P_{\bs{r}}(\bs{a})$,成立
	$$
	f(\bs{z})
	=\sum_{\bs{\alpha}\in\N^n}\frac{f^{(\bs{\alpha})}(\bs{a})}{\bs{\alpha}!}(\bs{z}-\bs{a})^{\bs{\alpha}}
	$$
\end{theorem}

\begin{theorem}{Cauchy不等式}{Cauchy不等式}
	对于域$\Omega\sub\C^n$上的全纯函数$f:\Omega\to\C$,如果$\overline{P}_{\bs{r}}(\bs{a})\sub\Omega$,那么
	$$
	\left|f^{(\bs{\alpha})}(\bs{a})\right|
	\le
	\frac{\bs{\alpha}!}{\bs{r}^{\bs{\alpha}}}\sup_{\bs{z}\in \partial_DP_{\bs{r}}(\bs{a})}|f(\bs{z})|
	$$
\end{theorem}

\begin{proof}
	由多圆柱的Cauchy积分公式\ref{thm:多圆柱的Cauchy积分公式}
	$$
	f(\bs{z})=
	\frac{1}{(2\pi i)^n}\int_{\partial_DP_{\bs{r}}(\bs{a})}\frac{f(\bs{\zeta})}{\prod\limits_{k=1}^{n}(\zeta_k-z_k)}\dd\bs{\zeta}
	$$
	那么
	$$
	f^{(\bs{\alpha})}(\bs{a})=\frac{\bs{\alpha}!}{(2\pi i )^n}
	\int_{\partial_DP_{\bs{r}}(\bs{a})}
	\frac{f(\bs{\zeta})}{\prod\limits_{k=1}^{n}(\zeta_k-z_k)^{\alpha_k-1}}\dd\bs{\zeta}
	$$
	记$\displaystyle M=\sup_{\bs{z}\in \partial_DP_{\bs{r}}(\bs{a})}|f(\bs{z})|$,进而
	\begin{align*}
		\left|f^{(\bs{\alpha})}(\bs{a})\right|
		& \le \frac{\bs{\alpha}!}{(2\pi)^n}
		\int_{\partial_DP_{\bs{r}}(\bs{a})}
		\frac{|f(\bs{\zeta})|}{\prod\limits_{k=1}^{n}|\zeta_k-z_k|^{\alpha_k1}}|\dd\bs{\zeta}|\\
		& \le \frac{\bs{\alpha}!}{(2\pi)^n}
		\frac{M}{\prod\limits_{k=1}^{n}r_k^{\alpha_k-1}}(2\pi)^nr_1\cdots r_n\\
		& = M\frac{\bs{\alpha}!}{\bs{r}^{\bs{\alpha}}}
	\end{align*}
\end{proof}

\subsection{Weierstrass定理}

\begin{definition}{相对紧集}
	称$G$为相对于$\Omega$的紧集,并记作$G\sub\sub \Omega$,如果$\overline{G}\sub\Omega$,且$\overline{G}$为紧集。
\end{definition}

\begin{theorem}
	对于域$\Omega\sub\C^n$上的全纯函数$f:\Omega\to\C$,如果紧集$K$成立$K\sub G\sub\sub \Omega$,那么存在域$K,G$与$\bs{\alpha}$有关的常数$C_{\bs{\alpha}}$,使得成立
	$$
	\sup_{\bs{z}\in K}|f^{\bs{\alpha}}(\bs{z})|
	\le C_{\bs{\alpha}}\sup_{\bs{z}\in G}|f(\bs{z})|
	$$
\end{theorem}

\begin{proof}
	
\end{proof}

\begin{theorem}{Weierstrass定理}{Weierstrass定理}
	对于域$\Omega\sub\C^n$上的全纯函数序列$\{f_n:\Omega\to\C\}_{n=1}^{\infty}$,如果其在$\Omega$上内闭一致收敛于$f$,那么$f$为$\Omega$上的全纯函数,且$\{f_n^{(\bs{\alpha})}\}_{n=1}^{\infty}$在$\Omega$上内闭一致收敛于$f^{(\bs{\alpha})}$。
\end{theorem}

\subsection{Montel定理}

\begin{definition}{正规族}
	称域$\Omega\sub\C^n$上的全纯函数族$\mathscr{F}$为正规族,如果对于任意序列$\{f_n:\Omega\to\C\}_{n=1}^{\infty}\sub \mathscr{F}$,存在子列$\{f_{n_k}:\Omega\to\C\}_{k=1}^{\infty}$,使得其在$\Omega$上内闭一致收敛。
\end{definition}

\begin{definition}{局部一致有界性}
	域$\Omega\sub\C^n$上的全纯函数族$\mathscr{F}$为局部一致有界的,如果对于任意紧集$K\sub\Omega$,存在常数$M$,使得对于任意$\bs{z}\in K$与$f\in\mathscr{F}$,成立$|f(\bs{z})|\le M$。
\end{definition}

\begin{theorem}{Montel定理}{Montel定理}
	对于域$\Omega\sub\C^n$上的全纯函数族$\mathscr{F}$,成立
	$$
	\mathscr{F}\text{ 为正规族}
	\iff 
	\mathscr{F}\text{ 局部一直有界}
	$$
\end{theorem}

\subsection{Hurwitz定理}

\begin{theorem}{Hurwitz定理}{Hurwitz定理}
	对于域$\Omega\sub\C^n$上的处处非零的全纯函数序列$\{f_n:\Omega\to\C\}_{n=1}^{\infty}$,如果其在$\Omega$上内闭一致收敛于$f$,那么或$f=0$,或$f$处处非零。
\end{theorem}

\section{Hartogs现象}

\subsection{Hartogs现象}

\begin{definition}{全纯延拓}
	对于域$\Omega_0\subsetneq\Omega\sub\C^n$,称$\Omega$上的全纯函数$F$为$\Omega_0$上的全纯函数$f$的全纯延拓,如果$F|_{\Omega_0}=f$。
\end{definition}

\begin{theorem}{Hartogs现象}{Hartogs现象}
	如果$n\ge 2$,那么存在域$\Omega\sub\C^n$,使得对于任意$\Omega$上的全纯函数存在全纯延拓。
\end{theorem}

\subsection{Reinhardt域上的展式}

\begin{definition}{Reinhardt域}
	称域$\Omega\sub\C^n$为Reinhardt域,如果对于任意$(z_1,\cdots,z_n)\in\Omega$与$(\theta_1,\cdots,\theta_n)\in\R^n$,成立$(\mathrm{e}^{i\theta_1}z_1,\cdots,\mathrm{e}^{i\theta_n}z_n)\in\Omega$。
\end{definition}

\begin{theorem}{Reinhardt域上的展式}{Reinhardt域上的展式}
	如果函数$f$为Reinhardt域$\Omega\sub\C^n$上的全纯函数,那么对于任意$\bs{z}\in \Omega$,存在幂级数
	$$
	f(\bs{z})
	=\sum_{\bs{\alpha}\in\Z^n}C_{\bs{\alpha}}\bs{z}^{\bs{\alpha}}
	$$
	且该级数在$\Omega$上内闭一致收敛。
\end{theorem}

\begin{theorem}{}{Reinhardt域上的幂级数}
	对于Reinhardt域$\Omega\sub\C^n$上的全纯函数$f:\Omega\to\C$,如果对于任意$1\le k\le n$,$\Omega$中存在第$k$个坐标为$0$的点,那么存在幂级数
	$$
	f(\bs{z})
	=\sum_{\bs{\alpha}\in\Z^n}C_{\bs{\alpha}}\bs{z}^{\bs{\alpha}},\qquad \bs{z}\in\Omega
	$$
	且该级数在$\Omega$上内闭一致收敛。
\end{theorem}

\begin{corollary}
	单位球$B_n$上的全纯函数$f:B_n\to\C$存在幂级数
	$$
	f(\bs{z})
	=\sum_{\bs{\alpha}\in\Z^n}C_{\bs{\alpha}}\bs{z}^{\bs{\alpha}},\qquad \bs{z}\in B_n
	$$
	且该级数在$B_n$上内闭一致收敛。
\end{corollary}

\begin{theorem}{全纯延拓定理}{全纯延拓定理}
	对于Reinhardt域$\Omega\sub\C^n$,如果对于任意$1\le k\le n$,$\Omega$中存在第$k$个坐标为$0$的点,那么任意Reinhardt域$\Omega\sub\C^n$上的全纯函数$f:\Omega\to\C$均可延拓到域
	$$
	\Upsilon=\{ (\rho_1z_1,\cdots,\rho_nz_n):(z_1,\cdots,z_n)\in\Omega,0\le \rho_k \le 1,1\le k \le n \}
	$$
\end{theorem}

\begin{proof}
	由定理\ref{thm:Reinhardt域上的幂级数},$f$在$\Omega$存在幂级数
	$$
	f(\bs{z})
	=\sum_{\bs{\alpha}\in\Z^n}C_{\bs{\alpha}}\bs{z}^{\bs{\alpha}},\qquad \bs{z}\in\Omega
	$$
	任取$\bs{w}=(w_1,\cdots,w_n)\in\Upsilon$,那么存在$\bs{z}=(z_1,\cdots,z_n)\in\Omega$与$0\le \rho_k \le 1$,使得成立$w_k=\rho_k z_k$,从而$|w_k|\le |z_k|$,其中$1\le k \le n$。由于$\displaystyle \sum_{\bs{\alpha}\in\Z^n}C_{\bs{\alpha}}\bs{z}^{\bs{\alpha}}$收敛,那么$\displaystyle \sum_{\bs{\alpha}\in\Z^n}C_{\bs{\alpha}}\bs{w}^{\bs{\alpha}}$收敛,且在$\Upsilon$中内闭一致收敛。定义函数
	$$
	F(\bs{w})=\sum_{\bs{\alpha}\in\Z^n}C_{\bs{\alpha}}\bs{w}^{\bs{\alpha}},\qquad \bs{w}\in\Upsilon
	$$
	那么$F$为$\Upsilon$上的全纯函数,且$F|_\Omega=f$,进而$F$为$f$在$\Upsilon$上的全纯延拓。
\end{proof}

\begin{corollary}{}{全纯延拓定理的推论}
	对于$0<r<R$,域
	$$
	\Omega=\{ \bs{z}\in\C^n:r<|\bs{z}|<R \}
	$$
	上的全纯函数可延拓到球
	$$
	B_R(\bs{0})=\{ \bs{z}\in\C^n:|\bs{z}|<R \}
	$$
\end{corollary}

\begin{proof}
	由全纯延拓定理\ref{thm:全纯延拓定理},命题显然!
\end{proof}

\begin{theorem}{}{零点的非孤立性}
	如果$n\ge 2$,那么域$\Omega\sub\C^n$上的全纯函数$f:\Omega\to\C$的零点非孤立。
\end{theorem}

\begin{proof}
	如果$\bs{a}\in\Omega$为$f$的孤立零点,那么存在$\varepsilon>0$,使得$f$在$B_\varepsilon(\bs{a})$中除$\bs{a}$外无零点。令
	$$
	g(z)=\frac{1}{f(z)}
	$$
	那么$g$在$B_\varepsilon(\bs{a})-\overline{B}_{\varepsilon/2}(\bs{a})$中全纯。由全纯延拓定理的推论\ref{cor:全纯延拓定理的推论},$g$在$B_\varepsilon(\bs{a})$中全纯,从而$f(\bs{a})\ne 0$,导出矛盾!
\end{proof}

\chapter{全纯映射}

\section{全纯映射的导数}

\subsection{全纯映射的导数}

\begin{definition}{全纯映射}
	对于域$\Omega\sub\C^n$,称映射$F=(f_1,\cdots,f_m):\Omega\to\C^m$为全纯映射,如果对于任意$1\le k\le n$,$f_k:\Omega\to\C$为全纯函数。
\end{definition}

\begin{definition}{映射的导数}
	对于域$\Omega\sub\C^n$上的映射$F=(f_1,\cdots,f_m):\Omega\to\C^m$,称$F$在$\bs{z}\in\Omega$处可微,如果存在线性算子$A:\C^n\to\C^m$,使得成立
	$$
	\lim_{\bs{h}\to \bs{0}}\frac{|F(\bs{z}+\bs{h})-F(\bs{z})-A(\bs{h})|}{|\bs{h}|}=0
	$$
\end{definition}

\begin{theorem}{全纯映射的导数}
	域$\Omega\sub\C^n$上的全纯映射$F=(f_1,\cdots,f_m):\Omega\to\C^m$在$\Omega$上处处可微,且
	$$
	F'(\bs{z}_0)=\begin{pmatrix}
		\frac{\partial f_1(\bs{z})}{\partial z_1}|_{\bs{z}=\bs{z}_0} & \cdots & \frac{\partial f_1(\bs{z})}{\partial z_n}|_{\bs{z}=\bs{z}_0}\\
		\vdots & \ddots & \vdots\\
		\frac{\partial f_m(\bs{z})}{\partial z_1}|_{\bs{z}=\bs{z}_0} & \cdots & \frac{\partial f_m(\bs{z})}{\partial z_n}|_{\bs{z}=\bs{z}_0}
	\end{pmatrix}
	$$
\end{theorem}

\begin{theorem}
	对于域$\Omega\sub \C^{l}$与$\Upsilon\sub \C^{m}$上的全纯映射$F:\Omega\to\Upsilon$与$G:\Upsilon\to\C^n$,其复合$H=G\circ F:\Omega\to\C^n$为全纯映射,且$H'(\bs{z})=G'(\bs{z})F'(\bs{z})$。
\end{theorem}

\subsection{复Jacobian和实Jacobian的关系}

\begin{definition}{复Jacobian}
	定义域$\Omega\sub\C^n$上的全纯映射$F:\Omega\to\C^n$在$\bs{z}_0$处的复Jacobian为%
	$$
	J_F^{(\C)}(\bs{z}_0)=\begin{vmatrix}
		\frac{\partial f_1(\bs{z})}{\partial z_1}|_{\bs{z}=\bs{z}_0} & \cdots & \frac{\partial f_1(\bs{z})}{\partial z_n}|_{\bs{z}=\bs{z}_0}\\
		\vdots & \ddots & \vdots\\
		\frac{\partial f_m(\bs{z})}{\partial z_1}|_{\bs{z}=\bs{z}_0} & \cdots & \frac{\partial f_m(\bs{z})}{\partial z_n}|_{\bs{z}=\bs{z}_0}
	\end{vmatrix}
	$$
\end{definition}

\begin{definition}{实Jacobian}
	定义域$\Omega\sub\C^n$上的全纯映射$F=(f_1,\cdots,f_n):\Omega\to\C^n$在$\bs{z}_0=(\bs{x}_0,\bs{y}_0)$处的实Jacobian为%
	$$
	J_F^{(\R)}(\bs{x}_0,\bs{y}_0)=\begin{vmatrix}
		\frac{\partial u_1(\bs{x},\bs{y})}{\partial x_1}|_{(\bs{x}_0,\bs{y}_0)}
		& \cdots &
		\frac{\partial u_1(\bs{x},\bs{y})}{\partial x_n}|_{(\bs{x}_0,\bs{y}_0)}
		&
		\frac{\partial u_1(\bs{x},\bs{y})}{\partial y_1}|_{(\bs{x}_0,\bs{y}_0)}
		& \cdots &
		\frac{\partial u_1(\bs{x},\bs{y})}{\partial y_n}|_{(\bs{x}_0,\bs{y}_0)}\\
		\vdots & \ddots & \vdots & \vdots & \ddots & \vdots\\
		\frac{\partial u_n(\bs{x},\bs{y})}{\partial x_1}|_{(\bs{x}_0,\bs{y}_0)}
		& \cdots &
		\frac{\partial u_n(\bs{x},\bs{y})}{\partial x_n}|_{(\bs{x}_0,\bs{y}_0)}
		&
		\frac{\partial u_n(\bs{x},\bs{y})}{\partial y_1}|_{(\bs{x}_0,\bs{y}_0)}
		& \cdots &
		\frac{\partial u_n(\bs{x},\bs{y})}{\partial y_n}|_{(\bs{x}_0,\bs{y}_0)}\\
		\frac{\partial u_1(\bs{x},\bs{y})}{\partial x_1}|_{(\bs{x}_0,\bs{y}_0)}
		& \cdots &
		\frac{\partial u_1(\bs{x},\bs{y})}{\partial x_n}|_{(\bs{x}_0,\bs{y}_0)}
		&
		\frac{\partial u_1(\bs{x},\bs{y})}{\partial y_1}|_{(\bs{x}_0,\bs{y}_0)}
		& \cdots &
		\frac{\partial u_1(\bs{x},\bs{y})}{\partial y_n}|_{(\bs{x}_0,\bs{y}_0)}\\
		\vdots & \ddots & \vdots & \vdots & \ddots & \vdots\\
		\frac{\partial u_n(\bs{x},\bs{y})}{\partial x_1}|_{(\bs{x}_0,\bs{y}_0)}
		& \cdots &
		\frac{\partial u_n(\bs{x},\bs{y})}{\partial x_n}|_{(\bs{x}_0,\bs{y}_0)}
		&
		\frac{\partial u_n(\bs{x},\bs{y})}{\partial y_1}|_{(\bs{x}_0,\bs{y}_0)}
		& \cdots &
		\frac{\partial u_n(\bs{x},\bs{y})}{\partial y_n}|_{(\bs{x}_0,\bs{y}_0)}\\
		\frac{\partial v_1(\bs{x},\bs{y})}{\partial x_1}|_{(\bs{x}_0,\bs{y}_0)}
		& \cdots &
		\frac{\partial v_1(\bs{x},\bs{y})}{\partial x_n}|_{(\bs{x}_0,\bs{y}_0)}
		&
		\frac{\partial v_1(\bs{x},\bs{y})}{\partial y_1}|_{(\bs{x}_0,\bs{y}_0)}
		& \cdots &
		\frac{\partial v_1(\bs{x},\bs{y})}{\partial y_n}|_{(\bs{x}_0,\bs{y}_0)}\\
		\vdots & \ddots & \vdots & \vdots & \ddots & \vdots\\
		\frac{\partial v_n(\bs{x},\bs{y})}{\partial x_1}|_{(\bs{x}_0,\bs{y}_0)}
		& \cdots &
		\frac{\partial v_n(\bs{x},\bs{y})}{\partial x_n}|_{(\bs{x}_0,\bs{y}_0)}
		&
		\frac{\partial v_n(\bs{x},\bs{y})}{\partial y_1}|_{(\bs{x}_0,\bs{y}_0)}
		& \cdots &
		\frac{\partial v_n(\bs{x},\bs{y})}{\partial y_n}|_{(\bs{x}_0,\bs{y}_0)}
	\end{vmatrix}
	$$
	其中对于任意$1\le k\le n$,$f_k=u_k+iv_k$,且$z=x+iy$。
\end{definition}

\begin{theorem}{复Jacobian和实Jacobian的关系}
	对于域$\Omega\sub\C^n$上的全纯映射$F:\Omega\to\C^n$,成立%
	$$
	J_F^{(\R)}(\bs{z})
	=|J_F^{(\C)}(\bs{z})|^2
	$$
\end{theorem}

\section{双全纯映射}

\begin{definition}{单叶全纯映射}
	称域$\Omega\sub \C^{n}$上的全纯映射$F:\Omega\to\C^m$为单叶全纯映射,如果成立
	$$
	F(\bs{z})=F(\bs{w})\implies
	\bs{z}=\bs{w}
	$$
\end{definition}

\begin{definition}{双全纯映射}
	称域$\Omega\sub \C^{n}$上的全纯映射$F:\Omega\to\C^m$为双全纯映射,如果存在域$\Upsilon\sub\C^m$上的全纯映射$G:\Upsilon\to\C^n$,使得成立
	$$
	G\circ F=\mathbbm{1}_{\Omega},\qquad 
	G\circ F=\mathbbm{1}_{\Upsilon}
	$$
\end{definition}

\begin{definition}{全纯等价}
	称域$\Omega\sub \C^{n}$与$\Upsilon\sub\C^m$全纯等价,如果存在双全纯映射$F:\Omega\to\Upsilon$。
\end{definition}

\begin{definition}{全纯自同构}
	称域$\Omega\sub \C^{n}$上的双全纯映射$F:\Omega\to\Omega$为$\Omega$的全纯自同构映射。
\end{definition}

\section{H.Cartan定理与球的全纯自同构}

\subsection{H.Cartan定理}

\begin{definition}{圆型域}
	称域$\Omega\sub\C^n$为圆型域,如果对于任意$\bs{z}\in\Omega$与$\theta\in\R$,成立$\mathrm{e}^{i\theta}\bs{z}\in \Omega$。
\end{definition}

\begin{theorem}{H.Cartan定理}{H.Cartan定理2}
	对于有界域$\Omega\sub\C^n$上的全纯映射$F:\Omega\to\Omega$,如果存在$\bs{z}_0\in \Omega$,使得成立$F(\bs{z}_0)=\bs{z}_0$,且$F'(\bs{z}_0)=I_n$,那么$F=\mathbbm{1}_{\Omega}$。
\end{theorem}

\begin{remark}
	域$\Omega$的有界性条件不可取消。例如,对于域
	$$
	\Omega=\{ (z_1,z_2)\in\C^2:|z_1z_2|<1 \}
	$$
	记$\mathbb{D}=\{ z\in\C:|z|<1 \}$,$h:\mathbb{D}\to\C$为无零点的全纯函数,且$h(0)=1$。构造映射
	\function{F_h}{\Omega}{\Omega}{(z_1,z_2)}{\left(z_1h(z_1z_2),\frac{z_2}{h(z_1z_2)}\right)}
	注意到
	$$
	F_h\circ F_{\frac{1}{h}}=F_{\frac{1}{h}}\circ F_h=\mathbbm{1}_\Omega
	$$
	因此$F_h$为双全纯映射。注意到$F_h(\bs{0})=\bs{0}$,且$F'_h(\bs{0})=I_2$,但是$F_h$不为线性映射。
\end{remark}

\begin{theorem}{H.Cartan定理}{H.Cartan定理2}
	对于圆型域$\Omega\sub\C^n$与$\Upsilon\sub\C^m$,如果$\bs{0}\in \Omega\cap\Upsilon$,且$\Omega$为有界域,那么对于双全纯映射$F:\Omega\to\Upsilon$,若$F(\bs{0})=\bs{0}$,则$F$为线性映射。
\end{theorem}

\begin{corollary}
	对于有界圆型域$\Omega\sub\C^n$,如果$\bs{0}\in \Omega$,那么对于双全纯映射$F:\Omega\to\Omega$,若$F(\bs{0})=\bs{0}$,则$F$为线性映射。
\end{corollary}

\begin{remark}
	圆型域$\Omega$的有界性条件不可取消。例如,对于圆型域
	$$
	\Omega=\{ (z_1,z_2)\in\C^2:|z_1z_2|<1 \}
	$$
	记$\mathbb{D}=\{ z\in\C:|z|<1 \}$,$h:\mathbb{D}\to\C$为无零点的全纯函数。构造映射
	\function{F_h}{\Omega}{\Omega}{(z_1,z_2)}{\left(z_1h(z_1z_2),\frac{z_2}{h(z_1z_2)}\right)}
	注意到
	$$
	F_h\circ F_{\frac{1}{h}}=F_{\frac{1}{h}}\circ F_h=\mathbbm{1}_\Omega
	$$
	因此$F_h$为双全纯映射,但是$F_h$不为线性映射。
\end{remark}

\subsection{球的全纯自同构}

\begin{theorem}{单位多圆柱上的全纯自同构映射}
	对于任意单位多圆柱$U^n$上的全纯自同构映射$f:U^n\to U^n$,存在$(a_1,\cdots,a_n)\in U^n$,与$\theta_1,\cdots,\theta_n\in\R$,以及置换$\tau:\N^*\to\N^*$,使得成立
	$$
	f(z_1,\cdots,z_n)
	=\left(
	\mathrm{e}^{i\theta_1}\frac{z_{\tau(1)}-a_1}{1-\overline{a}_1z_{\tau(1)}},\cdots,\mathrm{e}^{i\theta_n}\frac{z_{\tau(n)}-a_n}{1-\overline{a}_nz_{\tau(n)}}
	\right)
	$$
\end{theorem}

\begin{theorem}{单位球上的全纯自同构映射}
	对于单位球柱$B_n$上的全纯自同构映射$f:B_n\to B_n$,如果$f(\bs{0})=\bs{0}$,那么存在且存在唯一酉矩阵$\bs{U}$,使得成立
	$$
	f(\bs{z})=\bs{U}\bs{z},\qquad \bs{z}\in B_n
	$$
\end{theorem}

\begin{theorem}
	对于$\bs{a}\in B_n$,记$s^2=1-|\bs{a}|^2$,以及
	$$
	\bs{P}=\begin{cases}
		\displaystyle \frac{1}{|\bs{a}|^2}\bs{a}\bs{a}^H,\qquad & \bs{a}\ne \bs{0}\\
		\bs{0},\qquad & \bs{a}=\bs{0}
	\end{cases},\qquad 
	\bs{A}=s\bs{I}_n+(1-s)\bs{P}
	$$
	定义映射
	$$
	\varphi_{\bs{a}}(\bs{z})=\frac{\bs{a}-\bs{A}\bs{z}}{1-\bs{z}^T\overline{\bs{a}}}
	$$
	那么映射$\varphi_{\bs{a}}$具有如下性质。
	\begin{enumerate}
		\item $\varphi_{\bs{a}}(\bs{0})=\varphi_{\bs{a}}(\bs{a})=\bs{0}$
		\item $\varphi_{\bs{a}}'(\bs{0})=\overline{\bs{a}}\bs{a}^T-\bs{A}^T,\varphi_{\bs{a}}'(\bs{a})=-\bs{A}^T/s^2$
		\item 对于任意$\bs{z}\in\overline{B}_n$,成立
		$$
		1-|\varphi_{\bs{a}}(\bs{z})|^2
		=\frac{(1-|\bs{a}|^2)(1-|\bs{z}|^2)}{|1-\bs{z}^T\overline{\bs{a}}|^2}
		$$
		\item $\varphi_{\bs{a}}\circ \varphi_{\bs{a}}=\mathbbm{1}_{B_n}$
		\item $\varphi_{\bs{a}}$为双全纯函数。
	\end{enumerate}
\end{theorem}

\begin{theorem}{单位球上的全纯自同构映射}
	对于单位球柱$B_n$上的全纯自同构映射$f:B_n\to B_n$,如果$f(\bs{a})=\bs{0}$,那么存在且存在唯一酉矩阵$\bs{U}$,使得成立
	$$
	f(\bs{z})=\bs{U}\varphi_{\bs{a}}(\bs{z}),\qquad \bs{z}\in B_n
	$$
\end{theorem}

\chapter{Cauchy积分公式}

\section{球的Cauchy积分公式}

\begin{definition}{Cauchy核}
	定义函数$f:\C^n\times\C^n\to \C$在单位球$B_n$中的Cauchy核为%
	$$
	C(\bs{z},\bs{\zeta})=\frac{1}{(1-\bs{z}^T\overline{\bs{\zeta}})^n}
	$$
\end{definition}

\begin{definition}{Cauchy积分}
	对于$f\in L(\sigma)$,定义其Cauchy积分为%
	$$
	C_f(\bs{z})=\int_{\partial B_n}C(\bs{z},\bs{\zeta})f(\bs{\zeta})\dd \sigma(\bs{\zeta}),\qquad \bs{z}\in B_n
	$$
\end{definition}

\begin{theorem}{球的Cauchy积分公式}
	对于在$B_n$上全纯且在$\overline{B}_n$上连续的函数$f$,成立%
	$$
	f(\bs{z})=\int_{\partial B_n}\frac{f(\bs{\zeta})}{(1-\bs{z}^T\overline{\bs{\zeta}})^n}\dd \sigma(\bs{\zeta}),\qquad \bs{z}\in B_n
	$$
\end{theorem}

\section{$\C$上的非齐次Cauchy积分公式及其应用}

\subsection{非齐次Cauchy积分公式}

\begin{theorem}
	对于具有光滑定向边界的有界域$\Omega\sub\C$,如果$f$为$\overline{\Omega}$上的连续可微函数,那么对于任意$z\in\Omega$,成立%
	$$
	f(z)=\frac{1}{2\pi i}\int_{\partial\Omega}\frac{f(\zeta)}{\zeta-z}\dd\zeta+\frac{1}{2\pi i}\int_{\Omega}\frac{1}{\zeta-z}\frac{\partial f(\zeta)}{\partial \overline{\zeta}}\dd\zeta\wedge\dd\overline{\zeta}
	$$
\end{theorem}

\subsection{平面上$\overline{\partial}$问题的解}

\begin{theorem}
	对于有界域$\Omega\sub\C$,如果$f$为$\Omega$上的有界连续可微函数,令%
	$$
	u(z)=\frac{1}{2\pi i}\int_{\Omega}\frac{1}{\zeta-z}\frac{\partial f(\zeta)}{\partial \overline{\zeta}}\dd\zeta\wedge\dd\overline{\zeta},\qquad z\in\Omega
	$$
	那么$u$在$\Omega$上连续可微,且存在$C\in\R$,使得成立%
	$$
	\frac{\partial u}{\partial \overline{z}}=f,\qquad 
	\sup_{\Omega}|u|\le C\sup_{\Omega}|f|
	$$
\end{theorem}

\appendix

\chapter{单复变函数定理扩展}

\begin{theorem}{Bieberbach定理}{Bieberbach定理}
	对于单位圆盘$\mathbb{D}$上的单的全纯函数$f$,如果$f(0)=0$,且$f'(0)=1$,那么作Taylor展式%
	$$
	f(z)=\sum_{n=0}^{\infty}a_nz^n
	$$
	成立%
	$$
	|a_n|\le n,\qquad n\in\N
	$$
\end{theorem}

\begin{theorem}{Koebe定理 $1/4$掩盖定理}{Koebe定理}
	对于单位圆盘$\mathbb{D}$上的单的全纯函数$f$,如果$f(0)=0$,且$f'(0)=1$,那么$f(\mathbb{D})\supset\mathbb{D}/4$。
\end{theorem}

\begin{proof}
	作Taylor展式%
	$$
	f(z)=\sum_{n=0}^{\infty}a_nz^n
	$$
	任取$w\notin f(\mathbb{D})$,令%
	$$
	g(z)
	=\frac{wf(z)}{w-f(z)}
	=\sum_{n=0}^{\infty}\frac{g^{(n)}(0)}{n!}z^n
	=z+\left(a_2+\frac{1}{w}\right)z^2+\sum_{n=3}^{\infty}\frac{g^{(n)}(0)}{n!}z^n
	$$
	由Bieberbach定理\ref{thm:Bieberbach定理},$|a_2|$且$|a_2+1/w|\le 2$,因此%
	$$
	\frac{1}{|w|}
	\le\left|a_2+\frac{1}{w}\right|+|a_2|
	\le 4
	$$
	从而$|w|\ge 1/4$,进而$f(\mathbb{D})\supset\mathbb{D}/4$。
\end{proof}

\begin{lemma}{Schwartz引理}{Schwartz引理}
	对于单位开圆盘$\mathbb{D}$,如果$f:\mathbb{D}\to\mathbb{D}$为全纯函数,且$f(0)=0$,那么
	$$
	|f'(0)|\le 1,\qquad 
	|f(z)|\le|z|,\qquad
	z\in\mathbb{D}
	$$
	当且仅当存在$\theta\in\R$,使得$f(z)=\mathrm{e}^{i\theta}z$时等号成立。
\end{lemma}

\begin{proof}
	构造%
	$$
	g(z)=\begin{cases}
		f(z)/z,\qquad & z\in\mathbb{D}\setminus\{0\}\\
		f'(0),\qquad & z=0
	\end{cases}
	$$
	那么$g(z)$在$\mathbb{D}$内全纯。由最大模原理,对于任意$0<r<1$,成立
	$$
	\max_{|z|<r}|g(z)|\le \max_{\theta\in\R}|g(r\mathrm{e}^{i\theta})|
	=\max_{\theta\in\R}\frac{|f(r\mathrm{e}^{i\theta})|}{r}\le\frac{1}{r}
	$$
	令$r\to 1$,那么$|g(z)|\le 1$,于是
	$$
	|f'(0)|\le 1,\qquad 
	|f(z)|\le|z|,\qquad
	z\in\mathbb{D}
	$$
	若存在$z\ne 0$,使得成立$|f(z)|=|z|$或$|f'(0)|=1$,则由最大模原理,$g$为常函数,因此存在$\theta\in\R$,使得$f(z)=\mathrm{e}^{i\theta}z$。
\end{proof}

\begin{lemma}{Schwartz-Pick引理}{Schwartz-Pick引理}
	对于单位开圆盘$\mathbb{D}$,如果$f:\mathbb{D}\to\mathbb{D}$为全纯函数,那么
	$$
	\left| \frac{f(z)-f(w)}{1-\overline{f(z)}f(w)} \right|
	\le
	\left| \frac{z-w}{1-\overline{z}w} \right|,\qquad
	z,w\in\mathbb{D}
	$$
\end{lemma}

\begin{proof}
	首先容易证明对于$z,w\in\overline{\mathbb{D}}$,当$\overline{w}z\ne1$时,成立
	$$
	\left|\frac{w-z}{1-\overline{w}z}\right|\le1
	$$
	当且仅当$|z|=1$或$|w|=1$时等号成立。
	
	对于$w\in\mathbb{D}$,定义映射
	\function{\varphi_w}{\mathbb{D}}{\mathbb{D}}{z}{\frac{w-z}{1-\overline{w}z}}
	我们来证明$\varphi_w$为全纯双射。注意到
	$$
	\lim_{h\to0}\frac{\varphi_w(z+h)-\varphi_w(z)}{h}=\lim_{h\to0}\frac{|w|^2-1}{(1-\overline{w}(z+h))(1-\overline{w}z)}=\frac{|w|^2-1}{(1-\overline{w}z)^2}
	$$
	因此$\varphi_w$为全纯映射。同时注意到
	$$
	(\varphi_w\circ\varphi_w)(z)=z
	$$
	因此$\varphi_w$为双射。
	
	由于$\varphi_w(w)=0$,那么$\varphi^{-1}_w(0)=w$。考察映射
	$$
	\psi_w=\varphi_{f(w)}\circ f \circ \varphi_w^{-1}
	$$
	由于$\varphi_w$和$f$均为$\mathbb{D}\to\mathbb{D}$上的全纯函数,那么$\psi_w$为为$\mathbb{D}\to\mathbb{D}$上的全纯函数,且
	$$
	\psi_w(0)=(\varphi_{f(w)}\circ f \circ \varphi_w^{-1})(0)=0
	$$
	于是由Schwartz引理\ref{lem:Schwartz引理},对于任意$z\in\mathbb{D}$,成立
	$$
	|\psi_w(z)|\le|z|
	$$
	即
	$$
	|(\varphi_{f(w)}\circ f \circ \varphi_w^{-1})(z)|\le|z|
	$$
	而$\varphi_w$为双射,因此存在$z'\in\mathbb{D}$,使得成立$z=\varphi_w(z')$,因此
	$$
	|(\varphi_{f(w)}\circ f )(z')|\le|\varphi_w(z')|
	$$
	进而
	$$
	\left| \frac{f(w)-f(z')}{1-\overline{f(w)}f(z')} \right|
	\le
	\left| \frac{w-z'}{1-\overline{w}z'} \right|
	$$
	由$z'$与$w$的任意性,原命题得证!
\end{proof}

\begin{lemma}{}{Landou引理的引理1}
	对于单位圆盘$\mathbb{D}$上的全纯函数$f$,如果$f(\mathbb{D})\sub M\mathbb{D}$,$|f(0)|\ne 0$,那么当$|z|=r<|f(0)|<M$时,成立%
	$$
	|f(z)|\ge\frac{M(|f(0)|-Mr)}{M-r|f(0)|}
	$$
\end{lemma}

\begin{proof}
	当$M=1$时,由Schwartz-Pick引理\ref{lem:Schwartz-Pick引理}%
	$$
	|z|\ge \left| \frac{f(z)-f(0)}{1-\overline{f(z)}f(0)} \right|
	,\qquad
	z\in\mathbb{D}
	$$
	从而%
	$$
	1-|z|^2\le
	1-\left| \frac{f(z)-f(0)}{1-\overline{f(z)}f(0)} \right|^2
	=\frac{(1-|f(z)|^2)(1-|f(0)|^2)}{|1-\overline{f(z)}f(0)|^2}
	\le\frac{(1-|f(z)|^2)(1-|f(0)|^2)}{(1-|f(z)||f(0)|)^2}
	$$
	因此%
	$$
	|z|^2\ge 
	1-\frac{(1-|f(z)|^2)(1-|f(0)|^2)}{(1-|f(z)||f(0)|)^2}
	=\frac{(|f(z)|-|f(0)|)^2}{(1-|f(z)||f(0)|)^2}
	$$
	进而%
	$$
	|z|\ge \frac{||f(z)|-|f(0)||}{1-|f(z)||f(0)|}
	$$
	解之%
	$$
	|f(z)|\ge\frac{|f(0)|-|z|}{1-|z||f(0)|}
	=\frac{|f(0)|-r}{1-r|f(0)|}
	$$
	
	当$M\ne 1$时,令$g=f/M$,从而由
	$$
	|g(z)|\ge\frac{|g(0)|-|z|}{1-|z||g(0)|}
	=\frac{|g(0)|-r}{1-r|g(0)|}
	$$
	可得%
	$$
	\frac{|f(z)|}{M}\ge\frac{\frac{|f(0)|}{M}-|z|}{1-|z|\frac{|f(0)|}{M}}
	=\frac{\frac{|f(0)|}{M}-r}{1-r\frac{|f(0)|}{M}}
	\iff
	|f(z)|\ge\frac{M(|f(0)|-Mr)}{M-r|f(0)|}
	$$
\end{proof}

\begin{lemma}{}{Landou引理的引理2}
	对于单位圆盘$\mathbb{D}$上的全纯函数$f$,如果$f(\mathbb{D})\sub M\mathbb{D}$,且$f(0)=0$,$f'(0)=1$,那么$M\ge 1$,且$f$在$\eta\mathbb{D}$中为单射,其中$\eta=1/(M+\sqrt{M^2-1})$。
\end{lemma}

\begin{proof}
	作Taylor展式%
	$$
	f(z)=\sum_{n=0}^{\infty}\frac{g^{(n)}(0)}{n!}z^n=\sum_{n=0}^{\infty}a_nz^n
	$$
	由Cauchy不等式%
	$$
	|f^{(n)}(0)| \le \frac{n!}{r^n}\sup_{|z|=r}|f(z)|
	< \frac{n!}{r^n}M,\qquad r<1
	$$
	从而%
	$$
	|a_n|=\frac{|f^{(n)}(0)|}{n!}\le\frac{M}{r^n},\qquad n\in\N,r<1
	$$
	令$r\to 1^-$,从而%
	$$
	|a_n|\le M,\qquad n\in\N
	$$
	而$|a_1|=|f'(0)|=1$,从而$M\ge 1$。
	
	若$f$在$\eta\mathbb{D}$中不为单射,则存在$z_1\ne z_2\in\mathbb{D}$,使得成立$f(z_1)=f(z_2)=\beta$。不妨$|z_1|\le |z_2|=\rho<1/M$。令%
	$$
	g(z)=\frac{\frac{\beta}{M}-\frac{f(z)}{M}}{1-\frac{\rho}{M}\frac{f(z)}{M}}
	=\frac{M(\beta-f(z))}{M^2-\beta f(z)}
	$$
	则$|g|<M$,且$g(z_1)=g(z_2)=0$。再令%
	$$
	h(z)=\frac{g(z)(1-\overline{z}_1z)(1-\overline{z}_2z)}{(z-z_1)(z-z_2)}
	$$
	则$h$在$\mathbb{D}$内全纯。断言:$|h|<M$。事实上,由最大模原理,$|h|$在$\partial\mathbb{D}$上取到;而$z\to\partial\mathbb{D}$,$|g(z)|<M$,从而$|h|<M$。因此
	$$
	|h(0)|=\frac{|g(0)|}{|z_1z_2|}<M
	$$
	而$|g(0)|\le \beta$,则$\beta<M|z_1z_2|<M\rho^2$。令%
	$$
	\varphi(z)=\begin{cases}
		f(z)/z,\qquad & z\ne 0\\
		f'(0)=1,\qquad & z=0
	\end{cases}
	$$
	则$\varphi$在$\mathbb{D}$内全纯,且$|\varphi|<M$。由引理\ref{lem:Landou引理的引理1},当$|z|=\rho<1/M$时%
	$$
	|\varphi(z)|\ge\frac{M(\varphi(0)-M\rho)}{M-\varphi(0)\rho}
	=\frac{M(1-M\rho)}{M-\rho}\implies
	|f(z)|\ge\frac{M(1-M\rho)}{M-\rho}|z|
	$$
	结合%
	$$
	\beta=|f(z_2)|\ge\frac{M(1-M\rho)}{M-\rho}|z_2|
	=\frac{M(1-M\rho)}{M-\rho}\rho\implies
	M\rho^2\ge \frac{M(1-M\rho)}{M-\rho}\rho
	\implies
	\rho\ge\frac{1}{M+\sqrt{M^2-1}}
	$$
	可得要使得$f$在$\rho\mathbb{D}$中不为单射,从而当$\rho<1/(M+\sqrt{M^2-1})$时,$f$在$\rho\mathbb{D}$中为单射。
\end{proof}

\begin{theorem}{Landou引理}{Landou引理}
	对于单位圆盘$\mathbb{D}$上的全纯函数$f$,如果$f(0)=0$,$f(\mathbb{D})\sub\mathbb{D}$,$0<f'(0)=\alpha\le 1$,那么$f$在$\eta\mathbb{D}$上为单射,且$\eta^2\mathbb{D}\sub f\left(\eta\mathbb{D}\right)$,其中$\eta=\alpha/(1+\sqrt{1-\alpha^2})$。
\end{theorem}

\begin{proof}
	令$F(z)=f(z)/\alpha$,则$F(0)=0$,$F'(0)=1$,且$F(\mathbb{D})\sub\mathbb{D}/\alpha$。由引理\ref{lem:Landou引理的引理2},则$F$在$\eta\mathbb{D}$中为单射,其中%
	$$
	\eta=\frac{1}{M+\sqrt{M^2-1}}
	=\frac{1}{\frac{1}{\alpha}+\sqrt{\frac{1}{\alpha^2}-1}}
	=\frac{\alpha}{1+\sqrt{1-\alpha^2}}
	$$
	当$|z|=\eta$时,由引理\ref{lem:Landou引理的引理2}%
	$$
	|F(z)|\ge\frac{M(1-M\eta)}{M-\eta}|z|
	$$
	从而%
	$$
	|f(z)|\ge\frac{1-\frac{1}{\alpha}\eta}{\frac{1}{\alpha}-\eta}\eta=\frac{\alpha-\eta}{1-\alpha\eta}\eta\ge\eta^2
	$$
	由Rouché定理,$\eta^2\mathbb{D}\sub f\left(\eta\mathbb{D}\right)$。
\end{proof}

\end{document}
