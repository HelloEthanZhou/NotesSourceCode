\documentclass{beamer}

% 导入metropolis主题
\usetheme{metropolis}

% 使用xeCJK宏包支持中文
\usepackage{xeCJK}
% 设置中文字体
\setCJKmainfont{Microsoft YaHei}

% 导入unicode-math宏包
\usepackage{unicode-math}
% 设置数学公式字体为Computer Modern
\setmathfont{Latin Modern Math}

% 引入宏包
\usepackage[all]{xy}
\usepackage{bbm, svg, graphicx, float, extpfeil, amsmath, amssymb, mathrsfs, mathalpha, hyperref}


% 定义命令
\newcommand{\N}{\mathbb{N}}            % 自然数集合
\newcommand{\R}{\mathbb{R}}            % 实数集合
\newcommand{\C}{\mathbb{C}}  		   % 复数集合
\newcommand{\Q}{\mathbb{Q}}            % 有理数集合
\newcommand{\Z}{\mathbb{Z}}            % 整数集合
\newcommand{\sub}{\subset}             % 包含
\newcommand{\im}{\mathrm{im }}           % 像
\newcommand{\lang}{\langle}            % 左尖括号
\newcommand{\rang}{\rangle}            % 右尖括号
\newcommand{\bs}{\boldsymbol}          % 向量加黑
\newcommand{\dd}{\mathrm{d}}           % 微分d
\newcommand{\pll}{\kern 0.56em/\kern -0.8em /\kern 0.56em} % 平行
\newcommand{\function}[5]{
	\begin{align*}
		#1:\begin{aligned}[t]
			#2 &\longrightarrow #3\\
			#4 &\longmapsto #5
		\end{aligned}
	\end{align*}
}                                     % 映射

% 设置演示文稿标题、副标题、日期、作者和机构信息
\title{范畴论介绍}
\date{\today}
\author{周梦轩}
\institute{河北工业大学}

\begin{document}

% 创建标题页
\maketitle

% 创建目录页
\begin{frame}{目录}
	% 设置目录样式,显示章节编号
	\setbeamertemplate{section in toc}[sections numbered]
	% 显示目录,隐藏所有子章节
	\tableofcontents[hideallsubsections]
\end{frame}


\section{引言}

\begin{frame}{从{\bf 同构}讲起}
	在分析学中,称Banach空间$X$与$Y$同构,并记作$X\cong Y$,如果存在双射$f:X\to Y$,使得$f$与$f^{-1}$均为有界线性映射。
	
	在代数学中,称群$X$与$Y$同构,并记作$X\cong Y$,如果存在双射$f:X\to Y$,使得对于任意$x,y\in X$,成立$f(xy)=f(x)f(y)$。
	
	在拓扑学中,称拓扑空间$X$与$Y$同构,并记作$X\cong Y$,如果存在双射$f:X\to Y$,使得$f$与$f^{-1}$均为连续映射。
\end{frame}

\section{范畴的定义与示例}

\begin{frame}{范畴 category}
	范畴$\mathsf{C}$包含:
	\begin{itemize}
		\item {\bf 对象(object)}:$\mathrm{Obj}(\mathsf{C})$
		\item {\bf 态射(morphism)}:$\mathrm{Hom}_\mathsf{C}(A,B)$,其中称$A$为源(source),$B$为目标(target)。
	\end{itemize}
\end{frame}

\begin{frame}
	成立如下公理:
	\begin{enumerate}
		\item {\bf 态射复合}:对于任意对象$A,B,C\in\mathrm{Obj}(\mathsf{C})$,存在态射复合
		\begin{align*}
			\circ :\begin{aligned}[t]
				\mathrm{Hom}_\mathsf{C}(A,B)\times \mathrm{Hom}_\mathsf{C}(B,C)&\longrightarrow \mathrm{Hom}_\mathsf{C}(A,C)\\
				(f,g)&\longmapsto g\circ f
			\end{aligned}
		\end{align*}
		\item {\bf 恒等态射}:对于任意对象$S\in\mathrm{Obj}(\mathsf{C})$,存在恒等态射$\mathbbm{1}_S\in\mathrm{Hom}_\mathsf{C}(S,S)$,使得对于任意态射$f\in \mathrm{Hom}_\mathsf{C}(A,B)$,成立
		$$
		f\circ\mathbbm{1}_A=\mathbbm{1}_B\circ f=f
		$$
		\item {\bf 结合律}:对于任意态射$f\in \mathrm{Hom}_\mathsf{C}(A,B),g\in \mathrm{Hom}_\mathsf{C}(B,C),h\in \mathrm{Hom}_\mathsf{C}(C,D)$,成立
		$$
		(h\circ g)\circ f=h\circ (g\circ f)
		$$
	\end{enumerate}
\end{frame}

\begin{frame}{范畴示例}
	集合范畴$\mathsf{Set}$:
	\begin{itemize}
		\item $\mathrm{Obj}(\mathsf{Set})=\{ \text{集合} \}$
		\item $\mathrm{Hom}_{\mathsf{Set}}(A,B)=\{ \text{映射 }f:A\to B \}$
	\end{itemize}

	矩阵范畴$\mathsf{V}$:
	\begin{itemize}
		\item $\mathrm{Obj}(\mathsf{V})=\N^*$
		\item $\mathrm{Hom}_{\mathsf{V}}(m,n)=\{ \{a_{ij}\}_{m\times n}\mid a_{ij}\in\R \}$
	\end{itemize}

	群范畴$\mathsf{Grp}$:
	\begin{itemize}
		\item $\mathrm{Obj}(\mathsf{Grp})=\{ \text{群 }(G,*) \}$
		\item $\mathrm{Hom}_{\mathsf{Grp}}((G,*),(H,\star))=\{ \text{映射 }\varphi:G\to H\mid \varphi(x*y)=\varphi(x)\star \varphi(y) \}$
	\end{itemize}
\end{frame}

\section{态射的定义}

\begin{frame}{左逆态射,右逆态射}
	\textbf{左逆态射:}
	对于范畴$\mathsf{C}$,以及对象$A,B\in\mathrm{Obj}(\mathsf{C})$,称态射$g\in\mathrm{Hom}_\mathsf{C}(B,A)$为态射$f\in\mathrm{Hom}_\mathsf{C}(A,B)$的左逆态射,如果成立$g\circ f=\mathbbm{1}_A$。
	
	\textbf{右逆态射:}
	对于范畴$\mathsf{C}$,以及对象$A,B\in\mathrm{Obj}(\mathsf{C})$,称态射$g\in\mathrm{Hom}_\mathsf{C}(B,A)$为态射$f\in\mathrm{Hom}_\mathsf{C}(A,B)$的右逆态射,如果成立$f\circ g=\mathbbm{1}_B$。
\end{frame}

\begin{frame}{同构态射,同构}
	\textbf{同构态射:}
	对于范畴$\mathsf{C}$,以及对象$A,B\in\mathrm{Obj}(\mathsf{C})$,称态射$f\in\mathrm{Hom}_\mathsf{C}(A,B)$为同构态射,如果存在态射$g\in\mathrm{Hom}_\mathsf{C}(B,A)$,使得成立
	$$
	g\circ f=\mathbbm{1}_A,\qquad f\circ g=\mathbbm{1}_B
	$$
	
	\textbf{同构:}
	对于范畴$\mathsf{C}$,称对象$A,B\in\mathrm{Obj}(\mathsf{C})$是同构的,且记作$A\cong B$,如果存在同构态射$f\in\mathrm{Hom}_\mathsf{C}(A,B)$。
\end{frame}

\begin{frame}{单态射,满态射}
	\textbf{单态射:}
	对于范畴$\mathsf{C}$,以及对象$A,B\in\mathrm{Obj}(\mathsf{C})$,称态射$f\in\mathrm{Hom}_\mathsf{C}(A,B)$为单态射,如果对于任意对象$Z\in\mathrm{Obj}(\mathsf{C})$,以及任意态射$\alpha_1,\alpha\in\mathrm{Hom}_\mathsf{C}(Z,A)$,成立
	$$
	f\circ \alpha_1=f\circ \alpha_2
	\implies
	\alpha_1=\alpha_2
	$$
	
	\textbf{满态射:}
	对于范畴$\mathsf{C}$,以及对象$A,B\in\mathrm{Obj}(\mathsf{C})$,称态射$f\in\mathrm{Hom}_\mathsf{C}(A,B)$为满态射,如果对于任意对象$Z\in\mathrm{Obj}(\mathsf{C})$,以及任意态射$\beta_1,\beta\in\mathrm{Hom}_\mathsf{C}(B,Z)$,成立
	$$
	\beta_1\circ f=\beta_2\circ f
	\implies
	\beta_1=\beta_2
	$$
\end{frame}

\section{集合范畴中的一个命题}

\begin{frame}{一个命题}
	对于映射$f:A\to B$,如下命题等价。
	\begin{enumerate}
		\item $f$为单射。
		\item $f$存在左逆。
		\item $f$为单态射。
	\end{enumerate}
\end{frame}

\begin{frame}{单射$\implies$存在左逆}
	定义映射
	\function{g}{B}{A}{f(a)}{a}
	
	首先来验证$g$的定义是良好的。取$a_1,a_2\in A$,满足$f(a_1)=f(a_2)$,由$f$的单射性,$a_1=a_2$,于是$g$定义良好。
	
	其次来验证$g\circ f=\mathbbm{1}_A$。任取$a\in A$,注意到
	$$
	(g\circ f)(a)=g(f(a))=a
	$$
	那么$g\circ f=\mathbbm{1}_A$。
	
	综合这两点,$f$存在左逆$g$。
\end{frame}

\begin{frame}{存在左逆$\implies$单态射}
	记$f$的左逆为$g$,那么$g\circ f=\mathbbm{1}_A$。
	
	任取$\alpha_1,\alpha_2$,满足$f\circ \alpha_1=f\circ \alpha_2$,那么
	$$
	\alpha_1=\mathbbm{1}_A\circ \alpha_1=g\circ f\circ \alpha_1=g\circ f\circ \alpha_2=\mathbbm{1}_A\circ \alpha_2=\alpha_2
	$$
	于是$f$是单态射。
\end{frame}

\begin{frame}{单态射$\implies$单射}
	任取$a_1,a_2\in A$,满足$f(a_1)=f(a_2)$。
	
	定义$\alpha_1:Z\to \{a_1\}$和$\alpha_2:Z\to \{a_2\}$,任取$z\in Z$,注意到
	$$
	(f\circ \alpha_1)(z)=f(\alpha_1(z))=f(a_1)=f(a_2)=f(\alpha_2(z))=(f\circ \alpha_2)(z)
	$$
	因此$f\circ\alpha_1=f\circ\alpha_2$,于是$\alpha_1=\alpha_2$,即$a_1=a_2$,进而$f$是单射。
\end{frame}

\begin{frame}{总结}
	\begin{columns}[T]
		
		\metroset{block=fill}
		
		\column{0.5\textwidth}
		
		\begin{block}{定理1}
			对于映射$f:A\to B$,如下命题等价。
			\begin{enumerate}
				\item $f$为单射。
				\item $f$存在左逆。
				\item $f$为单态射。
			\end{enumerate}
		\end{block}
		
		\column{0.5\textwidth}
		
		\begin{block}{定理2}
			对于映射$f:A\to B$,如下命题等价。
			\begin{enumerate}
				\item $f$为满射。
				\item $f$存在右逆。
				\item $f$为满态射。
			\end{enumerate}
		\end{block}
		
	\end{columns}

\end{frame}

\section{命题的反例}

\begin{frame}{反例}
	在群范畴$\mathsf{Grp}$中,群同态映射
	\begin{align*}
		\varphi:\begin{aligned}[t]
			\Z/3\Z &\longrightarrow S_3\\
			[0]_3 &\longmapsto (1)\\
			[1]_3 &\longmapsto (132)\\
			[2]_3 &\longmapsto (123)
		\end{aligned}
	\end{align*}
	为单态射,但是不存在左逆,因为群同态映射$S_3\to \Z/3\Z$只能为平凡映射。
\end{frame}

\begin{frame}[standout]
	Questions?
\end{frame}

\begin{frame}[standout]
	Thank you!
\end{frame}

\end{document}
