\documentclass[lang = cn, scheme = chinese, thmcnt = section]{elegantbook}
% elegantbook      设置elegantbook文档类
% lang = cn        设置中文环境
% scheme = chinese 设置标题为中文
% thmcnt = section 设置计数器


%% 1.封面设置

\title{范畴论Presentation}                % 文档标题

\author{若水}               % 作者

\extrainfo{上善若水任方圆}   % 箴言

\date{\today}               % 日期

\logo{PiCreatures_happy.pdf}        % 设置Logo

\cover{阿基米德螺旋曲线.pdf}          % 设置封面图片

% 修改标题页的色带
\definecolor{customcolor}{RGB}{135, 206, 250} 
% 定义一个名为customcolor的颜色,RGB颜色值为(135, 206, 250)

\colorlet{coverlinecolor}{customcolor}     % 将coverlinecolor颜色设置为customcolor颜色

%% 2.目录设置
\setcounter{tocdepth}{3}  % 目录深度为3

%% 3.引入宏包
\usepackage[all]{xy}
\usepackage{bbm, svg, graphicx, float, extpfeil, amsmath, amssymb, mathrsfs, mathalpha, hyperref, graphicx, centernot}


%% 4.定义命令
\newcommand{\N}{\mathbb{N}}            % 自然数集合
\newcommand{\R}{\mathbb{R}}            % 实数集合
\newcommand{\C}{\mathbb{C}}  		   % 复数集合
\newcommand{\Q}{\mathbb{Q}}            % 有理数集合
\newcommand{\Z}{\mathbb{Z}}            % 整数集合
\newcommand{\sub}{\subset}             % 包含
\newcommand{\im}{\text{im }}           % 像
\newcommand{\lang}{\langle}            % 左尖括号
\newcommand{\rang}{\rangle}            % 右尖括号
\newcommand{\function}[5]{
	\begin{align*}
		#1:\begin{aligned}[t]
			#2 &\longrightarrow #3\\
			#4 &\longmapsto #5
		\end{aligned}
	\end{align*}
}                                     % 函数

\newcommand{\lhdneq}{%
	\mathrel{\ooalign{$\lneq$\cr\raise.22ex\hbox{$\lhd$}\cr}}} % 真正规子群

\newcommand{\rhdneq}{%
	\mathrel{\ooalign{$\gneq$\cr\raise.22ex\hbox{$\rhd$}\cr}}} % 真正规子群

\newcommand{\upiff}{\mathrel{\rotatebox[origin=c]{90}{$\iff$}}} % 竖着的等价


\begin{document}

\maketitle       % 创建标题页

\frontmatter     % 开始前言部分

\tableofcontents % 创建目录

\mainmatter      % 开始正文部分

\chapter{范畴论Presentation}

\section{导入}

从“同构”讲起。

在分析学中,称Banach空间$X$与$Y$同构,并记作$X\cong Y$,如果存在双射$f:X\to Y$,使得$f$与$f^{-1}$均为有界线性映射。

在代数学中,称群$X$与$Y$同构,并记作$X\cong Y$,如果存在双射$f:X\to Y$,使得对于任意$x,y\in X$,成立$f(xy)=f(x)f(y)$。

在拓扑学中,称拓扑空间$X$与$Y$同构,并记作$X\cong Y$,如果存在双射$f:X\to Y$,使得$f$与$f^{-1}$均为连续映射。

\section{范畴论}

\begin{definition}{范畴 category}
	范畴$\mathsf{C}$包含:
	\begin{itemize}
		\item {\bf 对象(object)}:$\mathrm{Obj}(\mathsf{C})$
		\item {\bf 态射(morphism)}:$\mathrm{Hom}_\mathsf{C}(A,B)$,其中称$A$为源(source),$B$为目标(target)。
	\end{itemize}
	成立如下公理:
	\begin{enumerate}
		\item {\bf 态射复合}:对于任意对象$A,B,C\in\mathrm{Obj}(\mathsf{C})$,存在态射复合
		\begin{align*}
			\circ :\begin{aligned}[t]
				\mathrm{Hom}_\mathsf{C}(A,B)\times \mathrm{Hom}_\mathsf{C}(B,C)&\longrightarrow \mathrm{Hom}_\mathsf{C}(A,C)\\
				(f,g)&\longmapsto g\circ f
			\end{aligned}
		\end{align*}
		\item {\bf 恒等态射}:对于任意对象$S\in\mathrm{Obj}(\mathsf{C})$,存在恒等态射$\mathbbm{1}_S\in\mathrm{Hom}_\mathsf{C}(S,S)$,使得对于任意态射$f\in \mathrm{Hom}_\mathsf{C}(A,B)$,成立
		$$
		f\circ\mathbbm{1}_A=\mathbbm{1}_B\circ f=f
		$$
		\item {\bf 结合律}:对于任意态射$f\in \mathrm{Hom}_\mathsf{C}(A,B),g\in \mathrm{Hom}_\mathsf{C}(B,C),h\in \mathrm{Hom}_\mathsf{C}(C,D)$,成立
		$$
		(h\circ g)\circ f=h\circ (g\circ f)
		$$
	\end{enumerate}
\end{definition}

\begin{example}
	集合范畴:
	\begin{itemize}
		\item $\mathrm{Obj}(\mathsf{Set})=\{ \text{集合} \}$
		\item $\mathrm{Hom}_{\mathsf{Set}}(A,B)=\{ \text{映射 }f:A\to B \}$
	\end{itemize}
\end{example}

\begin{example}
	矩阵范畴:
	\begin{itemize}
		\item $\mathrm{Obj}(\mathsf{V})=\N^*$
		\item $\mathrm{Hom}_{\mathsf{V}}(m,n)=\{ \{a_{ij}\}_{m\times n}:a_{ij}\in\R \}$
	\end{itemize}
\end{example}

\begin{example}
	群范畴:
	\begin{itemize}
		\item $\mathrm{Obj}(\mathsf{Grp})=\{ \text{群 }(G,*) \}$
		\item $\mathrm{Hom}_{\mathsf{Grp}}((G,*),(H,\star))=\{ \text{映射 }\varphi:G\to H:\varphi(x*y)=\varphi(x)\star \varphi(y) \}$
	\end{itemize}
\end{example}

\begin{definition}{左逆态射 left-inverse morphism}
	对于范畴$\mathsf{C}$,以及对象$A,B\in\mathrm{Obj}(\mathsf{C})$,称态射$g\in\mathrm{Hom}_\mathsf{C}(B,A)$为态射$f\in\mathrm{Hom}_\mathsf{C}(A,B)$的左逆态射,如果成立$g\circ f=\mathbbm{1}_A$。
\end{definition}

\begin{definition}{右逆态射 right-inverse morphism}
	对于范畴$\mathsf{C}$,以及对象$A,B\in\mathrm{Obj}(\mathsf{C})$,称态射$g\in\mathrm{Hom}_\mathsf{C}(B,A)$为态射$f\in\mathrm{Hom}_\mathsf{C}(A,B)$的右逆态射,如果成立$f\circ g=\mathbbm{1}_A$。
\end{definition}

\begin{definition}{同构态射 isomorphism}
	对于范畴$\mathsf{C}$,以及对象$A,B\in\mathrm{Obj}(\mathsf{C})$,称态射$f\in\mathrm{Hom}_\mathsf{C}(A,B)$为同构态射,如果存在态射$g\in\mathrm{Hom}_\mathsf{C}(B,A)$,使得成立
	$$
	g\circ f=\mathbbm{1}_A,\qquad f\circ g=\mathbbm{1}_B
	$$
\end{definition}

\begin{definition}{同构的 isomorphic}
	对于范畴$\mathsf{C}$,称对象$A,B\in\mathrm{Obj}(\mathsf{C})$是同构的,且记作$A\cong B$,如果存在同构态射$f\in\mathrm{Hom}_\mathsf{C}(A,B)$。
\end{definition}

\begin{definition}{单态射 monomorphism}
	对于范畴$\mathsf{C}$,以及对象$A,B\in\mathrm{Obj}(\mathsf{C})$,称态射$f\in\mathrm{Hom}_\mathsf{C}(A,B)$为单态射,如果对于任意对象$Z\in\mathrm{Obj}(\mathsf{C})$,以及任意态射$\alpha_1,\alpha\in\mathrm{Hom}_\mathsf{C}(Z,A)$,成立
	$$
	f\circ \alpha_1=f\circ \alpha_2
	\implies
	\alpha_1=\alpha_2
	$$
\end{definition}

\begin{definition}{满态射 epimorphism}
	对于范畴$\mathsf{C}$,以及对象$A,B\in\mathrm{Obj}(\mathsf{C})$,称态射$f\in\mathrm{Hom}_\mathsf{C}(A,B)$为满态射,如果对于任意对象$Z\in\mathrm{Obj}(\mathsf{C})$,以及任意态射$\beta_1,\beta\in\mathrm{Hom}_\mathsf{C}(B,Z)$,成立
	$$
	\beta_1\circ f=\beta_2\circ f
	\implies
	\beta_1=\beta_2
	$$
\end{definition}

\begin{proposition}{存在左逆$\implies$单态射}{存在左逆则为单态射}
	对于范畴$\mathsf{C}$,以及对象$A,B\in\mathrm{Obj}(\mathsf{C})$,如果态射$f\in\mathrm{Hom}_\mathsf{C}(A,B)$存在左逆态射,那么态射$f\in\mathrm{Hom}_\mathsf{C}(A,B)$为单态射。
\end{proposition}

\begin{proof}
	如果态射$f\in\mathrm{Hom}_\mathsf{C}(A,B)$存在左逆态射$g\in\mathrm{Hom}_\mathsf{C}(B,A)$,那么任取$\alpha_1,\alpha_2$,满足$f\circ \alpha_1=f\circ \alpha_2$,由于
	$$
	\alpha_1=\mathbbm{1}\circ \alpha_1=g\circ f\circ \alpha_1=g\circ f\circ \alpha_2=\mathbbm{1}\circ \alpha_2=\alpha_2
	$$
	于是$f$是单态射。
\end{proof}

\begin{proposition}{存在右逆$\implies$满态射}
	对于范畴$\mathsf{C}$,以及对象$A,B\in\mathrm{Obj}(\mathsf{C})$,如果态射$f\in\mathrm{Hom}_\mathsf{C}(A,B)$存在右逆态射,那么态射$f\in\mathrm{Hom}_\mathsf{C}(A,B)$为满态射。
\end{proposition}

\begin{proof}
	如果态射$f\in\mathrm{Hom}_\mathsf{C}(A,B)$存在右逆态射$g\in\mathrm{Hom}_\mathsf{C}(B,A)$,那么任取$\beta_1,\beta_2$,满足$\beta_1\circ f=\beta_2\circ f$,由于
	$$
	\beta_1=\beta_1\circ \mathbbm{1}=\beta_1\circ f\circ g=\beta_2\circ f\circ g=\beta_2\circ \mathbbm{1}=\beta_2
	$$
	于是$f$是满态射。
\end{proof}

\begin{example}
	在群范畴$\mathsf{Grp}$中,群同态映射
	\begin{align*}
		\varphi:\begin{aligned}[t]
			\Z/3\Z &\longrightarrow S_3\\
			[0]_3 &\longmapsto (1)\\
			[1]_3 &\longmapsto (132)\\
			[2]_3 &\longmapsto (123)\\
		\end{aligned}
	\end{align*}
	为单态射,但是不存在左逆,因为群同态映射$S_3\to \Z/3\Z$只能为平凡映射。
\end{example}

\section{函数}

\begin{definition}{函数 function}
	称$f$为定义域为$A$,陪域为$B$的函数,如果对于任意$a\in A$,存在且存在唯一$b\in B$,使得成立$f(a)=b$。记作
	\begin{align*}
		f: \begin{aligned}[t]
			A&\longrightarrow B\\
			a&\longmapsto f(a)
		\end{aligned}
	\end{align*}
\end{definition}

\begin{definition}{单射 injection}
	称函数$f:A\to B$是单的,如果$f(a_1)=f(a_2)$,那么$a_1=a_2$。单射记作$f:A\hookrightarrow B$。
\end{definition}

\begin{definition}{满射 surjection}
	称函数$f:A\to B$是满的,如果对于任意$b\in B$,存在$a\in A$,使得成立$f(a)=b$。满射记作$f:A\twoheadrightarrow B$。
\end{definition}

\begin{definition}{双射 bijection}
	称函数$f:A\to B$是双射,并记作$f:A\xrightarrow{\sim}B$,如果其既单又满。
\end{definition}

\begin{definition}{同构的 isomorphic}
	称集合$A$与$B$为同构的,并记做$A\cong B$,如果存在双射$f:A\to B$。
\end{definition}

\begin{definition}{逆 inverse}
	对于双射$f:A\to B$,定义其逆为
	\begin{align*}
		f^{-1}:\begin{aligned}[t]
			B&\longrightarrow A\\
			f(a)&\longmapsto a
		\end{aligned}
	\end{align*}
\end{definition}

\begin{definition}{左逆 left-inverse}
	称函数$g:B\to A$为函数$f:A\to B$的左逆,如果成立$g\circ f=\mathbbm{1}_A$。
\end{definition}

\begin{definition}{右逆 right-inverse}
	称函数$g:B\to A$为函数$f:A\to B$的右逆,如果成立$f\circ g=\mathbbm{1}_B$。
\end{definition}

\begin{definition}{单态射 monomorphism}
	称函数$f:A\to B$是单态射,如果对于任意集合$Z$,以及任意函数$\alpha_1,\alpha_2:Z\to A$,成立
	$$
	f\circ \alpha_1=f\circ \alpha_2\implies\alpha_1=\alpha_2
	$$
\end{definition}

\begin{definition}{满态射 epimorphism}
	称函数$f:A\to B$是满态射,如果对于任意集合$Z$,以及任意函数$\beta_1,\beta_2:B\to Z$,成立
	$$
	\beta_1\circ f=\beta_2\circ f\implies \beta_1=\beta_2
	$$
\end{definition}

\begin{theorem}{}{1.2.2}
	对于函数$f:A\to B$,如下命题等价。
	\begin{enumerate}
		\item $f$为单射。
		\item $f$存在左逆。
		\item $f$为单态射。
	\end{enumerate}
\end{theorem}

\begin{proof}
	1$\implies$2:如果$f$为单射,定义函数
	\function{g}{\text{im }f}{A}{f(a)}{a}
	首先来验证$g$的定义是良好的。取$a_1,a_2\in A$,满足$f(a_1)=f(a_2)$,由$f$的单射性,$a_1=a_2$,于是$g$定义良好。其次来验证$g\circ f=\mathbbm{1}$。任取$a\in A$,注意到$(g\circ f)(a)=g(f(a))=a$,那么$g\circ f=\mathbbm{1}$。综合这两点,$f$存在左逆$g$。
	
	1$\implies$3:如果$f$为单射,任取$\alpha_1,\alpha_2:Z\to A$,满足$f\circ \alpha_1=f\circ \alpha_2$。任取$z\in Z$,注意到
	$$
	f(\alpha_1(z))=(f\circ \alpha_1)(z)=(f\circ \alpha_2)(z)=f(\alpha_2(z))
	$$
	于是$\alpha_1=\alpha_2$,进而$f$是单态射。
	
	2$\implies $3:如果$f$存在左逆$g$,任取$\alpha_1,\alpha_2$,满足$f\circ \alpha_1=f\circ \alpha_2$,那么
	$$
	\alpha_1=\mathbbm{1}\circ \alpha_1=g\circ f\circ \alpha_1=g\circ f\circ \alpha_2=\mathbbm{1}\circ \alpha_2=\alpha_2
	$$
	于是$f$是单态射。
	
	2$\implies $1:如果$f$存在左逆$g$,任取$a_1,a_2\in A$,满足$f(a_1)=f(a_2)$,那么
	$$
	a_1=\mathbbm{1}(a_1)=(g\circ f)(a_1)=g(f(a_1))=g(f(a_2))=(g\circ f)(a_2)=\mathbbm{1}(a_2)=a_2
	$$
	于是$f$是单射。
	
	3$\implies$1:如果$f$是单态射,任取$a_1,a_2\in A$,满足$f(a_1)=f(a_2)$。定义$\alpha_1:Z\to \{a_1\}$和$\alpha_2:Z\to \{a_2\}$,任取$z\in Z$,注意到
	$$
	(f\circ \alpha_1)(z)=f(\alpha_1(z))=f(a_1)=f(a_2)=f(\alpha_2(z))=(f\circ \alpha_2)(z)
	$$
	因此$f\circ\alpha_1=f\circ\alpha_2$,于是$\alpha_1=\alpha_2$,即$a_1=a_2$,进而$f$是单射。
\end{proof}

\begin{theorem}{}{1.2.3}
	对于函数$f:A\to B$,如下命题等价。
	\begin{enumerate}
		\item $f$为满射。
		\item $f$存在右逆。
		\item $f$为满态射。
	\end{enumerate}
\end{theorem}

\begin{proof}
	1$\implies$2:如果$f$是满射,定义函数
	\function{g}{B}{A}{b}{a}
	这里要说明的是对于特别的$b\in B$,$f^{-1}(b)$中的元素可能不唯一,这时候任取其一即可,此时便说明$g$定义良好。然后我们来验证$f\circ g=\mathbbm{1}$。任取$b\in B$,注意到$(f\circ g)(b)=f(g(b))=f(f^{-1}(b))=b$,那么$f\circ g=\mathbbm{1}$,进而$f$存在右逆$g$。
	
	1$\implies$3:如果$f$为满射,任取$\beta_1,\beta_2$,满足$\beta_1\circ f=\beta_2\circ f$。任取$b\in B$,存在$a\in A$,使得成立$f(a)=b$,因此
	$$
	\beta_1(b)=\beta_1(f(a))=(\beta_1\circ f)(a)=(\beta_2\circ f)(a)=\beta_2(f(a))=\beta_2(b)
	$$
	于是$\beta_1=\beta_2$,进而$f$是满态射。
	
	2$\implies $3:如果$f$存在右逆$g$,任取$\beta_1,\beta_2$,满足$\beta_1\circ f=\beta_2\circ f$,那么
	$$
	\beta_1=\beta_1\circ \mathbbm{1}=\beta_1\circ f\circ g=\beta_2\circ f\circ g=\beta_2\circ \mathbbm{1}=\beta_2
	$$
	于是$f$是满态射。
	
	2$\implies$1:如果$f$存在右逆$g$,任取$b\in B$,注意到$g(b)\in A$,且$f(g(b))=(f\circ g)(b)=\mathbbm{1}(b)=b$,因此$f$为满射。
	
	3$\implies$1:如果$f$是满态射,任取$b\in B$。定义$\beta_1:B\to\{1\}$以及
	\begin{align*}
		\beta_2: \begin{aligned}[t]
			B &\longrightarrow \{0,1\}\\
			b &\longmapsto \begin{cases}
				1,\quad & b\in\mathrm{im}f\\
				0,\quad & b\in B\setminus\mathrm{im}f
			\end{cases}
		\end{aligned}
	\end{align*}
	任取$a\in A$,注意到
	$$
	(\beta_1\circ f)(a)=\beta_1(f(a))=1=\beta_2(f(a))=(\beta_2\circ f)(a)
	$$
	因此$\beta_1\circ f=\beta_2\circ f$,于是$\beta_1=\beta_2$,即$b\in\mathrm{im}f$,进而$f$是满射。
\end{proof}








	
\end{document}